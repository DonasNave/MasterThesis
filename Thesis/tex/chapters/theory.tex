\cast{Teoretická část}

%%%%%%%%%%%%%%%%%%%%%%%%%%%%%%%%%%%%%%%%%%%%%%%%%%%%%%%%%%%%%%%%%%%%%%%%%%%%%%%%%
%                                    .NET                                     %
%%%%%%%%%%%%%%%%%%%%%%%%%%%%%%%%%%%%%%%%%%%%%%%%%%%%%%%%%%%%%%%%%%%%%%%%%%%%%%%%%

\n{1}{Platforma .NET}

Platforma .NET od společnosti Microsoft představuje sadu nástrojů k vývoji aplikací v jazyce C\# a jeho derivátech. Tato platforma je multiplatformní a umožňuje vývoj aplikací pro operační systémy jako Windows, Linux, macOS ale i pro mobilní platformy. Vývojáři mohou využívat nástroje pro vývoj webových aplikací, desktopových aplikací, mobilních aplikací a dalších. Platforma .NET je postavena na dvou hlavních nástrojích. Prvním z nich je \textit{Common Language Runtime} (dále jen CLR), runtime prostředí zodpovídající za běh aplikací. Druhým nástrojem je \textit{dotnet CLI}, konzolový nástroj-rozhraní, zodpovědné za interakci s dílčími nástroji v platformě. \cite{Richter2012}

\n{2}{Historie}

Využití runtime prostředí, respektive v originální podobě virtuálního stroje, má historický původ. V dřívějších dobách byly programátoři limitování nutností kompilace kódu do nativní reprezentace přímo pro architekturu systému. Kód vytvořen pro jednu konkrétní architekturu se zpravidla neobešel bez modifikací, pokud měl fungovat i na odlišné architektuře.

V průběhu 90. let 20. století představila společnost Sun-Microsystems virtuální stroj Java Virtual Machine (JVM). Jedná se o komponentu runtime prostředí Javy, která zprostředkovává spuštění specifického kódu, správu paměti, vytváření tříd a typů a další. Kompilací Javy do tzv. Bytecode (Intermediate Language, zkráceně IL), tedy provedením mezikroku v procesu transformace zdrojového kódu do strojového kódu, je získána reprezentace programu, jenž běží na každém zařízení s implementovaným JVM. V rámci JVM dochází k finálnímu kroku a to interpretaci (JIT kompilaci) Bytecode do cílové architektury systému. 

Microsoft v reakci na JVM vydal v roce 2000 první .NET Framework, který umožňoval spouštět kód v jazyce C\# na operačním systému Windows. Cílem prvních verzí .NET Framework nebylo primárně umožnit vývoj pro různé zařízení a operační systémy, ale zprostředkovat lepší nástroje pro vývoj aplikací. KV roce 2014 byla vydána první multiplatformní verze .NETu. Byl vydán .NET Core, který umožňoval spouštět kód v jazyce C\# na operačních systémech Windows, Linux a macOS. \cite{Richter2012}

\n{2}{Nástroje .NET}

Platforma .NET zprostředkovává širokou sadu nástrojů za účelem tvorby, sestavení a spuštění aplikace. Mezi nejdůležitější lze zařadit následující:

\begin{itemize}
    \item \textbf{CLR} - běhové prostředí pro programy kompilované do IL
    \item \textbf{.NET CLI} (konzolové rozhraní) - konzolové rozhraní pro interakci s nástroji .NET
    \item \textbf{MSBuild} - buildovací engine pro kompilaci, testování a balíčkování aplikací
    \item \textbf{.NET SDK nástroje} - soubor nástrojů pro testování, debuggování a migraci .NET aplikací
    \item \textbf{Roslyn} - kompilační nástroj
    \item \textbf{NuGet} - balíčkovací manager 
\end{itemize}

\n{2}{Kompilace zdrojového kódu}

Kompilace je proces transformace zdrojového kódu do jiné podoby. Kód je zpravidla kompilován do podoby bližší cílové architektuře, ať je touto architekturou OS, případně konkrétní HW, nebo runtime prostředí (virtuální stroj). V rámci platformy .NET jsou k dispozici 2 hlavní fundamentálně odlišné režimy kompilace zdrojového kódu: kompilace pro běhové prostředí (CLR) do tzv. \emph{assembly} a kompilace do nativního kódu přímo pro cílovou architekturu (Native AoT).

\n{3}{Kompilace pro CLR}

Standartním výstupem sestavení aplikace v .NETu je transformace zdrojového kódu z vybraného podporovaného jazyka do assembly v jazyce IL. Tento výstupní IL se v .NET konkrétně navývaná Common Intermediate language (CIL) nebo také Microsoft Intermediate Language (MSIL). V případě jazyka C\# na platformě Windows slouží ke kompilaci spustitelný soubor \emph{csc.exe}. 

\n{4}{Výstup} 

Assembly s popisnými metadaty a IL (v případě režimu R2R i částečně nativním) kódem. Assembly typicky disponují příponou \emph{.dll}, případně jsou zabaleny do spustilného souboru dle cílové platformy a výstupu. Takovýto výstup je následně připraven buďto ke spuštění za pomocí CLR, případně pro využití a referenci při tvorbě dalšího .NET kódu.

Kód IL je sada instrukcí nezávislá na procesoru, kterou může spustit běhové prostředí .NET (CLR).

\n{3}{Kompilace do nativního kódu}

Přímá nativní AoT kompilace je proces, při kterém je kód kompilován do podoby sytémově nativního kódu při sestavení programu ze zdrojového kódu. V případě .NETu je tato funkcionalita dostupná při použití jazyka C\# a speciálních projektových atributů. 

Jedná se o funkcionalitu, jenž prošla několika iteracemi. První možnosti sestavení aplikace v nativním kódu na .NET platformě byly aplikace Universal Windows Platform. Jednalo se o aplikace využívající specifické rozhraní, nativní pro produkty Microsoftu. S verzí .NET framework 7 byly rozšířeny možnosti sestavení aplikace jako do podoby nativního kódu i pro další architektury a typy aplikací. Tato nová funkcionalita získala vyráznější podporu v roce 2023 s vydáním dotent framework 8. Filozofie Microsoftu ohledně AoT kompilace je, že vývojáři by měli mít možnost využít AoT kompilace v .NETu, pokud je to pro daný scénář vhodné. Scénáře kladoucí takovéto požadavky se vyskytují především v cloudovém nasazení, na které v současné filosofii apelují. 

\n{4}{Výstup}

Výstupem nativní AoT kompilace .NET aplikace je spustitelný soubor ve formátu podporovaném operačním systémem konfigurovaným v procesu kompilace. Takto vytvořený soubor je možné spustit přímo bez potřeby CLR. 

\n{3}{Popis procesu}

\begin{enumerate}
    \item Analýza zdrojového kódu, provádí kontrolu syntaxe
    \item Kontrola syntaxe
    \item Transformace vybrané syntaxe - například pro C\# je to transformace "High level C\#" na "Low level C\#"
    \item Generování pre-build kódu 
    \item Překlad kódu
\end{enumerate}

\n{3}{Cíle kompilace}

TODO

\n{2}{Běh kódu}

Spuštění, respektive běh kódu na HW počítačového zařízení vyžaduje instrukční sadu, které daná architektura rozumí, tedy nativní kód. V případě nativní AoT kompilace v .NET tento kód získáme již při sestavení aplikace. Při využití kompilace do IL je nutné kód získat pomocí jednoho z kompilačních způsobů podporovaného CLR. Výsledná nativní reprezentace se v obou případech spouští zavoláním vstupní metody v binárním souboru dle specifikace architektury.

\n{3}{CLR}

Common Language Runtime (CLR) je běhové prostředí frameworku .NET. Poskytuje spravované prostředí pro spouštění aplikací .NET. Podporuje více programovacích jazyků, včetně jazyků C\#, VB.NET a F\#, a umožňuje jejich bezproblémovou spolupráci. Spravuje paměť prostřednictvím automatického garbage collection, který pomáhá předcházet únikům paměti a optimalizuje využití prostředků. CLR také zajišťuje typovou bezpečnost a ověřuje, zda jsou všechny operace typově bezpečné, aby se minimalizovaly chyby při programování.

\n{4}{Funkce}

CLR je zodpovědný za několik důležitých funkcí, které zvyšují produktivitu vývojářů a výkon aplikací.

\begin{itemize}
    \item Správa paměti - spravuje alokaci paměti, obsluhuje GC
    \item Bezpečnost
    \item Zpracování vyjímek - obsluhuje zpracování chyb/vyjímek v programu
    \item Generování typů 
    \item Reflexe
\end{itemize}

Klíčovými vlasnostmi jsou CLR jsou multiplatformnost kódu, reflexe, optimalizace kódu pro konkrétní architekturu a bezpečnost. CLR nabízí mechanismy, jako je zabezpečení přístupu ke kódu (CAS), které zabraňují neoprávněným operacím. Kompilace JIT (just-in-time) znamená, že kód zprostředkujícího jazyka je zkompilován do nativního kódu těsně před spuštěním, což zajišťuje optimální výkon na cílovém hardwaru. CLR usnadňuje zpracování chyb v různých jazycích a poskytuje konzistentní přístup k řešení výjimek. Navíc obsahuje nástroje pro ladění a profilování, které vývojářům pomáhají efektivně identifikovat a odstraňovat problémy s výkonem.

Aby mohl být kód z IL reprezentace spuštěn na systému, respektive HW stroje, musí být dodatečně kompilován. Za tímto účelem existuje v CLR několik technik, které s sebou přínáší různé benefity a negativa a mají využití v specifických scénářích.

\n{4}{JIT kompilace}

Při JIT kompilaci v rámci CLR dochází ke kompilaci kódu do nativní podoby těsně před spuštěním kódu. Kompilace veškerého kódu aplikace JIT umožňuje optimalizovat běh programu současnému stavu systému. Za běhu je prováděna inspekce kódu, dochází ke kontrole validity, typování a adresace IL kódu. Kompilovány jsou pouze ty části kódu, jenž jsou relevantní pro aktuální stav programu.

\n{4}{R2R kompilace}

Zdrojový kód je při sestavení zkompilován do podoby nativního kódu pomocí nástroje crossgen, čímž vzniknou sestavy R2R. Za běhu se sestavy R2R načtou a spustí s minimální kompilací JIT, protože většina kódu je již v nativní podobě. CLR může přesto JIT kompilovat některé části kódu, které nelze staticky zkompilovat předem. Využití je v aplikacích, které potřebují zkrátit dobu spouštění, ale zachovat určitou funkcionalitu nebo úroveň optimalizace poskytovanou JIT kompilací.

\n{3}{Nativní kód}

Běh nativního kódu je závislý na konkrétní architektuře systému, pro které jsou nativní programové soubory vytvořeny. Nepodléhá další úpravě ze strany .NET nástrojů.

\n{2}{Tvorba programu v dotnet}

Následující část popisuje obecnou koncepci a strukturu projektu aplikace v dotnet. Součástí je postup pro tvorbu a vydání projektu. Blížší pozornost bude věnována tvorbě nativního AoT projektu.

\n{3}{Struktura aplikačních zdrojů}

Základním strukturovaným prvkem v .NET aplikaci je projektový soubor. Jedná se o XML soubor disponující příponou \emph{.csproj}. V rámci něj dochází ke konfiguraci a deklaraci, jak bude .NET CLI s projektem pracovat. Zároveň jsou zde definováný závislosti na další projekty a knihovny. Mezi základní charakteristiky běžně určené v projektovém souboru patří verze .NET, verze projektu/assembly, seznam závislostí, konfigurace pro buildování, testování a publikaci.

Pro tvorbu složitějších aplikací je možné využít více projektových souborů, které jsou následně propojeny. Tento způsob je využíván především v případě větších aplikací, které jsou rozděleny do více částí. Propojení a vazby mezi více projekty v aplikaci je definované pomocí tzv. solution souboru. Jedná se o kontejnerový soubor s příponou \emph{.sln}, jenž popisuje závislosti mezi projektovými soubory, konfigurace sestavení a nasazení a správu pomocných souborů.

\n{3}{Obecný postup}

\begin{enumerate}
    \item \textbf{Nastavení vývojového prostředí}: Sestává z instalace sady nástrojů .NET SDK.
    
    \item \textbf{Vytvoření projektu}: Pomocí příkazu \texttt{dotnet new} nebo skrze GUI IDE je vytvořen nový projekt a solution soubor. Součástí je výběr typu projektu, jazyka, frameworku a dalších konfiguračních parametrů.
    
    \item \textbf{Programování}: Sestává z tvorby kódu aplikace, testování a ladění.

    \item \textbf{Správa závislostí}: Pomocí nástrojů .NET CLI je možno referencovat balíčky a knihovny v rámci projektu.
    
    \item \textbf{Kompilace}: Kompilace aplikace probíhá pomocí příkazu \texttt{dotnet build}, který převede vysokoúrovňový kód do IL. V případě AoT dochází k dodatečné kompilace do nativního kódu dle cílové architektury.
    
    \item \textbf{Publikování}: Použitím příkazu příkazu \texttt{dotnet publish} dochází k vydání aplikace, tedy specifickému sestavení v konfigurovaném nastavení.
\end{enumerate}

\n{3}{Tvorba nativního programu}

Pro tvorbu nativního programu v .NET je nutné využít speciálního atributu \emph{PublishAoT} v projektovém souboru. Tento atribut je zodpovědný za konfiguraci projektu pro nativní AoT kompilaci. Při jeho použití je nutné specifikovat cílovou architekturu, pro kterou je nativní kód vytvářen. Po kompilaci kódu do IL dochází k dodatečné kompilaci do nativního kódu, která dodává další konzolový výstup s informacemi o průběhu kompilace.

Vzhledem k tomu, že nativní AoT kompilace je v .NETu stále vývojově nezralá, samotný proces kompilace, tak jako analýza kompilovaného kódu není dostatečně informativní. Za účelem přenesení vysokoúrovňových konceptů a formálních zápisů v C\# je při kompilace prováděno široké spektrum transformací a gerování kódu.

\n{4}{Deklarace unmanaged rozhraní}

\n{4}{Trimming}

\n{3}{Přehled podpory}

\n{4}{Funkcionalita}

Následující přehled představuje rozsah funkcionality implementované v rámci .NET frameworku 8.0, konkrétně APS.NET k datu zvěřejnění práce.

\begin{itemize}
    \item \textbf{REST minimal API}
    \item \textbf{gRPC API}
    \item \textbf{JWT Authentication}
    \item \textbf{CORS}
    \item \textbf{HealthChecks}
    \item \textbf{HttpLogging}
    \item \textbf{Localization}
    \item \textbf{OutputCaching}
    \item \textbf{RateLimiting}
    \item \textbf{RequestDecompression}
    \item \textbf{ResponseCaching}
    \item \textbf{ResponseCompression}
    \item \textbf{Rewrite}
    \item \textbf{StaticFiles}
    \item \textbf{WebSockets}
\end{itemize}

\n{4}{Cíle kompilace}

.NET poskytuje podporu pro kompilaci zdrojového kódu v režimu AoT pouze pro určité operační systémy:

\begin{itemize}
    \item \textbf{Windows} - plná podpora
    \item \textbf{Linux} - plná podpora
    \item \textbf{macOS} - plná podpora
    \item \textbf{Android} - částečná podpora
    \item \textbf{iOS} - částečná podpora
    \item \textbf{WebAssembly} - částečná podpora
\end{itemize}

\n{2}{Srovnání}

\n{3}{JIT}

\n{4}{Výhody}

Mezi hlavní výhody se řadí zprostředkování následujícího:

\begin{itemize}
    \item  \textbf{Reflexe} - CLR umožňuje využívat reflexi, která umožňuje získat informace o kódu za běhu aplikace. Tímto je umožněno vytvářet aplikace, které jsou schopny měnit své chování za běhu.
    \item \textbf{Dynamické načítání} - CLR umožňuje dynamicky načítat knihovny za běhu aplikace. Tímto je umožněno vytvářet aplikace, které jsou schopny měnit své chování za běhu.
    \item \textbf{Větší bezpečnost} - CLR zajišťuje, že aplikace nemůže přistupovat k paměti, která jí nebyla přidělena. Tímto je zajištěna bezpečnost aplikace a zabráněno chybám, které by mohly vést k pádu aplikace.
    \item \textbf{Správa paměti} - CLR zajišťuje správu paměti pomocí GC. Tímto je zajištěno, že paměť je uvolněna vždy, když ji aplikace již nepotřebuje. Tímto je zabráněno tzv. memory leakům, které by mohly vést k pádu aplikace.
    \item \textbf{Větší přenositelnost} - CLR zajišťuje, že aplikace je spustitelná na všech operačních systémech, na kterých je dostupné běhové prostředí CLR.
\end{itemize}

\n{4}{Nevýhody}

Zatímco za nevýhody CLR se dá považovat:

\begin{itemize}
    \item  \textbf{Výkonnost} - I když určité optimalizace jsou prováděny pro konkrétní systém a architekturu, výkon CLR je nižší než výkon nativního kódu. Dalším výkonnostním měřítkem je rychlost startu aplikace, která je pro CLR vyšší než v případě nativního kódu.
    \item \textbf{Operační paměť} - CLR využívá více operační paměti, jak pro aplikaci, tak i pro běhové prostředí.
    \item \textbf{Velikost aplikace} - Přítomnost CLR nehraje zásádní roli v případě monolitických aplikací, ale v případě mikroslužeb je nutné CLR přidat ke každé službě. Tímto se zvyšuje velikost jedné aplikační instance.
\end{itemize}

\n{3}{AoT}

\n{4}{Výhody}

Mezi hlavní výhody se řadí zprostředkování následujícího:

\begin{itemize}
    \item  \textbf{Výkonnost} - CLR umožňuje využívat reflexi, která umožňuje získat informace o kódu za běhu aplikace. Tímto je umožněno vytvářet aplikace, které jsou schopny měnit své chování za běhu.
    \item \textbf{Paměťová zátěž} - CLR umožňuje dynamicky načítat knihovny za běhu aplikace. Tímto je umožněno vytvářet aplikace, které jsou schopny měnit své chování za běhu.
\end{itemize}

\n{4}{Nevýhody}

Zatímco za nevýhody CLR se dá považovat:

\begin{itemize}
    \item  \textbf{Absence nástrojů z CLR} - Mnoho nástrojů, které jsou dostupné v CLR, nejsou dostupné v AoT kompilaci. Mezi tyto nástroje patří například reflexe, dynamické načítání knihoven a další.
    \item \textbf{Absence dynamického načítání} - například Assembly.LoadFile.
    \item \textbf{Bez generování kódu za běhu} - například System.Reflection.Emit.
    \item \textbf{Žádné C++/CLI} - např. System.Runtime.InteropServices.WindowsRuntime
    \item \textbf{Windows: absence COM} - např. System.Runtime.InteropServices.ComTypes
    \item \textbf{Vyžaduje trimming (ořezávání)} - má určitá omezení, je však klíčový pro rozumnou velikost výsledného programu
    \item \textbf{Kompilace do jediného souboru} 
    \item \textbf{Připojení běhových knihoven} - požadované běhové knihovny jsou součástí výsledného aplikačního souboru. To zvyšuje velikost samoteného programu ve srovnání s aplikacemi závislými na frameworku.
    \item \textbf{System.Linq.Expressions} - výsledný kód používá svou interpretovanou podobu, která je pomalejší než run-time generovaný kompilovaný kód.
    \item \textbf{Kompatibilita knihoven s AoT} - né všechny knihovny runtime jsou plně anotovány tak, aby byly kompatibilní s Native AoT. To znamená, že některá varování v knihovnách runtime nejsou pro koncové vývojáře použitelná.
\end{itemize}

\n{2}{Závěr}

xx

%%%%%%%%%%%%%%%%%%%%%%%%%%%%%%%%%%%%%%%%%%%%%%%%%%%%%%%%%%%%%%%%%%%%%%%%%%%%%%%%%
%                                 MICROSERVICE                                  %
%%%%%%%%%%%%%%%%%%%%%%%%%%%%%%%%%%%%%%%%%%%%%%%%%%%%%%%%%%%%%%%%%%%%%%%%%%%%%%%%%

\n{1}{Microservice architektura}
Při vývoji softwaru je možné využít z několika architektur. Jednou z těchto architektur je monolitická architektura. V monolitické architektuře je celá aplikace rozdělena do několika vrstev, které jsou využívány k oddělení logiky aplikace.

Oproti tomu microservice architektura je založena na principu oddělení aplikace do několika samostatných služeb. Každá z těchto služeb je zodpovědná za určitou část funkcionality aplikace. Služby jsou navzájem nezávislé a komunikují mezi sebou pomocí definovaných rozhraní. \cite{Richardson2018}

\n{2}{Historie}
Původ microservice architektury nelze přesně definovat, důležitý moment však nastal v roce 2011, kdy Martin Fowler publikoval článek \textit{Microservices} na svém blogu. V tomto článku popsal výhody a nevýhody této architektury a zároveň popsal způsob, jakým je možné tuto architekturu využít. Dalším popularizačním momentem pro popularizaci bylo vydání knihy \textit{Building Microservices} od Sama Newmana v roce 2015. Tato kniha popisuje způsob, jakým je možné využít microservice architekturu v praxi.

Opravdový přelom přišel postupně, nástupem a popularizací virtualizace a kontejnerizace v průběhu let 2013 až 2015. Tímto bylo umožněno vytvářet a spouštět mikroslužby v izolovaných prostředích. Tímto bylo umožněno vytvářet mikroslužby, které jsou nezávislé na operačním systému a hardwaru, na kterém jsou spouštěny. Nejdůležitější v tomto ohledu je nepochybně projekt Docker, který byl vydán v roce 2013. Díky Dockeru bylo možno jednoduše definovat, vytvářet a spouštět kontejnerizované aplikace.

\n{2}{Popis}

Architektura mikroslužeb rozděluje složité softwarové aplikace na menší, spravovatelné části, které lze vyvíjet, nasazovat a škálovat nezávisle.

\n{3}{Virtualizace a kontejnerizace}

Virtualizace a kontejnerizace jsou klíčové technologie, které umožňují architekturu mikroslužeb. Virtualizace umožňuje provozovat více operačních systémů na jednom fyzickém hardwarovém hostiteli, čímž se snižuje počet potřebných fyzických strojů a zvyšuje efektivita využití zdrojů. Kontejnerizace jde ještě o krok dále tím, že zabalí aplikaci a její závislosti do kontejneru, který může běžet na libovolném serveru Linux nebo Windows. Tím je zajištěno, že aplikace funguje jednotně i přes rozdíly v prostředí nasazení.

Kontejnerizace je obzvláště důležitá pro mikroslužby, protože zapouzdřuje každou mikroslužbu do vlastního kontejneru, což usnadňuje její nasazení, škálování a správu nezávisle na ostatních. Synonymem kontejnerizace se staly nástroje jako Docker, které nabízejí ekosystém pro vývoj, odesílání a provoz kontejnerových aplikací.

\n{3}{Orchestrace}

S rozšiřováním mikroslužeb a kontejnerů se jejich správa stává složitou. Nástroje pro orchestraci pomáhají automatizovat nasazení, škálování a správu kontejnerů. Mezi oblíbené orchestrační nástroje patří Kubernetes, Docker Swarm a Mesos. Zejména Kubernetes se stal de facto standardem, který poskytuje robustní rámec pro nasazení, škálování a provoz kontejnerových aplikací v clusteru strojů. Řeší vyhledávání služeb, vyvažování zátěže, sledování přidělování prostředků a škálování na základě výkonu pracovní zátěže.

\n{3}{Základní principy}

\n{4}{Komunikace}

Mikroslužby spolu komunikují prostřednictvím rozhraní API, obvykle prostřednictvím protokolů HTTP/HTTPS, i když pro aplikace citlivější na výkon lze použít i jiné protokoly, například gRPC. Komunikační vzory zahrnují synchronní požadavky (např. RESTful API) a asynchronní zasílání zpráv (např. pomocí brokerů zpráv jako RabbitMQ nebo Kafka). Tím je zajištěno volné propojení mezi službami, což umožňuje jejich nezávislý vývoj a nasazení.

\n{4}{Škálování}

Architektura mikroslužeb zvyšuje škálovatelnost. Služby lze škálovat nezávisle, což umožňuje efektivnější využití zdrojů a zlepšuje schopnost systému zvládat velké objemy požadavků. Běžně se používá horizontální škálování (přidávání dalších instancí služby), které usnadňují nástroje pro kontejnerizaci a orchestraci.

\n{4}{Odolnost}

Robustnosti mikroslužeb je dosaženo pomocí strategií, jako jsou přerušovače, záložní řešení a opakované pokusy, které pomáhají zabránit tomu, aby se selhání jedné služby kaskádově přeneslo na ostatní. Izolace služeb také znamená, že problémy lze omezit a vyřešit s minimálním dopadem na celý systém. Kromě toho jsou kontroly stavu a monitorování nezbytné pro včasné odhalení a řešení problémů.

\n{4}{Vývoj}

Mikroslužby umožňují agilní vývojové postupy. Týmy mohou vyvíjet, testovat a nasazovat služby nezávisle, což umožňuje rychlejší iteraci a zpětnou vazbu. Nedílnou součástí jsou pipelines pro kontinuální integraci a doručování (CI/CD), které umožňují automatizované testování a nasazení. Tento přístup podporuje kulturu DevOps a podporuje užší spolupráci mezi vývojovými a provozními týmy.

\n{3}{Serverless a mikroslužby}

Serverless je model vývoje aplikací, který umožňuje vývojářům psát a nasazovat kód bez starostí o infrastrukturu. Tento model je založen na konceptu funkcí jako služby (FaaS), které jsou jednotlivé kusy kódu, které jsou spouštěny na základě událostí. Serverless a mikroslužby se často používají společně, protože oba modely podporují škálovatelnost, agilitu a odolnost. Serverless může být výhodný pro mikroslužby, které jsou založeny na událostech, jako jsou zpracování obrázků, zpracování zpráv nebo plánování úloh.

\n{2}{Testování}

Testování mikroslužeb je klíčové pro zajištění kvality a spolehlivosti systému. Mikroslužby lze testovat na několika úrovních, včetně jednotkových testů, integračních testů a testů end-to-end. Jednotkové testy se zaměřují na testování jednotlivých komponent služby, zatímco integrační testy testují komunikaci mezi službami. Testy end-to-end testují celý systém z pohledu uživatele. Automatizované testování je klíčové pro rychlé a spolehlivé nasazení.

\n{2}{Výhody a nevýhody}

\n{3}{Výhody}

\n{4}{Zvýšená agilita} 

Mikroslužby umožňují rychlé, časté a spolehlivé poskytování rozsáhlých a komplexních aplikací. Týmy mohou aktualizovat určité oblasti aplikace, aniž by to mělo dopad na celý systém, což umožňuje rychlejší iterace.

\n{4}{Škálovatelnost}

Služby lze škálovat nezávisle, což umožňuje přesnější přidělování zdrojů na základě poptávky. To usnadňuje zvládání proměnlivého zatížení a může zlepšit celkovou efektivitu aplikace.

\n{4}{Odolnost} 

Decentralizovaná povaha mikroslužeb pomáhá izolovat selhání na jedinou službu nebo malou skupinu služeb, čímž zabraňuje selhání celé aplikace. Techniky, jako jsou jističe, zvyšují odolnost systému.

\n{4}{Technologická rozmanitost}

Týmy si mohou vybrat nejlepší nástroj pro danou práci a podle potřeby používat různé programovací jazyky, databáze nebo jiné nástroje pro různé služby, což vede k potenciálně optimalizovanějším řešením.

\n{4}{Flexibilita nasazení}

Mikroslužby lze nasazovat nezávisle, což je ideální pro kontinuální nasazení a integrační pracovní postupy. To také umožňuje průběžné aktualizace, modrozelené nasazení a kanárkové verze, což snižuje prostoje a rizika.

\n{4}{Modularita}

Tato architektura zvyšuje modularitu, což usnadňuje pochopení, vývoj, testování a údržbu aplikací. Týmy se mohou zaměřit na konkrétní obchodní funkce, což zvyšuje produktivitu a kvalitu.

\n{3}{Nevýhody}

\n{4}{Komplexnost} 

Správa více služeb na rozdíl od monolitické aplikace přináší složitost při nasazování, monitorování a řízení komunikace mezi službami.

\n{4}{Správa dat}

Konzistence dat mezi službami může být náročná, zejména pokud si každá mikroslužba spravuje vlastní databázi. Implementace transakcí napříč hranicemi vyžaduje pečlivou koordinaci a vzory jako Saga.

\n{4}{Zpoždění sítě}

Komunikace mezi službami po síti přináší zpoždění, které může ovlivnit výkonnost aplikace. Ke zmírnění tohoto jevu jsou nutné efektivní komunikační protokoly a vzory.

\n{4}{Provozní režie}

S počtem služeb roste potřeba orchestrace, monitorování, protokolování a dalších provozních záležitostí. To vyžaduje další nástroje a odborné znalosti.

\n{4}{Složitost vývoje a testování}

Mikroslužby sice zvyšují flexibilitu vývoje, ale také komplikují testování, zejména pokud jde o testování end-to-end, které zahrnuje více služeb.

\n{4}{Integrace služeb} Zajištění bezproblémové spolupráce služeb vyžaduje robustní správu API, řízení verzí a strategie zpětné kompatibility.

\n{2}{Závěr}

Architektura mikroslužeb je metoda vývoje softwarových systémů, které jsou rozděleny do malých, nezávislých služeb komunikujících prostřednictvím přesně definovaných rozhraní API. Tyto služby jsou vysoce udržovatelné a testovatelné, volně provázané, nezávisle nasaditelné a organizované podle obchodních schopností. Tento přístup k architektuře umožňuje organizacím dosáhnout větší agility a škálování jejich aplikací.

%%%%%%%%%%%%%%%%%%%%%%%%%%%%%%%%%%%%%%%%%%%%%%%%%%%%%%%%%%%%%%%%%%%%%%%%%%%%%%%%%
%                                  MONITORING                                   %
%%%%%%%%%%%%%%%%%%%%%%%%%%%%%%%%%%%%%%%%%%%%%%%%%%%%%%%%%%%%%%%%%%%%%%%%%%%%%%%%%

\n{1}{Monitorování aplikace}

Monitorování aplikací je klíčovým aspektem moderního vývoje a provozu softwaru, který týmům umožňuje sledovat výkon, stav a celkové chování aplikací v reálném čase. Zahrnuje shromažďování, analýzu a interpretaci různých typů dat a informací, které zajišťují hladký a efektivní chod aplikací a umožňují rychle identifikovat a řešit případné problémy.

\n{2}{Druhy dat}

Pro efektivní monitorování aplikace je nezbytné porozumět různým typům dat a informací, které lze shromažďovat:

\n{3}{Logy}

Protokoly jsou záznamy o událostech, ke kterým dochází v rámci aplikace nebo jejího provozního prostředí. Poskytují podrobné, časově označené záznamy o činnostech, chybách a transakcích, které mohou vývojáři a provozní týmy použít k řešení problémů, pochopení chování aplikace a zlepšení spolehlivosti systému.

\n{3}{Traces}

Trasy se používají ke sledování toku požadavků v aplikaci, zejména v distribuovaných systémech, kde jedna transakce může zahrnovat více služeb nebo komponent. Sledování pomáhá identifikovat úzká místa, pochopit problémy s latencí a zlepšit celkový výkon aplikací.

\n{3}{Metriky}

Metriky jsou kvantitativní údaje, které poskytují přehled o výkonu a stavu aplikace. Mezi běžné metriky patří doba odezvy, využití systémových prostředků (CPU, paměť, diskové I/O), chybovost a propustnost. Sledování těchto metrik pomáhá při proaktivním ladění výkonu a plánování kapacity.

\n{2}{Sběr dat}

Efektivita monitorování aplikací do značné míry závisí na schopnosti efektivně shromažďovat relevantní data.

\n{3}{Collectory}

Kolektory jsou nástroje nebo agenti, kteří shromažďují data z různých zdrojů v rámci aplikace a jejího prostředí. Mohou být nasazeny jako součást infrastruktury aplikace nebo mohou být provozovány jako externí služby. Kolektory jsou zodpovědné za shromažďování protokolů, stop a metrik a za předávání těchto dat do monitorovacích řešení, kde je lze analyzovat a vizualizovat. Efektivní sběr dat je nezbytný pro monitorování v reálném čase a pro zajištění toho, aby shromážděná data přesně odrážela stav a výkon aplikace.

\n{2}{Vizualizace dat}

Vizualizace dat je klíčovým aspektem monitorování aplikací, který umožňuje rychle porozumět stavu a chování aplikací. Vizualizace může zahrnovat různé typy grafů, tabulek, dashboardů a dalších nástrojů, které umožňují zobrazit data v uživatelsky přívětivé podobě. Vizualizace dat umožňuje týmům identifikovat vzory, problémy a příležitosti, které by jinak mohly zůstat skryty v datech.

\n{2}{Implementace monitorování}

Implementace monitorování aplikací zahrnuje několik klíčových kroků, včetně definice klíčových metrik, výběru monitorovacích nástrojů, nasazení kolektorů a vizualizaci dat. Týmy by měly také vytvořit procesy pro řešení problémů, které byly identifikovány prostřednictvím monitorování, a pro využití dat k plánování kapacity a optimalizaci výkonu.

\n{3}{Sběr dat v monitorovaných službách}

Implementace sběru dat zahrnuje inkorporaci funkcionality monitorování a zprostředkování dat v rámci předdefinovaného rozhraní. Sběr je realizován zpravidla sérií čítačů a zapisovačů, které jsou využívány k získávání dat z různých zdrojů. Takto sbíraná datá jsou kategorizována a značkována pro identifikaci.

Realizace monitorování je zajištěna buďto použitím existujících implementací v rámci sw knihoven nebo vytvořením vlastní implementace dle potřeb aplikace a monitorovacích protokolů.

\n{3}{Nasazení služeb pro správu a kolekci dat}

Nasazení služeb pro správu a kolekci dat je zajištěno pomocí nástrojů, které jsou schopny zprostředkovat sběr dat z různých zdrojů a zároveň zajišťují jejich zpracování a zobrazení. Tímto je zajištěno, že data jsou zpracována a zobrazena v reálném čase.

\n{3}{Vizualizace dat}

Vizualizace dat je zajištěna pomocí nástrojů, které jsou schopny zobrazit data v uživatelsky přívětivé podobě. Tímto je zajištěno, že data jsou zobrazena v reálném čase a jsou přehledná a srozumitelná.

\n{2}{Konfigurace}

Konfigurace monitorování je zajištěna pomocí konfiguračních souborů, které definují chování monitorovacích nástrojů a sběr dat. Ovlivnit chování monitorovacího systému může být provedeno jak na straně monitorovacích nástrojů, respektive služeb, tak i na straně aplikací a služeb, které jsou monitorovány.

\n{2}{Závěr}

Monitorování aplikací je nezbytným nástrojem pro vývoj a provoz moderních softwarových systémů. Zahrnuje shromažďování, analýzu a interpretaci různých typů dat a informací, které umožňují týmům sledovat výkon, stav a chování aplikací v reálném čase. Tímto je zajištěno, že aplikace jsou spolehlivé, výkonné a efektivní.
