\nn{Závěr}

V rámci diplomové práce byly analyzovány kompilační režimy JIT a nativní AOT na platformě .NET. První částí byla rešerše, ve které byly popsány základní principy fungování platformy .NET, jejich kompilačních režimů a cílů kompilace. Následně byla popsána architektura microservice, která slouží jako primární zacílení nativních AOT aplikací a která poskytuje vzor pro testovací nasazení. V neposlední řadě byla popsána problematika testování, telemetrie a monitorovacích řešení.

Praktická část se zabývala vývojem testovacích služeb a testovací platformy.Dále byly popsány nástroje, které umožňují vytváření nativních AOT aplikací na platformě .NET. V rámci rešerše byly také popsány nástroje, které umožňují měření výkonu aplikací.

V analytické části byly výsledky praktické části popsány a vyhodnoceny. Zhodnocení vývoje probíhalo ve třech režimech: analýza vývoje, výstupu a výkonu.

Výsledkem práce je komplexní analýza použití kompilačních režimů JIT a nativní AOT. Vývojový proces při kompilaci do nativního AOT kódu se ukázal nepřívětivý. Primárně podpora knihoven 3. stran a princip interceptorů a generátorů má za vinu subjektivně neintuitivní proces debugování kódu. Samotný programový výstup vyšel dle očekávání. V porovnání byly obrazy nativních služeb výrazně paměťově efektivnější. Výsledky testování ukázaly, že na platformě .NET nativní AOT aplikace mají obecně srovnatelný výkon jako aplikace v režimu JIT. Rozdíl je znatelný v situacích, kdy je nutno využít velké množství instancí stejné služby (plyne z velikosti obrazu) a v situacích, kdy je pro systém rozhodující rychlost zpracování služby včetně spuštění (Serverless platformy). Výsledky výkonnostního testování byly zaznamenány a zpracovány do dashboardů a grafů, které jsou připraveny v uživatelské rozhraní platformy Grafana.

Dále byla vytvořena sada testovacích služeb, které slouží jako ukázka možností platformy. V neposlední řadě pro účely analýzy byla vytvořena testovací platforma, která umožňuje vytváření a nasazování testovacích služeb v kompilačním režimu JIT a nativní AOT.

Služby kompilované do nativního AOT kódu přináší specifické výkonnostní výhody za cenu kompatibility API. Vývoj kódu je s ohledem na zažité postupy a praktiky v .NET nestandartní. Využití interceptorů a generátorů je odebrána část inciativy z rukou vývojáře a vytváří se na pozadí kompilace v .NET další úroveň abstrakce. Podpora knihoven a frameworků třetích stran je omezena a nelze se spolehnout na jejich plnou funkčnost. Tím připadá na vývojáře zodpovědnost za implementaci vlastních řešení, která by jinak byla dostupná.

Většina výhod, jenž z .NET plyne souvisí s možnostmi jeho runtime prostředí. Nativní AOT kompilace má smysl ve specifických případech, jenž plynou z nutnosti rychlosti spuštění a velikosti výstupu aplikace (s přihlednutím k velikosti .NET runtime). Případy konkurenční výhody pro AOT kompilaci staví na předpokladu a tím je zájem či potřeba mít zdrojové kódy v .NET, respektive jazyce C\#. Při rozmanitém technologickém přístupu, kdy je vývojář, respektive zapojený tým schopen přijmout jiný jazyk a framework, jsou výhody AOT kompilace ztraceny, zatímco nedostatky jsou zvýrazněny. Tento předpoklad je relativně v rozporu s požadavky na poskytování cloudových služeb, kdy je očekáváno silné technické a vědomostní zázemí a flexibilní přístup k technologiím.

Nativní AOT kompilace má oproti JIT kompilaci nesporné výhody za splnění uričtých podmínek na požadavky vůči nasazení, kódu a vývojového týmu. Zaplňuje specifickou díru v portfoliu technologií platformy .NET, která je klíčová pro kompletní řešení cloudové platformy pouze s použitím této platformy. Vývojáři, kteří se rozhodnou pro AOT kompilaci, by měli být obeznámeni s těmito specifiky a měli by být schopni je zohlednit v návrhu a implementaci řešení.

V návaznosti na platformu, která v práci vznikla v rámci výkonnostního testování služeb, je možné doplnit implementaci dalších služeb, případně rozšířit stávající. Podle vzoru současného řešení lze dodat další funkcionalitu, nastavit další zdroje telemetrie, případně rozšířit možnosti vizualizace dat. Z pohledu uživatelské přívětivosti se nabízí rozšíření o webovou aplikaci využívající princip Docker outside of Docker (DooD). Tímto by bylo možné zjednodušit spouštění konkrétních testovacích scénářů v GUI. V rámci webové aplikace by bylo možné nastavit parametry testování, spustit konkrétní test a prokliknout se odkazem na relevantní dashboard v Grafaně. S ohledem na citlivé data a přístupy, které aplikace zprostředkovává, se nabízí rozšíření o autentizaci a autorizaci v případě vystavení stacku v síti. Jelikož grafické rozhraní aplikace je založeno na aplikaci Grafana, jež je schopna připojit se k externím zprostředkovatelům autentizace, bylo by vhodné zapojit službu jako Keycloak pro sjednocení autentifikace napříč stackem.


XXX Petr radí


tahle XXX XXX

