
\n{1}{Doporučení pro použití AOT kompilace v dotnet}

Služby kompilované do nativního AOT kódu přináší specifické výkonnostní výhody za cenu kompatibility API. Vývoj kódu je s ohledem na zažité postupy a praktiky v .NET nestandartní. Využití interceptorů a generátorů je odebrána část inciativy z rukou vývojáře a vytváří se na pozadí kompilace v .NET další úroveň abstrakce. Podpora knihoven a frameworků třetích stran je omezena a nelze se spolehnout na jejich plnou funkčnost. Tím připadá na vývojáře zodpovědnost za implementaci vlastních řešení, která by jinak byla dostupná.

Většina výhod, jenž z .NET plyne souvisí s možnostmi jeho runtime prostředí. Nativní AOT kompilace má smysl ve specifických případech, jenž plynou z nutnosti rychlosti spuštění a velikosti výstupu aplikace (s přihlednutím k velikosti .NET runtime). Případy konkurenční výhody pro AOT kompilaci staví na předpokladu a tím je zájem či potřeba mít zdrojové kódy v .NET, respektive jazyce C\#. Při rozmanitém technologickém přístupu, kdy je vývojář, respektive zapojený tým schopen přijmout jiný jazyk a framework, jsou výhody AOT kompilace ztraceny, zatímco nedostatky jsou zvýrazněny. Tento předpoklad je relativně v rozporu s požadavky na poskytování cloudových služeb, kdy je očekáváno silné technické a vědomostní zázemí a flexibilní přístup k technologiím.

Nativní AOT kompilace má oproti JIT kompilaci nesporné výhody za splnění uričtých podmínek na požadavky vůči nasazení, kódu a vývojového týmu. Zaplňuje specifickou díru v portfoliu technologií platformy .NET, která je klíčová pro kompletní řešení cloudové platformy pouze s použitím této platformy. Vývojáři, kteří se rozhodnou pro AOT kompilaci, by měli být obeznámeni s těmito specifiky a měli by být schopni je zohlednit v návrhu a implementaci řešení.
