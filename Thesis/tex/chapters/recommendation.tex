
\n{1}{Doporučení pro použití AoT kompilace v dotnet}

AOT kód přináší jasné výhody výkonnostní výhody za cenu kompatibility. Řešení tvorby toho kódu, je však s ohledem na běžný postup velmi nešťastné. Využití interceptorů a generátorů bere inciativu z rukou vývojáře a vytváří naprosto nový program. Toto chování není natolik odlišné od průběhu kompilace do nativního systémového kódu v jiných jazycích, v případě .NET avšak bylo dodáno znatelně "post mortem".

Valná většina konkurenčních výhod, jenž z .NET plyne souvisí s možnostmi jeho runtime prostředí. Nativní AOT kompilace má smysl jen ve velice specifický situacích, jenž lze blíže identifikovat jako poskytování cludové infrastruktury a s tím spojenou potřebu běhu velkého množství instancí. Dalším příkladem je využití Serverless nebo také jako lambda funkce, kdy je poskytována funkcionalita a běh spuštění aplikace pro její vykonání je dílčí režie.

Případy konkurenční výhody pro AOT kompilaci staví na jednom předpokladu a to je zájem či potřeba mít zdrojové kódy v .NET, respektive jazyce C\#. Při unimodálním přístupu, kdy je vývojář, respektive zapojený tým schopen přijmout jiný jazyk, jsou výhody AOT kompilace značně zmenšeny, zatímco nedostatky jsou zvýrazněny.

Nativní AOT kompilace má nesporné výhody za splnění uričtých podmínek na požadavky vůči nasazení, codebase a vývojového týmu. Zaplňuje určitou díru na trhu, která je však vzhledem k výše uvedeným podmínkám velice specifická. Pro běžné vývojáře usilující o výkonostní výhody, je však AOT kompilace v současné podobě nevhodná.
