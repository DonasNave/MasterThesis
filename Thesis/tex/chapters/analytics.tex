\n{1}{Analýza aplikace}

Tato kapitola se zabývá analýzou aplikace z hlediska vývoje, výstupu a výkonu. Využívá k tomu definovanou metodiku a scénáře testování. Výsledky jsou důkladně analyzovány a závěry shrnuty v jednotlivých sekcích.

\n{2}{Analýza vývojového procesu}

Následující sekce analyzuje vývojový proces .NET služeb pro režimy kompilace nativní AOT a JIT. Vývojový proces byl založen na experimentaci a snaze vyzkoušet dostupné knihovny a nástroje. Bylo tak provedeno za cenu řešení problémů z nekompatibility a vynucených změn implementace.

\obr{Ukázka kódu s vizualizací direktiv dle konfigurace}{fig:codesample}{0.7}{graphics/images/code-visual-sample.png}

\n{3}{Vývojové prostředí}

K vývoji byl použit IDE Rider od společnosti JetBrains. Vyzkoušena byla rovněž i práce ve Visual Studio 2022 Community Edition a Visual Studio Code s doporučenými rozšířeními od Microsoft. Všechna vývojová prostředí jsou kompatibilní, co se týče procesu kompilace, respektive sestavení, jelikož tyto procesy jsou spouštěny pomocí .NET CLI. Samotný vývoj služeb byl značně zjednodušen vizualizacemi pomocí definice direktiv kompilačních režimů. Ty umožnili vytvořit program s místy odlišným API v rámci jedné kódové báze. Grafické rozlišení sekcí bylo k dispozici ve vývojových prostředích Rider a Visual Studio. Obecně byla v těchto IDE zjednodušena i analýza a hledání chyb díky integraci referencí na kód tvořený na pozadí generátory. V tomto ohledu Visual Studio Code zaostávalo. S ohledem na aktivní vývoj a podporu, jenž je ze strany Microsoft poskytována pro vývoj .NET aplikací ve Visual Studio Code (po vyřazení produktu Visual Studio pro Mac), lze očekávat, že se tato situace v budoucnu změní.

\obr{Vizualizace zachycení metody interceptorem v kódu}{fig:codemethodintercept}{0.8}{graphics/images/code-method-interception.png}

\obr{Ukázka kódu nové implementace metody vygenerované na pozadí}{fig:codegeneratedmethod}{1}{graphics/images/code-generated-method.png}

Příprava nástrojů a sestavení projektů bylo vyzkoušeno jak na platformě macOS, tak ve Windows. Návod uvedený v oficiální dokumentaci pro platformu macOS úspěšně umožnil připravit vývojové prostředí a sestavit aplikace. \cite{netdocsnativeguide} V případě Windows a návodu z dokumentace nastala chyba, která vyžadovala značný zásah do konfigurace nástrojů. Pro sestavení projektů je nutné mít nainstalovanou komponentu pro vývoj C++ z dostupných komponent Visual Studio. Tato komponenta zajistí instalaci kompilátoru jazyka C a linker, jenž jsou nutné pro vytvoření nativní aplikace. Pro správné spuštění těchto nástrojů, tak jak jsou implicitně volány v procesu nativní AOT kompilace, se musejí nacházet na adrese souborového systému, jenž neobsahuje mezery nebo speciální znaky. Bohužel výchozí instalační cesta k nástroji Visual Studio obsahuje mezeru, což způsobuje chybu při jejich spuštění. Tomuto problému lze předejít instalací nástroje na validní adresu bez mezer a speciálních znaků. V případě již nainstalovaného nástroje je potom potřeba jak reinstalace Visual Studio, tak i zmíněné komponenty pro vývoj C++. Celý problém je zvýrazněn výchozím nastavením OS Windows instalovat programy do složky \emph{Program Files}, která sama o sobě obsahuje mezeru a způsobuje problém.

\n{3}{JIT}

Vývojový proces pro kompilaci služeb JIT se zacílením na .NET runtime probíhal standartním způsobem. Veškeré dostupné knihovny a nástroje byly plně kompatibilní s JIT kompilací. Znatelný rozdíl oproti běžnému vývoji byl při výběru technologií, který přihlížel k potencionální kompatibilitě s AOT. Knihovny a řešení, které vyžadovaly funkce závislé na .NET runtime, byly ihned zavrženy.

\n{3}{AOT}

Vývoj se zacílením na nativní AOT kompilaci probíhal s průběžnými problémy. Podpora ze strany knihoven 3. stran ve spoustě případů neodpovídala deklarovaným možnostem. Vývojový proces byl značně zpomalován řešením problémů, které byly způsobeny nedostatečnou podporou. Experimentace s knihovnami často vyústila v nutnost změny implementace, případně v implementaci zcela vlastní. Obecně se při vývoji nedalo spolehnout na systém AOT varování, jenž je součástí kompilačního procesu. 

\n{3}{Knihovny třetích stran}

Pro zjednodušení procesu vývoje a využití existující funkcionality byly do aplikace zapojeny knihovny třetích stran. Následující seznam obsahuje knihovny, které byly použity v rámci vývoje a zda byly kompatibilní s AOT kompilací.

\begin{itemize}
  \item \textbf{Entity Framework} - Entity framework (dále EF) se pyšní vysokou kompatibilitou s AOT kompilací. V rámci vývoje nebyly zaznamenány problémy, avšak následné testování se ukázalo problematické. EF jakožto plnohodnotný ORM framework stopuje stav objektu a jeho změny. Toto chování bohužel vyžaduje dynamické generování kódu, což je v rozporu s možnostmi AOT kompilovaného kódu. Vypnout tuto funkcionalitu lze pouze částečně, neboť EF stále vyžaduje reflexi při vkládání nových entit do databáze. Knihovna se nedá považovat za plně kompatibilní s AOT kompilací.
  \item \textbf{Fluent Migrator} - Fluent Migrator je knihovna, která umožňuje verzování databáze pomocí kódu. V rámci testování bylo zjištěno, že knihovna využívá reflexi pro načítání migrací. Toto chování je v rozporu s AOT kompilací a výsledkem je chyba při spuštění migrace. Problém byl vyřešen vytvořením vlastního minimalistického migrátoru, který nepoužívá reflexi. Knihovna není kompatibilní s AOT kompilací.
  \item \textbf{Grpc} - Vytváření rozhrání a modelů pro gRPC komunikaci vyžadovalo využití přístupu model first. Tento přístup využívá generátorů pro tvorbu .NET rozhraní. Tím je dosaženo vygenerování veškerého potřebného kódu v době kompilace a je zajištěna kompatibilita s AOT. Pro definici modelu přístupem code first ovšem kompatibilita s AOT není zajištěna.
  \item \textbf{Párování konfigurace} - V rámci systémové .NET knihovny je umožněno volání API, jenž načte data ze sjednocení stavu proměnných prostředí a konfiguračního souboru. Součástí API je volání metody mapující tuto konfiguraci na předem definovaný objekt. Použití této standartní systémové metody při AOT kompilaci nevyvolalo varování. Z testování však vyplynulo, že mapování konfigurace ne objekt využívá funkcionality runtime a neproběhne správně. Z tohoto důvodu je v případě AOT kompilace použité přímé načtení jednotlivých hodnot z konfigurace dle klíče.
\end{itemize}

V rámci služeb jsou dále využity knihovny Npgsql, RabbitMQ a OpenTelemetry jenž poskytují plnou kompatibilitu s AOT kompilací. Knihovna SwaggerUI je také součástí řešení, nicméně z podstaty své funkcionality není kompatibilní s AOT kompilací a nebyla využita v rámci AOT kompilace služeb.

\n{2}{Výstup služeb}

Samotný proces nativní AOT a JIT kompilace je různě výkonnostně náročný. Při tvorbě obrazu služeb, ale i kompilace je hlavní náročná operace \emph{restore}, která stahuje potřebné závislosti a balíčky pro projekt. Proces kompilace je vysoce závislý na specifickém HW, SW a přítomnosti závislostí. Pro účely testování byly potřebné NuGet balíčky nacachovány v systému. Následující tabulka zobrazuje přehled časové náročnosti kompilace služeb pro oba kompilační cíle. Data byla získána na základě průměru z 6 pokusů. Nejdelší a nejkratší hodnota kompilace byla zahozena pro dodatečné omezení vlivu cachování. Jelikož je hodnota ovlivněna externím zatížením systému, má výsledek pouze orientační charakter.

K získání času výstupu bylo využito diagnostického režimu příkazu \emph{dotnet}. Pro AOT byl použit příkaz \emph{dotnet publish -v d -c Release-AOT -r osx-x64}, pro získání výstupu JIT byl použit příkaz \emph{dotnet publish -v d -c Release-JIT -r osx-x64 --self-contained false}.

\tab{Čas kompilace služeb}{tab:priklad}{0.65}{|l|c|c|r|}{
  \hline
    Služba & JIT (s) & AOT (s) & AOT \% nárůst \\ \hline \hline
  \emph{SRS} & 01.99 & 19.49 & 979.3 \\ \hline
  \emph{FUS} & 03.85 & 30.36 & 788.5 \\ \hline
  \emph{BPS} & 02.02 & 20.74 & 1026.7 \\ \hline
  \emph{EPS} & 01.85 & 20.05 & 1083.7 \\ \hline
}

Výstup poukazuje na dodatečnou režii interceptorů, generátorů, nástroje pro generování nativního kódu a linkeru. Výsledek se shoduje s předpoklady větší náročnosti AOT kompilace. Pohled na velikost výstupního souboru služeb v kompilačních režimech AOT a JIT je zobrazen v následující tabulce. Služby cílí na architekturu ARM (arm64) a macOS. Pro vytvoření výstupů na základě JIT byl použit příkaz \emph{dotnet publish -c Release-JIT -r osx-x64 /p:PublishSingleFile=true --self-contained false}, pro vytvoření výstupů AOT byl použit příkaz \emph{dotnet publish -c Release-AOT -r osx-x64}.

\tab{Velikost programu služeb pro architekturu osx-arm64}{tab:priklad}{0.65}{|l|c|c|r|}{
  \hline
    Služba & JIT (MB) & AOT (MB) & AOT \% nárůst \\ \hline \hline
  \emph{SRS} & 5.7 & 21.4 & 375.4 \\ \hline
  \emph{FUS} & 12.4 & 28.4 & 229.0 \\ \hline
  \emph{BPS} & 6.0 & 21.8 & 363.3 \\ \hline
  \emph{EPS} & 6.0 & 21.7 & 361.6 \\ \hline
}

Velikost samotného výstupního programu vyšla dle očekávání definovaného v hypotéze 2. Program má výrazně menší velikost v případě JIT kompilace. To je dáno tím, že výstupní program této kompilace je závislý na .NET runtime, který poskytuje dodatečnou obecnou funkcionalitu a vytváří nativní kód včetně generování typů až za běhu aplikace.

Z pohledu velikosti výstupního programu nativní AOT kompilace má smysl porovnat výstup napříč architekturami a OS. Následující tabulka ukazuje velikost výstupu pro Windows v architektuře x64, macOS v architektuře arm64 a Linux v architekturách x64 a arm64. Pro vytvoření výstupů na základě JIT byl použit příkaz \emph{dotnet publish -c Release-AOT -r <target\_OS>-<target\_architecture>}. V příkazu \emph{<target\_OS>} a \emph{<target\_architecture>} představuje vybraný operační systém a architekturu.

\tab{Velikost výstupu programu AOT služeb pro různé OS}{tab:priklad}{1}{|l|c|c|c|c|}{
  \hline
    Služba & win-x64 (MB) & osx-arm64 (MB) & linux-x64 (MB) & linux-arm64 (MB) \\ \hline \hline
  \emph{SRS} & 17.2 & 21.4 & 19.4 & 20.2 \\ \hline
  \emph{FUS} & 22.9 & 28.4 & 25.8 & 26.8 \\ \hline
  \emph{BPS} & 17.5 & 21.8 & 19.8 & 20.6 \\ \hline
  \emph{EPS} & 17.4 & 21.7 & 19.7 & 20.5 \\ \hline
}

Velikost výstupního programu naznačuje větší optimalizaci výstupu pro Windows. Tato skutečnost může být způsobena dostupností specifických knihoven ve Windows, respektive rozsáhlejšího API. Následkem toho je menší množství závislostí v rámci výstupního programu z důvodu zaručení větší části funkcionality v rámci OS. Obecně výsledek vychází lépe pro architekturu x64 než ARM. Cílové využití nativních AOT kompilovaných služeb je vhodné pro cloudové nasazení, které ve většině případů běží na Linux serverech. Také je značně populární architektura ARM, která obecně vykazuje efektivnější provoz k poměru nákladů. Lze tedy předpokládat, že tato architektura byla optimalizována. Z tohoto předpokladu plyne že výsledek Linux x64 a Linux arm64 ukazuje roli architektury na velikost výstupního programu, kdy kompletní instrukční sada x64 architektury umožňuje menší výstupní program.

Z pohledu microservice architektury má velký význam měřit velikost nekomprimovaného výstupního obrazu .NET služeb. Sestavení obrazu je závislé na prostředí, vyhodnocení a stažení závislostí, kompilaci a publikování aplikace. Následující tabulka zobrazuje nekomprimovanou velikost obrazů služeb pro architekturu arm64. Označení \emph{JIT-full} značí obraz JIT služby založený na plnohodnotným .NET runtime a programu publikovaném v režimu \emph{runtime dependent}, tedy ve verzi závislé na OS s přítomným runtime. \emph{JIT-trimmed} označuje redukovaný obraz JIT služby definovaný v kapitole tvorby testovací aplikace. \emph{AOT} označuje obraz AOT služby. Použitý příkaz k sestavení obrazu je \emph{docker build -t <service>:<tag> -f <target>.Dockerfile .}. V příkazu \emph{<target>} představuje vybranou kompilační metodu AOT, JIT nebo JIT.Trimmed, zatímco \emph{<service>} reprezentuje jméno služby. Konkrétní Dockerfile soubory jsou součástí adresáře jednotlivých služeb.

\tab{Nekomprimovaná velikost obrazu služeb}{tab:priklad}{0.9}{|l|c|c|c|}{
  \hline
   Služba & JIT-full (MB) & JIT-trimmed (MB) & AOT (MB) \\ \hline
  \emph{SRS} & 121.97 & 31.03 & 31.41 \\ \hline
  \emph{FUS} & 134.36 & 36.27 & 38.32 \\ \hline
  \emph{BPS} & 122.39 & 31.25 & 31.40 \\ \hline
  \emph{EPS} & 122.26 & 31.22 & 31.74 \\ \hline
}

Ve výsledku jsou obrazy služeb využívajících plnohodnotného .NET runtime výrazně větší než je tomu u kontejnerů nativních AOT služeb. Nicméně \emph{self-contained} varianty obrazu JIT služeb s trimming poskytují ekvivalentní výstup vůči nativním AOT. Použití procesu trimming s sebou nese určitá rizika a nejedná se o běžnou formu nasazení služby. Vývoj služeb s tímto způsobem nasazení nese jisté podobnosti jako vývoj nativních AOT služeb. Přesto je množina omezení v .NET, které použití trimming ukládá na vyvíjenou aplikaci menší, než vyžaduje nativní AOT vývoj. Výsledek ukazuje, že v případě snahy dosáhnout co nejmenší velikosti obrazu, AOT kompilace produkuje obdobnou velikost obrazu jako zacílení na JIT kompilaci s optimalizací velikosti. Výsledek vyvrací hypotézu 2, kdy bylo předpokládáno, že velikost obrazu AOT služeb bude znatelně menší, než tomu je v případě JIT služeb, nezávisle na použití optimalizace.

Základní obraz \emph{dotnet/runtime-deps:8.0-jammy-chiseled} teoreticky obsahuje nepotřebné závislosti a jeho nekomprimovaná velikost 10.23 MB by neměla představovat nejmenší možný základ pro nativní AOT aplikaci. Za účelem tvorby menšího obrazu byly provedeny experimenty s vytvořením vlastního obrazu na bázi Alpine OS s kompatibilitou pro knihovnu \emph{glibc}. Výsledkem tvorby vlastního základního obrazu byl obraz větší než využitý \emph{runtime-deps}. Rovněž byly prozkoumány možnosti využití různých obrazů s knihovnou \emph{glibc}. Na architektuře x64 byl prozkoumán \emph{frolvlad/alpine-glibc}, který představuje jeden z nejmenších spustitelných OS. Základem tohoto obrazu jsou konkrétní verze Apline OS doplněny o specifické verze \emph{glibc}. Služby pro svůj provoz vyžadují \emph{glibc} ve verzi 2.29. Bohužel verze obrazů disponující touto závislostí dosahují větší velikosti než dříve zmíněný \emph{runtime-deps} obraz. Experiment redukce tedy skončil neúspěchem, kdy se nepodařilo vytvořit menší obraz nativní AOT služby. Problematika tvorby a minimalizace obrazu služeb je značně komplexní a závislá na mnoha faktorech. Její úplné pokrytí a prozkoumání všech možností optimalizace obrazů .NET služeb je nad rámec této práce.

\n{2}{Analýza testování}

Následující sekce se zabývá analýzou testovacích scénářů a výsledků testování. Testování bylo provedeno na základě předem definované metodiky. Testy byly vytvořeny s ohledem na funkční a nefunkční požadavky dle podkladů. Samotné testy a výsledky z nich získané mají spíše informativní charakter a představují vzor pro další testování. Na to má vliv variabilita externích vlivů a přidaná hodnota v testování specifický doménových problémů na konkrétním HW. Při testování byl nezávisle na spuštěný test zaznamenáván stav hostitelského systému s ohledem na spuštěné kontejnery a využití systémových prostředků. Samotné služby využívaly předem definované metry ve frameworku ASP.NET pro dodatečnou diagnostiku a monitorování. Výsledky testování byly zaznamenány a analyzovány.

\n{3}{Scénář 1 - Výkonnost komunikace}

První scénář se zabíral jednoduchou funkcionalitou dotazu na healthcheck endpoint a měřením výkonu Kestrel serveru u odpovědí na požadavky skrze REST API. Následující tabulka zobrazuje průměrné využití zdrojů a dobu odpovědi služeb v testovacím scénáři 1 pro oba kompilační režimy.

\tab{Průměrné využití zdrojů a doba odpovědi healthcheck služeb}{tab:service_metrics}{1.0}{|l|r|r|r|r|}{
  \hline
    Služba - Režim & CPU (ms) & IO (ns) & Paměť (MB) & Doba požadavku (ms) \\ \hline \hline
  \emph{SRS-AOT} & 3.41 & 0.550 & 41.1 & 1.61 \\ \hline
  \emph{SRS-JIT} & 9.69 & 0.453 & 41.3 & 3.84 \\ \hline
  \emph{FUS-AOT} & 1.99 & 0.825 & 52.5 & 1.27 \\ \hline
  \emph{FUS-JIT} & 7.62 & 0.458 & 39.3 & 2.22 \\ \hline
  \emph{BPS-AOT} & 1.21 & 0.425 & 37.9 & 2.57 \\ \hline
  \emph{BPS-JIT} & 9.24 & 0.550 & 36.3 & 1.96 \\ \hline
  \emph{EPS-AOT} & 2.47 & 0.451 & 36.5 & 2.07 \\ \hline
  \emph{EPS-JIT} & 6.63 & 0.686 & 35.3 & 3.09 \\ \hline
}

Testování přineslo rozdílné výkonnostní výsledky mezi JIT a AOT kompilací. Dle předpokladu AOT služby využívaly méně času CPU. Paměťová stopa však u nich byla větší. Konečně, AOT služby byly schopné v průměru rychleji odpovídat. Výsledek poukazuje na dodatečnou režii od CLR na CPU v případě JIT služeb. Větší využití paměti u AOT služeb poukazuje na režii vygenerovaných typů a přítomnost dalších závislostí v programu načteného v paměti. Rozdíl využití IO je s ohledem na dobu využití a počet požadavků zanedbatelný. Čistá rychlost zpracování požadavku ukázala na rychlejší zpracování AOT službami. I když tato hodnota není v mnoha případech kritickým faktorem, v případě velkého množství využitých služeb pro splnění požadavku, se může rozdíl v řádech milisekund nasčítat do znatelné prodlevy.

\n{3}{Scénář 2 - Přístup k perzistenci}

Scénář se zabýval výkonností přístupu k perzistenci, respektive simulací běžného scénáře, kdy jsou data získávána a ukládána do databáze. Faktorem byla jak samotná rychlost služby v ohledu komunikace a serializace dat, tak rychlost zpracování požadavku databází. Následující tabulka zobrazuje průměrné využití zdrojů a dobu odpovědi služby FUS v testovacím scénáři 2 pro oba kompilační režimy.

\tab{Průměrné využití zdrojů službou FUS a doba odpovědi stažení a nahrání souboru}{tab:service_metrics}{1.0}{|l|r|r|r|r|}{
  \hline
  Služba - Režim & CPU (\%) & IO (ns) & Paměť (MB) & Doba požadavku (ms) \\ \hline \hline
  \emph{FUS-AOT} & 0.289 & 2.132 & 47.2 & 49.26 \\ \hline
  \emph{FUS-JIT} & 0.602 & 2.384 & 75.9 & 63.38 \\ \hline
}

Ve výsledku je vidět výrazný rozdíl ve využití zdrojů mezi AOT a JIT verzi služby. Oproti předchozímu měření je AOT paměťově efektivnější. Poukazuje na méně efektivní datovou manipulaci při serializaci v přístupu k perzistenci, ale také při serializaci a kompresi souboru na rozhrání REST API. V případě doby odpovědi JIT služby na první požadavek vyžadující generování typů nebo serializace je obecně znatelná prodleva. To je následkem JIT kompilace potřebného kódu při prvním volání. Oproti tomu první volání totožného požadavku na AOT variantě služby má obdobnou rychlost odpovědi jako průměrný požadavek.

\tab{Průměrné využití GC službou FUS}{tab:service_metrics}{1.0}{|l|r|r|r|}{
  \hline
  Služba - Režim & Alokovaná paměť (MB) & Doba běhu (ms) & Velikost objektů (MB) \\ \hline \hline
  \emph{FUS-AOT} & 253.0 & 107.0 & 15.6 \\ \hline
  \emph{FUS-JIT} & 237.0 & 134.0 & 16.6  \\ \hline
}

Vysoká paměťová režie na zpracování požadavků pro nahrání a stažení souboru má za výsledek výraznou alokaci a běh GC. Náročnost scénáře odfiltrovala nepatrné rozdíly mezi využití GC dané podstatou jejich běhového prostředí a způsobem zpracování požadavku. Výsledek ukazuje, že při dostatečné zátěži jsou výsledky AOT a JIT služeb srovnatelné.

\n{3}{Scénář 3 - Výpočetní zátěž}

Za účelem zjištění výkonnosti služeb, jejich potencionálně odlišné využití systémového API byl otestován scénář výpočetní zátěže. Na jednotlivé služby byly vysílány požadavky na výpočet 40. Fibonacciho čísla rekurzivní metodou. Následující tabulka zobrazuje průměrné využití zdrojů a dobu odpovědi služeb v testovacím scénáři 3 pro oba kompilační režimy.

\tab{Průměrné využití zdrojů službou BPS a doba odpovědi výpočtu Fibonacciho čísla}{tab:service_metrics}{1.0}{|l|r|r|r|r|}{
  \hline
  Služba - Režim & CPU (\%) & IO (ns) & Paměť (MB) & Doba požadavku (s) \\ \hline \hline
  \emph{BPS-AOT} & 2.596 & 0.550 & 24.5 & 5.80 \\ \hline
  \emph{BPS-JIT} & 2.783 & 0.637 & 32.1 & 6.07  \\ \hline
}

Výsledky testování ukázaly že při náročné výpočetní zátěži žádná z kompilací nebyla výrazně výkonnější. Oproti dřívějším situacím využila AOT kompilovaná služba více času CPU než JIT kompilovaná. Tato skutečnost naznačuje, že výkonnostní výhody AOT služeb jsou při větší zátěži srovnány. V ostatních měřených hodnotách byly rozdíly zanedbatelné. 

\tab{Průměrné využití GC službou BPS}{tab:service_metrics}{1.0}{|l|r|r|r|}{
  \hline
  Služba - Režim & Alokovaná paměť (MB) & Doba běhu (ms) & Velikost objektů (MB) \\ \hline \hline
  \emph{BPS-AOT} & 19.0 & 18.3 & 9.6 \\ \hline
  \emph{BPS-JIT} & 9.1 & 3.1 & 8.5  \\ \hline
}

Dle očekávání z dřívějších výsledků i zde AOT varianta služby si vyžádala více běhu GC a alokované paměti. Na výslednou rychlost odpovědi služby to však nemělo vliv.

\n{3}{Scénář 4 - Vzájemná komunikace služeb}

Komplexnější situace pro aplikaci byla simulována ve čtvrtém scénáři. Situace simulovala kombinaci synchronní a asynchronní komunikace mezi službami, přístup k perzistenci a výpočetní zátěž. 

\tab{Průměrné využití zdrojů službami dle nasazení v kompilačních režimu}{tab:service_metrics}{1.0}{|l|r|r|r|}{
  \hline
  Režim & CPU (\%) & IO (ns) & Paměť (MB) \\ \hline \hline
  \emph{AOT} & 3.732 & 4.830 & 124.3 \\ \hline
  \emph{JIT} & 3.035 & 7.560 & 138.5 \\ \hline
}

Výsledky ukázaly očekávané chování dle předchozích výsledků a hypotéz. Zpracování požadavku skrze více služeb je obdobně náročné pro oba kompilační režimy. Samotné využití zdrojů neposkytuje opověď pro vhodnější kompilační režim pro množinu minimalistických služeb zpracovávajících požadavek v rámci aplikace.

\n{3}{Scénář 5 - Rychlost odpovědi služby po startu}

V tomto scénáři byla vyvolána simulace Serveless nasazení. Jednotlivé varianty služby SRS byly v rámci testu spuštěny, kontrolovány než se dostaly do stavu \emph{healthy} a následně nad nimi byl zavolán dotaz pro získání generovaných dat. Pro tento specifický scénář bylo doplněno testování i trimmed verze JIT služby SRS. Tabulka níže zobrazuje průměrné využití paměti a dobu odpovědi služby SRS v testovacím scénáři 5 pro oba kompilační režimy. 

\tab{Průměrné využití zdrojů službou SRS a doba odpovědi včetně startu služby}{tab:service_metrics}{1.0}{|l|r|r|r|r|}{
  \hline
  Služba - Režim & Doba startu služby a požadavku (s) & Paměť (MB) \\ \hline \hline
  \emph{SRS-AOT} & 0.945 & 12.40 \\ \hline
  \emph{SRS-JIT} & 2.434 & 33.07 \\ \hline
  \emph{SRS-JIT-Trimmed} & 2.464 & 35.84 \\ \hline
}

Výsledky ukázaly, že služba kompilovaná nativním AOT způsobem je výrazně rychleji dostupná a odpovídá na požadavky dříve než služba kompilovaná pro CLR. Celková doba startu kontejneru však nepoměrně odpovídá propagované rychlosti startu samotného programu. To je dáno režií kontejneru, tedy jeho OS. Rychlost odpovědi je tedy velkým poměrem závislá na konkrétním prostředí, ve kterém je služba spuštěna a scénáři nasazení. V případě microservice architektury a kontejnerizovaného nasazení je kritickým faktorem právě doba odpovědi včetně kontejneru. Přesto výsledek ukazuje, že AOT kompilovaná služba je mnohem vhodnější varianta pro scénáře, kdy není vhodné aby odpovídající kontejner neustále běžel. Ať už je důvodem pro tento stav úspora zdrojů nebo bezpečnostní požadavky.

\n{2}{Závěr analýzy}

Na základě výsledků vývoje, výstupu a testování služeb lze zhodnotit definované hypotézy následujícím způsobem:

\begin{itemize}
  \item \textbf{Hypotéza 1} - Hypotéza, že vývoj služeb s jak AOT, tak JIT kompilací je v rámci podporované funkcionality systémových knihoven a ASP.NET možný s podobným API se ukázal jako ne zcela pravdivý. Při vývoji nastaly komplikace se serializací konfigurace, na které bylo nutné reagovat využitím odlišného API. Další problémy nastaly s využitím Entity Framework. Tento ORM využívá pro provádění operací nad databází tzv. tracking, který zaznamená změny nad aplikačními objekty a podle nich tvoří výsledné databázové operace. Vypnutím trackingu bylo umožněno se na datové entity dotázat a aktualizovat je. Operace vložení nové entity však bez trackingu nebyla možná. Pro knihovny 3. stran lze obecně říci, že podpora AOT kompilace není vždy úplně zřejmá. I v situacích, kdy AOT varování jsou implementovány, neznamenají nutně, že využité API nebude funkční, ale jejich absence ve výpisu kompilace nezaručí bezproblémový běh aplikace.
  \item \textbf{Hypotéza 2} - Výsledky ukazují, že služby kompilované nativně AOT se výrazněji rychleji spouští jak na hostitelských systémech, tak v kontejnerizovaném prostředí. Ovšem při započtení režie startu virtualizovaného OS, není tento rozdíl až tak markantní. Binární velikosti samotných aplikací jsou mnohonásobně větší, než je tomu u služeb vyžadující CLR. Vytvořené obrazy nativních AOT služeb disponují menší velikostí než obrazy JIT služeb s plnohodnotným runtime, ovšem jejich velikost je obdobná jako u specificky zmenšených JIT služeb. Výsledné obrazy AOT služeb jsou ve výsledku ekvivalentní velikostí a rychleji spustitelné. Hypotéza nebyla potvrzena.
  \item \textbf{Hypotéza 3} - Na základě dostupných metrik bylo potvrzeno, že obecně služby kompilované AOT mají menší využití CPU. Využití paměti a IO se ukázalo být srovnatelné jak u služeb kompilovaných JIT v CLR. Toto chování je způsobeno rozdílem v době, kdy se generují potřebné typy a části funkcionality aplikace. JIT kompilace umožňuje za provozu využít méně paměti, nicméně ani tento fakt nebyl pravidlem ve všech scénářích. Zároveň bylo ale pozorováno zvýšené využití GC v případě služeb kompilovaných AOT. I přes tuto dodatečnou režii byly obecně efektivnější a hypotéza byla potvrzena. Rozdíly však nebyly tak signifikantní u náročnějších procesů a významný rozdíl mezi kompilacemi byl znatelný primárně při požadavcích s malým zatížením na službu.
\end{itemize}