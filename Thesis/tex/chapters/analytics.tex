\n{1}{Analýza aplikace}

Tato kapitola se zabývá analýzou aplikace z hlediska vývoje, výstupu a výkonu. Využívá k tomu definovanou metodiku a scénáře testování. Výsledky jsou důkladně analyzovány a závěry shrnuty v jednotlivých sekcích.

\n{2}{Architektura}

Výsledná architektura aplikace je založena na mikroslužbách. Splňuje předem definované funkční a nefunkční požadavky. V případě testovaných služeb, zapojuje základní množinu systémových knihoven a knihoven 3. stran. Po straně telemetrie, implementuje sběr a zpracování dat z různých zdrojů. Výsledná data jsou následně zpracována a uložena do databáze, dle druhu dat. Veškeré dostupné zdroje jsou uživatelsky přívětivě vizualizovány v rámci webového aplikace Grafana. Aplikační stack je testovatelný a nasaditelný na všech hlavních platformách (po sestavení se zacílením na vybranou architekturu a využitím variant služeb třetích stran s cílovou architekturou).


\n{2}{Vývojový proces}

Následující sekce popisuje vývojový proces, tak jak se týkal testovaných služeb. Vývojový proces byl založen na experimentaci a snaze využít co nejvíc dostupných knihoven a nástrojů, za cenu nutnosti řešení problémů, případně změny implementace.

\obr{Ukázka kódu s vyzualizací direktiv dle konfigurace}{fig:codesample}{1}{graphics/images/code-visual-sample.png}

\n{3}{JIT}

Vývojový proces pro kompilaci služeb JIT se zacílením na .NET runtime probíhal standarntím způsobem. Veškeré dostupné knihovny a nástroje byly plně kompatibilní s JIT kompilací. Nedošlo k žádným nepředpokládaným problémům.

Znatelný rozdíl oproti běžnému vývoji byl výběr technologií, který přihlížel k potencionální kompatibilitě s AOT a tedy řešení, které inherntně vyžadovala funkce rezervované pro využití .NET runtime, byly ihned zavrženy.

\n{4}{Výhody}

Mezi hlavní výhody se řadí zprostředkování následujícího:

\begin{itemize}
    \item  \textbf{Reflexe} - CLR umožňuje využívat reflexi, která umožňuje získat informace o kódu za běhu aplikace. Tímto je umožněno vytvářet aplikace, které jsou schopny měnit své chování za běhu.
    \item \textbf{Dynamické načítání} - CLR umožňuje dynamicky načítat knihovny za běhu aplikace. Tímto je umožněno vytvářet aplikace, které jsou schopny měnit své chování za běhu.
    \item \textbf{Větší bezpečnost} - CLR zajišťuje, že aplikace nemůže přistupovat k paměti, která jí nebyla přidělena. Tímto je zajištěna bezpečnost aplikace a zabráněno chybám, které by mohly vést k pádu aplikace.
    \item \textbf{Správa paměti} - CLR zajišťuje správu paměti pomocí GC. Tímto je zajištěno, že paměť je uvolněna vždy, když ji aplikace již nepotřebuje. Tímto je zabráněno tzv. memory leakům, které by mohly vést k pádu aplikace.
    \item \textbf{Větší přenositelnost} - CLR zajišťuje, že aplikace je spustitelná na všech operačních systémech, na kterých je dostupné běhové prostředí CLR.
\end{itemize}

\n{4}{Nevýhody}

Zatímco za nevýhody CLR se dá považovat:

\begin{itemize}
    \item  \textbf{Výkonnost} - I když určité optimalizace jsou prováděny pro konkrétní systém a architekturu, výkon CLR je nižší než výkon nativního kódu. Dalším výkonnostním měřítkem je rychlost startu aplikace, která je pro CLR vyšší než v případě nativního kódu.
    \item \textbf{Operační paměť} - CLR využívá více operační paměti, jak pro aplikaci, tak i pro běhové prostředí.
    \item \textbf{Velikost aplikace} - Přítomnost CLR nehraje zásádní roli v případě monolitických aplikací, ale v případě mikroslužeb je nutné CLR přidat ke každé službě. Tímto se zvyšuje velikost jedné aplikační instance.
\end{itemize}

\n{3}{AOT}

Kompilace do nativního kódu probíhala s průběžnými problémy. Podpora ze strany knihoven 3. stran ve spoustě případů neodopvídala deklarovaným možnostem. Vývojový proces byl značně zpomalován nutností řešení problémů, které byly způsobeny nedostatečnou podporou. Experimentace s řešeními často vyústila v nutnost změny implementace, případě v implementaci zcela vlastní.

\n{4}{Výhody}

Mezi hlavní výhody se řadí zprostředkování následujícího:

\begin{itemize}
    \item  \textbf{Nezávislost na CLR} - AOT kompilace umožňuje vytvořit aplikaci, která je schopna běžet bez nutnosti běhového prostředí. Tímto je zajištěno, že aplikace je schopna běžet na jakémkoliv operačním systému, na kterém je dostupný běhové prostředí.
    \item \textbf{Efektivní využití zdrojů} - Aplikace využívající nativní AOT kompilaci efektivněji využívají zdroje jako CPU a operační paměť.
    \item \textbf{Rychlejší start aplikace} - Aplikace zkompilované do nativního kódu cílové architektury se spouští daleko rychleji než aplikace využívající běhové prostředí.
    \item \textbf{Rychlá odpověď aplikace} - Díky tomu, že aplikace musí mít všechny typy a funkcionality vygenerovány ve chvíli kompilace, je rychlost první odpovědi aplikace vyšší než v případě aplikace využívající běhové prostředí.
\end{itemize}

\n{4}{Nevýhody}

Následující představují hlavní nevýhody AOT kompilace:

\begin{itemize}
    \item \textbf{Absence dynamického načítání} - Například funkce Assembly.LoadFile, jenž umožňuje dynamicky načítat knihovny za běhu aplikace, není dostupná v AOT kompilaci.
    \item \textbf{Bez generování kódu za běhu} - Není možno použít knihovnu System.Reflection.Emit pro generování kódu za běhu aplikace.
    \item \textbf{Vyžaduje trimming (ořezávání)} - Trimming vyžaduje, aby veškeré nepřímo používané části byly explicitně deklarovány, jinak budou vyřazeny z výsledného kódu.
    \item \textbf{Připojení běhových knihoven} - Veškeré potřebné knihovny jsou součástí výsledného aplikačního souboru. To zvyšuje velikost samoteného programu ve srovnání s aplikacemi závislými na runtime protředí.
    \item \textbf{Kompatibilita knihoven s AOT} - né všechny knihovny runtime jsou plně anotovány tak, aby byly kompatibilní s Native AOT. To znamená, že některá varování v knihovnách runtime nejsou pro koncové vývojáře použitelná.
\end{itemize}

\n{2}{Výstup služeb}

Samotný proces nativní AOT a JIT kompilace je různě výkonnostně náročný. Při tvorbě obrazu službeb, ale i kompilace je hlavní náročná operace \emph{restore}, která stahuje potřebné závislosti a balíčky pro projekt. Proces kompilace je vysoce závislý na specifickém HW, SW a přítomnosti závislostí. Pro účely testování byly potřebné NuGet balíčky nacachovány v systému. Následující tabulka zobrazuje přehled časové náročnosti kompilace služeb pro oba kompilační cíle. K získání času výstupu bylo využity diagnostického režimu příkazu \emph{dotnet}. Pro AOT byl použit příkaz \emph{dotnet publish -v d -c Release-AOT -r osx-x64}, pro získání výstupu JIT byl použit příkaz \emph{dotnet publish -v d -c Release-JIT -r osx-x64 --self-contained false}.


\tab{Čas kompilace služeb}{tab:priklad}{0.65}{|l|c|c|r|}{
  \hline
    & JIT (s) & AOT (s) & AOT \% nárůst \\ \hline
  \emph{SRS} & 01.99 & 19.49 & 979.3 \\ \hline
  \emph{FUS} & 03.85 & 30.36 & 788.5 \\ \hline
  \emph{BPS} & 02.02 & 20.74 & 1026.7 \\ \hline
  \emph{EPS} & 01.85 & 20.05 & 1083.7 \\ \hline
}

Velikost samotného výstupního programu je dle očekávání výrazně menší v případě JIT kompilace. To je dáno tím, že výstupní program je závislý na .NET runtime, který poskytuje dodatečnou obecnou funkcionalitu a vytváří nativní kód včetně generování typů až za běhu aplikace. Následující tabulka zobrazuje velikost služeb pro oba kompilační cíle. Pro vytvoření výstupů na základě JIT byl použit příkaz \emph{dotnet publish -c Release-JIT -r osx-x64 /p:PublishSingleFile=true --self-contained false}, pro vytvoření výstupů AOT byl použit příkaz \emph{dotnet publish -c Release-JIT -r osx-x64 /p:PublishSingleFile=true --self-contained false}.

\tab{Velikost programu služeb}{tab:priklad}{0.65}{|l|c|c|r|}{
  \hline
    & JIT (MB) & AOT (MB) & AOT \% nárůst \\ \hline
  \emph{SRS} & 05.70 & 21.40 & 375.4 \\ \hline
  \emph{FUS} & 12.40 & 28.40 & 229.0 \\ \hline
  \emph{BPS} & 06.00 & 21.80 & 363.3 \\ \hline
  \emph{EPS} & 06.00 & 21.70 & 361.6 \\ \hline
}

Sestavení obrazu je závislé na přípravu prostředí, vyhodnocení a stažení závislostí, kompilaci a publikování aplikace. Výstupné obrazy jsou založené na linuxovém systému, Alpine s .NET runtime v případě JIT výstupu služby, zredukované Ubuntu v případě nativního AOT výstupu. Z pohledu použitelnosti výsledného obrazu služeb má smysl měřit velikost výstupního obrazu. Následující tabulka zobrazuje velikost obrazu služeb pro oba kompilační cíle. Použitý příkaz je \emph{docker build -t <service>:<tag> -f Dockerfile-<target> .}, kdy \emph{<target>} představuje vybranou kompilační metodu AOT nebo JIT. Před každým sestavením byl obraz a cache smazány. I přes toto opatření není zaručena konzistentní časová náročnost sestavení obrazu.

\tab{Velikost obrazu služeb}{tab:priklad}{0.65}{|l|c|c|r|}{
  \hline
    & JIT (MB) & AOT (MB) & AOT \% zmenšení \\ \hline
  \emph{SRS} & 121.97 & 31.41 & 74.3 \\ \hline
  \emph{FUS} & 134.36 & 38.32 & 71.5 \\ \hline
  \emph{BPS} & 122.39 & 31.40 & 74.3 \\ \hline
  \emph{EPS} & 122.26 & 31.74 & 74.0 \\ \hline
}

\n{3}{Vývojové prostředí}

K vývoji byl použit IDE Rider od společnosti JetBrains. Vyzkoušena byla rovněž i práce ve Visual Studio 2022 Community Edition a Visual Studio Code s doporučenými rozšířeními od Microsoft. Všechna vývojová prostředí jsou kompatibilní, co se týče procesu kompilace respektive sestavení, jelikož to se odehrává pomocí CLI .NET.

Samotný vývoj s ohledem na práci s direktivami pro různé kompilace byl značně zjednodušen vizualicemi, jenž poskytovala vývojová prostředí Rider a Visual Studio. Obdobně byla v těchto IDE zjednodušena i analýza a hledání chyb díky integraci referencí na kód generovaný na pozadí pro kompatibilitu s AOT. V tomto ohledu Visual Studio Code zaostávalo. S ohledem na aktivní vývoj a podporu, jenž je ze strany Microsoft poskytována podpoře vývoje .NET ve Visual Studio Code (po diskontuaci produktu Visual Studio pro Mac), lze očekávat, že se tato situace v budoucnu změní.

\n{3}{Knihovny třetích stran}

Pro zjednodušení procesu vývoje a využití existující funkcionality byly využity knihovny třetích stran. Následující seznam obsahuje knihovny, které byly využity použity v rámci vývoje a zda byly kompatibilní s AOT kompilací.

\begin{itemize}
  \item \textbf{Entity Framework} - Entity framework se pyšní vysokou kompatibilitou s AOT kompilací. V rámci vývoje nebyly zaznamenány problémy, avšak následné testování se ukázalo problematické. EF jakožto plnohodnotný ORM framework stopuje stav objektu a jeho změny. Toto chování bohužel vyžaduje dynamické generování kódu, což je v rozporu s možnostmi AOT kompilovaného kódu. Vypnutí této funkcionality je pouze částečné, neb EF stále vyžaduje reflexi při vkládání nových entit do databáze.
  \item \textbf{Fluent Migrator} - Fluent Migrator je knihovna, která umožňuje verzování databáze pomocí kódu. V rámci testování bylo zjištěno, že knihovna využívá reflexi pro načítání migrací. Toto chování je v rozporu s AOT kompilací a výsledkem je chyba při spuštění migrace. Problém byl vyřešen vytvořením vlastního minimalistického migrátoru, který nepoužívá reflexi.
  \item \textbf{Grpc} - Vytváření rozhrání a modelů pro gRPC komunikaci vyžadovalo využití přístupu model first. Tento přístup využívá generátorů pro tvorbu kódu, definijucího kódového rozhraní pro .NET. Tímto je dosaženo vygenerování veškerého potřebného kódu v době kompilace a je zajištěna kompatibila s AOT. Pro definici modelu code first ovšem kombatibila s AOT není zajištěna.
  \item \textbf{Párování konfigurace} - V rámci systémové .NET knihovny je umožněno volání API, jenž načte data ze sjednocení stavu proměnných prostředi a konfiguračního souboru. Součástí API je volání metody mapující tuto konfiguraci na předem definovaný objekt. Toto chování dle dostupných informací není v rozporu s AOT kompilací a volání relevantního kódu neprodukuje AOT warning. Z testování však vyplynulo, že mapování konfigurace ne objekt bylo problematické a neprobíhalo správně. Z toho důvodu je v případě AOT kompilace za pomocí deriktivy použité přímé načtení jednotlivých hodnot z konfigurace, dle stromomvého klíče.
\end{itemize}

\n{2}{Analýza testování}

Následující sekce se zabývá analýzou testovacích scénářů a výsledků testování. Testování bylo provedeno na základě předem definované metodiky. Podkladem testů byly definované scénáře, které byly vytvořeny s ohledem na funkční a nefunkční požadavky. Při testování byl nezávisle na spuštěný test zaznaménáván stav hostitelského systému s ohledem na spuštěné kontejnery a využití systémových prostředků. Samotné služby využívaly předem definované metry ve frameworku ASP.NET pro dodatečnou diagnostiku a monitorování. Výsledky testování byly zaznamenány a analyzovány.

\n{3}{Scénář 1 - Výkonnost komunikace}

První scénář se zabíral jednoduchou funkcionalitou dotazu na healthcheck endpoint a meřením výkonu kestrel serveru u odpovědí na požadavky skrze REST API. Testování přineslo rozdílné výkonostní výsledky mezi JIT a AOT kompilací. Dle předpokladu služby s nativním kódem využívaly méně času procesoru. Paměťová stopa však u nich byla větší. Konečně, AOT služby byly schopné v průměru rychleji odpovídat. Čistá rychlost zpracování požadavku a odpovědi není v mnoha případech kritickým faktorem. Avšak v případě velkého množství požadavků, může být rozdíl v řádech milisekund zásadní.

\tab{Průměrné využití zdrojů a doba odpovědi healthcheck služeb}{tab:service_metrics}{1.0}{|l|r|r|r|r|}{
  \hline
    Služba - Režim & CPU (ms) & IO (ns) & Paměť (MB) & Doba požadavku (ms) \\ \hline \hline
  \emph{SRS-AOT} & 3.41 & 0.550 & 41.1 & 1.61 \\ \hline
  \emph{SRS-JIT} & 9.69 & 0.453 & 41.3 & 3.84 \\ \hline
  \emph{FUS-AOT} & 1.99 & 0.825 & 52.5 & 1.27 \\ \hline
  \emph{FUS-JIT} & 7.62 & 0.458 & 39.3 & 2.22 \\ \hline
  \emph{BPS-AOT} & 1.21 & 0.425 & 37.9 & 2.57 \\ \hline
  \emph{BPS-JIT} & 9.24 & 0.550 & 36.3 & 1.96 \\ \hline
  \emph{EPS-AOT} & 2.47 & 0.451 & 36.5 & 2.07 \\ \hline
  \emph{EPS-JIT} & 6.63 & 0.686 & 35.3 & 3.09 \\ \hline
}

\n{3}{Scénář 2 - Přístup k perzistenci}

Scénář se zabýval výkonností přístupu k persistenci, respektive zachystením reálného scénáře, kdy jsou data získávána a ukládána do databáze. Faktorem byla jak samotná rychlost služby v ohledu komunikace a serializace dat, tak rychlost zpracování požadavku databází. Ve výsledku je vidět výrazný rozdíl ve využití zdrojů mezi AOT a JIT verzi služby, kdy první jmenovaná je výrazně efektivnější.

\tab{Průměrné využití zdrojů službou FUS a doba odpovědi stažení a nahrání souboru}{tab:service_metrics}{1.0}{|l|r|r|r|r|}{
  \hline
  Služba - Režim & CPU (ms) & IO (ns) & Paměť (MB) & Doba požadavku (ms) \\ \hline \hline
  \emph{FUS-AOT} & 1.9 & 2.208 & 29.3 & 4.18 \\ \hline
  \emph{FUS-JIT} & 16.7 & 2.000 & 60.9 & 8.05 \\ \hline
}

V případě doby odpovědi služby je velmi znatelný rozdíl služby kompilované JIT kdy její hodnota činila 93.6 ms. Následkem JIT kompilace potřebného kódu při prvním volání byla tato doba výrazně vyšší než v dalších voláních. Oproti tomu AOT varianta služby měla i při prvním volání odpoveď srovnatelnou s průměrným voláním a to 11.8 ms.


\tab{Průměrné využití GC službou FUS}{tab:service_metrics}{1.0}{|l|r|r|r|}{
  \hline
  Služba - Režim & Alokovaná paměť (MB) & Doba běhu (ms) & Velikost objektů (MB) \\ \hline \hline
  \emph{FUS-AOT} & 25.8 & 25.9 & 15.2 \\ \hline
  \emph{FUS-JIT} & 14.0 & 5.3 & 12.9  \\ \hline
}

Přítomnost vygenerovaných typů a funkcionality v nativní AOT verzi služby má za výsledek větší alokace paměti, jenž jsou následně uvolněny, a větší doba běhu GC.

\n{3}{Scénář 3 - Výpočetní zátěž}

Za účelem zjištění výkonnosti služeb, jejich potencionálně odlišné využití systémového API byl otestován scénář výpočetní zátěže. Na jednotlivé služby byly vysílány požadavky na výpočet 40-tého Fibonacciho čísla rekurzivní metodou. Výsledky testování ukázaly že při náročné výpočetní zátěži žádná z kompilací nebyla výrazně výkonnější.

\tab{Průměrné využití zdrojů službou BPS a doba odpovědi výpočtu Fibonacciho čísla}{tab:service_metrics}{1.0}{|l|r|r|r|r|}{
  \hline
  Služba - Režim & CPU (ms) & IO (ns) & Paměť (MB) & Doba požadavku (s) \\ \hline \hline
  \emph{BPS-AOT} & 58.4 & 2.940 & 46.1 & 5.80 \\ \hline
  \emph{BPS-JIT} & 48.8 & 1.164 & 44.8 & 6.07  \\ \hline
}

Dle očekávání z předchozích výsledků i zde AOT varianta služby si vyžádala více běhu GC a alokované paměti. Na výsledek to však nemělo vliv.


\tab{Průměrné využití GC službou BPS}{tab:service_metrics}{1.0}{|l|r|r|r|}{
  \hline
  Služba - Režim & Alokovaná paměť (MB) & Doba běhu (ms) & Velikost objektů (MB) \\ \hline \hline
  \emph{BPS-AOT} & 19.0 & 18.3 & 9.6 \\ \hline
  \emph{BPS-JIT} & 9.1 & 3.1 & 8.5  \\ \hline
}

\n{3}{Scénář 4 - Vzájemná komunikace služeb}

Komplexnější situace pro aplikaci byla simulována ve čtvrtém scénáři. Zde na základě požadavku na EPS byla vyvolána událost do RabbitMQ, načež byla zpracována službou BPS. Ta na jejím základě stáhla patřičný záznam pomocí RPC z FUS a provedla simulaci zpracování dat výpočtem Fibonacciho čísla. Situace simulovala kombinaci synchronní a asynchronní komunikace mezi službami doplněnou o výpočetní zátěž. Výsledky přiblížily služby v obou kompilačních režimech, vyjma využití CPU. Bylo náhlednuto na situaci bližší reálnému nasazení, kdy čistě provozní výkonnostní rozdíly služeb v kompilacích AOT a JIT nehrají zásadní roli. 

\tab{Průměrné využití zdrojů službami dle nasazení v kompilačních režimu}{tab:service_metrics}{1.0}{|l|r|r|r|}{
  \hline
  Režim & CPU (ms) & IO (ns) & Paměť (MB) \\ \hline \hline
  \emph{AOT} & 8.1 & 0.683 & 30.4 \\ \hline
  \emph{JIT} & 24.3 & 1.634 & 36.7 \\ \hline
}

\n{3}{Scénář 5 - Rychlost odpovědi služby po startu}

Simulaci serveless nasazení byla vyvolána v tomto scénáři. Jednotlivé varianty služby SRS byly v rámci testu zpuštěny, kontrolovány než se dostaly do stavu \emph{healthy} a následně nad nimi zavolán dotaz pro získání generovaných dat. Výsledky ukázaly, že služba kompilovaná nativním AOT způsobem je mnohem rychleji dostupná a odpovídá na požadavky dříve, než služba kompilovaná pro .NET runtime. Oproti propagovaným zrychlení v dokumentaci .NET však nebylo dosaženo tak výrazného zrychlení s ohledem na režii kontejneru.

\tab{Průměrné využití zdrojů službou SRS a doba odpovědi včetně startu služby}{tab:service_metrics}{1.0}{|l|r|r|r|r|}{
  \hline
  Služba - Režim & Doba startu + požadavku (s) & Paměť (MB) \\ \hline \hline
  \emph{SRS-AOT} & 0.91 & 9.59 \\ \hline
  \emph{SRS-JIT} & 1.40 & 8.46 \\ \hline
}

\n{2}{Závěr analýzy}

Na základě výsledků vývoje, výstupu a testování služeb lze odpovědět na definované hypotézy následujícím způsobem:

\begin{itemize}
  \item \textbf{Hypotéza 1} - Hypotéza, že vývoj služeb s jak AOT, tak JIT kompilací je v rámci podporované funkcionality systémových knihoven a ASP.NET možný s podobným API se ukázal jako ne zcela pravdivý. Při vývoji nastaly komplikace se serializací konfigurace, na které bylo nutné reagovat využitím odlišného API. Zároveň tento způsob serializace nebyl kompilátorem označen jako potencionálně problematický. Další problémy nastaly s využitím Entity Framework. Tento ORM využívá pro provádění operací nad databází tzv. tracking, který zaznamená změny nad aplikačními objekty a podle nich tvoří výsledné databázové operace. Vypnutím trackingu bylo umožněno se na datové entity dotázat a aktualizovat je. Operace vložení nové entity však bez trackingu nebyla možná. Pro knihovny 3. stran lze obecně říci, že podpora AOT kompilace není vždy úplně zřejmá a i v situacích kdy AOT varování jsou implementovány, lze očekávat chybné chování.
  \item \textbf{Hypotéza 2} - Výsledky ukazují, že služby napsané v nativním kódu se výrazněji rychleji spouští jak na hostitelských systémech, tak ve virtualizovaném prostředí. Zároveň binární velikosti samotných aplikací jsou mnohonásobně větší, než je tomu u služeb vyžadující .NET runtime. To je ovšem kompenzováno při virtualizovaném spuštění, kdy obraz služby pro vytvoření plnohodnotného kontejneru vyžaduje mnohem méně závislotí z hlediska paměti. Výsledné obrazy jsou tedy menší a rychleji spustitelné. Hypotéza byla potvrzena.
  \item \textbf{Hypotéza 3} - Na základě dostupných metrik bylo potvrzeno, že obecně služby kompilované do nativního kódu poskytují vyšší výkon a jsou méně paměťově náročné než služby kompilované pro .NET runtime. Tento fakt je způsoben rozdílem v době, kdy se část generují typy a část funkcionality aplikace. Pro .NET runtime za běhu a pro nativní AOT při sestavení. Zároveň bylo ale pozorováno zvýšené využití GC v případě služeb kompilovaných do nativního kódu. I přes tuto dodatečnou režii byly nativní služby efektivnější a hypotéza byla potvrzena.
\end{itemize}