\cast{Praktická část}

\n{1}{Tvorba tech stacku}
Na této stránce je k vidění způsob tvorby různých úrovní nadpisů.

\n{2}{Požadavky na SW}

Aplikace pro svůj účel nezávislého testování výkonu a škálovatelnosti mikroslužeb vyžaduje několik požadavků, které jsou rozděleny na funkční a nefunkční.

\n{3}{Funkční požadavky}

\n{4}{Sběr a vizualizace dat}

Aplikace musí být schopna sbírat a vizualizovat data o výkonu a škálovatelnosti mikroslužeb. To zahrnuje sběr a vizualizaci metrik, protokolů a tras.

\n{4}{Testování scénářů}

Aplikace musí být schopna provádět testování scénářů, které simuluje zátěž na mikroslužby a zjišťuje, jak se chovají za různých podmínek.

\n{4}{Konfigurace aplikace}

Aplikace musí být schopna konfigurovat testovací scénáře, které se mají provést, a způsob, jakým se mají provést.

\n{3}{Nefunkční požadavky}

\n{4}{Výkon}

Implementace aplikace, respektive jejich služeb, musí být schopna zvládnout zátěž, která je na ně kladena. To zahrnuje schopnost zvládnout požadavky na výkon a škálovatelnost.

\n{2}{Požadavky na HW}

Hardware, na kterém bude aplikace provozována, musí výkonnostně dostačovat pro provozování testovacích scénářů a sběr a vizualizaci dat. Týká se to primárně počtu jader, velikosti paměti a rychlosti diskového I/O. Provozované služby mají určitou základní režii, která se musí brát v potaz.

\n{2}{Výběr technlogií}

\n{3}{Kontejnerizace a orchestrace}

\n{3}{Persistenční vrstva}

\n{3}{Komunikační protokoly}

\n{3}{Monitorovací nástroje}

\n{2}{Návrh a implementace služeb}

\n{3}{Předpoklady služeb}

\n{3}{Implementace služeb}

\n{4}{SRS - Signal Readings Service}

Služba v systému hraje roli čtecího zařízení, které čte data ze zdroje a poskytuje je ostatním službám. Tato služba simulu základní kámen celého systému, značně ovlivňuje výkon a škálovatelnost celého systému. Očekává se velké množství požadavků na tuto službu.

Za účelem zjednodušení implementace není využito čtení dat ze skutečného zdroje, ale jsou generována náhodná data. Načež data jsou následně poskytována se simulovaným zdržením, časově založenému na měření skutečného zdržení systému při čtení dat ze vzdáleného zdroje u obdobného systému. Tato služba je implementována jako REST API (TODO: pokud konečná implementace bude gRPC, změň tuto sekci), které poskytuje data ve formátu JSON. (TODO: Obrázek návrhu architektury a rozhraní služby).

\n{4}{FUS - File Upload Service}

Služba v systému hraje roli zapisovacího zařízení, které zapisuje data do zdroje. Tato služba hraje roli méně vytíženého služby, která nemá značný vliv na fungování systému jako celku. Požadavky, jenž musí vyřídit nejsou kritické a nutné řešit s minimální odezvou.

Služba je implementována s REST API rozhraním. (TODO: Obrázek návrhu architektury a rozhraní služby).

\n{4}{TWS - Test Workers Service}

\n{2}{Konfigurace aplikace}

\n{3}{Konfigurace služeb}

\n{4}{Grafana}

\n{4}{Prometheus}

\n{4}{Loki}

\n{4}{Tempo}

\n{4}{OpenTelemetry}

\n{4}{Postgres}

\n{4}{SRS - Signal Readings Service}

\n{4}{FUS - File Upload Service}

\n{4}{TWS - Test Workers Service}

\n{3}{Konfigurace monitorovacích nástrojů}

\n{3}{Nastavení uživatelského rozhraní}

\n{1}{Testování scénářů}
Na této stránce je k vidění způsob tvorby různých úrovní nadpisů.

\n{2}{Požadavky na scénáře}

\n{2}{Popis scénářů}

\n{3}{Scénář 1 - TBS}

\n{3}{Scénář 2 - TBS}

\n{2}{Zpracování a vizualizace dat}

\n{3}{Monitorování v reálném čase}

\n{3}{Sběr historických dat}

% \n{2}{Obrázek}
% Obrázek \ref{fig:logo} prezentuje logo Fakulty aplikované informatiky.

% % Obrázek lze vkládat pomocí následujícího zjednodušeného stylu, nebo klasickým LaTex způsobem
% % Pozor! Obrázek nesmí obsahovat alfa kanál (průhlednost). Jde to totiž proti požadovanému standardu PDF/A.
% \obr{Popisek obrázku}{fig:logo}{0.5}{graphics/logo/fai_logo_cz.png}


% \n{2}{Tabulka}
% Tabulka \ref{tab:priklad} obsahuje dva řádky a celkem 7 sloupců.

% % Tabulku lze vkládat pomocí následujícího zjednodušeného stylu, nebo klasickým LaTex způsobem
% \tab{Popisek tabulky}{tab:priklad}{0.65}{|l|c|c|c|c|c|r|}{
%   \hline
%    & 1 & 2 & 3 & 4 & 5 & Cena [Kč] \\ \hline
%   \emph{F} & (jedna) & (dva) & (tři) & (čtyři) & (pět) & 300 \\ \hline
% }


% \n{2}{Citování}
% Následuje ukázka odkazování na různé zdroje:
% \begin{itemize}
% 	\item kniha \cite{HRW1997},
% 	\item kapitola v knize \cite{Delorme2006},
% 	\item článek v odborném žurnálu \cite{Bourreau2006},
% 	\item konferenční příspěvek \cite{Judish1999},
% 	\item doktorská práce \cite{Valente2005},
% 	\item technická zpráva \cite{Fralick1997},
% 	\item webová stránka \cite{WWWCST}.
% \end{itemize}