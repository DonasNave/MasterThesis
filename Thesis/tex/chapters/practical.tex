\cast{Praktická část}

\n{1}{Tvorba tech stacku}
Na této stránce je k vidění způsob tvorby různých úrovní nadpisů.

\n{2}{Požadavky na SW}

Aplikace pro svůj účel nezávislého testování výkonu a škálovatelnosti mikroslužeb vyžaduje několik požadavků, které jsou rozděleny na funkční a nefunkční.

\n{3}{Funkční požadavky}

\n{4}{Sběr a vizualizace dat}

Aplikace musí být schopna sbírat a vizualizovat data o výkonu a škálovatelnosti mikroslužeb. To zahrnuje sběr a vizualizaci metrik, protokolů a tras.

\n{4}{Testování scénářů}

Aplikace musí být schopna provádět testování scénářů, které simuluje zátěž na mikroslužby a zjišťuje, jak se chovají za různých podmínek.

\n{4}{Konfigurace aplikace}

Aplikace musí být schopna konfigurovat testovací scénáře, které se mají provést, a způsob, jakým se mají provést.

\n{3}{Nefunkční požadavky}

\n{4}{Výkon}

Implementace aplikace, respektive jejich služeb, musí být schopna zvládnout zátěž, která je na ně kladena. To zahrnuje schopnost zvládnout požadavky na výkon a škálovatelnost.

\n{2}{Požadavky na HW}

Hardware, na kterém bude aplikace provozována, musí výkonnostně dostačovat pro provozování testovacích scénářů a sběr a vizualizaci dat. Týká se to primárně počtu jader, velikosti paměti a rychlosti diskového I/O. Provozované služby mají určitou základní režii, která se musí brát v potaz.

\n{2}{Výběr technlogií}

Součástí tvorby tech stacku je výběr technologií, které budou použity pro implementaci aplikace. Výběr technologií je závislý na požadavcích na aplikaci a HW, na kterém bude aplikace provozována.

\n{3}{Kontejnerizace a orchestrace}

Základním prvkem nasazení aplikace je kontejnerizace a orchestrace. Kontejnerizace zajišťuje, že aplikace bude spouštěna v izolovaném prostředí, které je nezávislé na hostitelském systému. Orchestrace zajišťuje, že aplikace bude spouštěna na dostupných zdrojích a bude schopna zvládnout zátěž, která je na ni kladena.

Pro kontejnerizaci byla zvolena technologie Docker. Docker je open-source platforma pro kontejnerizaci aplikací, která umožňuje vytvářet, spouštět a spravovat kontejnery.

Pro orchestraci byla vybrána technologie Kubernetes. Kubernetes je open-source platforma pro orchestraci kontejnerů, která umožňuje automatizovat nasazování, škálování a správu aplikací. Kubernetes je schopný pracovat s kontejnery, které jsou vytvořeny pomocí Dockeru.

\n{3}{Persistenční vrstva}

Pro persistenci relačních dat byla zvolena technologie PostgreSQL. PostgreSQL je open-source relační databázový systém, který je schopný zvládnout velké množství dat a zároveň poskytovat vysoký výkon.

\n{3}{Komunikační protokoly}

Pro komunikaci mezi službami byl zvolen protokol HTTP. Verze HTTP/2 byla zvolena pro její schopnost zvládnout vysoký počet požadavků a zároveň poskytovat vysoký výkon.

\n{3}{Monitorovací nástroje}

Pro monitorování aplikace byl zvolen Grafana observability stack pro jeho pokrytí komplexní škály monitorovacích dat. Grafana observability stack zahrnuje nástroje pro sběr, vizualizaci a analýzu dat.

\n{4}{Grafana}

je open-source platforma pro vizualizaci a analýzu dat. Grafana umožňuje vizualizovat data z různých zdrojů, včetně časových řad, protokolů a tras.

\n{4}{Prometheus}
 
je open-source systém pro sběr a vizualizaci metrik. Prometheus umožňuje sbírat metriky z různých zdrojů, včetně aplikací, systémů a služeb.

\n{4}{Loki}

je open-source systém pro sběr a vizualizaci protokolů. Loki umožňuje sbírat protokoly z různých zdrojů, včetně aplikací, systémů a služeb.

\n{4}{Tempo}

je open-source systém pro sběr a vizualizaci tras. Tempo umožňuje sbírat trasy z různých zdrojů, včetně aplikací, systémů a služeb.

\n{4}{OpenTelemetry}

je open-source systém pro sběr a vizualizaci metrik, protokolů a tras. OpenTelemetry umožňuje sbírat metriky, protokoly a trasy z různých zdrojů, včetně aplikací, systémů a služeb.

\n{4}{K6}

K6 je nástroj pro výkonové testování, který umožňuje vývojářům testovat výkon svých aplikací. K6 umožňuje vývojářům vytvářet a spouštět testy, které simuluji reálné uživatelské scénáře. Tímto je zajištěno, že aplikace je schopna zvládnout požadavky uživatelů. K6 je nástroj, který je možné využít pro testování mikroslužeb, protože umožňuje vývojářům vytvářet testy, které simuluji reálné uživatelské scénáře.

\n{3}{Testovací služby}

Pro implementaci testovacích služeb byl zvolena technologie dotnet, konkrétně jazyk C\#. Dotnet je open-source platforma pro vývoj a provozování aplikací, která umožňuje vytvářet výkonné a škálovatelné aplikace. Služby budou implementovány jako mikroslužby, které budou spouštěny v kontejnerech. Služby jsou vytvořeny ve dvou verzích, které se liší v použitém způsobu kompilace, a to JIT a AoT.

\n{2}{Návrh a implementace testovacích služeb}

\n{3}{Předpoklady služeb}

Služby musí být implementovány tak, aby v obou kompilačních verzích poskytovaly totožnou funkcionalitu. Jejich chování musí být konfigurovatelné na úrovni kontejneru, který je spouští. Zároveň musí sbírat data o svém chování a poskytovat je monitorovacím nástrojům.

AoT kompilované služby budou otestovány s ohledem na možné kompilační optimalizace, které ovlivňují výsledný program. Toto chování je ovlivňeno atributem OptimizationPreference, který je součástí konfigurace služby.

\n{3}{Implementace služeb}

\n{4}{SRS - Signal Readings Service}

Služba v systému hraje roli čtecího zařízení, které čte data ze zdroje a poskytuje je ostatním službám. Tato služba simulu základní kámen celého systému, značně ovlivňuje výkon a škálovatelnost celého systému. Očekává se velké množství požadavků na tuto službu.

Za účelem zjednodušení implementace není využito čtení dat ze skutečného zdroje, ale jsou generována náhodná data. Načež data jsou následně poskytována se simulovaným zdržením, časově založenému na měření skutečného zdržení systému při čtení dat ze vzdáleného zdroje u obdobného systému. Tato služba je implementována jako REST API (TODO: pokud konečná implementace bude gRPC, změň tuto sekci), které poskytuje data ve formátu JSON. (TODO: Obrázek návrhu architektury a rozhraní služby).

\n{4}{FUS - File Upload Service}

Služba v systému hraje roli zapisovacího zařízení, které zapisuje data do zdroje. Tato služba hraje roli méně vytíženého služby, která nemá značný vliv na fungování systému jako celku. Požadavky, jenž musí vyřídit nejsou kritické a nutné řešit s minimální odezvou.

Služba je implementována s REST API rozhraním. (TODO: Obrázek návrhu architektury a rozhraní služby).

\n{2}{Konfigurace aplikace}

\n{3}{Konfigurace služeb}

\n{4}{Grafana}

\n{4}{Prometheus}

\n{4}{Loki}

\n{4}{Tempo}

\n{4}{OpenTelemetry}

\n{4}{Postgres}

\n{4}{K6}

\n{4}{SRS - Signal Readings Service}

\n{4}{FUS - File Upload Service}

\n{4}{TWS - Test Workers Service}

\n{3}{Konfigurace monitorovacích nástrojů}

\n{3}{Nastavení uživatelského rozhraní}

\n{1}{Testování scénářů}
Na této stránce je k vidění způsob tvorby různých úrovní nadpisů.

\n{2}{Požadavky na scénáře}

\n{2}{Popis scénářů}

\n{3}{Scénář 1 - TBS}

\n{3}{Scénář 2 - TBS}

\n{2}{Zpracování a vizualizace dat}

\n{3}{Monitorování v reálném čase}

\n{3}{Sběr historických dat}

% \n{2}{Obrázek}
% Obrázek \ref{fig:logo} prezentuje logo Fakulty aplikované informatiky.

% % Obrázek lze vkládat pomocí následujícího zjednodušeného stylu, nebo klasickým LaTex způsobem
% % Pozor! Obrázek nesmí obsahovat alfa kanál (průhlednost). Jde to totiž proti požadovanému standardu PDF/A.
% \obr{Popisek obrázku}{fig:logo}{0.5}{graphics/logo/fai_logo_cz.png}


% \n{2}{Tabulka}
% Tabulka \ref{tab:priklad} obsahuje dva řádky a celkem 7 sloupců.

% % Tabulku lze vkládat pomocí následujícího zjednodušeného stylu, nebo klasickým LaTex způsobem
% \tab{Popisek tabulky}{tab:priklad}{0.65}{|l|c|c|c|c|c|r|}{
%   \hline
%    & 1 & 2 & 3 & 4 & 5 & Cena [Kč] \\ \hline
%   \emph{F} & (jedna) & (dva) & (tři) & (čtyři) & (pět) & 300 \\ \hline
% }


% \n{2}{Citování}
% Následuje ukázka odkazování na různé zdroje:
% \begin{itemize}
% 	\item kniha \cite{HRW1997},
% 	\item kapitola v knize \cite{Delorme2006},
% 	\item článek v odborném žurnálu \cite{Bourreau2006},
% 	\item konferenční příspěvek \cite{Judish1999},
% 	\item doktorská práce \cite{Valente2005},
% 	\item technická zpráva \cite{Fralick1997},
% 	\item webová stránka \cite{WWWCST}.
% \end{itemize}