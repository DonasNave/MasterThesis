\nn{Úvod}
Programovací jazyky jsou základním kamenem softwarovévo vývoje respektive celého moderního světa v období informací. Představují způsob, kterým vývojář komunikuje s virtuálním prostředím OS a následně HW rozhraním. Vývoj výkonu HW, znalostí a zkušeností vývojářů a požadavků na vyvíjené systémy byl hnacím strojem technologického rozvoje. Postupným vývojem přicházeli další a další variace programovacích jazyků, některé rozdílné inkrementálně, jiné zcela diametrálně. Významným mezníkem v přístupu k tvorbě a běhu strojového kódu je vznik virtuálních strojů, které umožňují běh kódu nezávisle na HW. Tento přístup umožňuje vývojářům psát kód v jazyce, který je jim přirozený a následně jej spouštět na různých platformách.

Dotnet je platforma od společnosti Microsoft, která umožňuje vytvářet kód určený pro následnou kompilaci za běhu (Just-in-Time, dále JIT) a spuštění pomocí tzv. běhového prostředí (Common Language Runtime, dále CLR), jenž operuje jako virtuální stroj. Jedná se o relativně vyvinutou a zkušenou platformu s využitím v mnoha projektech a firmách. Přesto právě na této platformě byla dodatečně vyvinuta možnost pro PC platformy kompilace do nativního kódu (Ahead-of-Time, dále AOT), který je spouštěn přímo na OS a konkrétní architektuře HW. Tato funkce přichází do období rozmachu vývoje a migrace nativních cloudových řešení. Ty charakterizuje snaha dodávat pouze nezbytnou část infrastruktury a zpoplatnit reálnou dobu běhu systému s režií. Právě v prostředí cloudu mají nastávat situace, kdy bude využití služeb zkompilovaných do nativního kódu výhodnější. V kterých případech však opravdu takto napsaný program exceluje či selhává? A lze kvantifikacovat rozdíly mezi JIT kopilací pro běhové prostředí a nativní AOT kompilací?

Tato práce se zabývá porovnáním vývojového procesu, charakteristik a výkonu JIT a AOT kompilovaných služeb na platformě .NET. Cílem je zjistit, zda a v jakých případech je možné využít AOT kompilace pro zvýšení výkonu a zlepšení chování aplikací. Výsledkem práce je kvantifikace, respektive srovnání výkonu a chování JIT a AOT kompilace na platformě Dotnet. Na základě těchto výsledků je možné posoudit a doporučit vhodné případy pro využití AOT kompilace.