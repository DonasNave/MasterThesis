%%%%%%%%%%%%%%%%%%%%%%%%%%%%%%%%%%%%%%%%%%%%%%%%%%%%%%%%%%%%%%%%%%%%%%%%%%%%%%%%%
%                             Tvorba tech stacku                                %
%%%%%%%%%%%%%%%%%%%%%%%%%%%%%%%%%%%%%%%%%%%%%%%%%%%%%%%%%%%%%%%%%%%%%%%%%%%%%%%%%

\n{1}{Tvorba aplikačního stacku}

Za účelem důkladného testování výkonu a škálovatelnosti mikroslužeb byl vytvořen tech stack, který zahrnuje technologie pro kontejnerizaci, orchestraci, persistenci, komunikaci, monitorování a testování. 

\n{2}{Požadavky na SW}

Na aplikaci jsou pro splnění účelu analýzy vývoje, výstupu a výkonu služeb kladeny přímé i nepřímé požadavky. Je klíčové navrhnout řešení, vybrat technologie a provést implementaci a konfiguraci s ohledem na tyto požadavky, jenž rozděleny na funkční a nefunkční.

\n{3}{Funkční požadavky}

Funkční požadavky definují chování, funkce a vlastnosti, které musí systém poskytovat. Přímo souvisejí s doménovými požadavky a zahrnují specifikace, jako je zpracování dat, provádění výpočtů nebo podpora konkrétních procesů. Funkční požadavky v podstatě popisují očekávané operace systému, včetně vstupů, chování a výstupů, a jsou tak klíčové pro vývoj a testování.

\begin{itemize}
  \item \textbf{Mikroslužby} - Každá aplikace musí poskytovat rozhraní REST API s healtcheck enpointem pro informování celého systému o svém stavu. Dalším požadavkem je obecná komunikace mezi službami pomocí vybraných protokolů. Aplikace musí být schopna sbírat a vizualizovat data o výkonu a škálovatelnosti mikroslužeb. To zahrnuje sběr a vizualizaci metrik, logů a traces.
  \item \textbf{Stack} - Aplikační stack jakožto celek musí zahrnovat komunikaci pomocí protokolů HTTP/2 a gRPC. Je nutné aby implementoval publish - subscribe pattern pro komunikaci mezi vybranými službami. Stack musí zprostředkovat přístup a ukládání dat do relační a timeseries databáze. Musí poskytovat nutné rozhraní pro sběr, uchování a vizualizaci metrik a testovacích dat. Stack musí být schopen konfigurovat testovací scénáře, které se mají provést a také je spouštět v manuálním a automatizovaném režimu.
  \item \textbf{Sběr a vizualizace dat} - Aplikace musí být schopna sbírat a vizualizovat data o výkonu a škálovatelnosti mikroslužeb. To zahrnuje sběr a vizualizaci metrik, logů a traces. Data musí být dostupná v reálném čase a musí být možné je analyzovat a porovnávat. Zároveň je klíčové aby vizualizace byla konfigurovatelná a bylo umožněno vytvářet vlastní dashboardy pomocí komponent.
  \item \textbf{Testování scénářů} - Aplikace musí být schopna provádět testování scénářů, které simulují fungování systému a zátěž na mikroslužby. Testovací scénáře musí být konfigurovatelné a spustitelné v manuálním a automatizovaném režimu. 
  \item \textbf{Konfigurace aplikace} - V rámci aplikace musí být možnost konfigurovat chování nasazených služeb. To se týká jak konfigurace komunikace mezi službami, tak i konfigurace monitorovacích nástrojů. Konfigurace musí být uložena v konfiguračních souborech ve standardním formátu dle konvencí dané služby.
\end{itemize}

\n{3}{Nefunkční požadavky}

Nefunkční požadavky specifikují celkové vlastnosti systému. Definují atributy kvality, které musí systém splňovat. Nefunkční požadavky mohou zahrnovat omezení týkající se návrhu a implementace systému, jako jsou bezpečnostní standardy, soulad s právními a regulačními směrnicemi, doba odezvy při zpracování dat, kapacita pro souběžné uživatele, integrita dat a mechanismy převzetí služeb při selhání. Mají zásadní význam pro zajištění životaschopnosti a efektivity systému v jeho provozním prostředí a často ovlivňují celkovou uživatelskou zkušenost, výkonnost systému a splnění regulačních podmínek.

\begin{itemize}
  \item \textbf{Použitelnost} - Aplikace musí být snadno použitelná a přístupná pro uživatele. To zahrnuje snadnou konfiguraci a spuštění testovacích scénářů.
  \item \textbf{Udržitelnost} - Aplikace musí být udržitelná a snadno rozšiřitelná. To zahrnuje schopnost přidávat nové služby a rozšiřovat stávající služby.
  \item \textbf{Výkon} - Implementace aplikace, respektive jejich služeb, musí být schopna zvládnout zátěž, která je na ně kladena. To zahrnuje schopnost zvládnout požadavky na výkon a škálovatelnost.
\end{itemize}

\n{2}{Požadavky na HW}

Hardware, na kterém bude aplikace provozována, musí výkonnostně dostačovat pro provozování testovacích scénářů a sběr a vizualizaci dat. Týká se to primárně počtu jader, velikosti paměti a rychlosti diskového I/O. Provozované služby mají určitou základní režii, která se musí brát v potaz.

\n{2}{Cíle monitorování}

Efektivní monitorování hraje klíčovou roli při vyhodnocení a porovnání výkonnosti a technických vlastností kompilace JIT a nativní AOT. Hlavní cíle tohoto monitorování jsou následující:

\begin{itemize}
  \item \textbf{Zkušenosti s vývojem} - Jedním z klíčových cílů je zachytit a analyzovat dopad různých kompilačních strategií na proces vývoje. To zahrnuje sledování doby sestavení, cyklů nasazení a celkové snadnosti integrace a nasazení v rámci architektury mikroslužeb. Posouzením těchto faktorů můžeme poskytnout subjektivní i objektivní náhled na to, jak jednotlivé metody kompilace ovlivňují každodenní zkušenosti vývojářů, včetně potenciálních problémů nebo efektivity, které přinášejí přístupy JIT nebo AOT.
  \item \textbf{Srovnání výstupů} - Tento cíl se zaměřuje na přímé porovnání hmatatelných výstupů metod kompilace JIT a AOT. Konkrétně se bude sledovat velikost vytvořených spustitelných souborů, inicializační časy (jak rychle jsou služby po nasazení funkční) a využití zdrojů během běhu (využití procesoru a paměti). Pochopení těchto aspektů pomůže vymezit provozní efektivitu nebo režijní náklady spojené s každou kompilační strategií.
  \item \textbf{Výkonnostní metriky} - Pro tuto studii je rozhodující porovnání výkonnostních ukazatelů za podobných provozních podmínek. Mezi sledované metriky patří doba odezvy, propustnost (počet požadavků, které je služba schopna zpracovat za jednotku času), chybovost a stabilita systému při zatížení. Tyto ukazatele poskytnou kvantitativní základ pro porovnání účinnosti kompilací JIT a AOT při zvládání reálné provozní zátěže.
\end{itemize}

K dosažení těchto cílů bude zavedena komplexní monitorovací sestava zahrnující nástroje a postupy, které poskytují údaje a poznatky v reálném čase. Tento přístup zajistí, že shromážděná data budou robustní, spolehlivá a vhodná pro provedení důkladné srovnávací analýzy, a podpoří tak informovaná rozhodnutí týkající se optimálního využití kompilace JIT a AOT při nasazení mikroslužeb.

\n{2}{Výběr technlogií}

Součástí tvorby tech stacku je výběr technologií, které budou použity pro implementaci aplikace. Výběr technologií je závislý na požadavcích na aplikaci a HW, na kterém bude aplikace provozována.

\n{3}{Organizace a správa zdrojů}

Pro správu souborů práce byl zvolen SCM Git. Git je open-source verzovací, který umožňuje vytvářet, spravovat a sdílet soubory. Git je schopný pracovat s větvemi, které umožňují vytvářet paralelní vývojové větve.

Za účelem jednoduché organizace souborů bylo zvoleno řešení monorepozitáře. Monorepozitář je repozitář, který obsahuje veškeré soubory projektu, ale také relevantní dokumentaci, obrázky, podpůrné nástroj a zdrojové soubory diplomové práce. Následující struktura adresářů byla zvolena pro organizaci souborů.

\begin{itemize}
    \item \textbf{Documentation} - adresář obsahující dokumentaci aplikace.
    \item \textbf{Source} - adresář obsahující zdrojové soubory aplikace.
    \item \textbf{Thesis} - adresář obsahující zdrojové soubory textu diplomové práce a práci samotnou ve formátu pdf.
\end{itemize}

Pro zaručení dostupnosti a sdílení veškerých prostředků souvisejících s prací byl vybrán GitHub, jakožto server pro hostování repozitáře. GitHub je open-source platforma pro verzování souborů a projektů. Navíc poskytuje rozšířující možnosti jako je CI/CD, správa dokumentace a další. Repozitář projektu je veden jako veřejný s licencí MIT.

\n{3}{Kontejnerizace a orchestrace}

Základním prvkem nasazení aplikace je kontejnerizace a orchestrace. Kontejnerizace zajišťuje, že aplikace bude spouštěna v izolovaném prostředí, které je nezávislé na hostitelském systému. Orchestrace zajišťuje, že aplikace bude spouštěna na dostupných zdrojích a bude schopna zvládnout zátěž, která je na ni kladena.

Pro kontejnerizaci byla zvolena technologie Docker. Docker je open-source platforma pro kontejnerizaci aplikací, která umožňuje vytvářet, spouštět a spravovat kontejnery.

Pro orchestraci byla vybrána technologie Kubernetes. Kubernetes je open-source platforma pro orchestraci kontejnerů, která umožňuje automatizovat nasazování, škálování a správu aplikací. Kubernetes je schopný pracovat s kontejnery, které jsou vytvořeny pomocí Dockeru.

\n{3}{Konfigurace nasazení}

Pro konfiguraci nasazení byla zvolena technologie Helm. Helm je open-source platforma pro správu balíčků, která umožňuje vytvářet, spravovat a nasazovat balíčky. Helm je schopný pracovat s balíčky, které jsou vytvořeny pomocí Kubernetes.

Definice balíčků je řešena pomocí konfiguračních souborů, které jsou použity již při tvorbě obecného obrazu. V rámci Helm je základním prvkem chart, který obsahuje definici balíčku a šablonu, která je použita pro generování konfigurace.

\n{3}{Persistenční vrstva}

Jako relační databáze byla vybrána Postgres. Je schopná pracovat s relačními daty, které jsou uloženy v tabulkách. Poskytuje základní klientský balíček pro .NET, jenž umožňuje komunikaci s databází. Tento balíček je kompatibilní s nativní AOT kompilací.

Za účelem persistence a zprostředkování dat z testování je použita InfluxDB. Tato timeseries databáze umožňuje ukládat a spravovat časové řady. Využití InfluxDb je pragramtické z důvodu nativní podpory napojení InfluxDB v1 z nástroje K6 pro export testovacích dat.

\n{3}{Komunikační metody}

Za účelem analýzy možností komunikace klienta se službami, ale i interní komunikace, bylo vybráno k implementaci hned několik protokolů. 

\begin{itemize}
  \item \textbf{REST API} - V rámci Kestrel serveru každé služby je využit protokol HTTP/1 a komunikace pomocí REST API. Toto rozhraní slouží pro komunikaci klienta se službou a poskytuje data ve formátu JSON.
  \item \textbf{gRPC} - Vybrané služby implementují komunikaci pomocí protokolu HTTP/2 a gRPC. Za tímto účelem mají zmíněné služby otevřené rozhraní na dodatečném portu. gRPC protokol je využit přístupem model first, tedy rozhraní je definováno pomocí protobuf souboru a následně je vygenerován kód pro komunikaci.
  \item \textbf{RabbitMQ} - Pro implementaci komunikace prodle vzoru Publish - Subscribe byl vybrán message broker RabbitMQ. Umožňuje službám odebírat data z jiných služeb a zároveň poskytovat data jiným službám. Tím je zajištěna asynchronní messaging mezi službami.
\end{itemize}

\n{3}{Monitorovací nástroje}

Pro monitorování aplikace byl zvolen Grafana observability stack pro jeho pokrytí komplexní škály monitorovacích dat. Grafana observability stack zahrnuje nástroje pro sběr, vizualizaci a analýzu dat. Následující nástroje byly vybrány pro implementaci monitorovacího stacku:

\begin{itemize}
  \item \textbf{Grafana} - Grafana je open source webová aplikace pro analýzu a interaktivní vizualizaci dat. Poskytuje možnost sestavit dashboard z komponent jako jsou grafy, tabulky a další. Jedná se o velmi populární technologii v doménách serverové infrastruktury a monitorování. Grafana umožňuje sjednotit monitorovací služby a zobrazit data v reálném čase. Podporuje širokou škálu datových zdrojů, jako jsou Prometheus, InfluxDB, Tempo, Loki nebo Elasticsearch, což umožňuje jednoduchou konfiguraci a připojení cílových dat. Kombinací dat z různých zdrojů umožňuje vytvářet komplexní pohled na celý systém. To je obzlvášť cenné pro analýzu systému pomocí kombinací metrických dat.
  \item \textbf{Prometheus} - Open-source monitorovací systém. Shromažďuje a ukládá metriky jako time-series data a umožňuje se na ně dotazovat pomocí vlastního výkonného jazyka PromQL. Prometheus je zvláště vhodný pro monitorování microservice architektur díky své schopnosti automaticky objevovat cíle. Jeho architektura podporuje více modelů získávání dat, stahování metrik z cílových služeb nebo collectorů, odesílání metrik přes gateway a zprostředkování notifikací.
  \item \textbf{Loki} - Škálovatelný agregátor logů. Na rozdíl od obdobných systémů pro agregaci logů, jenž indexují všechna data, Loki indexuje pouze metadata, přičemž ukládá celá data logu efektivním způsobem. Loki je navržen tak, aby jednoduše spolupracoval s Grafanou a umožňuje rychle vyhledávat a vizualizovat logy.
  \item \textbf{Tempo} - Je snadno ovladatelný open-source backend pro distribuované sledování požadavků. Tempo podporuje ukládání a načítání traces, které jsou přijímány ze zdrojů jako Jaeger, Zipkin a OpenTelemetry. Na rozdíl od mnoha jiných systémů pro traces nevyžaduje Tempo žadné předem definované schéma. Je navržen tak, aby se bezproblémově integroval s Prometheus a Loki.
  \item \textbf{OpenTelemetry} - Open source kolektor telemetrických dat. Poskytuje jednotný, vendor-agnostic způsob sběru, zpracování a exportu telemetrických dat  Je konfigurovatelný a podporuje více pipeline, které mohou upravovat telemetrická data při jejich průchodu. Výrazně zjednodušuje instrumentaci služeb, protože umožňuje agregovat a exportovat metriky, taces a logy do různých analytických a monitorovacích nástrojů. Poskytuje podporu pro export dat do Prometheus, Tempo i Loki.
  \item \textbf{Cadvisor} - Open-source monitorovací nástroj pro kontejnery. Poskytuje informace o využití zdrojů a výkonu. Cadvisor je schopen monitorovat kontejnery běžící na Dockeru, Kubernetes nebo jiných kontejnerových platformách. Poskytuje informace o využití procesoru, paměti, sítě a diskového I/O.
  \item \textbf{NodeExporter} - Nástroj pro sběr metrik z hostitelského systému. Poskytuje informace o využití procesoru, paměti, sítě a diskového I/O.
\end{itemize}

\n{3}{Testovací nástroje}

Za účelem testování monitorovacího stacku byl vybrán nástroj K6. Jedná se o moderní open-source nástroj pro testování zátěže. Slouží k vytváření, provádění a analýze výkonnostních testů softwarových aplikací. Nabízí čisté skriptovací rozhraní s jazykem JavaScript, které umožňuje psát, jak komplexní testovací scénáře napodobující reálný provoz systému, tak i vytvářet nereálné nebo hraniční situace. K6 podporuje různé systémové metriky, jako je doba odezvy, propustnost a chybovost. Nabízí šiřoké možnosti rozšíření skrze API, což umožňuje přizpůsobení a integraci s dalšími nástroji pro komplexní sledování výkonu.

\n{3}{Testovací služby}

Pro implementaci testovacích služeb z podstaty práce zvolena technologie .NET, konkrétně jazyk C\#. Služby budou implementovány jako mikroslužby a budou podporovat kontejnerizované nasazení v microservice architektuře. Služby budou vytvořeny tak, že každou dílčí službu reprezentuje projektový soubor s doménových kódem. Celé řešení spolu s dílčími knihovnami bude součástí jednoho solution souboru.

Pro řešení byla vybrána nejnovější verze .NET SDK 8.0, která poskytuje nejrozsáhlejší impelmentaci a podporu pro nativní AOT kompilaci. Jakožto nástroj pro vývoj a správu projektů byl zvolen JetBrains Rider. Rider je IDE, které poskytuje širokou škálu funkcí pro vývoj aplikací v .NET.

Konkrétní knihovny použité v rámci implementace budou záviset na konkrétních požadavcích na služby a popsány v následující sekci.

\n{2}{Návrh a implementace testovacích služeb}

Následující pasáž se zabírá návrhem a implementací testovacích služeb, které budou využity pro analýzu vývoje a výkonu jednotlivých kompilací AOT a JIT v rámci .NET.

\n{3}{Architektura}

Pro implementaci požadované funkcionality bylo zvoleno následující rozdělení zodpovědnosti služeb:

\begin{itemize}
    \item \textbf{SRS - Signal reading service} - Služba simuluje roli čtecího zařízení. Generuje data a poskytuje je ostatním službám. Poskytuje REST API rozhraní.
    \item \textbf{FUS - File Upload Service} - Zprostředkovává datové persistentní zapisovací zařízení. Čte nebo zapisuje data do PostgreSQL databáze. Poskytuje REST API a gRPC rozhraní.
    \item \textbf{BPS - Batch Processing Service} - služba, která zpracovává data z jiných služeb. Reaguje na požadavek o hromadném zpracování při předem definovaném splnění podmínek. Poskytuje REST API a gRPC rozhraní. Je přihlášena do RabbitMQ jako subscriber.
    \item \textbf{EPS - Event Publishing Service} - slouží k vyvolání události, která je následně zpracována jinými službami. Poskytuje REST API rozhraní. Je přihlášena do RabbitMQ jako publisher.
\end{itemize}

obrázek architektury

Kompilaci do nativního AOT kódu je deklarována použitím atributu PublishAoT v projektovém souboru. Za účelem zajištění co největší podobnosti služeb zacílených na AOT a JIT kompilaci, bude využito zadefinování konstantních hodnot v rámci projektu. Konstanty \emph{JIT} a \emph{AOT} budou využity pro rozlišení chování služeb v rámci obou kompilačních verzí. S použitím direktiv kompilátoru a zmíněných konstant bude v nutných případech docíleno rozdílného volání API při snaze zachovat totožnou funkcionalitu.

\n{3}{Očekávání vývojového procesu}

Na základě podporované funkcionality, tak jak je definována týmem .NET a popsána v rámci rešerše, je očekáváno, že vývojový proces bude probíhat bez výrazných problémů a bude možné vytvořit služby, které budou schopny zvládnout definované funkční a nefunkční požadavky. Podpora 3. stran byla předem prozkoumána v rámci dostupných dokumentací vybraných knihoven .NET. Konkrétní podoba a rozsah této podpory bude plně ověřitelná až v rámci implementace a testování služeb.

foto využití konstant v kódu

\n{3}{Organizace zdrojových souborů služeb}

Organizace zdrojových souborů služeb, knihoven a pomocných souborů je řešena v rámci hlavního adresáře obsahujícího .NET solution soubor, pomocné soubory a solution složky s konkrétními projekty služeb a knihoven. Následující
stromový graf představuje adresářovou strukturu projektu.
    
\begin{forest}
    for tree={
      font=\ttfamily,
      grow'=0,
      child anchor=west,
      parent anchor=south,
      anchor=west,
      calign=first,
      edge path={
        \noexpand\path [draw, \forestoption{edge}]
        (!u.south west) +(7.5pt,0) |- node[fill,inner sep=1.25pt] {} (.child anchor)\forestoption{edge label};
      },
      before typesetting nodes={
        if n=1
          {insert before={[,phantom]}}
          {}
      },
      fit=band,
      before computing xy={l=15pt},
    }
  [DTA
    [Common
      [DTA.Abstractions]
      [DTA.Extensions.Common]
      [DTA.Extensions.Postgre]
      [DTA.Extensions.Swagger]
      [DTA.Extensions.Telemetry]
      [DTA.Migrator]
      [DTA.Models]
    ]
    [Services
      [DTA.BPS]
      [DTA.EPS]
      [DTA.FUS]
      [DTA.SRS]
    ]
    [SolutionItems
      [Directory.Build.props]
      [NuGet.config]
    ]
    [.gitignore]
    [DTA.sln]
  ]
\end{forest}


\n{3}{Společná struktura služeb}

Každá z vyvinutých služeb využívá konkrétní .NET SDK \emph{Microsoft.NET.Sdk.Web}, které umožňuje využít WebApplication pro registraci a konfiguraci funkcionality služby a zároveň poskytuje konfigurovatelný Kestrel server. Pro zajištění jednotného přístupu k logování, metrikám a konfiguraci byly vytvořeny společné knihovny, které jsou využity ve všech službách.

\begin{forest}
    for tree={
      font=\ttfamily,
      grow'=0,
      child anchor=west,
      parent anchor=south,
      anchor=west,
      calign=first,
      edge path={
        \noexpand\path [draw, \forestoption{edge}]
        (!u.south west) +(7.5pt,0) |- node[fill,inner sep=1.25pt] {} (.child anchor)\forestoption{edge label};
      },
      before typesetting nodes={
        if n=1
          {insert before={[,phantom]}}
          {}
      },
      fit=band,
      before computing xy={l=15pt},
    }
  [DTA.SampleService
    [Api
      [Rest]
      [Grpc]
      [Rabbit]
    ] 
    [Extensions
      [ProgramExtensions.cs]
    ]
    [Monitoring
      [AppMonitor.cs]
    ]
    [Properties
      [launchSettings.json]
    ]
    [Services
      [Interfaces
        [ISampleService.cs]
      ]
      [SampleService.cs]
    ]
    [apssettings.json]
    [Dockerfile-AOT]
    [Dockerfile-JIT]
    [DTA.SampleService.csproj]
    [Program.cs]
  ]
  \end{forest}


  \begin{itemize}
    \item \textbf{Api} - obsahuje implementaci rozhraní služby
    \item \textbf{Extensions} - implementuje extension metody specifické pro doménu služby
    \item \textbf{Monitoring} - obsahuje statickou třídu, která drží reference na počítadla metrik
    \item \textbf{Service} - ve složce jsou implementovány služby, které provádějí doménovou logiku služby
    \item \textbf{Properties} - drží konfiguraci pro spuštění služby
    \item \textbf{Program.cs} - vstupní bod služby
    \item \textbf{appsettings.json} - konfigurace služby
    \item \textbf{Dockerfile-AOT} - soubor pro tvorbu Docker obrazu pro AOT kompilaci
    \item \textbf{Dockerfile-JIT} - soubor pro tvorbu Docker obrazu pro JIT kompilaci
\end{itemize}

Specifické služby obsahují dodatečné adresáře a soubory nutné pro implementaci jejich doménové funkce.

\n{3}{Knihovny 3. stran}

Pro implementaci funkcionality aplikace byly využity následující knihovny třetích stran:

\begin{itemize}
  \item \textbf{Npgsql} - Npgsql je open-source ADO.NET provider pro PostgreSQL, který umožňuje komunikaci s PostgreSQL databází. Npgsql poskytuje základní balíček funkcí pro vytvoření připojení na základě standardizovaného řetězce pro připojení. Tento balíček sice není plně kompatibilní s AOT kompilací, funkce které jsou využity v rámci aplikace jsou avšak kompatibilní.
  \item \textbf{Dapper} - ORM knihovna pro .NET, která umožňuje mapovat databázové struktury na C\# objekty a vytvářet a provádět dotazy na databázi. \emph{Dapper.AOT} je dílčí knihovna, která umožňuje vytvářet a provádět dotazy na databázi v rámci AOT kompilace. Toho je zajištěno tím, že Dapper.AOT generuje kód pro dotazy na databázi v době kompilace. Využívá k tomu interceptorů a generátorů. Samotný balíček Dapper.AOT obsahuje další knihovnu - \emph{Dapper.Advisor}, která pomáhá s analýzou zdrojového kódu a generováním kódu pro dotazy na databázi.
  \item \textbf{OpenTelemetry} - OpenTelemetry zprostředkovává množinu knihoven pro sběr, zpracování a export telemetrických dat. V rámci knihovny je umožněno registrace vlastních metrik, logů a traces, ale také nastavení exportu vybraných systémových dat sbíraných v rámci knihoven .NET.
  \item \textbf{Grpc} - Knihovny pro implementaci komunikace pomocí protokolu HTTP/2 a gRPC. Konkrétně jsou využity \emph{Grpc.AspNetCore} v případě serveru, \emph{Grpc.Net.Client} pro klienta a \emph{Google.Protobuf} s \emph{Grpc.Tools} pro generování modelů v přístupu model first.
  \item \textbf{RabbitMQ} - Komunikace a implementace publish subscribe vzoru je umožněna knihovnou \emph{RabbitMQ.Client}. S její pomocí jsou vytvářeny fronty, dochází k přihlášení k odběru zpráv a jejich publikování.
  \item \textbf{Swagger} - Grafické rozhraní pro vizualizaci a testování REST API služeb. Swagger je využit pouze v kombinaci konfigurací \emph{JIT Debug}. K tomuto účelou jsou využity knihovny \emph{Swashbuckle.AspNetCore} a \emph{Microsoft.AspNetCore.OpenApi}.
\end{itemize}

\n{3}{Společné knihovny}

V rámci zjednodušení tvorby služeb, jednotné implementaci a konfiguraci, ale také z důvodu zajištění některé základní ale klíčové funkcionality, byly vytvořeny společné knihovny. Tyto knihovny obsahují společné třídy, rozhraní a konfigurace, které jsou použity ve všech službách.

\begin{itemize}
  \item \textbf{Persistence} - Pro implementaci persistence byla vytvořena pomocná knihovna \\ \emph{DTA.Extensions.Postgres}, která poskytuje pomocnou funkcionalitu pro zajištění existence databáze pro službu, dle konfigurace v řetězci pro připojení.
  \item \textbf{Migrace} - Zajištění migrace databáze bylo implementováno po vlastní ose minimalistickým migrátorem v knihovně \emph{DTA.Migrator}. Tato knihovna poskytuje základní funkcionalitu pro vytvoření databáze, vytvoření tabulek a indexů, ale také zajištění migrace dat a verzování změn.
  \item \textbf{Telemetrie} - Knihovna \emph{DTA.Extensions.Telemetry} zprostředkovává extensions metody pro jednotnou a jednoduchou registraci sběru a export telemetrických dat napříč službami.
  \item \textbf{Modely} - Knihovna \emph{DTA.Models} obsahuje společné modely, které jsou využity ve službách. Je tím docílena viditelnost na datové struktury rozhraní aplikace napříč všemi službami, jež knihovnu referencují.
  \item \textbf{Obecná funkcionalita} - Za účelem sjednocení funkcionality využité napříč všemi službami jsou implementovány extension metody v knihovně \emph{DTA.Extensions.Common}. Zde je poskytnuta funkcionalita pro sestavení názvů pro službu.
\end{itemize}

\n{3}{Společná konfigurace}

Součástí řešení je společná konfigurace, která je využita ve všech službách. Ta je řešena jedna na úrovni solution souboru, tak i Directory.Build.props souboru. Týká se jednotné distribuce projektových atributů pro verzi, kompatibilitu s AOT, vynucení konkrétních pravidel pro kód a analyzéry.

\n{3}{SRS - Signal reading service}

Za účelem simulace funkce čtecího zařízení byla vytvořena služba SRS. Tato služba poskytuje základní rozhraní pro získání dat signálu včetně jednotek a značek formou REST API. Pro zjednodušení implementace není využito čtení dat ze skutečného zdroje, ale jsou generována náhodná data. Načež data jsou následně poskytována se zdržením simulujícím čtení dat ze vzdáleného zdroje.

Služba poskytuje následující rozhraní

\begin{itemize}
    \item \textbf{GET /api/signals/\{int:amount\}} - Vygeneruje zadané množství náhodných signálů
\end{itemize}

\n{3}{FUS - File Upload Service}

Služba v systému hraje roli zapisovacího zařízení, které zapisuje a čte data z perzistentního úložiště. Jakožto úložiště je využito PostgreSQL databáze. Služba využívá vlastní databázovou instanci, spravuje vlastní tabulky pomocí migrací.

Poskytuje rozhraní formou REST API pro zápis a čtení dat. Daty je myšlen libovolný soubor v libovolném formátu. Samotná podstata nahraných dat není pro službu důležitá, ale je zpracována a uložena do databáze. Za účelem sehrání testovacích scénářů poskytuje SRS také gRPC rozhraní, které je zajištěno na dedikovaném portu. V rámci gRPC komunikace slouží služba jako server, který splňuje volání vzdálené procudery.

Služba poskytuje následující rozhraní

\begin{itemize}
    \item \textbf{GET /api/file/download/\{int:id\}} - Stáhne soubor podle zadaného ID.
    \item \textbf{POST /api/file/upload} - Nahraje soubor do systému.
    \item \textbf{gRPC Operation FileServer.GetFile} - Stáhne soubor podle zadaného objektu s ID.
\end{itemize}

\n{3}{BPS - Business Processing Service}

Pro splnění role a požadavků na zpracování dat z jiných služeb byla vytvořena služba BPS. Tato služba získává data, provádí náročné výpočetní operace, sloužící k simulaci obtížných doménových operací. Konkrétně implementaváno je neefiktivní rekurzivní výpočet Fibonacciho posloupnosti a faktoriálu.

Služba se po spuštění přihlašuje k odběru zpráv na předem definovaný kanál \emph{simulated} na službe \emph{RabbitMQ}. Po získání zprávy získává data ze služby FUS pomocí volání vzdálené procedury. Po získání dat provádí náročné výpočetní operace, které jsou simulovány náhodným čekáním.

Služba poskytuje následující rozhraní

\begin{itemize}
    \item \textbf{GET /api/processFibonacci/\{int:degree\}} - Vypočítá číslo z Fibonacciho posloupnosti na zadané pozici náročným rekurzivním způsobem.
    \item \textbf{Event subscribed: \{queue-name\}\_simulated} - Přihlášení k odběru zpráv v rámci kanálu na službě RabbitMQ.
\end{itemize}

\n{3}{EPS - Event Publishing Service}

Jednoduchá službami umožňující vyvolat událost v systému a docílit spuštění dodatečných operací v systému. V systému simuluje roli vydavatele událostí Poskytuje REST API rozhraní pro vyvolání události. Po vyvolání události je zpráva publikována do RabbitMQ kanálu, kde je zpracována jinými službami.

Služba poskytuje následující rozhraní

\begin{itemize}
    \item \textbf{GET api/simulateEvent/\{int:id\}} - Vyvolá simulovanou událost s daným ID.
    \item \textbf{Event published: \{queue-name\}\_simulated} - Vyvolá údalost se zprávou obsahující identifikator na konfigurovaném kanálu do služby RabbitMQ.
\end{itemize}

\n{3}{Přehled řešení}

Řešení aplikace sestává z následujících sekcí a jednotlivých obrazů služeb a verzí, definovaných v rámci Docker Compose souboru.

\begin{itemize}
    \item \textbf{Testovací služby} - Aplikace obsahuje testovací služby FUS, SRS, BPS a EPS. Tyto služby jsou vytvořeny ve dvou kompilačních verzích - AOT a JIT. Každá služba je vytvořena jako obraz s názvem \emph{dta-\{service-name\}:\{compilatioinMode\}-latest}.
    \item \textbf{Komunikace} - Komunikační kanál mezi službami je zajištěn pomocí RabbitMQ. Pro RabbitMQ je využit obraz \emph{rabbitmq:3-management-alpine}.
    \item \textbf{Monitorovací nástroje} - Monitorování zajišťuje Grafana Observability stack implementovaný v rámci obrazu \emph{dta-lgtm:latest}, jenž obsahuje OpenTelemetry, Prometheus, Loki, Tempo a Grafanu. Pro měření výkonu hostitelského systému a export těchto dat jsou využity obrazy \emph{node-exporter:latest} a \emph{cadvisor-arm64:v0.49.1}.
    \item \textbf{Persistence} - Pro persistenci dat je využita PostgreSQL databáze. Pro PostgreSQL je využit obraz \emph{postgres:latest}. Ukládání metrik je zajištěno pomocí InfluxDB a obrazu \emph{influxdb:1.8.10}.
    \item \textbf{Směrování} - Funkci reverzní proxy zajišťuje Nginx ve verzi obrazu \emph{nginx:latest}.
\end{itemize}

Následující diagram znázorňuje přímé závislosti testovacích služeb na další nástroje.

\obr{Diagram .NET služeb a závislých služeb}{fig:logo}{0.9}{graphics/diagrams/services-architecture.png}

\n{2}{Konfigurace aplikace}

Za účelem běhu aplikace je klíčové správné nastavení konfigurace. Konfigurace je řešena na různých úrovních. Základní úroveň představuje soubor \emph{compose.yaml}, který inforumje Docker orchestrátor jaká infrastrukura má být vytvořena a jak má být nastavena, které služby mají být vytvořeny a jaké zdroje jim mají být přiděleny. Dále jsou v tomto souboru nastaveny základní parametry jako je jméno sítě, názvy volume (persistentní úložiště orchestrátoru), závislosti mezi službami a další. V neposlední řadě je dílčí konfigurace jednotlivých služeb řešena v rámci konfiguračních souborů a proměnných prostředí.

\n{3}{Konfigurace infrastruktury}

Jak již bylo zmíněno, základní konfigurace je řešena v souboru \emph{compose.yaml} a je rozdělena do následujících částí:

\begin{itemize}
  \item \textbf{volumes} - V této sekci jsou definovány všechny volume, které jsou využity v rámci stacku. Volume jsou definovány názvem a využity pro ukládání dat služeb. Konkrétně jsou využity úložiště pro data Grafany, OpenTelemetry, Prometheus a InfluxDB.
  \item \textbf{networks} - Konfigurace pro vnitřní síť aplikace, která je využita pro komunikaci mezi službami. Síť je definována názvem a typem. Pro potřeby aplikace se využívá síť s názvem \emph{stack-network} a typ \emph{bridge}, jenž funguje jako síťový most.
  \item \textbf{services} - Definuji sekci pro jednotlivé služby, které jsou součástí stacku. Každá služba je definována názvem služby - \emph{container\_name}, názvem obrazu - \emph{image}, v případě lokálně sestavených služeb také definicí sestavení - \emph{build}. Dále je definováno, jaké porty jsou mapovány z kontejneru do hostitelského systému - \emph{ports}, jaké volume jsou připojeny k kontejneru - \emph{volumes}, použité síťové rozhraní - \emph{networks}, závislosti služby - \emph{depends\_on} a proměnné prostředí - \emph{environment}. V případech testovaných služeb je uvedena dodatečná konfigurační sekce nasazení - \emph{deploy}, jenž limituje dostupné zdroje paměti a procesoru.
\end{itemize}

\n{3}{Konfigurace směrování}

Nastavení směrování v rámci stacku je řešeno konfigurací proxy služby Nginx. To je řešeno odděleně dodatečným konfiguračním souborem nginx.conf jenž je připojen do kontejneru služby. Soubor obsahuje následující směrovací pravidla:

\begin{itemize}
    \item / - cesta na statickou hlavní stránku-rozcestník aplikace
    \item /grafana - směrování na Grafanu
\end{itemize}

Rozcestník představuje soubor html.index a je připojen do virualizovaného repozitáře kontejneru obdobným způsobem jako konfigurační soubor. 

\n{3}{Konfigurace telemetrie}

Nastavení telemetrie spočívá v definici rozhraní, nastavení chování služeb a systémů z niž se telemetrická data sbírají, jejich cíl pro zpracování, správu a vizualizaci. Značnou část konfigurace představuje propojení nástrojů stacku LGTM, k čemuž slouží konfigurační soubory. Ty obsahují výchozí minimalistickou konfiguraci pro jednotlivé nástroje. Výstupem LGTM je individuální kontejner a zmíněná konfigurace je náturou interní a není potřeba s ní manipulovat s ohledem na požadavky práce uživateli. Mezi dodatečné konfigurace telemetrie napříč službami patří:

\begin{itemize}
    \item \textbf{Testované služby} - Veškeré testovací služby mají nastavený endpoint pro export telemetrických dat. Toto nastavení je zprostředkováno proměnnou prostředí \emph{OpenTelemetrySettings\_ExporterEndpoint}.
    \item \textbf{LGTM} - Konfigurace pro LGTM je řešena pomocí proměnných prostředí. Je definována adresa a cesta, z které je k dispozici Grafana. Dále je umožněno anonymní přihlášení do Grafany.
    \item \textbf{Cadvisor} - Nastavení služby představuje dodatečné nastavení volumes, které jsou připojeny k kontejneru. Připojením systémových souborů je zajištěno sběr dat o využití systémových zdrojů. Toto nastavení je závislé na operačním systému.
\end{itemize}

\n{3}{Konfigurace testovacích služeb}

Jednotlivé služby mají vlastní dodatečné konfigurace, které jsou řešeny pomocí kombinace dle standardizovaného postupu pro konfiguraci .NET aplikací, a to konfiguračního souboru \emph{appsettings.json} a proměnných prostředí. V první řadě řadě je použita konfigurace ze souboru, načež je přepsána odpovídajícími hodnotami proměnných prostředí. Každá služba má definovaný specifický prefix pro identifikaci proměnných prostředí. Následující seznam popisuje dodatečné konfigurace jednotlivých služeb.

\begin{itemize}
  \item \textbf{FUS - File Upload Service} - Obsahuje connection string (přístupový řetězec) pro připojení do databáze PostgreSQL.
  \item \textbf{BPS - Batch Processing Service} - Disponuje konfigurací pro připojení k RabbitMQ a konkrétní frontě.
  \item \textbf{EPS - Event Publishing Service} - Drží informaci o rozhraní RabbitMQ, konkrétní frontě a gRPC rozhraní služby FUS.
\end{itemize}

\n{3}{Konfigurace persistence}

Služby využité pro perzistentní ukládání dat jsou konfigurovány pomocí proměnných prostředí. Jedná se o následující služby:

\begin{itemize}
  \item \textbf{PostgreSQL} - Jsou definovány údaje pro uživatele databáze a název databáze.
  \item \textbf{InfluxDB} - Proměnné prostředí definují název databáze, uživatelské údaje a povolení přihlášení pomocí http.
\end{itemize}

\n{3}{Nastavení uživatelského rozhraní}

Definice uživatelského rozhraní, respektive dostupných dashboardů, je dána při sestavení obrazu LGTM. V rámci něj jsou předdefinovány hodnoty pro připojení zdrojů dat, tj. Prometheus, Loki, Tempo a InfluxDb. Patřičné dashboardy zobrazující relevantní data pro různé scénáře systému byly předem připraveny a jsou k dispozici po otevření Grafany anonymním uživatelem. Následující seznam popisuje klíčové soubory konfigurace uživatelského rozhraní.

\begin{itemize}
  \item \textbf{grafana-dashboards.json} - Definuje dostupné dashboardy v Grafaně. Dashboardy jsou definovány v JSON formátu a obsahují definici panelů, zdrojů dat a dalších parametrů.
  \item \textbf{grafana-datasources.json} - Obsahuje  zdroje dat z kterých Grafana, respektive dashboardy, čerpají data.
\end{itemize}
