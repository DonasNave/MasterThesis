%%%%%%%%%%%%%%%%%%%%%%%%%%%%%%%%%%%%%%%%%%%%%%%%%%%%%%%%%%%%%%%%%%%%%%%%%%%%%%%%%
%                   Testovací aplikace                              %
%%%%%%%%%%%%%%%%%%%%%%%%%%%%%%%%%%%%%%%%%%%%%%%%%%%%%%%%%%%%%%%%%%%%%%%%%%%%%%%%%

\n{1}{Testovací aplikace}

Praktická část této práce se zaměřuje na vytvoření testovací aplikace postavené na microservice architektuře. Cílem je vytvořit, otestovat a analyzovat služby využívající JIT a nativní AOT kompilace. Rozsah funkcionality a chování aplikace je definováno množinou funkčních a nefunkčních požadavků. Dále jsou vybrány konkrétní technologie a nástroje, jenž v aplikaci doplní .NET služby. Jejich účelem je zprostředkování platformy monitorování, dodání perzistence, hostování webového rozhraní a zprostředkování testování. Samotný postup testování je definován metodikou, která zahrnuje také definici hypotéz a je podrobně popsána v následující kapitole. Tyto hypotézy jsou následně ověřeny v analytické sekci práce. Ověření probíhá v rámci experimentů, které jsou provedeny na testovací aplikaci. Data z testů jsou zprostředkovány pro analýzu a vyhodnocení. Aplikace je nasazena v kontejnerizovaném prostředí a vytvořena s ohledem na rozšiřitelnost pro otestování konkrétní doménové problematiky.

\n{2}{Požadavky na SW}

Na aplikaci jsou pro splnění účelu analýzy vývoje, výstupu a výkonu služeb kladeny přímé i nepřímé požadavky. Je klíčové navrhnout řešení, vybrat technologie a provést implementaci včetně konfigurace s ohledem na tyto požadavky. Následující část této sekce je kategorizována do funkčních a nefunkčních požadavků na SW.

\n{3}{Funkční požadavky}

Funkční požadavky definují chování, funkce a vlastnosti, které musí aplikace poskytovat. Přímo souvisejí s doménovými požadavky a zahrnují specifikace, jako je zpracování dat, provádění výpočtů nebo podpora konkrétních procesů. Funkční požadavky popisují očekávané operace systému, včetně vstupů, chování a výstupů. Jsou tak klíčové pro vývoj a testování. V případě testovací aplikace jsou požadavky se zacílením na splnění testovacích scénářů a minimalistickou simulaci scénářů. Následující seznam funkčních požadavků byl definován pro testovací aplikaci.

\begin{itemize}
  \item \textbf{Healthchecks} - Služby musí implementovat na REST API healtcheck endpoint, který bude poskytovat informace o stavu služby. Endpoint musí být dostupný na standardní adrese \emph{/health}. Návratová hodnota bude triviálně formou řetězce \emph{Healthy} a vracet HTTP Code 200 v případě, že je služba dostupná. Dostupnost je definována schopností služby přijímat požadavky.
  \item \textbf{SwaggerUI} - Pro vizualizaci a testování REST API služeb musí být implementováno grafické rozhraní SwaggerUI. SwaggerUI musí být dostupné na standardní adrese \emph{/swagger} a musí zobrazovat dostupné endpointy a umožňovat jejich testování. SwaggerUI je implementováno pouze v konfiguraci JIT kompilace a režimu Debug.
  \item \textbf{Perzistence souborů} - Aplikace musí umožňovat ukládání libovolného souboru do perzistentního úložiště. Soubor musí být uložen do PostgreSQL databáze a musí být možné ho následně stáhnout. Pro ukládání a čtení souborů musí být implementováno rozhraní REST API. Specificky pro čtení souborů musí být implementováno i gRPC rozhraní.
  \item \textbf{Generování signálů} - Aplikace musí být schopna generovat náhodné signály. Signál musí obsahovat název, jednotku a hodnotu. Pro získání generovaných signálů musí být implementováno rozhraní REST API.
  \item \textbf{Výpočet n-tého Fibonacciho čísla} - Je požadováno, aby aplikace poskytovala funkcionalitu výpočtu Fibonacciho čísla rekurzivní metodou. Tato neefektivní metoda má za účel vytvořit zátěž na systém. Pro volání výpočtu a získání výsledku musí být implementováno rozhraní API.
  \item \textbf{Asynchronní komunikace} - Aplikace musí být schopna asynchronně zpracovávat data z jiných služeb. To zahrnuje vyvolání události a její následnou konzumaci vzorem publish - subscribe. Pro implementaci asynchronní komunikace bude využito RabbitMQ. Samotné vyvolání události musí být k dispozici pomocí požadavku na REST API.
  \item \textbf{Sběr telemetrických dat z .NET služeb} - Aplikace musí být schopna sbírat a ukládat telemetrická data z .NET služeb. To zahrnuje metriky, logy a traces. Data musí být dostupná v reálném čase. Veškerá data budou strukturována dle zásad OpenTelemetry a sbírány na gRPC rozhraní této služby. Sbíraná data budou určena podstatou funkcionality služby a doplněna o množinu dostupných a relevantních metrik zprostředkovaných knihovnami OpenTelemetry.
  \item \textbf{Monitorování kontejnerů a systému} - Aplikace musí být schopna sbírat a vizualizovat data o výkonu, škálovatelnosti kontejnerů a hostitelském systému. To zahrnuje sběr a vizualizaci výkonnostních metrik. Data musí být dostupná v reálném čase a musí být perzistentně ukládána.
  \item \textbf{Testování scénářů} - Aplikace musí být schopna provádět testování scénářů, které simulují běh systému a zátěž na mikroslužby. Testovací scénáře musí být jednoduše vytvořitelné pomocí skriptů. Spouštění scénářů nad aplikací musí být možno více způsoby, napřímo pomocí nástroje nativně běžícího na hostitelském sytému, ale také kontejnerizovaným nasazením nástroje. Testovací scénáře musí být konfigurovatelné a spustitelné v manuálním a automatizovaném režimu. 
  \item \textbf{Vizualizace dat} - V rámci aplikace musí být dostupné grafické rozhraní pro vizualizaci metrik a testovacích dat. Vizualizace musí být dostupná v reálném čase a umožnit zobrazení historických dat. Je nutné, aby aplikace podporovala seskupení a filtraci dat podle druhu, značek a času. Zároveň je požadováno, aby uživatelé mohli jednoduše připojit různé zdroje dat a vytvářet vizualizace ve vlastní režii. Přístup k funkcionalitě vizualizace musí být řešen přes webové rozhraní.
  \item \textbf{Směrování} - Přístup k aplikaci bude řešen pomocí reverzní proxy. Ta bude mít vystavené rozhraní z orchestračního nástroje a její indexovací stránka bude odkazovat na vizualizační nástroj monitorovacích dat.
  \item \textbf{Konfigurace aplikace} - V rámci aplikace musí být možnost konfigurovat chování nasazených služeb. To se týká jak konfigurace komunikace mezi službami, tak i konfigurace monitorovacích nástrojů. Nastavení musí být uloženo v konfiguračních souborech ve standardním formátu dle konvencí dané služby nebo nástroje.
\end{itemize}

\n{3}{Nefunkční požadavky}

Nefunkční požadavky specifikují celkové vlastnosti systému. Definují atributy kvality, které musí systém splňovat. Nefunkční požadavky mohou zahrnovat omezení týkající se návrhu a implementace aplikace, jako jsou bezpečnostní standardy, soulad s právními a regulačními směrnicemi, doba odezvy při zpracování dat, kapacita pro souběžné uživatele, integrita dat a mechanismy převzetí služeb při selhání. Mají zásadní význam pro zajištění životaschopnosti a efektivity aplikace v provozním prostředí. Ovlivňují celkový uživatelský dojem, výkonnost systému a splnění regulačních podmínek. Následující seznam nefunkčních požadavků byl definován pro testovací aplikaci.

\begin{itemize}
  \item \textbf{Použitelnost} - Aplikace musí být snadno použitelná a přístupná pro uživatele. To zahrnuje snadnou konfiguraci a nasazení aplikace na specifickém HW a OS. Aplikace musí být dostupná na webovém rozhraní a standardních portech.
  \item \textbf{Udržitelnost} - Aplikace musí být udržitelná a snadno rozšiřitelná. To zahrnuje dodržení praktik čistého kódu a vhodných návrhových vzorů. Implementace služeb musí být založena na principu SOLID a Don't Repeat Yourself (dále DRY). Dodržování SOLID principu zajišťuje testovatelnost, rozšiřitelnost a dlouhodobou udržitelnost aplikace. Zatímco použitím DRY principu je zabráněno tvorbě duplicitního kódu. Vytvořený kód, konfigurace a skripty musí být řádně dokumentovány.
  \item \textbf{Testovatelnost} - Aplikace musí být snadno testovatelná. To zahrnuje zprostředkování nástrojů a API pro možnost definice a konfigurace vlastních testovacích scénářů. Testování musí být automatizovatelné a poskytovat možnost perzistentního ukládání výsledků testů.
  \item \textbf{Přívětivost} - Aplikace musí být přívětivá pro uživatele. To zahrnuje snadnou navigaci, přehlednost a intuitivní ovládání. Vizuální stránka aplikace musí být jasná a přehledná.
\end{itemize}

\n{2}{Požadavky na HW}

Hardware, na kterém bude aplikace provozována, musí výkonnostně dostačovat pro běh testovacích scénářů a sběr a vizualizaci dat. Týká se to primárně počtu jader, velikosti paměti a rychlosti diskového I/O. Provozované služby mají určitou základní režii, která se musí brát v potaz.

\n{2}{Organizace a správa zdrojů}

Pro správu souborů práce byl zvolen verzovací systém (Source Control Management, dále SCM) Git. Git je open-source nástroj, který umožňuje spravovat a sdílet soubory. Byl zvolen pro svou schopnost vést historii v rámci větví a s ohledem na dostupná serverová úložiště. Za účelem jednoduché organizace souborů bylo zvoleno řešení monorepozitáře. Monorepozitář je repozitář, který obsahuje veškeré soubory projektu, ale také relevantní dokumentaci, obrázky, podpůrné nástroje a zdrojové soubory diplomové práce. Následující struktura adresářů byla zvolena pro organizaci souborů.

\begin{itemize}
    \item \textbf{Documentation} - Zahrnuje podpůrné soubory aplikační dokumentace.
    \item \textbf{Source} - Obsahuje zdrojové soubory aplikace, nasazení a konfigurace.
    \item \textbf{Thesis} - Uchovává text diplomové práce, zdrojové soubory pro vytvoření práce v LaTeX a práci samotnou ve formátu pdf.
\end{itemize}

Pro sdílení veškerých souborů souvisejících s prací a jejich sdílení byl vybrán GitHub, jakožto server pro hostování repozitáře. GitHub je platforma pro verzování souborů a projektů. Navíc poskytuje dodatečné funkce jako je CI/CD, správa dokumentace a další. Repozitář projektu je veden jako veřejný s licencí MIT a je dostupný na adrese \url{https://github.com/DonasNave/MasterThesis}.

\n{2}{Návrh a implementace testovacích služeb}

Následující pasáž se zabírá návrhem a implementací testovacích služeb, které budou využity pro analýzu vývoje a výkonu jednotlivých kompilací AOT a JIT v rámci .NET. Služby jsou implementovány jako mikroslužby a podporují kontejnerizované nasazení v microservice architektuře. Každá služba reprezentuje jednu dílčí funkcionalitu a má definované rozhraní pro komunikaci s ostatními službami. Pro implementaci služeb byla vybrána z podstaty práce technologie .NET, konkrétně jazyk C\#. Verze frameworku byla zvolena .NET 8.0 (konkrétně 8.0.4) jakožto jediná verze oficiálně podporující nativní AOT kompilaci pro framework ASP.NET. Jazyk C\# je použit ve verzi 12.0.

\n{3}{Architektura}

Architektura testovacích služeb byla vytvořena s cílem minimalisticky simulovat scénáře v aplikaci. Pro splnění funkčních požadavků bylo zvoleno následující rozdělení zodpovědnosti .NET služeb:

\begin{itemize}
    \item \textbf{SRS - Signal reading service} - Simuluje roli čtecího zařízení. Generuje signály a poskytuje je ostatním službám.
    \item \textbf{FUS - File Upload Service} - Zprostředkovává datové perzistentní zapisovací zařízení. Čte nebo zapisuje soubory do PostgreSQL databáze.
    \item \textbf{BPS - Business Processing Service} - Reaguje na události publikované v RabbitMQ, do kterého je napojena jako subscriber. Provádí výpočet Fibonacciho čísla.
    \item \textbf{EPS - Event Publishing Service} - Slouží k vyvolání události, která je následně zpracována jinými službami. Je přihlášena do RabbitMQ jako publisher.
\end{itemize}

Nativního AOT kompilace kódu je deklarována použitím atributu \emph{PublishAoT} v projektovém souboru. Za účelem zajištění co největší podobnosti služeb zacílených na AOT a JIT kompilaci, bude využito vymezení konstantních hodnot v rámci projektu. Konstanty \emph{JIT} a \emph{AOT} budou využity pro rozlišení chování služeb v rámci obou kompilačních verzích. S použitím direktiv kompilátoru a zmíněných konstant bude v nutných případech docíleno rozdílného volání API při snaze zachovat totožnou funkcionalitu.

\n{3}{Očekávání vývojového procesu}

Na základě podporované funkcionality, tak jak je definována týmem .NET a popsána v rámci rešerše, je očekáváno, že vývojový proces bude probíhat bez výrazných problémů a bude možné vytvořit služby, které budou schopny zvládnout definované funkční a nefunkční požadavky. Podpora třetích stran byla předem prozkoumána v rámci dostupných dokumentací vybraných knihoven .NET. Konkrétní podoba a rozsah této podpory budou plně ověřitelné až po implementaci a otestování služeb.

\n{3}{Organizace souborů}

Organizace zdrojových souborů služeb, knihoven a pomocných souborů je řešena v rámci hlavního adresáře \emph{DTA} obsahujícího .NET solution soubor, pomocné soubory a solution složky s konkrétními projekty služeb a knihoven. Následující graf popisuje strukturu adresáře projektu.

\obr{Struktura adresáře projektu}{fig:projectstructure}{0.3}{graphics/images/folder-structure-app.png}

Každá z vyvinutých služeb využívá konkrétní .NET SDK \emph{Microsoft.NET.Sdk.Web}, které umožňuje využít třídu \emph{WebApplication} pro registraci a konfiguraci funkcionality služby a zároveň poskytuje konfigurovatelný Kestrel server pro běh programu. Za účelem zajištění jednotného přístupu k logování, metrikám a konfiguraci byly vytvořeny společné knihovny, které jsou využity ve všech službách. Následující graf vizualizuje ukázkou strukturu adresáře služby. Následný seznam popisuje vybrané složky a soubory.

\obr{Struktura adresáře služby}{fig:servicestructure}{0.3}{graphics/images/folder-structure-service.png}

\begin{itemize}
  \item \textbf{Api} - Obsahuje implementaci rozhraní služby.
  \item \textbf{Extensions} - Implementuje extension metody specifické pro doménu služby.
  \item \textbf{Monitoring} - Obsahuje statickou třídu, která drží reference na počítadla metrik.
  \item \textbf{Service} - Ve složce jsou implementovány služby, které provádějí doménovou logiku služby.
  \item \textbf{Properties} - Drží konfiguraci pro spuštění služby.
  \item \textbf{Program.cs} - Obsahuje definici a konfiguraci služby, včetně jejího vstupního bodu.
  \item \textbf{appsettings.json} - Konfigurační soubor služby.
  \item \textbf{Dockerfile-AOT} - Soubor pro tvorbu Docker obrazu pro AOT kompilaci.
  \item \textbf{Dockerfile-JIT} - Soubor pro tvorbu Docker obrazu pro JIT kompilaci.
\end{itemize}

Součástí řešení je společná konfigurace, která je využita ve všech službách. Ta je řešena jedna na úrovni solution souboru a Directory.Build.props souboru. Týká se jednotné distribuce projektových atributů pro verzi, kompatibilitu s AOT, vynucení konkrétních pravidel pro kód a analyzéry.

\n{3}{Knihovny třetích stran}

Pro implementaci funkcionality aplikace byly využity následující knihovny třetích stran:

\begin{itemize}
  \item \textbf{Npgsql} - Npgsql je open-source ADO.NET provider pro PostgreSQL, který umožňuje komunikaci s PostgreSQL databází. Npgsql poskytuje základní balíček funkcí pro vytvoření připojení na základě standardizovaného řetězce pro připojení. Tento balíček sice není plně kompatibilní s AOT kompilací, funkce, které jsou využity v rámci aplikace, jsou avšak kompatibilní.
  \item \textbf{Dapper} - ORM knihovna pro .NET, která umožňuje mapovat databázové struktury na C\# objekty a provádět dotazy na databázi. Dapper.AOT je dílčí knihovna, která umožňuje provádět dotazy na databázi s podporou AOT kompilace. Toho je zajištěno tím, že Dapper.AOT generuje kód pro mapování objektů v době kompilace. Využívá k tomu interceptorů a generátorů pro zachování totožného API jak v případě kódu pro JIT kompilaci. Samotný balíček Dapper.AOT obsahuje další knihovnu Dapper.Advisor, která pomáhá s analýzou zdrojového kódu včetně dotazů na databázi.
  \item \textbf{OpenTelemetry} - OpenTelemetry zprostředkovává množinu knihoven pro sběr, zpracování a export telemetrických dat. V rámci dodaného API je možno registrovat vlastní metriky, logy a traces, ale také nastavit export vybraných dat systémových knihoven a třetích stran. To se týká vybraných knihoven, které zprostředkovávají vlastní implementaci metrik OpenTelemetry.
  \item \textbf{Grpc} - Knihovny pro implementaci komunikace pomocí protokolu HTTP/2 a gRPC. Konkrétně jsou využity Grpc.AspNetCore v případě serveru, Grpc.Net.Client pro klienta a Google.Protobuf s Grpc.Tools pro generování modelů v přístupu code first.
  \item \textbf{RabbitMQ} - Asynchronní komunikace a implementace publish subscribe vzoru je umožněna knihovnou RabbitMQ.Client. S její pomocí aplikace komunikují s brokerem, vytváří fronty, dochází k přihlášení nebo odběru zpráv a jejich publikování.
  \item \textbf{Swagger} - Grafické rozhraní pro vizualizaci a testování REST API služeb. Swagger je využit pouze v kombinaci konfigurací \emph{JIT Debug}. K tomuto účelu jsou využity knihovny Swashbuckle.AspNetCore a Microsoft.AspNetCore.OpenApi.
\end{itemize}

\n{3}{Společné knihovny}

V rámci zjednodušení tvorby služeb, jednotné implementace a konfigurace, ale také z důvodu zajištění některé základní klíčové funkcionality, byly vytvořeny společné knihovny. Tyto knihovny obsahují společné třídy, rozhraní a konfigurace, které jsou použity ve všech službách. Následující výčet popisuje oblasti funkcionality, které jsou zprostředkovány společnými knihovnami.

\begin{itemize}
  \item \textbf{Perzistence} - Pro implementaci perzistence byla vytvořena pomocná knihovna DTA.Extensions.Postgres, která poskytuje pomocnou funkcionalitu pro zajištění existence databáze pro službu, dle konfigurace v řetězci pro připojení.
  \item \textbf{Migrace} - Zajištění migrace databáze bylo implementováno po vlastní ose minimalistickým migrátorem v knihovně DTA.Migrator. Tato knihovna poskytuje základní funkcionalitu pro vytvoření databáze, vytvoření tabulek a indexů, ale také zajištění migrace dat a verzování změn.
  \item \textbf{Telemetrie} - Knihovna DTA.Extensions.Telemetry zprostředkovává extensions metody pro jednotnou a jednoduchou registraci sběru a export telemetrických dat napříč službami.
  \item \textbf{Modely} - Knihovna DTA.Models obsahuje společné modely, které jsou využity ve službách. Je tím docílena dostupnost datových struktur a rozhraní aplikace napříč všemi službami.
  \item \textbf{Obecná funkcionalita} - Za účelem sjednocení funkcionality využité napříč všemi službami jsou v rámci DTA.Extensions.Common knihovny implementovány specifické extension metody. Poskytnuta je funkcionalita pro extrakci názvů a verzí z metadat služby.
\end{itemize}

\n{3}{SRS - Signal reading service}

Za účelem simulace funkce čtecího zařízení byla vytvořena služba SRS. Tato služba poskytuje základní rozhraní pro získání dat signálu včetně jednotek formou REST API. Pro zjednodušení implementace není využito čtení dat ze skutečného zdroje, ale jsou generována náhodná data. Načež data jsou následně poskytována se zdržením simulujícím čtení dat ze vzdáleného zdroje. Služba poskytuje následující rozhraní

\begin{itemize}
    \item \textbf{GET /api/signals/\{int:amount\}} - Vygeneruje zadané množství náhodných signálů
\end{itemize}

\n{3}{FUS - File Upload Service}

Služba v systému hraje roli rozhraní k persistentnímu uložišti, v rámci kterého čte a zapisuje data. Jakožto úložiště je využito PostgreSQL databáze. Služba využívá vlastní databázovou instanci a spravuje vlastní tabulky pomocí migrací definovaných SQL skripty. Pro přístup k perzistence dat je využito knihovny Dapper, která umožňuje mapování databázových struktur na C\# objekty a vytváření a provádění dotazů na databázi. SRS poskytuje rozhraní formou REST API pro zápis a čtení dat. Daty je myšlen libovolný soubor v libovolném formátu. Samotná podstata nahraných dat není pro službu důležitá, ale to že jsou uložena do databáze. Za účelem sehrání testovacích scénářů poskytuje služba také gRPC rozhraní, které je zajištěno na dedikovaném portu. V rámci gRPC komunikace slouží FUS jako server, který zpracovává volání vzdálené procedury. Služba poskytuje následující rozhraní:

\begin{itemize}
    \item \textbf{GET /api/file/download/\{int:id\}} - Stáhne soubor podle zadaného ID.
    \item \textbf{POST /api/file/upload} - Nahraje soubor do systému. Soubor je předán jako multipart/form-data.
    \item \textbf{gRPC Operation FileServer.GetFile} - Stáhne soubor podle zadaného objektu s ID.
\end{itemize}

\n{3}{BPS - Business Processing Service}

Pro splnění role vytvoření zátěže na systém je vytvořena BPS. Tato služba získává data pomocí volání gRPC, konzumuje jako subscriber událost a provádí náročnou výpočetní operaci, kterou simuluje obtížnou doménovou operaci. Konkrétně implementován je neefektivní rekurzivní výpočet zadaného čísla Fibonacciho posloupnosti. BPS se po spuštění přihlašuje k odběru zpráv na předem definovaný kanál \emph{simulated} na službe RabbitMQ. Po získání zprávy volá vzdálenou proceduru nad FUS. Následně provádí výpočet 40-tého Fibonacciho čísla. Tato hodnota je pevná s ohledem na její jediné využití a zacílení pro specifický testovací scénář. Služba poskytuje následující rozhraní:

\begin{itemize}
    \item \textbf{GET /api/processFibonacci/\{int:degree\}} - Vypočítá číslo z Fibonacciho posloupnosti na zadané pozici náročným rekurzivním způsobem.
    \item \textbf{Event subscribed: <queue-name>\_simulated} - Přihlášení k odběru zpráv v rámci kanálu na službě RabbitMQ.
\end{itemize}

\n{3}{EPS - Event Publishing Service}

Jednoduchá služba umožňující vyvolat událost ve frontě a docílit spuštění dodatečných operací v aplikaci. V systému simuluje roli uživatele vyvolávajícího událost. Služba poskytuje následující rozhraní:

\begin{itemize}
    \item \textbf{GET api/simulateEvent/\{int:id\}} - Vyvolá simulovanou událost s daným ID.
    \item \textbf{Event published: <queue-name>\_simulated} - Vyvolá údalost se zprávou obsahující identifikátor na konfigurovaném kanálu do služby RabbitMQ.
\end{itemize}

Následující diagram znázorňuje přímé závislosti testovacích služeb na další nástroje.

\obr{Diagram .NET služeb a závislých služeb}{fig:logo}{0.75}{graphics/diagrams/services-architecture.png}

\n{2}{Monitorování aplikace}

Za účelem monitorování aplikace byla vybraná množina nástrojů. Tyto nástroje umožňují sběr, perzistenci a vizualizaci metrik, traces a logů. Klíčové bylo zajistit možnost sledovat dění uvnitř aplikace, ale i v rámci hostitelského systému. Následující pasáž se zabývá výběrem a implementací monitorovacích nástrojů.

\n{3}{Grafana observability stack}

Pro monitorování aplikace byl zvolen Grafana Observability stack pro jeho pokrytí komplexní škály monitorovacích dat. Zahrnuje specifické nástroje pro sběr, vizualizaci a analýzu dat. Zprostředkovává jednoduchou možnost propojení dílčích nástrojů a konfiguraci datových zdrojů. V neposlední řadě poskytuje rozsáhlé možnosti vizualizace. Následující nástroje jsou součástí Grafana Observability stacku.

\begin{itemize}
  \item \textbf{Grafana} - Grafana je open source webová aplikace pro analýzu a interaktivní vizualizaci dat. Poskytuje možnost sestavit dashboard z komponent jako jsou grafy, tabulky a další. Jedná se o velmi populární technologii v doménách serverové infrastruktury a monitorování. Grafana umožňuje sjednotit monitorovací služby a zobrazit data v reálném čase. Podporuje širokou škálu datových zdrojů, jako jsou Prometheus, InfluxDB, Tempo, Loki nebo PostgreSQL, což umožňuje jednoduchou konfiguraci a připojení monitorovacích dat aplikace. Zároveň nativně podporuje datový zdroj, kde ukládá data testovací nástroj aplikace. Kombinací dat z různých zdrojů umožňuje vytvářet komplexní pohled na celý systém.
  \item \textbf{Prometheus} - Open-source monitorovací systém. Shromažďuje a ukládá metriky jako time-series data a umožňuje se na ně dotazovat pomocí vlastního výkonného jazyka PromQL. Jeho architektura podporuje více modelů získávání dat, což je využito při napojení více zdrojů v aplikaci. Dále umožňuje přímé stahování metrik z cílových služeb nebo kolektorů, odesílání metrik přes gateway a zprostředkování notifikací.
  \item \textbf{Loki} - Škálovatelný agregátor logů. Na rozdíl od obdobných systémů pro agregaci logů, jenž indexují všechna data, Loki indexuje pouze metadata, přičemž ukládá celá data logu efektivním způsobem. Loki je navržen tak, aby jednoduše spolupracoval s Grafanou a umožňuje rychle vyhledávat a vizualizovat logy.
  \item \textbf{Tempo} - Poskytuje jednoduše ovladatelný open-source nástroj pro sledování distribuovaných požadavků. Tempo podporuje ukládání a načítání traces. Na rozdíl od mnoha jiných nástrojů pro traces, nevyžaduje Tempo žádné předem definované schéma. Je navržen tak, aby se bezproblémově integroval s Prometheus, Grafanou a aby jednoduše přijímal data z OpenTelemetry kolektoru.
  \item \textbf{OpenTelemetry} - Open source kolektor monitorovacích dat. Poskytuje jednotný způsob sběru, zpracování a exportu dat. Je konfigurovatelný a podporuje více pipeline, které mohou upravovat telemetrická data při jejich sběru. Výrazně zjednodušuje instrumentaci služeb, protože umožňuje agregovat a exportovat metriky, traces a logy do různých analytických a monitorovacích nástrojů. Poskytuje podporu pro export dat do Prometheus, Tempo i Loki.
\end{itemize}

Implementace Grafana Observability stacku je zajištěna pomocí obrazu nazvaného dta-lgtm a sestrojeného po vzoru Grafana LGTM (Loki, Grafana, Tempo a Mimir). Grafana LGTM kombinuje množinu monitorovacích nástrojů v rámci jediného obrazu s předpřipravenou konfigurací. Tím je odstíněna část konfigurace monitorovacího stacku a zjednodušen proces nasazení. Obraz použitý v rámci práce využívá kombinaci dříve zmíněných technologií (Loki, Tempo, Grafana, Prometheus a OpenTelemetry) zabalených a předem konfigurovaných v rámci Docker obrazu. Tím je v aplikaci zajištěno, že potřebná konfigurace pro vzájemné propojení nástrojů a datových zdrojů je předpřipravena. Stejně tak jsou součástí vytvořeného obrazu vlastní monitorovací dashboardy pro vizualizace.

V rámci aplikace mají jednotlivé služby nastaven export svých logů, traces a metrik do OpenTelemetry, respektive na adekvátní rozhraní dta-lgtm. Služby využívají existujících metrů a logů, ale také vytváří vlastní metriky a logy. Vlastní metriky zahrnují informace o počtu a druhu provedených operací. Z předpřipravených metrik, ať systémových nebo třetích strán jsou využity následující instrumentace:

\begin{itemize}
  \item \textbf{System.Runtime} - Metriky běhového prostředí .NET.
  \item \textbf{System.Net.Http} - Metriky HTTP dotazů.
  \item \textbf{Microsoft.AspNetCore.Hosting} - Metriky hostovacího prostředí ASP.NET Core.
  \item \textbf{Microsoft.AspNetCore.Server.Kestrel} - Metriky serveru Kestrel.
  \item \textbf{Npgsql} - Metriky klientské knihovny PostgreSQL.
\end{itemize}

\n{3}{Monitorování hostitelského systému}

Monitorování hostitelského systému poskytuje pro aplikaci klíčové informace o využití zdrojů a výkonu, jak ze samotného systému, tak i z jednotlivých kontejnerů. Pro monitorování Docker kontejnerů je využit nástroj CAdvisor. CAdvisor je schopen monitorovat kontejnery běžící na Dockeru, Kubernetes nebo jiných kontejnerových platformách. Poskytuje informace o využití procesoru, paměti, sítě a diskového I/O z pohledu hostitelského systému. Dalším využitým nástrojem je NodeExporter. NodeExporter je nástroj pro sběr metrik z hostitelského systému. Obdobně jako CAdvisor poskytuje informace o využití procesoru, paměti, sítě a diskového I/O. Data z obou zmíněných nástrojů jsou exportována do Prometheus, kde jsou následně zpracována a zprostředkována Grafaně.

\n{2}{Testovací nástroje}

Za účelem testování monitorovacího stacku byl vybrán nástroj K6. Jedná se o moderní open-source nástroj pro zátěžové testování. Slouží k vytváření a spouštění výkonnostních testů nad aplikací. Nabízí bohaté API pro těchto testů v jazyce JavaScript. Umožňuje psát komplexní scénáře napodobující reálný provoz systému nebo simulovat hraniční situace. K6 podporuje různé systémové metriky, jako je doba odezvy, propustnost a chybovost. Nabízí široké možnosti rozšíření skrze API, což umožňuje přizpůsobení a integraci s dalšími nástroji pro komplexní sledování výkonu. K6 obsahuje nativní integraci exportu výsledku testování do databáze InfluxDB ve verzi 1. Pro zjednodušení procesu testování a zajištění opakovatelnosti testovacích scénářů byly vytvořeny skripty pro spuštění testů. Tyto skripty zajišťují opakovatelné spouštění testů v rámci Docker kontejneru s nastavitelnými parametry. Skripty jsou vytvořeny v jazyce Bash a využívají nástroje K6 pro spuštění testů a zpracování výsledků. Zároveň pracují s Docker Compose orchestrací pro spouštění a vypínání služeb dle potřeby testů.

\n{2}{Nasazení aplikace}

Následující pasáž popisuje náležitosti nasazení aplikace, včetně obecného popisu finální struktury nasazení, využitých nástrojů kontejnerizace a orchestrace, konkrétních verzí obrazů a použitou konfiguraci.

\n{3}{Přehled řešení}

Řešení aplikace sestává z následujících sekcí a jednotlivých obrazů služeb a verzí, definovaných v rámci Docker Compose souboru.

\begin{itemize}
    \item \textbf{Testovací služby} - Aplikace obsahuje testovací .NET služby FUS, SRS, BPS a EPS. Tyto služby jsou vytvořeny ve dvou kompilačních verzích - AOT a JIT. Každá služba je vytvořena jako obraz s názvem \emph{dta-<service-name>} a značkou \emph{<compilatioinMode>-latest}.
    \item \textbf{Komunikace} - Komunikační kanál mezi službami je zajištěn pomocí RabbitMQ. Pro RabbitMQ je využit obraz \emph{rabbitmq:3-management-alpine}.
    \item \textbf{Monitorovací nástroje} - Monitorování zajišťuje Grafana Observability stack implementovaný v rámci obrazu \emph{dta-lgtm:latest}, jenž obsahuje OpenTelemetry, Prometheus, Loki, Tempo a Grafanu. Pro měření výkonu hostitelského systému a export těchto dat jsou využity služby NodeExporter a CAdvisor s obrazy \emph{node-exporter:latest} a \emph{cadvisor-arm64:v0.49.1}.
    \item \textbf{Perzistence} - Pro perzistenci dat je využita PostgreSQL databáze. Pro PostgreSQL je využit obraz \emph{postgres:latest}. Ukládání metrik je zajištěno pomocí InfluxDB a obrazu \emph{influxdb:1.8.10}.
    \item \textbf{Směrování} - Funkci reverzní proxy zajišťuje Nginx ve verzi obrazu \emph{nginx:latest}.
\end{itemize}

\obr{Diagram nasazení aplikace}{fig:app}{0.85}{graphics/images/dta.drawio.png}

\n{3}{Kontejnerizace a orchestrace}

Kontejnerizace služeb je zajištěna pomocí nástroje Docker. Docker je open-source platforma poskytující ekosystém pro správu kontejnerů. Jednotlivé služby jsou vytvořeny jako obrazy podle definicí Dockerfile. 

Orchestrace aplikace je zprostředkována rovněž nástrojem Docker, konkrétně formou \emph{compose} utility. Ta umožňuje jednoduše nasadit a spravovat větší množství služeb. Definice nasazení aplikace je sepsána v souboru \emph{compose.yaml}. V něm lze nalézt následující části:

\begin{itemize}
  \item \textbf{volumes} - V této sekci jsou definovány všechny volume, které jsou využity v rámci stacku. Volume jsou definovány názvem a využity pro ukládání dat služeb. Konkrétně jsou využity úložiště pro data Grafany, OpenTelemetry, Prometheus a InfluxDB.
  \item \textbf{networks} - Konfigurace pro vnitřní síť aplikace, která je využita pro komunikaci mezi službami. Síť je definována názvem a typem. Pro potřeby aplikace se využívá síť s názvem \emph{stack-network} a typ \emph{bridge}, jenž funguje jako síťový most.
  \item \textbf{services} - Definuji sekci pro jednotlivé služby, které jsou součástí stacku. Každá služba je definována názvem služby - \emph{container\_name}, názvem obrazu - \emph{image}, v případě lokálně sestavených služeb také definicí sestavení - \emph{build}. Dále je definováno, jaké porty jsou mapovány z kontejneru do hostitelského systému - \emph{ports}, jaké volume jsou připojeny k kontejneru - \emph{volumes}, použité síťové rozhraní - \emph{networks}, závislosti služby - \emph{depends\_on} a proměnné prostředí - \emph{environment}. V případech testovaných služeb je uvedena dodatečná konfigurační sekce nasazení - \emph{deploy}, jenž limituje dostupné zdroje paměti a procesoru.
\end{itemize}

\n{3}{Konfigurace nasazení}

Za účelem běhu aplikace je klíčové správné nastavení konfigurace. Konfigurace je řešena na různých úrovních. Základní úroveň představuje soubor \emph{compose.yaml}. V tomto souboru je dílčí konfigurace jednotlivých služeb řešena v rámci konfiguračních souborů, které jsou do služeb připojeny z hostitelského OS, a proměnných prostředí. 

Nastavení směrování v rámci stacku je řešeno konfigurací proxy služby Nginx. Za tímto účelem obsahuje dva klíčové soubory, rozcestník v podobně statického \emph{index.html} souboru a konfigurační \emph{nginx.conf} soubor se směrovacími pravidly. Oba zmíněné soubory jsou do služby připojeny formou mapování virtualizovaného repozitáře kontejneru. Směrovací pravidla jsou následující:

\begin{itemize}
    \item / - cesta na statickou hlavní stránku-rozcestník aplikace
    \item /grafana - směrování na Grafanu
\end{itemize}

Nastavení telemetrie spočívá v definici rozhraní, nastavení chování služeb a systémů z niž se telemetrická data sbírají, jejich cíl pro zpracování, správu a vizualizaci. Značnou část konfigurace představuje propojení nástrojů aplikace v rámci obrazu dta-lgtm, k čemuž slouží konfigurační soubory. Ty obsahují výchozí minimalistickou konfiguraci pro jednotlivé nástroje. Výstupem dta-lgtm je individuální kontejner a zmíněná konfigurace je z podstaty interní a není potřeba s ní manipulovat s ohledem na požadavky práce. Mezi dodatečné konfigurace telemetrie napříč službami patří:

\begin{itemize}
    \item \textbf{Testované služby} - Veškeré testovací služby mají nastavený endpoint pro export telemetrických dat. Toto nastavení je zprostředkováno proměnnou prostředí \emph{OpenTelemetrySettings\_ExporterEndpoint}.
    \item \textbf{LGTM} - Konfigurace pro LGTM je řešena pomocí proměnných prostředí. Je definována adresa a cesta, z které je k dispozici Grafana. Dále je umožněno anonymní přihlášení do Grafany.
    \item \textbf{Cadvisor} - Nastavení služby představuje dodatečné nastavení volumes, které jsou připojeny k kontejneru. Připojením systémových souborů je zajištěno sběr dat o využití systémových zdrojů. Toto nastavení je závislé na OS hostitelského stroje.
\end{itemize}

Jednotlivé služby mají vlastní dodatečné konfigurace, které jsou řešeny pomocí kombinace dle standardizovaného postupu pro konfiguraci .NET aplikací, a to konfiguračního souboru \emph{appsettings.json} a proměnných prostředí. V první řadě je použita konfigurace ze souboru, načež je přepsána odpovídajícími hodnotami proměnných prostředí. Každá služba má definovaný specifický prefix pro identifikaci proměnných prostředí. Následující seznam popisuje dodatečné konfigurace jednotlivých služeb.

\begin{itemize}
  \item \textbf{FUS - File Upload Service} - Obsahuje connection string (přístupový řetězec) pro připojení do databáze PostgreSQL.
  \item \textbf{BPS - Batch Processing Service} - Disponuje konfigurací pro připojení k RabbitMQ a konkrétní frontě.
  \item \textbf{EPS - Event Publishing Service} - Drží informaci o rozhraní RabbitMQ, konkrétní frontě a gRPC rozhraní služby FUS.
\end{itemize}

Služby využité pro perzistentní ukládání dat jsou konfigurovány pomocí proměnných prostředí. Jedná se o následující služby:

\begin{itemize}
  \item \textbf{PostgreSQL} - Jsou definovány údaje pro uživatele databáze a název databáze.
  \item \textbf{InfluxDB} - Proměnné prostředí definují název databáze, uživatelské údaje a povolení přihlášení pomocí http.
\end{itemize}

Definice uživatelského rozhraní, respektive dostupných dashboardů, je dána při sestavení obrazu LGTM. V rámci něj jsou předdefinovány hodnoty pro připojení zdrojů dat, tj. Prometheus, Loki, Tempo a InfluxDb. Patřičné dashboardy zobrazující relevantní data pro různé scénáře systému byly předem připraveny a jsou k dispozici po otevření Grafany anonymním uživatelem. Následující seznam popisuje klíčové soubory konfigurace uživatelského rozhraní.

\begin{itemize}
  \item \textbf{grafana-dashboards.json} - Definuje dostupné dashboardy v Grafaně. Dashboardy jsou definovány v JSON formátu a obsahují definici panelů, zdrojů dat a dalších parametrů.
  \item \textbf{grafana-datasources.json} - Obsahuje  zdroje dat z kterých Grafana, respektive dashboardy, čerpají data.
\end{itemize}