
%%%%%%%%%%%%%%%%%%%%%%%%%%%%%%%%%%%%%%%%%%%%%%%%%%%%%%%%%%%%%%%%%%%%%%%%%%%%%%%%%
%                                  MONITORING                                   %
%%%%%%%%%%%%%%%%%%%%%%%%%%%%%%%%%%%%%%%%%%%%%%%%%%%%%%%%%%%%%%%%%%%%%%%%%%%%%%%%%

\n{1}{Monitorování aplikace}

Monitorování aplikace je proces získávání zpětné vazby. Představuje klíčový aspekt vývoje a provozu software. \cite{Riedesel2021} Umožňuje sledovat stav, výkon a celkové chování aplikace. Zahrnuje shromažďování, analýzu a interpretaci různých typů dat a informací, které zajišťují hladký a efektivní chod aplikací. Umožňuje rychle identifikovat a řešit problémy vzniklé za provozu. 

\n{2}{Cíle monitorování}

Cílem monitorování v kontextu mikroslužeb je poskytnout informace v klíčových oblastech aplikace. Základní informací je dostupnost aplikace, případně jednotlivých mikroslužeb. Ta zajišťuje, že lze rychle identifikovat vznik problému a minimalizovat dobu nedostupnosti systému. Výkonnostní metriky, jako je odezva, propustnost a chybovost, pomáhají pochopit, jak aplikace funguje za normálních podmínek i při zátěži. \cite{Riedesel2021} Rovněž poskytují zpětnou vazbu o dopadu různých strategií nasazení a sestavení na výkon. Zahrnují údaje o míře úspěšnosti nasazení a problémy vyplývající z nových nasazení pomocí analýzy chování služeb v produkčním prostředí.

\n{2}{Druhy dat}

Monitorovací data hrají zásadní roli při údržbě a optimalizaci moderních softwarových systémů. Sběrem, analýzou a interpretací různých typů dat lze získat cenné informace o výkonu, stavu a celkovém chování aplikace. Následující část kategorizuje tři základní typy monitorovacích dat: logy, traces a metriky. \cite{Majors2022}

\n{3}{Logy}

Logy jsou chronologické záznamy o událostech, ke kterým dochází v rámci aplikace. Pomáhají určit hlavní příčiny konkrétních problémů nebo odhalit vzory rozsáhlých chyb. Logy generuje jak aplikace, tak dle použité technologie i prostředí jenž aplikaci spravuje. Ať se jedná o OS nebo běhové prostředí. Logy mohou nabývat různých struktur. Nejjednodušší forma je obyčejný textový řetězec. Některé nástroje však podporují i strukturované logy. To jsou datové záznamy nabývající podoby typicky JSON nebo XML formátu. Je běžné, že konkrétní záznam logu poskytuje pouze část informace o systémové události. A tedy, že celá událost sestává z více záznamů. \cite{Majors2022} Logy typicky obsahují data o systémových aktivitách, chybách, zprávách, změnách konfigurace a síťových požadavcích. Analýzou logů mohou vývojáři a správci systému porozumět kontextu aplikace a zajistit soulad s očekávaným chováním. \cite{Majors2022}

\n{3}{Traces}

Traces trasují cestu požadavku, při průchodu různými částmi distribuované aplikace. Každá trace se skládá z jednoho nebo více segmentů, které zaznamenávají cestu a latenci požadavku napříč různými službami a zdroji. Sledování je zvláště důležité v architektuře mikroslužeb, kde jedna transakce může zahrnovat více volně propojených služeb. Traces poskytují přehled o výkonu a chování jednotlivých služeb, ale také aplikace jako celku. Pomáhají identifikovat úzká místa aplikace a problémy s latencí. \cite{Molkova2023} Zároveň umožňují pochopit vztahy a závislosti mezi službami, což podporuje efektivnější ladění a optimalizaci aplikace. Obvykle jsou reprezentovány stromovou strukturou, jenž obsahuje hierarchii segmentů. Každý segment udržuje informace o době odezvy, latenci, chybách a dalších metrikách. \cite{Molkova2023}

\n{3}{Metriky}

Metrika představuje konkrétní číslo volitelně doplněné o značky (tagy) sloužící k identifikaci a seskupování. \cite{Majors2022} Jedná se o kvantitativní údaj, který reprezentuje informaci o stavu aplikace. Metriky jsou klíčovým prvkem monitorování, protože poskytují objektivní a měřitelné informace o výkonu a chování aplikace. Typicky jsou shromažďovány v pravidelných intervalech v reálném čase. Běžná data, jenž metriky reprezentují, jsou například využití systémových prostředků (CPU, paměť, I/O, ...), doba odezvy, chybovost anebo propustnost. Sledování těchto metrik pomáhá při ladění výkonu a správě aplikace.

\n{2}{Monitorovací nástroje}

Monitorovací data vyžadují nástroje, které umožňují jejich shromažďování, analýzu a vizualizaci. Obvykle se skládají z několika komponent včetně agentů, kolektorů, databází, vizualizačních nástrojů a nástrojů pro analýzu dat. \cite{Riedesel2021} Různé nástroje mohou poskytovat komplexní řešení nebo pouze část funkcionality vyžadované za účelem monitorování systému.

\n{2}{Sběr dat}

Efektivita monitorování aplikací do značné míry závisí na schopnosti shromažďovat relevantní data z různých zdrojů a na schopnosti zprostředkovat tato data do monitorovacích nástrojů. Kolektory jsou nástroje nebo agenti, kteří shromažďují data z různých zdrojů v rámci aplikace a jejího prostředí. \cite{Blanco2023} Mohou být nasazeny jako součást infrastruktury aplikace nebo mohou být provozovány jako externí služby. Kolektory jsou zodpovědné za shromažďování logů, traces a metrik. Dále zprostředkovávají předávání těchto dat do monitorovacích nástrojů, kde je následně možné provést analýzu a vizualizaci. Efektivní sběr dat je nezbytný pro monitorování v reálném čase a pro zajištění toho, aby shromážděná data nezaznamenala zkreslený stav a výkon aplikace. Příklad velmi populárního univerzálního kolektoru je OpenTelemetry. Tento nástroj mimo jiné definuje prostor monitorovacích technologií, protokoly a standardy využívané pro sběr a propagaci dat nezávisle na využitých nástrojích. \cite{Blanco2023}

\n{2}{Analýza a interpretace}

V oblasti monitorování systému je sběr dat pouze prvním krokem. Skutečná hodnota spočívá v tom, jak jsou tato data analyzována a interpretována. Analýza a interpretace transformují nezpracovaná data na praktické poznatky, které umožňují porozumět nejen tomu, co se děje v aplikacích, ale také tomu, proč k těmto událostem dochází. \cite{Majors2022} Tyto procesy jsou úzce propojeny. Vizualizace přibližuje pochopení komplexních dat a pomáhá uživatelům monitorovacích nástrojů rozpoznat trendy a anomálie na první pohled. Pokročilé analytické techniky poskytují hlubší porozumění dat, odhalují základní vzorce a předpovídají budoucí trendy, kterými podporují strategické rozhodování. Společně umožňují reagovat na aktuální stav aplikace, proaktivně ji spravovat a optimalizovat budoucí výkon. Grafickým znázorněním datových toků lze snadněji odhalit trendy a vzorce, které nemusí být patrné ze samotných nezpracovaných dat. Vizualizace mohou mít různé formáty, následující jsou příklady běžně používaných vizualizačních komponent:

\begin{itemize}
    \item \textbf{Grafy} - Graficky reprezentují data vůči času nebo jiným škálám. Různými typy jako např. spojnicové grafy, sloupcové grafy a bodové grafy, které mohou zobrazovat změny v čase, distribuci a korelaci.
    \item \textbf{Tabulky} - Prezentují nezpracovaná data zarovnaná do sloupců pro přímé srovnání.
    \item \textbf{Řídicí panely} - Integrují více vizualizací do jediného rozhraní a nabízejí detailní pohled na výkon a stav systému.
    \item \textbf{Heatmapy} - Zobrazují složité vztahy a toky mezi komponentami systému. Barevné škály ukazují intenzitu a frekvenci událostí.
\end{itemize}

Použitím a kombinací vizualizačních komponent vzniká unikátní pohled na dostupná monitorovací data. Tím je umožněno rychle identifikovat kritická místa systému, jako jsou bottlenecks (úzká místa výkonu) a řešit potenciální problémy dříve, než ovlivní stabilitu systému nebo uživatelský dojem z aplikace. \cite{Chapman2023} Konečným cílem analýzy a vizualizace dat je podpora informovaného rozhodování. Interpretovaná data poskytují užitečné poznatky, na základě kterých lze provádět strategická rozhodnutí, optimalizovat výkon a stabilitu systému. \cite{Majors2022} Následující seznam uvádí vybrané oblasti, ve kterých můžou informace získané monitorováním aplikací být využity:

\begin{itemize}
    \item \textbf{Přidělování zdrojů} - Úprava přidělování zdrojů na základě informací o výkonu. Za účelem optimalizace zdrojů a výkonu probíhá například škálování aplikace. V reakci na očekávanou zátěž je možné dynamicky přidělit nebo odebrat zdroje.
    \item \textbf{Optimalizace výkonu} - Identifikace a řešení překážek výkonu. Nalezením úzkých míst v aplikaci lze optimalizovat software (dále SW) s cílem zlepšit odezvu a spokojenost uživatelů.
    \item \textbf{Zabezpečení} - Rozpoznání vzorců indikujících bezpečnostní hrozby. Neočekávaným počtem požadavků nebo jiným využitím zdrojů lze identifikovat probíhající útok na aplikaci.
    \item \textbf{Vylepšení služeb} - Využití dat o interakci uživatelů s aplikací. Data popipsující chování uživatelů podporují kroky vedoucí k vylepšení aplikace dle očekávání zákazníků.
\end{itemize}

\n{2}{Implementace monitorování}

Implementace monitorování aplikace zahrnuje několik kroků, včetně definice klíčových metrik, výběru monitorovacích nástrojů, nasazení kolektorů a vizualizace dat. Zároveň by měly být vytvořeny procesy a postupy pro využití dat a řešení problémů, které byly identifikovány prostřednictvím monitorování. \cite{Blanco2023} Obecně implementace monitorování zahrnuje následující kroky:

\begin{enumerate}
    \item \textbf{Sběr dat v monitorovaných službách} - Implementace sběru dat zahrnuje inkorporaci funkcionality monitorování a zprostředkování dat v rámci předem definovaného rozhraní. Sběr je realizován zpravidla sérií čítačů a zapisovačů, které jsou využívány k získávání dat z různých zdrojů. Takto sbíraná data jsou kategorizována a označena pro identifikaci. Realizace monitorování je zajištěna buďto použitím existujících implementací v rámci SW knihoven nebo vytvořením vlastní implementace dle potřeb aplikace a monitorovacích protokolů.
    \item \textbf{Nasazení služeb pro správu a sběr dat} - Je zajištěno pomocí nasazení nástrojů, které jsou schopny zprostředkovat sběr a distribuci telemetrických dat z různých zdrojů. Zároveň mohou také zprostředkovat jejich zpracování a zobrazení. Klíčový je výběr nástrojů s ohledem na kompatibilní rozhraní pro sběr a následnou distribuci dat k vizualizaci. Splněním tohoto je dosaženo, že data mohou být zpracována, uložena a dále zprostředkována.
    \item \textbf{Vizualizace dat} - Vizualizace dat je implementována nasazením nástrojů, které jsou schopny zobrazit data z dostupných zdrojů v uživatelsky přívětivé podobě. Formát připojení na datové zdroje s monitorovacími daty je definován protokoly služeb spravujících tato data. Vizualiace konkrétních dat je předmětem implementace grafických komponent za pomocí dostupných rozhraní.
\end{enumerate}

Konfigurace monitorování v rámci aplikace obecně zahrnuje zmapování interakcí mezi monitorovanými komponentami a monitorovacími nástroji. To zahrnuje určení, které metriky, logy a traces jsou relevantní na základě architektury aplikace a doménových požadavků. Konfigurace musí zajistit, že shromážděná data budou relevatní a vyvarovat se nadměrné granularitě, která může vést k přetížení systému. \cite{Blanco2023} Obvykle tento proces zahrnuje nastavení agentů či integrací v rámci aplikace nebo služeb, které efektivně sbírají data a přenáší je dále do nástrojů monitorovacího systému. Konají tak aniž by došlo k narušení výkonu aplikace. Komunikace mezi aplikačními komponentami a monitorovacími nástroji využívá síťové protokoly a metody bezpečného přenosu dat. Proces konfigurace může dále zahrnovat nastavení hraničních hodnot pro výstrahy, definování pravidel retence dat a nastavení parametrů pro automatické reakce na určité typy událostí. Tyto prvky pomáhají udržovat celkový stav a výkon aplikace. \cite{Blanco2023} Implementace tohoto přístupu zajišťuje, že monitorovací systém poskytuje užitečné informace v souladu s doménovými požadavky a poskytuje komplexní bázi dat pro analýzu a optimalizaci systému.