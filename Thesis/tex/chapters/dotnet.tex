%%%%%%%%%%%%%%%%%%%%%%%%%%%%%%%%%%%%%%%%%%%%%%%%%%%%%%%%%%%%%%%%%%%%%%%%%%%%%%%%%
%                                    .NET                                       %
%%%%%%%%%%%%%%%%%%%%%%%%%%%%%%%%%%%%%%%%%%%%%%%%%%%%%%%%%%%%%%%%%%%%%%%%%%%%%%%%%

\n{1}{Platforma .NET}

Platforma .NET od společnosti Microsoft představuje komplexní sadu nástrojů k vývoji aplikací v podporovaných jazycích. Tato platforma je multiplatformní a umožňuje vývoj pro operační systémy jako Windows, Linux, macOS ale i pro mobilní platformy a zařízení Internet of Things (dále IoT). Vývojáři mohou využívat nástroje pro vývoj webových aplikací, desktopových aplikací, mobilních aplikací a dalších. Platforma .NET je postavena na dvou hlavních nástrojích. Prvním z nich je \textit{CLR}, běhové prostředí zodpovídající za běh aplikací. Druhým nástrojem je \textit{.NET CLI} (Command Line Interface, dále CLI), konzolový nástroj, zodpovědný za interakci s dílčími nástroji platformy .NET.

\n{2}{Historie}

Využití runtime prostředí, respektive v originální podobě virtuálního stroje, má historický původ. V dřívějších dobách byli programátoři limitování nutností kompilace kódu do nativní reprezentace přímo pro architekturu systému. Kód vytvořen pro jednu konkrétní architekturu se zpravidla neobešel bez modifikací, pokud měl fungovat i na odlišné architektuře.

V průběhu 90. let 20. století představila společnost Sun-Microsystems virtuální stroj Java Virtual Machine (dále JVM). Jedná se o komponentu runtime prostředí Javy, která zprostředkovává spuštění specifického kódu, správu paměti, vytváření tříd a typů a další. Kompilací Javy do tzv. bytecode (Intermediate Language, dále IL), tedy provedením mezikroku v procesu transformace zdrojového kódu do strojového kódu, je získána reprezentace programu, jenž běží na každém zařízení s implementovaným JVM. V rámci JVM dochází k finálním krokům mezi které patří interpretace (JIT kompilace) bytecode do nativního kódu pro cílovou architekturu systému. 

Microsoft v reakci na JVM vydal v roce 2000 první .NET Framework, který umožňoval spouštět kód v jazyce C\# na operačním systému Windows. Cílem prvních verzí .NET Framework nebylo primárně umožnit vývoj pro různé zařízení a operační systémy, ale zprostředkovat lepší nástroje pro vývoj aplikací. V roce 2014 byla vydána první multiplatformní verze platformy .NET. Ta nese název .NET Core a umožňovala spouštět kód v jazyce C\# na operačních systémech Windows, Linux a macOS. 

TODO: Cite

\n{2}{Architektura}

Platforma .NET je postavena na několika klíčových komponentách, které zajišťují běh aplikací a poskytují nástroje pro vývoj aplikací. \cite{netdocs} Mezi nejdůležitější komponenty patří:

\begin{itemize}
    \item \textbf{Common Language Runtime} - CLR je základním kamenem .NET a poskytuje běhové prostředí pro spouštění aplikací na platformě. Překládá IL do nativního kódu, spravuje alokaci paměti a garbage collection (dále GC), zajišťuje zpracování výjimek (exceptions). CLR také kontroluje datové typy, interoperabilitu a zprostředkovává služby nezbytné pro spouštění nejrůznějších aplikací .NET.
    \item \textbf{.NET CLI} - Všestranný nástroj pro vývoj, kompilaci a nasazení aplikací .NET prostřednictvím rozhraní příkazové řádky. Podporuje širokou škálu operací, od vytváření projektů a správy závislostí až po testování a publikování aplikací. Prostředí .NET CLI je multiplatformní a umožňuje sjednocení rozhraní uživatelských nástrojů pro vývoj aplikací .NET.
    \item \textbf{Microsoft Build} - Microsoft Build (dále MSBuild) je engine používaný v platformě .NET, který umožňuje sestavovat aplikace a vytvářet balíčky pro nasazení. Tento nástroj používá k organizaci a řízení procesu sestavení projektový soubor \emph{csproj} na bázi Extensible Markup Language (dále XML). Tím je zajištěna kontrola nad kompilací a průběhem sestavení. V rámci procesu sestavení lze doplnit vlastní úlohy a cíle kompilace, což poskytuje flexibilitu sestavení pro komplexní procesy ve velkých projektech.
    \item \textbf{nástroje .NET Software Development Kit} (dále SDK) - Soubor nástrojů a knihoven podporujících vývoj, debugging a testování aplikací .NET. Zahrnují různé CLI a GUI nástroje, které pomáhají vývojářům spravovat práci s kódem, optimalizovat výkon a zajistit kvalitní výstup programu v platformě .NET.
    \item \textbf{Roslyn} - Roslyn je sada kompilátorů platformy .NET, která poskytuje bohaté Application Programming Interface (dále API) pro analýzu kódu. Umožňuje vývojářům používat implementace kompilátorů jazyka C\# a VB.NET jako služby. Roslyn zlepšuje výkonnost vývojářů poskytnutím funkcí jako je refaktoring, generování kódu a sémantická analýza.
    \item \textbf{NuGet} - Správce balíčků pro platformu .NET. dodává standardizovanou metodu správy externích knihoven, na nichž závisí aplikace v .NET. Zjednodušuje proces inkorporace knihoven, systémových i třetích stran, do projektu. Rovněž spravuje závislosti, čímž zajišťuje, že projekty zůstávají aktuální a kompatibilní. Tento nástroj je téměř nezbytný pro vývoj na platformě .NET, neboť umožňuje modulární vývoj softwaru.
\end{itemize}


\n{2}{Frameworky a technologie}

Platforma .NET poskytuje mnoho frameworků a technologií pro vývoj aplikací. Jednotlivé frameworky plní různé role a poskytují různé úrovně funkcionality pro vývoj aplikací. Z hlediska struktury a účelu je lze kategorizovat následujícím způsobem. \cite{netdocs}

\begin{itemize}
    \item \textbf{.NET} - Hlavní framework platformy. .NET je robustní framework pro vývoj softwaru. Podporuje tvorbu a provoz moderních aplikací a služeb. Původně známý jako .NET Framework a primárně zaměřen na prostředí Windows, s příchodem .NET Core a novějších verzí se vyvinul v modulární platformu s open-source zdrojovým kódem známou jednoduše jako .NET. Umožňuje vývojářům vytvářet aplikace, které jsou škálovatelné, výkonné a multiplatformní.
    \item \textbf{ASP.NET} - Robustní framework pro vytváření webových aplikací a služeb. Je součástí ekosystému .NET navržený tak, aby umožňoval vývoj vysoce výkonných, dynamických webových stránek, API a webových aplikací v reálném čase. ASP.NET podporuje jak webové formuláře, tak architekturu Model-View-Controller (dále MVC). S uvedením ASP.NET Core byl framework přepracován pro cloudovou nasazením škálevatolnost, vývoj napříč platformami a vysoký výkon. Poskytuje komplexní základ pro vývoj moderních webových aplikací, které lze spustit jak na Linuxu, Windows tak macOS. ASP.NET Core také představuje Blazor, který umožňuje vývojářům používat C\# při vývoji webu, což dále zvyšuje všestrannost ekosystému. Vývojářům, kteří chtějí využít .NET pro vývoj webu, poskytuje ASP.NET komplexní a flexibilní sadu nástrojů pro vytváření všech řešení, od malých webů až po složité webové platformy.
    \item \textbf{MAUI} - Moderní specializovaný framework pro vývoj aplikací napříč platformami v rámci ekosystému .NET. Umožňuje vývojářům vytvářet aplikace pro Android, iOS, macOS a Windows z jedné kódové základny. Zakládá na populárních konceptech z Xamarin.Forms a zároveň rozšiřuje jeho možnosti na desktopové aplikace. .NET MAUI zjednodušuje vývojový proces tím, že poskytuje jednotnou sadu nástrojů pro vývoj uživatelského rozhraní na všech platformách s možností přístupu k funkcím specifickým pro platformu v případě potřeby. Podporuje moderní vývojové vzory a nástroje, včetně MVVM, datové vazby a asynchronního programování, což usnadňuje vytváření sofistikovaných a citlivých aplikací. Předchůdcem MAUI je platforma Xamarin, která sloužila pro vytváření mobilních aplikací na platformě .NET. \cite{Libery2023}
    \item \textbf{Blazor} - Specializovaný framework v rámci ekosystému .NET, který zprostředkovává vývojářům tvorbu interaktivních webových uživatelských rozhraní pomocí C\# namísto JavaScriptu. Blazor může běžet na serveru (Blazor Server), kde zpracovává požadavky a komunikuje s uživatelským rozhraním pomocí knihovny SignalR, která zabezpečuje websocket komunikaci. Nebo také v prohlížeči skrz WASM, kdy dochází k přeložení C\# kódu do nativního kódu WASM a je spouštěn přímo ve webovém prohlížeči vedle tradičních webových technologií, jako jsou HTML a CSS. Umožňuje vývojářům využít znalosti .NET pro komplexní vývoj webových aplikací a vytvářet bohaté webové aplikace běžící na straně klienta v prohlížeči. Architektura Blazor je založená na komponentách a usnadňuje jejich opětovné použití pro tvorbu uživatelského rozhraní. Zároveň Blazor podporuje modulární vývojový přístup a poskytuje možnost vyvíjet webové aplikace v ekosystému .NET.
\end{itemize}

\n{3}{Knihovny}

Knihovny představují soubor funkcí a tříd, které mohou být použity při vývoji ve více aplikacích. Typicky představují logicky oddělenou a obecnou část funkcionality aplikace. Umožňují distribuovat běžnou funkcionalitu napříč různými projekty. Knihovny v .NET mohou být tvořeny binárnimi Dynamic Link Library (DLL) soubory nebo organizované jako samostatný projekt v rámci kontejneru projektů zvaném solution. K distribuci knihoven obecně dochází pomocí balíčků NuGet. Pomocí stejnojmenného nástroje jsou knihovny zabaleny a sdíleny přes internetová úložiště. 

Běžnou praxí tvůrce platformy, programovacího jazyka nebo frameworku je poskytnutí sad knihoven, které usnadňují vývoj aplikací. Zároveň tyto knihovny zpravidla implementují nejběžnější funkcionality, které mohou programátoři vyžadovat. Typicky se jedná o přístup k souborovému systému, síti, databázím, grafickému rozhraní a další. \cite{Price2023}

Následující seznam obsahuje některé z nejběžněji používaných knihoven v .NET:

\begin{itemize}
    \item \textbf{System} - Poskytuje základní třídy, typy a rozhraní, které umožňují a podporují širokou škálu operací na úrovni systému, jako jsou vstupy a výstupy (IO), vlákna, kolekce, diagnostika a další. Je nezbytná prakticky pro každou aplikaci .NET.
    \item \textbf{System.IO} - Dodáva funkcionalitu čtení z datových proudů, souborů a zápis do nich a práci se souborovým systémem.
    \item \textbf{System.Net} - Obsahuje třídy a abstrakce pro síťovou komunikaci a elektronickou poštu.
    \item \textbf{System.Data} - Zprostředkovává přístup k datovým zdrojům, jako jsou databáze nebo XML soubory, a obsahuje ADO.NET pro přístup k vybraným databázovým serverům.
    \item \textbf{System.Collections} - Rozhraní a třídy, které definují různé kolekce objektů, jako jsou seznamy, fronty, bitová pole, hašovací tabulky a slovníky.
    \item \textbf{System.Linq} - Zaštiťuje dotazování nad kolekcemi objektů pomocí Language Integrated Query (dále LINQ).
    \item \textbf{System.Threading} - Umožňuje správu vláken, synchronizační primitiva nebo například thread pool. Podporuje vývoj paralelizovaných aplikacích.
    \item \textbf{System.Security} - Spravuje ověřování, autorizaci a šifrování, a je základem pro vývoj bezpečných aplikací.
    \item \textbf{Entity Framework} - Object-Relational Mapping (dále ORM) framework, který umožňuje vývojářům pracovat s databázemi pomocí objektově orientovaného přístupu. Poskytuje abstrakci nad databázovými systémy a umožňuje vývojářům pracovat s daty pomocí objektů a tříd. \cite{netdocs}
\end{itemize}

Kromě knihoven poskytovaných společností Microsoft existuje mnoho knihoven třetích stran. Za vývojem těchto knihoven mohou stát vývojařské komunity nebo být vydány velkými společnostmi. Běžně tyto knihovny navazují na sadu funkcí poskytovaných Microsoftem a rozšiřují je o další novou funkcionalitu, nebo portují známé existující projekty do platformy .NET. Mezi příklady nejznámějších knihovny třetích stran v .NET patří Dapper, AutoMapper, Newtonsoft.Json a další.

\n{2}{Nástroje .NET}

Platforma .NET zprostředkovává širokou sadu nástrojů za účelem tvorby, sestavení a spuštění aplikace. Mezi nejdůležitější lze zařadit následující:

\n{3}{IDE}

Neméně důležitým prvkem vývoje aplikací je integrované vývojové prostředí (dále IDE). I když není povinné, pro spoustu vývojářů je jeho použití neodmyslitené. IDE je nástroj, který zprostředkovává vývoj aplikací, správu projektů, debuggování a další. IDE poskytuje uživatelské rozhraní, které umožňuje vývojářům vytvářet, upravovat a testovat kód. Zprostředkovává nástroje pro správu projektů, jako jsou sestavení, testování a publikace. Umožňuje provádět různorodé operace nad aplikací, jako je refaktorování kódu, hledání chyb a ladění.

Jedním z nejpoužívanějších IDE je Visual Studio, vyvíjené společností Microsoft. Visual Studio poskytuje prvotřídní podporu pro vývoj na platformě .NET. Mezi další populární IDE patří Visual Studio Code a JetBrains Rider.

\n{3}{Balíčky}

Pro jednoduchou distribuci knihoven, jak systémových tak třetích stran, je využíván nástroj pro správu balíčků NuGet. Projekt, jenž má být distribuován je buďto opatřen atributem \emph{<PackageOnBuild>} a sestaven nebo je využito příkazu \emph{dotnet pack}.

Takto vytvořené balíčky lze distribuovat např. přes NuGet.org, což je veřejný repozitář knihoven, který je dostupný pro všechny vývojáře. Možná je také implementace vlastních řešení. Distribuované knihovny jsou jednoduše importovatelné do projektu a umožňují snadnou správu závislostí.

\n{3}{Dokumentace}

Dokumentace je důležitou součástí vývoje aplikací. Poskytuje informace o tom, jak používat nástroje a technologie, které jsou součástí platformy .NET. Dokumentace obsahuje informace o API, knihovnách, nástrojích a dalších součástech platformy .NET. Dokumentace je dostupná online na oficiálních webových stránkách platformy .NET a obsahuje podrobné informace o mnoha aspektech vývoje aplikací. K nalezení je na webu \url{https://docs.microsoft.com/en-us/dotnet/}.

\n{3}{Jazyky}

Základem aplikace je zdrojový kód, který je v případě platformy .NET reprezentován nejčastěji jedním z podporovaných jazyků, Mezi nejčastěji využívané patří VisualBasic.NET (dále VB.NET), C\# a F\#.

\begin{itemize}
    \item \textbf{C\#} - Představuje všestranný, objektově orientovaný jazyk navržený tak, aby umožnil vývojářům vytvářet širokou škálu bezpečných a robustních aplikací, které běží na platformě .NET. Kombinuje sílu a flexibilitu C++ s jednoduchostí jazyka Visual Basic.
    \item \textbf{F\#} - Funkční jazyk, který také podporuje imperativní a objektově orientované programování. Primárně je vhodný pro vědecké aplikace a aplikace náročné na data. Zakládá na silném typování, umožňuje stručný, robustní a výkonný kód.
    \item \textbf{VB.NET} - Historický programovací jazyk vyvinutý společností Microsoft. VB, představený v roce 1991, byl navržen jako uživatelsky přívětivé programovací prostředí založené na jazyce BASIC; jeho drag-and-drop rozhraní umožňovalo snadné vytváření GUI. Tento přístup zpřístupnil programování širšímu okruhu uživatelů a kladl důraz na rychlý vývoj aplikací (RAD). VB.NET je moderní verze jazyka Visual Basic, která je implementována na platformě .NET.
\end{itemize}


\n{2}{Aplikační struktura}

Základním stavebním prvkem aplikace v .NET je soubor projektu. Jedná se o soubor na bázi XML disponující příponou \emph{.csproj}. V rámci něj dochází ke konfiguraci a deklaraci, jak bude .NET CLI, respektive nástroje sestavení, s projektem pracovat. Zároveň jsou zde definované závislosti na další projekty a knihovny. Mezi základní charakteristiky běžně určené v projektovém souboru patří verze .NET, verze projektu/assembly, seznam závislostí, konfigurace pro buildování, testování a publikaci.

Pro tvorbu složitějších aplikací je možné využít více projektových souborů. Tyto souboury jsou seskupeny pomocí speciálního solution souboru. Jedná se o soubor sloužící ke kontejnerizaci a provázání veškerých projektových a pomocných souborů, jako jsou konfigurace setavení, pomocné skripty a další. Disponuje příponou \emph{.sln}. \cite{Price2023}

Mezi další běžně používané známé konfigurační soubory patří následující:

\begin{itemize}
    \item \textbf{appsettings.json} - obsahuje nastavení aplikace
    \item \textbf{launchsettings.json} - deklaruje konfiguraci pro spuštění aplikace
    \item \textbf{Directory.Build.props} - zprostředkovává globální nastavení atributů pro všechny projekty v solution
    \item \textbf{Directory.Build.targets} - obsahuje globální nastavení cílů kompilace pro všechny projekty v solution
    \item \textbf{NuGet.config} - nastavení pro balíčkovací správce NuGet
\end{itemize}

Dalším příkladem projektového souboru je \emph{.fsproj} pro F\# projekty, \emph{.vbproj} pro Visual Basic projekty a \emph{.nuspec} pro balíčkovací soubory NuGet. Speciálně pro IDE Visual Studio je využíván soubor \emph{.dcsproj}, který obsahuje nastavení pro spustění a debuggování aplikace spuštěné v Docker kontejneru. 

\n{2}{Kompilace zdrojového kódu}

Kompilace je proces transformace zdrojového kódu do jiné podoby. Kód je zpravidla kompilován do podoby bližší cílové architektuře, ať je touto architekturou OS, případně konkrétní HW, nebo runtime prostředí (virtuální stroj). V rámci platformy .NET jsou k dispozici dva hlavní režimy kompilace zdrojového kódu: kompilace pro běhové prostředí (CLR) a kompilace do nativního kódu přímo pro cílovou architekturu (Native AOT). \cite{Price2023}

\n{3}{Cíle kompilace}

Cílem kompilace je převést zdrojový kód do podoby, kterou je možné spustit na cílovém zařízení. Platforma .NET podporuje zacílit na několik cílových zařízení, jako jsou desktopové počítače, mobilní zařízení nebo cloudové služby. Mezi podporované cíle patří:

\begin{itemize}
    \item \textbf{Desktopové počítače} - Zacílení na platformy PC probíhá několika způsoby. Aplikace obvykle běží na CLR, kde je kód kompilován do jazyka IL a poté spouštěn prostřednictvím .NET runtime, přičemž je za běhu převeden na nativní kód. Pro situace, kdy .NET runtime není nebo nemůže být přítomen lze využít kompilace AOT pro vygenerování nativního kódu.
    \item \textbf{Mobilní zařízení} - Pro mobilní vývoj poskytuje .NET MAUI, sadu nástrojů, které umožňují vývojářům psát nativní aplikace pro Android, iOS a Windows. Využívá se zde stejné sdílené kódové základny .NET. Umožňuje vývojářům používat knihovny C\# a .NET k vytváření mobilních aplikací. Tyto aplikace jsou kompilovány specificky pro každou platformu a mohou využívat nativní funkce zařízení.
\end{itemize}

\n{3}{Obecné postupy kompilace}

Proces kompilace zahrnuje několik charakteristických postupů, které jsou přizpůsobeny optimalizaci výkonu aplikací při vývoji i za běhu. Jejich cílem je zvýšení výkonu a zabezpečení aplikací, ale také zajištění větší kompatibility a efektivity programu napříč platformami. Mezi tyto postupy kompilace patří:

\begin{itemize}
    \item \textbf{Linkování} - Kompilátor během procesu sestavení aplikace propojuje množství zdrojových souborů a dll, aby vytvořily jeden spustitelný soubor nebo knihovnu. To zahrnuje řešení odkazů mezi různými dll a integraci všech požadovaných zdrojů.
    \item \textbf{Optimalizace} - Různé optimalizace IL a nativního kódu pro zlepšení výkonu. Tyto optimalizace probíhají jak během kompilace IL, tak během kompilace JIT nebo AOT do nativního kódu.
    \item \textbf{Tree shaking a trimming} - V moderních aplikacích .NET odstraňují nástroje, jako je IL Linker, během procesu sestavování nepoužívaný kód z konečné sestavy. Toto \emph{protřepávání stromu} snižuje velikost aplikace tím, že vylučuje nepotřebné knihovny a cesty ke kódu.
    \item \textbf{Vytváření metadat a manifestů} - Překladače .NET také vytvářejí metadata a soubory manifestů, které popisují obsah a závislosti sestav, což má zásadní význam pro verzování, zabezpečení a rozlišení sestav za běhu.
\end{itemize}

\n{3}{Kompilace pro CLR}

Standartním výstupem sestavení aplikace v platformě .NET je transformace zdrojového kódu z vybraného podporovaného jazyka do assembly v jazyce IL. Tento výstupní IL se v .NET konkrétně navývá Common Intermediate language (CIL) nebo také Microsoft Intermediate Language (MSIL). IL je jazyk nezávislý na platformě, který je následně kompilován do nativního kódu za běhu aplikace.

CLR je zodpovědný za interpretaci IL kódu a jeho kompilaci do nativního kódu. Kompilace do nativního kódu je prováděna v rámci běhu aplikace, což zajišťuje, že kód je optimalizován pro konkrétní architekturu systému. V případě jazyka C\# na platformě Windows slouží ke kompilaci spustitelný soubor \emph{csc.exe}. 

Výstupem kompilace pro CLR je assembly s popisnými metadaty a IL (v případě režimu Ready to Run i částečně nativním) kódem. Assembly typicky disponují příponou \emph{.dll}, případně jsou zabaleny do spustilného souboru dle cílové platformy a výstupu. Takovýto výstup je následně připraven buďto ke spuštění za pomocí CLR, případně pro využití a referenci při tvorbě dalšího .NET kódu. Kód IL je sada instrukcí nezávislá na procesoru, kterou může spustit běhové prostředí .NET (CLR). \cite{Richter2012}

Speciálním případem JIT kompilace je aplikace R2R. Zdrojový kód je při sestavení zkompilován do podoby nativního kódu pomocí nástroje crossgen, čímž vzniknou sestavy Ready To Run (dále R2R). Za běhu se sestavy R2R načtou a spustí s minimální kompilací JIT, protože většina kódu je již v nativní podobě. CLR může přesto JIT kompilovat některé části kódu, které nelze staticky zkompilovat předem. Využití je v aplikacích, které potřebují zkrátit dobu spouštění, ale zachovat určitou funkcionalitu nebo úroveň optimalizace poskytovanou JIT kompilací.

\n{3}{Kompilace do nativního kódu}

Přímá nativní AOT kompilace je proces, při kterém je kód kompilován do nativního kódu cílové architektury. Děje se tak v procesu sestavení programu ze zdrojového kódu. V případě platformy .NET je tato funkcionalita dostupná při použití jazyka C\# a speciálních projektových atributů. 

Jedná se o funkcionalitu, jenž prošla několika iteracemi. První možnosti sestavení aplikace v nativním kódu na .NET platformě byly aplikace Universal Windows Platform. Jednalo se o aplikace využívající specifické rozhraní, nativní pro produkty Microsoft. S verzí frameworku .NET 7 byly rozšířeny možnosti sestavení aplikace jako do podoby nativního kódu i pro další architektury a typy aplikací. Tato nová funkcionalita získala vyráznější podporu v roce 2023 s vydáním frameworku .NET 8. Filozofie společnosti Microsoft ohledně AOT kompilace je, že vývojáři by měli mít možnost využít AOT kompilace v platformě .NET, pokud je to pro daný scénář vhodné. Scénáře kladoucí takovéto požadavky se vyskytují především v cloudovém nasazení, respektivně při implementaci cloudové infrastruky universálně s využitím platformy .NET. \cite{Pflug2023}

Výstupem nativní AOT kompilace je v rámci .NET spustitelný soubor. Soubor nabývá formátu dle cílové architektury, jenž byla definována v procesu sestavení. Takto vytvořený soubor je možné spustit přímo bez potřeby CLR.

\n{2}{Běh kódu}

Spuštění, respektive běh kódu na HW počítačového zařízení vyžaduje instrukční sadu, které daná architektura rozumí, tedy nativní kód. V případě nativní AOT kompilace v .NET tento kód získáme již při sestavení aplikace. Při využití kompilace do IL je nutné kód získat pomocí jednoho z kompilačních způsobů podporovaného CLR. Výsledná nativní reprezentace se v obou případech spouští zavoláním vstupní metody v binárním souboru dle specifikace architektury.

\n{3}{CLR}

 CLR je běhové prostředí frameworku .NET. Poskytuje spravované prostředí pro spouštění aplikací .NET. Podporuje více programovacích jazyků, včetně jazyků C\#, VB.NET a F\#, a umožňuje jejich bezproblémovou spolupráci. Spravuje paměť prostřednictvím automatického GC, který pomáhá předcházet únikům paměti a optimalizuje využití prostředků. CLR také zajišťuje typovou bezpečnost a ověřuje, zda jsou všechny operace typově bezpečné, aby se minimalizovaly chyby při programování.

CLR je zodpovědný za několik důležitých funkcí, které zvyšují produktivitu vývojářů a výkon aplikací.

\begin{itemize}
    \item \textbf{Správa paměti} - CLR spravuje alokaci a dealokalizaci paměti, čímž zajišťuje efektivní využití systémových prostředků a zabraňuje únikům paměti (memory leaks). Obsahuje GC, který automaticky přiděluje a sbírá paměť obsazenou objekty.
    \item \textbf{Bezpečnost} - CLR poskytuje komplexní model zabezpečení, který pomáhá chránit aplikace před neoprávněným přístupem, poškozením dat a dalšími bezpečnostními hrozbami. Vynucuje zásady zabezpečení, jako je zabezpečení přístupu ke kódu (Code Access Security, dále CAS) a zabezpečení založené na rolích. Zajišťuje, aby kód byl spouštěn s příslušnými oprávněními na základě svého původu a úrovně důvěryhodnosti, čímž chrání citlivá data a systémové prostředky.
    \item \textbf{Zpracování vyjímek} - Zpracování výjimek v CLR zahrnuje detekci, propagaci a zpracování chyb a stavů, které mohou nastat během provádění programu. Mechanismus vyjímek umožňuje elegantně řešit neočekávané situace a zachovat stabilitu aplikace.
    \item \textbf{Generování typů} - CLR podporuje dynamické generování typů za běhu, což aplikacím umožňuje za běhu dynamicky vytvářet a manipulovat typy dle potřeby. To dává možnost scénářům, jako je dynamické generování kódu, kompilace kódu za běhu a dynamické vytváření objektů.
    \item \textbf{Reflexe} - Reflexe umožňuje validaci a manipulaci typů, attributů a metadat načtených z dll za běhu. Umožňuje vývojářům dotazovat se a upravovat typy a jejich attributy za běhu, dynamicky volat metody a přistupovat k informacím o metadatech. Díky tomu mají aplikace .NET využívající běhové prostředí výrazné možnosti introspekce a přizpůsobení.
\end{itemize}

Klíčovými vlasnostmi CLR jsou multiplatformnost kódu, reflexe, optimalizace kódu pro konkrétní architekturu a bezpečnost. CLR nabízí mechanismy, jako je CAS, které zabraňují neoprávněným operacím. Kompilace JIT znamená, že kód zprostředkujícího jazyka je zkompilován do nativního kódu těsně před spuštěním, což zajišťuje optimální výkon na cílovém hardwaru. CLR usnadňuje zpracování chyb v různých jazycích a poskytuje konzistentní přístup k řešení výjimek. Navíc obsahuje nástroje pro ladění a profilování, které vývojářům pomáhají efektivně identifikovat a odstraňovat problémy s výkonem. Aby mohl být kód z IL reprezentace spuštěn na systému, respektive HW stroje, musí být dodatečně kompilován. Za tímto účelem existuje v CLR několik technik, které mají využití v specifických scénářích. \cite{Richter2012}

\n{3}{Nativní kód}

Běh nativního kódu je závislý na konkrétní architektuře systému, pro které jsou nativní programové soubory vytvořeny. Nepodléhá další úpravě ze strany .NET nástrojů. Spuštění probíhá nativním příkazem operačního systému, který zprostředkuje spuštění programu.

\n{2}{Tvorba programu v .NET}

Následující část popisuje obecnou koncepci a strukturu projektu aplikace v .NET. Součástí je postup pro tvorbu a vydání projektu. Blížší pozornost bude věnována tvorbě nativního AOT projektu.

\n{3}{Obecný postup}

Vytvoření aplikace v .NET sestává z několika kroků, které zahrnují nastavení vývojového prostředí, tvorbu projektu, programování, správu závislostí, kompilaci a publikaci. Postup je následující:

\begin{enumerate}
    \item \textbf{Nastavení vývojového prostředí}: Sestává z instalace sady nástrojů .NET SDK.
    
    \item \textbf{Vytvoření projektu}: Pomocí příkazu \texttt{dotnet new} nebo skrze GUI IDE je vytvořen nový projekt a solution soubor. Součástí je výběr typu projektu, jazyka, frameworku a dalších konfiguračních parametrů.
    
    \item \textbf{Programování}: Sestává z tvorby kódu aplikace, testování a ladění.

    \item \textbf{Správa závislostí}: Pomocí nástrojů .NET CLI je možno referencovat balíčky a knihovny v rámci projektu.
    
    \item \textbf{Kompilace}: Kompilace aplikace probíhá pomocí příkazu \texttt{dotnet build}, který převede vysokoúrovňový kód do IL. V případě AOT dochází k dodatečné kompilace do nativního kódu dle cílové architektury.
    
    \item \textbf{Publikování}: Použitím příkazu příkazu \texttt{dotnet publish} dochází k vydání aplikace, tedy specifickému sestavení v konfigurovaném nastavení. \cite{Price2023}
\end{enumerate}

\n{3}{Tvorba nativního programu}

Pro tvorbu nativního programu v .NET je nutné využít speciálního atributu \emph{PublishAoT} v projektovém souboru. Tento atribut je zodpovědný za konfiguraci projektu pro nativní AOT kompilaci. Při jeho použití je nutné specifikovat cílovou architekturu, pro kterou je nativní kód vytvářen. Po kompilaci kódu do IL dochází k dodatečné kompilaci do nativního kódu, která dodává další konzolový výstup s informacemi o průběhu kompilace. Výstupem je spustitelný soubor, který je možné spustit na cílovém zařízení bez potřeby CLR. Následující seznam obsahuje klíčové termíny pro tvorbu nativního AOT programu v .NET:

\begin{itemize}
    \item \textbf{EmitCompilerGeneratedFiles} - Pokud je v projektu zapnut atribut EmitCompilerGeneratedFiles, kompilátor generuje soubory, které obsahují podrobné informace o stavu zkompilované aplikace. Ty zahrnují i meziprodukty nebo výpisy strojového kódu, které jsou cenné pro procest ladění a analýzy výstupního produktu. Pomáhají při zacílení na kompilaci do nativního AOT lépe pochopit, jak je vysokoúrovňový kód překládán do strojového kódu.
    \item \textbf{Deklarace unmanaged rozhraní} - Deklarace unmanaged nebo také nespravovaného rozhraní zahrnuje definování způsobů jakým jednotlivá rozhraní komunikují. Spravované rozhraní představuje zdrojový kód vytvářené aplikace .NET. Nespravované rozhraní jsou funkce operující mimo .NET aplikaci, například v C/C++ knihovně. Pomocí deklarací je specifikován způsob volání. U nativních AOT aplikací je důležité, aby tato rozhraní byla přesně definována a dodržována, protože jakýkoli nesoulad nebo chyba v deklaraci může vést k chybám za běhu, které se hůře diagnostikují a opravují kvůli absenci runtime prostředí a dynamickým funkcím.
    \item \textbf{Trimming} - Nebo také ořezání je proces odstranění nepotřebného kódu a zdrojů z aplikace během sestavení, za účelem snížení velikosti a zvýšení výkonu. V rámci platformy .NET představuje trimming jednu z technik postupu \emph{tree shaking}, kdy překladač analyzuje, které části zdrojového kódu jsou skutečně používány, a zbytek vyloučí z výstupního produktu. To je zvláště důležité pro aplikace, kde je úložiště nebo paměť omezená. Ořezávání pomáhá při optimalizaci a zajišťuje, aby aplikace obsahovala pouze nezbytné části a závislosti. \cite{netdocs}
\end{itemize}

\n{3}{Přehled podpory}

Následující přehled představuje rozsah funkcionality implementované v rámci .NET frameworku k datu vydání verze 8.0.0. \cite{aspnetdocs}

\begin{itemize}
    \item \textbf{REST minimal API} - Tvorba minimalistých služeb implementujících REST API.
    \item \textbf{gRPC API} - Komunikace mezi službami pomocí protokolu gRPC.
    \item \textbf{JSON Web Token Authentication} - Autentizace a autorizace za pomocí JSON Web Token (dále JWT) tokenů.
    \item \textbf{CORS} - Konfigurace Cross-Origin Resource Sharing.
    \item \textbf{HealthChecks} - Monitorování stavu aplikace.
    \item \textbf{HttpLogging} - Logování HTTP požadavků.
    \item \textbf{Localization} - Lokalizace aplikace.
    \item \textbf{OutputCaching} - Cachování výstupu.
    \item \textbf{RateLimiting} - Omezení počtu požadavků.
    \item \textbf{RequestDecompression} - Dekomprese požadavků.
    \item \textbf{ResponseCaching} - Cachování odpovědí.
    \item \textbf{ResponseCompression} - Komprese odpovědí.
    \item \textbf{Rewrite} - Přepisování adresy Uniform Resource Locator (dále URL).
    \item \textbf{StaticFiles} - Poskytování statických souborů.
    \item \textbf{WebSockets} - Komunikace pomocí WebSockets.
\end{itemize}
