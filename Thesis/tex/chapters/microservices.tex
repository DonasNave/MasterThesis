
%%%%%%%%%%%%%%%%%%%%%%%%%%%%%%%%%%%%%%%%%%%%%%%%%%%%%%%%%%%%%%%%%%%%%%%%%%%%%%%%%
%                                 MICROSERVICE                                  %
%%%%%%%%%%%%%%%%%%%%%%%%%%%%%%%%%%%%%%%%%%%%%%%%%%%%%%%%%%%%%%%%%%%%%%%%%%%%%%%%%

\n{1}{Microservice architektura}

Při vývoji softwaru je možné aplikovat různé architektury a návrhové vzory. Mezi obecně populární a rozšířenou architekturu patří monolitická. V monolitické architektuře je celá aplikace rozdělena do několika vrstev, které jsou využívány k oddělení logiky. Tato architektura je jednoduchá na vývoj a nasazení, ale může být obtížné škálovat a udržovat s rostoucí složitostí aplikace. Monolitická architektura je založena na principu, že celá aplikace je spuštěna jako jediný proces a sdílí stejný kód a zdroje. To může způsobit problémy s výkonem, škálovatelností a odolností aplikace.

Principiálně opačná je architektura microservice. Ta je založena na principu oddělení aplikace do několika samostatných služeb. Každá z těchto služeb je zodpovědná za určitou část funkcionality aplikace. Služby jsou navzájem nezávislé a komunikují mezi sebou pomocí definovaných rozhraní. Tím je zajištěno, že každá služba může být vyvíjena, nasazována a škálována nezávisle na ostatních. Tato architektura umožňuje vývojářům pracovat na menších a jednodušších částech aplikace, což zvyšuje produktivitu a umožňuje rychlejší iterace. Díky nezávislosti služeb je také možné dosáhnout vyšší odolnosti a škálovatelnosti aplikace. \cite{Martin2018}

\n{2}{Historie}

Původ microservice architektury nelze přesně definovat, důležitý moment však nastal v roce 2011, kdy Martin Fowler publikoval článek \textit{Microservices} na svém blogu. V tomto článku popsal výhody a nevýhody této architektury a zároveň popsal způsob, jakým je možné tuto architekturu využít. Dalším popularizačním momentem pro popularizaci bylo vydání knihy \textit{Building Microservices} od Sama Newmana v roce 2015. Tato kniha popisuje způsob, jakým je možné využít microservice architekturu v praxi.

Opravdový přelom přišel postupně, nástupem a popularizací virtualizace a kontejnerizace v průběhu let 2013 až 2015. Tímto bylo umožněno vytvářet a spouštět mikroslužby v izolovaných prostředích. Tímto bylo umožněno vytvářet mikroslužby, které jsou nezávislé na operačním systému a hardwaru, na kterém jsou spouštěny. Nejdůležitější v tomto ohledu je nepochybně projekt Docker, který byl vydán v roce 2013. Díky Dockeru bylo možno jednoduše definovat, vytvářet a spouštět kontejnerizované aplikace.

\n{2}{Základní principy}

V oblasti architektury mikroslužeb existuje několik základních principů, které tento přístup odlišují od tradičnějších softwarových architektur. Tyto principy nejsou jen teoretické, mají přímý dopad na to, jak jsou služby vyvíjeny, nasazovány a udržovány. Jejich využití přispívá k tvorbě vysoce flexibilní, škálovatelné a odolné architektuře.

\n{3}{Decentralizace}

Jedním z hlavních principů microservice architektury je decentralizace. To znamená, že každá mikroslužba je zodpovědná za určitou část funkcionality aplikace. Mikroslužby jsou navzájem nezávislé a komunikují mezi sebou pomocí definovaných rozhraní. Tím je zajištěno, že každá mikroslužba může být vyvíjena, nasazována a škálována nezávisle na ostatních.

\n{3}{Virtualizace a kontejnerizace}

Virtualizace a kontejnerizace jsou klíčové technologie, které umožňují architekturu mikroslužeb. Virtualizace umožňuje provozovat více operačních systémů na jednom fyzickém hardwarovém hostiteli, čímž se snižuje počet potřebných fyzických strojů a zvyšuje efektivita využití zdrojů. Kontejnerizace jde ještě o krok dále tím, že zabalí aplikaci a její závislosti do kontejneru, který může běžet na libovolném serveru Linux nebo Windows. Tím je zajištěno, že aplikace funguje jednotně i přes rozdíly v prostředí nasazení.

Kontejnerizace je obzvláště důležitá pro mikroslužby, protože zapouzdřuje každou mikroslužbu do vlastního kontejneru, což usnadňuje její nasazení, škálování a správu nezávisle na ostatních. Synonymem kontejnerizace je nástroj Docker, který nabízí ekosystém pro vývoj, správu a provoz kontejnerových aplikací.

\n{3}{Orchestrace}

S rozšiřováním mikroslužeb a kontejnerů se jejich správa stává složitou. Nástroje pro orchestraci pomáhají automatizovat nasazení, škálování a správu kontejnerů. Mezi oblíbené orchestrační nástroje patří Kubernetes, Docker Swarm a Mesos. Zejména Kubernetes se stal de facto standardem, který poskytuje robustní rámec pro nasazení, škálování a provoz kontejnerových aplikací v clusteru strojů. Řeší vyhledávání služeb, vyvažování zátěže, sledování přidělování prostředků a škálování na základě výkonu pracovní zátěže.

\n{3}{Odolnost}

Robustnosti mikroslužeb je dosaženo pomocí strategií, jako jsou přerušení, záložní řešení a opakování, které pomáhají zabránit tomu, aby se selhání jedné služby kaskádově přeneslo na ostatní. Izolace služeb také znamená, že problémy lze omezit a vyřešit s minimálním dopadem na celý systém. Kromě toho jsou kontroly stavu a monitorování nezbytné pro včasné odhalení a řešení problémů.

\n{3}{Škálování}

Architektura mikroslužeb zvyšuje škálovatelnost. Služby lze škálovat nezávisle, což umožňuje efektivnější využití zdrojů a zlepšuje schopnost systému zvládat velké objemy požadavků. Běžně se používá horizontální škálování (přidávání dalších instancí služby), které usnadňují nástroje pro kontejnerizaci a orchestraci.

\n{2}{Komponenty}

Architektura mikroslužeb rozkládá aplikace do menších, oddělených služeb, z nichž každá plní samostatnou funkci. Pro efektivní správu těchto služeb, zejména v distribuovaném prostředí, se používá několik základních komponent. Tato část se zabývá klíčovými architektonickými komponentami, které usnadňují robustní provoz, komunikaci a škálovatelnost mikroslužeb.

\n{3}{API Gateway}

API Gateway je služba, která slouží jako vstupní bod pro komunikaci s mikroslužbami. Zajišťuje směrování požadavků, autentizaci, autorizaci, zabezpečení a další funkce, které jsou společné pro všechny služby. API Gateway může také poskytovat další funkce, jako jsou cachování, transformace zpráv a řízení toku dat. Tím zjednodušuje a centralizuje správu komunikace mezi klienty a mikroslužbami.

\n{3}{Service Discovery}

Service Discovery je mechanismus, který umožňuje mikroslužbám dynamicky najít a komunikovat s ostatními službami v systému. To je důležité pro dynamické škálování, nasazování a správu služeb. Service Discovery může být implementován pomocí centrálního registru služeb nebo distribuovaného protokolu, jako je DNS nebo Consul.

\n{3}{Load Balancer}

Load Balancer je služba, která rozděluje provoz mezi několik instancí služby, aby se zajistila rovnoměrná zátěž a zvýšila odolnost proti chybám. Load Balancer může být implementován jako hardwareové zařízení nebo softwarová služba, která poskytuje rozhraní pro konfiguraci a správu zátěže.


\n{3}{Komunikační systémy}

Mikroslužby spolu komunikují prostřednictvím rozhraní API, obvykle prostřednictvím protokolů HTTP/HTTPS, i když pro aplikace citlivější na výkon lze použít i jiné protokoly, například gRPC. Komunikační vzory zahrnují synchronní požadavky (např. RESTful API) a asynchronní zasílání zpráv (např. pomocí brokerů zpráv jako RabbitMQ nebo Kafka). Tím je zajištěno volné propojení mezi službami, což umožňuje jejich nezávislý vývoj a nasazení.

\begin{itemize}
    \item \textbf{REST API} - Představuje vysoce rozšířenou možnost komunikace mezi mikroslužbami. Využívají se při ní standardní metody HTTP, jako jsou GET, POST, PUT a DELETE, k provádění operací na rozhraní identifikovaným prostřednictvím adresy URL. Díky bezstavové povaze je rozhraní REST vysoce škálovatelné a vhodné pro veřejně přístupné služby. Má širokou podporu na různých platformách a v různých jazycích, což pomáhá zajistit interoperabilitu v rozmanitém ekosystému mikroslužeb.
    \item \textbf{RPC} - Komunikační metoda používaná v distribuovaných systémech, včetně mikroslužeb, kdy program způsobí, že se procedura spustí v jiném adresním prostoru (obvykle na jiné virtualizované ve sdílené síti). Tato technika abstrahuje složitost síťové komunikace do jednoduchosti volání lokální funkce nebo metody. Mezi běžné implementace RPC patří gRPC, Thrift a Apache Avro.
    \item \textbf{Message Broker} - Jedná se o komunikační vzor kdy broker - prostředník, spravuje asynchronní komunikaci mezi mikroslužbami pomocí front zpráv. Tato metoda odděluje mikroslužby tím, že jim umožňuje publikovat zprávy do fronty, aniž by znaly podrobnosti o tom, které služby je budou spotřebovávat. Mezi běžné zprostředkovatele zpráv patří RabbitMQ, Apache Kafka a AWS SQS. Tato komunikační architektura zvyšuje odolnost proti chybám, protože zprostředkovatel zpráv může zajistit, že zprávy nebudou ztraceny při přenosu, i když je spotřebitelská služba dočasně nedostupná. 
\end{itemize}

\n{3}{Databáze}

V architekturách mikroslužeb si každá služba obvykle spravuje vlastní databázi, což je přístup, který podtrhuje princip decentralizované správy dat. Tato izolace pomáhá službám být volně provázané a nezávisle nasaditelné, přičemž každé databázové schéma je přizpůsobeno konkrétním potřebám služby. V závislosti na případu použití mohou služby používat různé typy databází - SQL pro transakční data vyžadující silnou konzistenci a vlastnosti ACID nebo NoSQL pro flexibilnější možnosti ukládání dat, které umožňují horizontální škálování a podporu velkých objemů strukturovaných, částečně strukturovaných nebo nestrukturovaných dat. Tato různorodost databázových technologií přináší výzvy, jako je udržování konzistence dat napříč službami, což se často řeší pomocí strategií, jako je vzor Saga nebo případná konzistence pro zajištění integrity dat napříč distribuovanými transakcemi.

\n{3}{Bezpečnost}

Bezpečnost v architektuře mikroslužeb je velmi důležitá, protože distribuovaná povaha těchto systémů přináší mnoho zranitelných míst. Bezpečnostní prvky se zaměřují na ochranu dat při přenosu a v klidovém stavu a zajišťují, že ke službám a datům mají přístup pouze oprávněné subjekty. Mezi klíčové strategie patří implementace bran API s vestavěnými bezpečnostními prvky, jako je ověřování, autorizace a ukončení SSL. Zásadní význam mají systémy správy identit a přístupu (IAM), často integrované s tokeny OAuth a JWT pro správu identit uživatelů a řízení přístupu na základě definovaných zásad. Zajištění šifrované komunikace mezi službami pomocí protokolů, jako je TLS, může navíc chránit před odposlechem a manipulací. Zásadní jsou také účinné strategie protokolování a monitorování, které poskytují možnost odhalovat bezpečnostní hrozby, reagovat na ně a zmírňovat je v reálném čase. Každá z těchto složek hraje klíčovou roli při vytváření bezpečného ekosystému mikroslužeb a umožňuje robustní obranné mechanismy proti interním i externím bezpečnostním rizikům.

\n{2}{Vývoj, testování a nasazení}

Mikroslužby umožňují agilní vývojové postupy. Týmy mohou vyvíjet, testovat a nasazovat služby nezávisle, což umožňuje rychlejší iteraci a zpětnou vazbu. Nedílnou součástí jsou pipelines pro kontinuální integraci a doručování (CI/CD), které umožňují automatizované testování a nasazení. Tento přístup podporuje kulturu DevOps a podporuje užší spolupráci mezi vývojovými a provozními týmy.

Testování mikroslužeb je klíčové pro zajištění kvality a spolehlivosti systému. Mikroslužby lze testovat na několika úrovních, včetně jednotkových testů, integračních testů a testů end-to-end. Jednotkové testy se zaměřují na testování jednotlivých komponent služby, zatímco integrační testy testují komunikaci mezi službami. Testy end-to-end testují celý systém z pohledu uživatele. Automatizované testování je klíčové pro rychlé a spolehlivé nasazení.

Integrace kontinuální integrace a kontinuálního nasazování (CI/CD) je zásadní pro správu mikroslužeb, jelikož podporuje rychlé iterace a minimalizuje rizika spojená s nasazováním změn. CI/CD automatizuje procesy sestavení, testování a nasazení softwaru, což zajišťuje, že nový kód prochází důkladným testováním a je pravidelně a bezpečně nasazován do produkce. V prostředí mikroslužeb umožňuje CI/CD týmům aktualizovat služby nezávisle na ostatních částech systému, což vede k vyšší agilitě a rychlejší reakci na požadavky trhu nebo na potřeby zákazníků.

\n{2}{Výhody a nevýhody}

Mezi hlavní výhody microservice architekture lze zařadit:

\begin{itemize}

\item \textbf{Přizpůsobitelnost} - Mikroslužby umožňují rychlé, časté a spolehlivé poskytování rozsáhlých a komplexních aplikací. Týmy mohou aktualizovat určité oblasti aplikace, aniž by to mělo dopad na celý systém, což umožňuje rychlejší iterace.

\item \textbf{Škálovatelnost} - Služby lze škálovat nezávisle, což umožňuje přesnější přidělování zdrojů na základě poptávky. To usnadňuje zvládání proměnlivého zatížení a může zlepšit celkovou efektivitu aplikace.

\item \textbf{Odolnost} - Decentralizovaná povaha mikroslužeb pomáhá izolovat selhání na jedinou službu nebo malou skupinu služeb, čímž zabraňuje selhání celé aplikace. Techniky, jako jsou jističe, zvyšují odolnost systému.

\item \textbf{Technologická rozmanitost} - Týmy si mohou vybrat nejlepší nástroj pro danou práci a podle potřeby používat různé programovací jazyky, databáze nebo jiné nástroje pro různé služby, což vede k potenciálně optimalizovanějším řešením.

\item \textbf{Flexibilita nasazení} - Mikroslužby lze nasazovat nezávisle, což je ideální pro kontinuální nasazení a integrační pracovní postupy. To také umožňuje průběžné aktualizace, modrozelené nasazení a kanárkové verze, což snižuje prostoje a rizika.

\item \textbf{Modularita} - Tato architektura zvyšuje modularitu, což usnadňuje pochopení, vývoj, testování a údržbu aplikací. Týmy se mohou zaměřit na konkrétní obchodní funkce, což zvyšuje produktivitu a kvalitu.

\end{itemize}

Zatímco mezi nevýhody patří:

\begin{itemize}

\item \textbf{Komplexnost} - Správa více služeb na rozdíl od monolitické aplikace přináší složitost při nasazování, monitorování a řízení komunikace mezi službami.

\item \textbf{Správa dat} - Konzistence dat mezi službami může být náročná, zejména pokud si každá mikroslužba spravuje vlastní databázi. Implementace transakcí napříč rozhraními vyžaduje pečlivou koordinaci.

\item \textbf{Zpoždění sítě} - Komunikace mezi službami po síti přináší zpoždění, které může ovlivnit výkonnost aplikace. Ke zmírnění tohoto jevu jsou nutné efektivní komunikační protokoly a vzory.

\item \textbf{Provozní režie} - S počtem služeb roste potřeba orchestrace, monitorování, protokolování a dalších provozních záležitostí. To vyžaduje další nástroje a odborné znalosti.

\item \textbf{Složitost vývoje a testování} - Mikroslužby sice zvyšují flexibilitu vývoje, ale také komplikují testování, zejména pokud jde o testování end-to-end, které zahrnuje více služeb.

\item \textbf{Integrace služeb} - Zajištění bezproblémové spolupráce služeb vyžaduje robustní správu API, řízení verzí a strategie zpětné kompatibility.

\end{itemize}

\n{2}{Nasazení založené na mikroslužbách}

Efektivní nasazení mikroslužeb je klíčové pro využití jejich potenciálních výhod, jako je škálovatelnost, flexibilita a odolnost. Tato část se zabývá různými strategiemi nasazení, které jsou pro mikroslužby obzvláště vhodné, zejména v prostředí cloud-native. Tyto strategie zajišťují, že mikroslužby lze efektivně spravovat a škálovat, dynamicky reagovat na změny zatížení a minimalizovat prostoje.

\n{3}{Strategie}

Existuje několik strategií nasazení, které jsou pro mikroslužby obzvláště vhodné:

\begin{itemize}
    \item \textbf{Jedna služba na hostitele} - Strategie zahrnuje nasazení každé mikroslužby na vlastní server, ať už virtuální, nebo fyzický. Tento přístup zjednodušuje ladění a izolaci služeb, ale může vést k nedostatečnému využití zdrojů a vyšším nákladům.
    \item \textbf{Více služeb na jednoho hostitele} - Nasazení více služeb na jednom hostiteli maximalizuje využití zdrojů a snižuje náklady. Vyžaduje však pečlivou správu, aby nedocházelo ke konfliktům a aby se služby vzájemně nerušily.
    \item \textbf{Instance služby na kontejner} - Moderní nasazení mikroslužeb často používají kontejnery (například Docker) pro umístění jednotlivých služeb. Kontejnery poskytují odlehčené, konzistentní prostředí pro každou službu, zjednodušují nasazení a škálování v různých prostředích a zajišťují, že každá služba má splněny své závislosti bez konfliktů.
\end{itemize}

\n{3}{Cloud native nasazení}

Mikroslužby jsou obzvláště vhodné pro cloudová nativní prostředí, která podporují jejich dynamickou povahu:

\begin{itemize}
    \item \textbf{Kontejnery a orchestrace} - Nástroje jako Kubernetes orchestrují kontejnerové služby a řídí jejich životní cyklus od nasazení až po ukončení. Kubernetes se stará o škálování, vyrovnávání zátěže a obnovu, což usnadňuje vysokou dostupnost a efektivní využívání zdrojů.
    \item \textbf{Mikroslužby na platformě jako služba (PaaS)} - Nabídky PaaS, jako jsou AWS Elastic Beanstalk, Microsoft Azure App Service a Google App Engine, poskytují prostředí, kde lze mikroslužby snadno nasadit, škálovat a spravovat bez nutnosti starat se o základní infrastrukturu.
    \item \textbf{Serverless} - Bezserverové výpočetní modely umožňují nasazení mikroslužeb jako funkcí (FaaS), které se spouštějí v reakci na události, přičemž poskytovatel cloudu spravuje prostředí pro jejich provádění. Tento model je vysoce škálovatelný a nákladově efektivní, protože zdroje jsou spotřebovávány pouze během provádění funkcí.
\end{itemize}
