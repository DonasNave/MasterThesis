% ============================================================================ %
% Encoding: UTF-8 (žluťoučký kůň úpěl ďábelšké ódy)
% ============================================================================ %

% ============================================================================ %
\nn{Úvod}
První řádek prvního odstavce v kapitole či podkapitole se neodsazuje, ostatní ano. Vertikální odsazení mezy odstavci je typycké pro anglickou sazbu; czech babel toto respektuje, netřeba do textu přidávat jakékoliv explicitní formátování, viz ukázka sazby tohoto textu s následujícím odstavcem).

Formátování druhého odstavce. Text text text text text text text text text text text text.


% ============================================================================ %
\cast{Teoretická část}

\n{1}{Kompilace kódu v platformě dotnet}
Platforma dotnet od společnosti Microsoft představuje sadu nástrojů k vývoji aplikací v jazyce C\# a jeho derivátech. Tato platforma je multiplatformní a umožňuje vývoj aplikací pro operační systémy Windows, Linux a macOS. Vývojáři mohou využívat nástroje pro vývoj webových aplikací, desktopových aplikací, mobilních aplikací a dalších. Platforma dotnet je postavena na dvou hlavních principech. Prvním z nich je \textit{Common Language Runtime} (dále jen CLR), systémové prostředí zodpovídající za běh aplikací. Druhým principem je \textit{Common Language Infrastructure} (dále jen CLI), konzolový nástroj-rozhraní, zodpovědné za kompilaci a spouštění aplikací. \cite{dotnet}

Využití runtime prostředí má historický původ. V dřívějších dobách byly programátoři limitování možnostmi programovacích jazyků a nástrojů kompilujících kód do spustitelných binárních souborů. Ve snazu omezit tyto limity vzniklo několik projektů, které měly za cíl vytvořit prostředí, ve kterém by bylo možné spouštět kód v různých programovacích jazycích. Jedním z těchto projektů byl projekt \textit{Java Virtual Machine} (dále jen JVM), který vznikl v roce 1995. Díky tomy bylo umožňeno kompilovat kód v jazyce Java do univerzálního byte code, který je spustitelný na systémech s JVM. Zároveň tento proces tvorby a spouštění aplikací umožnil programátorům využít vyšší úroveň abstrakce a konceptů aplikační architektury.

Microsoft v reakci na JVM vydal v roce 2000 první .NET Framework, který umožňoval spouštět kód v jazyce C\# na operačním systému Windows. Cílem prvních verzí .NET Framework nebylo primárně umožnit vývoj pro různé zařízení a operační systémy, ale zprostředkovat lepší nástroje pro vývoj aplikací. Konečně, v roce 2014 se dostavila i multiplaformnost dotnetu. Byl vydán .NET Core, který umožňoval spouštět kód v jazyce C\# na operačních systémech Windows, Linux a macOS. \cite{dotnet}

\n{2}{JIT kompilace}
JIT kompilace je proces, při kterém je kód kompilován do určité univerzální podoby, jenž v době spuštění aplikace je předkládán v běhovém prostředí na strojový kód. V případě dotnetu je tímto jazykem IL (Intermediate language) a výstupem kompilace dotnet aplikace jsou soubory s příponou .dll (mohou být i jiné). Takto vytvořený dll soubor je možné referencovat z jiných .dll souborů nebo jej přímo spustit přes CLI příkazem dotnet, pokud obsahuje vstupní funkci. Po spuštění je obsah .dll souboru načten běhovým prostředím CLR a kompilován na strojový kód.\cite{jit}

\n{3}{Historie}
Text

\n{3}{CLR}

\n{3}{Výhody a nevýhody}

Mezi hlavní výhody se řadí zprostředkování následujícího:
\begin{itemize}
    \item  \textbf{Reflexe} - CLR umožňuje využívat reflexi, která umožňuje získat informace o kódu za běhu aplikace. Tímto je umožněno vytvářet aplikace, které jsou schopny měnit své chování za běhu.
    \item \textbf{Dynamické načítání} - CLR umožňuje dynamicky načítat knihovny za běhu aplikace. Tímto je umožněno vytvářet aplikace, které jsou schopny měnit své chování za běhu.
    \item \textbf{Větší bezpečnost} - CLR zajišťuje, že aplikace nemůže přistupovat k paměti, která jí nebyla přidělena. Tímto je zajištěna bezpečnost aplikace a zabráněno chybám, které by mohly vést k pádu aplikace.
    \item \textbf{Správa paměti} - CLR zajišťuje správu paměti pomocí GC. Tímto je zajištěno, že paměť je uvolněna vždy, když ji aplikace již nepotřebuje. Tímto je zabráněno tzv. memory leakům, které by mohly vést k pádu aplikace.
    \item \textbf{Větší přenositelnost} - CLR zajišťuje, že aplikace je spustitelná na všech operačních systémech, na kterých je dostupné běhové prostředí CLR.
\end{itemize}

Zatímco za nevýhody CLR se dá považovat:
\begin{itemize}
    \item  \textbf{Výkonnost} - I když určité optimalizace jsou prováděny pro konkrétní systém a architekturu, výkon CLR je nižší než výkon nativního kódu. Dalším výkonnostním měřítkem je rychlost startu aplikace, která je pro CLR vyšší než v případě nativního kódu.
    \item \textbf{Operační paměť} - CLR využívá více operační paměti, jak pro aplikaci, tak i pro běhové prostředí.
    \item \textbf{Velikost aplikace} - Přítomnost CLR nehraje zásádní roli v případě monolitických aplikací, ale v případě mikroslužeb je nutné CLR přidat ke každé službě. Tímto se zvyšuje velikost jedné aplikační instance.
\end{itemize}

\n{2}{AoT kompilace}
AoT kompilace je proces, při kterém je kód kompilován do podoby sytémově nativního kódu před spuštěním aplikace. V případě dotnetu je tímto jazykem C\# a výstupem kompilace dotnet aplikace je spustitelný soubor ve formátu podporovaném operačním systémem konfigurovaným v procesu kompilace. Takto vytvořený soubor je možné spustit přímo bez potřeby CLR nebo využití dotnet CLI. 

Jedná se o funkcionalitu vydanou bez plné podpory v roce 2022 s dotnet framework verzí 7. Vyráznější podporu získala v roce 2023 s vydáním dotent 8. \cite{aot}

Filozofie Microsoftu ohledně AoT kompilace je, že vývojáři by měli mít možnost využít AoT kompilace, pokud je to vhodné, aniž by museli použít jiný programovací jazyk a sadu nástrojů.

\n{2}{Princip}
Text

\n{3}{Výhody a nevýhody}

Mezi hlavní výhody se řadí zprostředkování následujícího:
\begin{itemize}
    \item  \textbf{Výkonnost} - CLR umožňuje využívat reflexi, která umožňuje získat informace o kódu za běhu aplikace. Tímto je umožněno vytvářet aplikace, které jsou schopny měnit své chování za běhu.
    \item \textbf{Paměťová zátěž} - CLR umožňuje dynamicky načítat knihovny za běhu aplikace. Tímto je umožněno vytvářet aplikace, které jsou schopny měnit své chování za běhu.
\end{itemize}

Zatímco za nevýhody CLR se dá považovat:
\begin{itemize}
    \item  \textbf{Absence nástrojů z CLR} - Mnoho nástrojů, které jsou dostupné v CLR, nejsou dostupné v AoT kompilaci. Mezi tyto nástroje patří například reflexe, dynamické načítání knihoven a další.
    \item \textbf{Transformace kódu na pozadí} - Za účelem zachování obdobné definice API využívají vybrané knihovny techniku transformace kódu na pozadí. Tím je zajištěno, že uživatel může jednoduše využít funkcionalitu, jako například routování REST endpointů stejným způsobem jako v CLR kódu. Tím je ale značně abstrahována podoba a funkce kódu, který je vytvořen.
\end{itemize}

\n{1}{Microservice architektura}
Při vývoji softwaru je možné využít několik architektur, které se liší v několika aspektech. Jednou z těchto architektur je monolitická architektura. V této architektuře je celá aplikace rozdělena do několika vrstev, které jsou využívány k oddělení logiky aplikace. \cite{monolith}

Microservice architektura je architektura, která je založena na principu oddělení aplikace do několika samostatných služeb. Každá z těchto služeb je zodpovědná za určitou část funkcionality aplikace. Služby jsou navzájem nezávislé a komunikují mezi sebou pomocí definovaných rozhraní. \cite{microservice}

\n{2}{Historie}
Původ microservice architektury nelze přesně definovat, důležitý moment však nastal v roce 2011, kdy Martin Fowler publikoval článek \textit{Microservices} na svém blogu. V tomto článku popsal výhody a nevýhody této architektury a zároveň popsal způsob, jakým je možné tuto architekturu využít. \cite{fowler} Dalším popularizačním momentem pro popularizaci bylo vydání knihy \textit{Building Microservices} od Sama Newmana v roce 2015. Tato kniha popisuje způsob, jakým je možné využít microservice architekturu v praxi. \cite{newman}

Opravdový přelom přišel postupně, nástupem a popularizací virtualizace a kontejnerizace v průběhu let 2013 až 2015. Tímto bylo umožněno vytvářet a spouštět mikroslužby v izolovaných prostředích. Tímto bylo umožněno vytvářet mikroslužby, které jsou nezávislé na operačním systému a hardwaru, na kterém jsou spouštěny. Nejdůležitější v tomto ohledu je nepochybně projekt Docker, který byl vydán v roce 2013. Díky Dockeru bylo možno jednoduše definovat, vytvářet a spouštět kontejnerizované aplikace. \cite{docker}

\n{2}{Popis}


\n{3}{Virtualizace a kontejnerizace}

\n{3}{Orchestrace}

\n{3}{Základní principy}

\n{4}{Komunikace}

\n{4}{Škálování}

\n{4}{Odolnost}

\n{4}{Vývoj}

\n{2}{Výhody a nevýhody}

\n{1}{Monitorování aplikace}
Na této stránce je k vidění způsob tvorby různých úrovní nadpisů.

\n{2}{Druhy dat}
Text

\n{3}{Metriky}
Text

\n{3}{Traces}
Text

\n{3}{Logy}
Text

\n{2}{Sběr dat}
Text

\n{3}{OpenTelemetry}
Text

\n{2}{Správa dat}


% ============================================================================ %
\cast{Praktická část}

\n{1}{Tvorba tech stacku}
Na této stránce je k vidění způsob tvorby různých úrovní nadpisů.

\n{2}{Požadavky na aplikaci}

\n{2}{Výběr technlogií}

\n{2}{Návrh a implementace služeb}

\n{2}{Konfigurace aplikace}

\n{1}{Testování scénářů}
Na této stránce je k vidění způsob tvorby různých úrovní nadpisů.

\n{2}{Popis scénářů}

\n{2}{Zpracování a vizualizace dat}

\n{3}{Monitorování v reálném čase}

\n{3}{Sběr historických dat}

% \n{2}{Obrázek}
% Obrázek \ref{fig:logo} prezentuje logo Fakulty aplikované informatiky.

% % Obrázek lze vkládat pomocí následujícího zjednodušeného stylu, nebo klasickým LaTex způsobem
% % Pozor! Obrázek nesmí obsahovat alfa kanál (průhlednost). Jde to totiž proti požadovanému standardu PDF/A.
% \obr{Popisek obrázku}{fig:logo}{0.5}{graphics/logo/fai_logo_cz.png}


% \n{2}{Tabulka}
% Tabulka \ref{tab:priklad} obsahuje dva řádky a celkem 7 sloupců.

% % Tabulku lze vkládat pomocí následujícího zjednodušeného stylu, nebo klasickým LaTex způsobem
% \tab{Popisek tabulky}{tab:priklad}{0.65}{|l|c|c|c|c|c|r|}{
%   \hline
%    & 1 & 2 & 3 & 4 & 5 & Cena [Kč] \\ \hline
%   \emph{F} & (jedna) & (dva) & (tři) & (čtyři) & (pět) & 300 \\ \hline
% }


% \n{2}{Citování}
% Následuje ukázka odkazování na různé zdroje:
% \begin{itemize}
% 	\item kniha \cite{HRW1997},
% 	\item kapitola v knize \cite{Delorme2006},
% 	\item článek v odborném žurnálu \cite{Bourreau2006},
% 	\item konferenční příspěvek \cite{Judish1999},
% 	\item doktorská práce \cite{Valente2005},
% 	\item technická zpráva \cite{Fralick1997},
% 	\item webová stránka \cite{WWWCST}.
% \end{itemize}


% ============================================================================ %

% Pokud Vaše práce obsahuje analytickou část, stačí odkomentovat nasledujících dva řádky
\cast{Analytická část}
\n{1}{Vyhodnocení výsledků}

\n{2}{Charakteristika testovacího prostředí}

\n{2}{Výsledky testování}

\n{2}{Doporučení pro použití AoT kompilace v platformě dotnet}

% ============================================================================ %
\nn{Závěr}
Text závěru.


% ============================================================================ %
