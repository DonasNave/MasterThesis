% ============================================================================ %
% Encoding: UTF-8 (žluťoučký kůň úpěl ďábelšké ódy)
% ============================================================================ %

% ============================================================================ %
\nn{Úvod}
První řádek prvního odstavce v kapitole či podkapitole se neodsazuje, ostatní ano. Vertikální odsazení mezy odstavci je typycké pro anglickou sazbu; czech babel toto respektuje, netřeba do textu přidávat jakékoliv explicitní formátování, viz ukázka sazby tohoto textu s následujícím odstavcem).

Formátování druhého odstavce. Text text text text text text text text text text text text.


% ============================================================================ %
\cast{Teoretická část}

\n{1}{Kompilace kódu v platformě dotnet}
Na této stránce je k vidění způsob tvorby různých úrovní nadpisů.

\n{2}{JIT kompilace}
Text

\n{3}{Historie}
Text

\n{3}{Princip}

\n{3}{Výhody a nevýhody}

\n{2}{AoT kompilace}
Text

\n{2}{Princip}
Text

\n{3}{Výhody a nevýhody}

\n{1}{Microservice architektura}
Na této stránce je k vidění způsob tvorby různých úrovní nadpisů.

\n{2}{Historie}
Text

\n{2}{Popis}

\n{3}{Virtualizace a kontejnerizace}

\n{3}{Orchestrace}

\n{3}{Základní principy}

\n{4}{Komunikace}

\n{4}{Škálování}

\n{4}{Odolnost}

\n{4}{Vývoj}

\n{2}{Výhody a nevýhody}

\n{1}{Monitorování aplikace}
Na této stránce je k vidění způsob tvorby různých úrovní nadpisů.

\n{2}{Druhy dat}
Text

\n{3}{Metriky}
Text

\n{3}{Traces}
Text

\n{3}{Logy}
Text

\n{2}{Sběr dat}
Text

\n{3}{OpenTelemetry}
Text

\n{2}{Správa dat}


% ============================================================================ %
\cast{Praktická část}

\n{1}{Tvorba tech stacku}
Na této stránce je k vidění způsob tvorby různých úrovní nadpisů.

\n{2}{Požadavky na aplikaci}

\n{2}{Výběr technlogií}

\n{2}{Návrh a implementace služeb}

\n{2}{Konfigurace aplikace}

\n{1}{Testování scénářů}
Na této stránce je k vidění způsob tvorby různých úrovní nadpisů.

\n{2}{Popis scénářů}

\n{2}{Zpracování a vizualizace dat}

\n{3}{Monitorování v reálném čase}

\n{3}{Sběr historických dat}

% \n{2}{Obrázek}
% Obrázek \ref{fig:logo} prezentuje logo Fakulty aplikované informatiky.

% % Obrázek lze vkládat pomocí následujícího zjednodušeného stylu, nebo klasickým LaTex způsobem
% % Pozor! Obrázek nesmí obsahovat alfa kanál (průhlednost). Jde to totiž proti požadovanému standardu PDF/A.
% \obr{Popisek obrázku}{fig:logo}{0.5}{graphics/logo/fai_logo_cz.png}


% \n{2}{Tabulka}
% Tabulka \ref{tab:priklad} obsahuje dva řádky a celkem 7 sloupců.

% % Tabulku lze vkládat pomocí následujícího zjednodušeného stylu, nebo klasickým LaTex způsobem
% \tab{Popisek tabulky}{tab:priklad}{0.65}{|l|c|c|c|c|c|r|}{
%   \hline
%    & 1 & 2 & 3 & 4 & 5 & Cena [Kč] \\ \hline
%   \emph{F} & (jedna) & (dva) & (tři) & (čtyři) & (pět) & 300 \\ \hline
% }


% \n{2}{Citování}
% Následuje ukázka odkazování na různé zdroje:
% \begin{itemize}
% 	\item kniha \cite{HRW1997},
% 	\item kapitola v knize \cite{Delorme2006},
% 	\item článek v odborném žurnálu \cite{Bourreau2006},
% 	\item konferenční příspěvek \cite{Judish1999},
% 	\item doktorská práce \cite{Valente2005},
% 	\item technická zpráva \cite{Fralick1997},
% 	\item webová stránka \cite{WWWCST}.
% \end{itemize}


% ============================================================================ %

% Pokud Vaše práce obsahuje analytickou část, stačí odkomentovat nasledujících dva řádky
\cast{Analytická část}
\n{1}{Vyhodnocení výsledků}

\n{2}{Charakteristika testovacího prostředí}

\n{2}{Výsledky testování}

\n{2}{Doporučení pro použití AoT kompilace v platformě dotnet}

% ============================================================================ %
\nn{Závěr}
Text závěru.


% ============================================================================ %
