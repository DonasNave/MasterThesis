% ============================================================================ %
%
%           Šablona bakalářské/diplomové práce
%
% Autor:    Ing. Jozef Říha (2006-05-04), od té doby šablonu udržuje
%           Ing. Pavel Tomášek, Ph.D. (tomasek@utb.cz)
%
% Verze:    2021-05-04
%
% Kódování: UTF-8 (kontrolní řetězec: žluťoučký kůň úpěl ďábelšké ódy)
%
% Sazba:    pdflatex prace.tex && pdflatex prace.tex
%           (nutné dvakrát pro korektní vložení citací a jiných referencí),
%           v případě umístění literatury do externího bib souboru je třeba volat
%           pdflatex prace.tex && bibtex prace && pdflatex prace.tex && pdflatex prace.tex
%
% Tip:      Ve správně vysázeném českém textu by na konci řádku neměla zůstant
%           samotná jednopísmenná předložka či spojka. Na takové místo se vkládá
%           nezalomitelná mezera pomocí symbolu ~. Existuje program, který umí
%           zpracovat celý TeX dokument najednou podle českých konvencí:
%           http://petr.olsak.net/ftp/olsak/vlna/
%
% Pozor:    Vzhledem k požadovanému standardu PDF/A nesmí vložené obrázky 
%           obsahovat alfa kanál (průhlednost).
%
% ============================================================================ %


\documentclass[a4paper,12pt]{article}

% Definice vzhledu a nastavení se načítá z následujícího souboru (netřeba editovat)
\input{tex/UTB.tex}

% Uživatelské definice -- upravte dle požadavků
\nastavfakultu{FAI}
	% FAI  -- pro Fakultu aplikované informatiky
	% FAME -- pro Fakultu managementu a ekonomiky
	% FHS  -- pro Fakultu humanitních studií
	% FLKR -- pro Fakultu logistiky a krizového řízení
	% FMK  -- pro Fakutlu mutimediálních komunikací
	% FT   -- pro Fakultu technologickou
	% UNI  -- pro Univerzitní institut
\nastavtyp{DP}
	% BP   -- bakalářská práce
	% DP   -- diplomová práce
\nastavrok{2024}
	% zadejte rok místo "xxxx"
\nastavjazyk{CZ}
	% CZ   -- práce bude v českém jazyce
	% EN   -- práce bude v anglickém jazyce

% Lze přidat vertikalni odsazeni nad (prvni parametr) a pod (druhy parametr)
% obrázky, tabulky i rovnice/soustavy rovnic
\nastavmezerukolemobrazku{0mm}{0mm}
\nastavmezerukolemtabulek{0mm}{0mm}
\nastavmezerukolemrovnic{0mm}{0mm}

\nastavautora{Bc. Noe Švanda}
\nastavnazevcz{Analýza služeb kompilovaných v režimu Ahead-of-Time a Just-In-Time na platformě .NET}
\nastavnazeven{Název práce anglicky (max. 2 řádky)} % Jen u anglicky psané práce
\nastavabstraktcz{Text abstraktu česky}
\nastavabstrakten{Text of the abstract}
\nastavklicovaslovacz{Přehled klíčových slov}
\nastavklicovaslovaen{Some keywords}

% Následující příkaz nastaví standard PDF/A-1b
\aplikujpdfa


% ============================================================================ %
\begin{document}

\titulnistrana

\zadani

\prohlaseni

\abstraktaklicovaslova

% ============================================================================ %

\clearpage

\thispagestyle{empty}
Děkuji svému vedoucímu práce, doc. Ing. Petru Šilhavému, Ph.D., za jeho cenné rady, trpělivost a ochotu věnovat mi svůj čas. Dále bych chtěl poděkovat své rodině a přátelům za podporu a pochopení během mého studia.

% ============================================================================ %
\obsah  % Obsah je generován automaticky


% ============================================================================ %
\OdsazovaniOdstavcuStart % Nastaví odsazování odstavců dle zvoleného jazyka

\nn{Úvod}
Programovací jazyky jsou základním kamenem softwarovévo vývoje respektive celého moderního světa v období informací. Představují způsob, kterým vývojář komunikuje s virtuálním prostředím OS a následně HW rozhraním. Vývoj výkonu HW, znalostí a zkušeností vývojářů a požadavků na vyvíjené systémy byl hnacím strojem technologického rozvoje. Postupným vývojem přicházeli další a další variace programovacích jazyků, některé rozdílné inkrementálně, jiné zcela diametrálně. Významným mezníkem v přístupu k tvorbě a běhu strojového kódu je vznik virtuálních strojů, které umožňují běh kódu nezávisle na HW. Tento přístup umožňuje vývojářům psát kód v jazyce, který je jim přirozený a následně jej spouštět na různých platformách.

Dotnet je platforma od společnosti Microsoft, která umožňuje vytvářet kód určený pro následnou kompilaci za běhu (Just-in-Time, dále JIT) a spuštění pomocí tzv. běhového prostředí (Common Language Runtime, dále CLR), jenž operuje jako virtuální stroj. Jedná se o relativně vyvinutou a zkušenou platformu s využitím v mnoha projektech a firmách. Přesto právě na této platformě byla dodatečně vyvinuta možnost pro PC platformy kompilace do nativního kódu (Ahead-of-Time, dále AOT), který je spouštěn přímo na OS a konkrétní architektuře HW. Tato funkce přichází do období rozmachu vývoje a migrace nativních cloudových řešení. Ty charakterizuje snaha dodávat pouze nezbytnou část infrastruktury a zpoplatnit reálnou dobu běhu systému s režií. Právě v prostředí cloudu mají nastávat situace, kdy bude využití služeb zkompilovaných do nativního kódu výhodnější. V kterých případech však opravdu takto napsaný program exceluje či selhává? A lze kvantifikacovat rozdíly mezi kopilací pro běhové prostředí JIT a AOT kompilací?

Tato práce se zabývá porovnáním vývojového procesu, charakteristik a výkonu JIT a AOT kompilovaných služeb na platformě .NET. Cílem je zjistit, zda a v jakých případech je možné využít AOT kompilace pro zvýšení výkonu a zlepšení chování aplikací. Výsledkem práce je kvantifikace, respektive srovnání výkonu a chování JIT a AOT kompilace na platformě Dotnet. Na základě těchto výsledků je možné posoudit a doporučit vhodné případy pro využití AOT kompilace.

\cast{Teoretická část}

%%%%%%%%%%%%%%%%%%%%%%%%%%%%%%%%%%%%%%%%%%%%%%%%%%%%%%%%%%%%%%%%%%%%%%%%%%%%%%%%%
%                                    .NET                                       %
%%%%%%%%%%%%%%%%%%%%%%%%%%%%%%%%%%%%%%%%%%%%%%%%%%%%%%%%%%%%%%%%%%%%%%%%%%%%%%%%%

\n{1}{Platforma .NET}

Platforma .NET od společnosti Microsoft představuje sadu nástrojů k vývoji aplikací v jazyce C\# a jeho derivátech. Tato platforma je multiplatformní a umožňuje vývoj aplikací pro operační systémy jako Windows, Linux, macOS ale i pro mobilní platformy. Vývojáři mohou využívat nástroje pro vývoj webových aplikací, desktopových aplikací, mobilních aplikací a dalších. Platforma .NET je postavena na dvou hlavních nástrojích. Prvním z nich je \textit{Common Language Runtime} (dále jen CLR), runtime prostředí zodpovídající za běh aplikací. Druhým nástrojem je \textit{dotnet CLI}, konzolový nástroj-rozhraní, zodpovědné za interakci s dílčími nástroji v platformě. \cite{Richter2012}

\n{2}{Historie}

Využití runtime prostředí, respektive v originální podobě virtuálního stroje, má historický původ. V dřívějších dobách byly programátoři limitování nutností kompilace kódu do nativní reprezentace přímo pro architekturu systému. Kód vytvořen pro jednu konkrétní architekturu se zpravidla neobešel bez modifikací, pokud měl fungovat i na odlišné architektuře.

V průběhu 90. let 20. století představila společnost Sun-Microsystems virtuální stroj Java Virtual Machine (JVM). Jedná se o komponentu runtime prostředí Javy, která zprostředkovává spuštění specifického kódu, správu paměti, vytváření tříd a typů a další. Kompilací Javy do tzv. Bytecode (Intermediate Language, zkráceně IL), tedy provedením mezikroku v procesu transformace zdrojového kódu do strojového kódu, je získána reprezentace programu, jenž běží na každém zařízení s implementovaným JVM. V rámci JVM dochází k finálnímu kroku a to interpretaci (JIT kompilaci) Bytecode do cílové architektury systému. 

Microsoft v reakci na JVM vydal v roce 2000 první .NET Framework, který umožňoval spouštět kód v jazyce C\# na operačním systému Windows. Cílem prvních verzí .NET Framework nebylo primárně umožnit vývoj pro různé zařízení a operační systémy, ale zprostředkovat lepší nástroje pro vývoj aplikací. KV roce 2014 byla vydána první multiplatformní verze .NETu. Byl vydán .NET Core, který umožňoval spouštět kód v jazyce C\# na operačních systémech Windows, Linux a macOS. \cite{Richter2012}

\n{2}{Nástroje .NET}

Platforma .NET zprostředkovává širokou sadu nástrojů za účelem tvorby, sestavení a spuštění aplikace. Mezi nejdůležitější lze zařadit následující:

\begin{itemize}
    \item \textbf{CLR} - běhové prostředí pro programy kompilované do IL
    \item \textbf{.NET CLI} (konzolové rozhraní) - konzolové rozhraní pro interakci s nástroji .NET
    \item \textbf{MSBuild} - buildovací engine pro kompilaci, testování a balíčkování aplikací
    \item \textbf{.NET SDK nástroje} - soubor nástrojů pro testování, debuggování a migraci .NET aplikací
    \item \textbf{Roslyn} - kompilační nástroj
    \item \textbf{NuGet} - balíčkovací manager 
\end{itemize}

\n{2}{Kompilace zdrojového kódu}

Kompilace je proces transformace zdrojového kódu do jiné podoby. Kód je zpravidla kompilován do podoby bližší cílové architektuře, ať je touto architekturou OS, případně konkrétní HW, nebo runtime prostředí (virtuální stroj). V rámci platformy .NET jsou k dispozici 2 hlavní fundamentálně odlišné režimy kompilace zdrojového kódu: kompilace pro běhové prostředí (CLR) do tzv. \emph{assembly} a kompilace do nativního kódu přímo pro cílovou architekturu (Native AoT).

\n{3}{Kompilace pro CLR}

Standartním výstupem sestavení aplikace v .NETu je transformace zdrojového kódu z vybraného podporovaného jazyka do assembly v jazyce IL. Tento výstupní IL se v .NET konkrétně navývaná Common Intermediate language (CIL) nebo také Microsoft Intermediate Language (MSIL). V případě jazyka C\# na platformě Windows slouží ke kompilaci spustitelný soubor \emph{csc.exe}. 

\n{4}{Výstup} 

Assembly s popisnými metadaty a IL (v případě režimu R2R i částečně nativním) kódem. Assembly typicky disponují příponou \emph{.dll}, případně jsou zabaleny do spustilného souboru dle cílové platformy a výstupu. Takovýto výstup je následně připraven buďto ke spuštění za pomocí CLR, případně pro využití a referenci při tvorbě dalšího .NET kódu.

Kód IL je sada instrukcí nezávislá na procesoru, kterou může spustit běhové prostředí .NET (CLR).

\n{3}{Kompilace do nativního kódu}

Přímá nativní AoT kompilace je proces, při kterém je kód kompilován do podoby sytémově nativního kódu při sestavení programu ze zdrojového kódu. V případě .NETu je tato funkcionalita dostupná při použití jazyka C\# a speciálních projektových atributů. 

Jedná se o funkcionalitu, jenž prošla několika iteracemi. První možnosti sestavení aplikace v nativním kódu na .NET platformě byly aplikace Universal Windows Platform. Jednalo se o aplikace využívající specifické rozhraní, nativní pro produkty Microsoftu. S verzí .NET framework 7 byly rozšířeny možnosti sestavení aplikace jako do podoby nativního kódu i pro další architektury a typy aplikací. Tato nová funkcionalita získala vyráznější podporu v roce 2023 s vydáním dotent framework 8. Filozofie Microsoftu ohledně AoT kompilace je, že vývojáři by měli mít možnost využít AoT kompilace v .NETu, pokud je to pro daný scénář vhodné. Scénáře kladoucí takovéto požadavky se vyskytují především v cloudovém nasazení, na které v současné filosofii apelují. 

\n{4}{Výstup}

Výstupem nativní AoT kompilace .NET aplikace je spustitelný soubor ve formátu podporovaném operačním systémem konfigurovaným v procesu kompilace. Takto vytvořený soubor je možné spustit přímo bez potřeby CLR. 

\n{3}{Popis procesu}

\begin{enumerate}
    \item Analýza zdrojového kódu, provádí kontrolu syntaxe
    \item Kontrola syntaxe
    \item Transformace vybrané syntaxe - například pro C\# je to transformace "High level C\#" na "Low level C\#"
    \item Generování pre-build kódu 
    \item Překlad kódu
\end{enumerate}

\n{3}{Cíle kompilace}

TODO

\n{2}{Běh kódu}

Spuštění, respektive běh kódu na HW počítačového zařízení vyžaduje instrukční sadu, které daná architektura rozumí, tedy nativní kód. V případě nativní AoT kompilace v .NET tento kód získáme již při sestavení aplikace. Při využití kompilace do IL je nutné kód získat pomocí jednoho z kompilačních způsobů podporovaného CLR. Výsledná nativní reprezentace se v obou případech spouští zavoláním vstupní metody v binárním souboru dle specifikace architektury.

\n{3}{CLR}

Common Language Runtime (CLR) je běhové prostředí frameworku .NET. Poskytuje spravované prostředí pro spouštění aplikací .NET. Podporuje více programovacích jazyků, včetně jazyků C\#, VB.NET a F\#, a umožňuje jejich bezproblémovou spolupráci. Spravuje paměť prostřednictvím automatického garbage collection, který pomáhá předcházet únikům paměti a optimalizuje využití prostředků. CLR také zajišťuje typovou bezpečnost a ověřuje, zda jsou všechny operace typově bezpečné, aby se minimalizovaly chyby při programování.

\n{4}{Funkce}

CLR je zodpovědný za několik důležitých funkcí, které zvyšují produktivitu vývojářů a výkon aplikací.

\begin{itemize}
    \item Správa paměti - spravuje alokaci paměti, obsluhuje GC
    \item Bezpečnost
    \item Zpracování vyjímek - obsluhuje zpracování chyb/vyjímek v programu
    \item Generování typů 
    \item Reflexe
\end{itemize}

Klíčovými vlasnostmi jsou CLR jsou multiplatformnost kódu, reflexe, optimalizace kódu pro konkrétní architekturu a bezpečnost. CLR nabízí mechanismy, jako je zabezpečení přístupu ke kódu (CAS), které zabraňují neoprávněným operacím. Kompilace JIT (just-in-time) znamená, že kód zprostředkujícího jazyka je zkompilován do nativního kódu těsně před spuštěním, což zajišťuje optimální výkon na cílovém hardwaru. CLR usnadňuje zpracování chyb v různých jazycích a poskytuje konzistentní přístup k řešení výjimek. Navíc obsahuje nástroje pro ladění a profilování, které vývojářům pomáhají efektivně identifikovat a odstraňovat problémy s výkonem.

Aby mohl být kód z IL reprezentace spuštěn na systému, respektive HW stroje, musí být dodatečně kompilován. Za tímto účelem existuje v CLR několik technik, které s sebou přínáší různé benefity a negativa a mají využití v specifických scénářích.

\n{4}{JIT kompilace}

Při JIT kompilaci v rámci CLR dochází ke kompilaci kódu do nativní podoby těsně před spuštěním kódu. Kompilace veškerého kódu aplikace JIT umožňuje optimalizovat běh programu současnému stavu systému. Za běhu je prováděna inspekce kódu, dochází ke kontrole validity, typování a adresace IL kódu. Kompilovány jsou pouze ty části kódu, jenž jsou relevantní pro aktuální stav programu.

\n{4}{R2R kompilace}

Zdrojový kód je při sestavení zkompilován do podoby nativního kódu pomocí nástroje crossgen, čímž vzniknou sestavy R2R. Za běhu se sestavy R2R načtou a spustí s minimální kompilací JIT, protože většina kódu je již v nativní podobě. CLR může přesto JIT kompilovat některé části kódu, které nelze staticky zkompilovat předem. Využití je v aplikacích, které potřebují zkrátit dobu spouštění, ale zachovat určitou funkcionalitu nebo úroveň optimalizace poskytovanou JIT kompilací.

\n{3}{Nativní kód}

Běh nativního kódu je závislý na konkrétní architektuře systému, pro které jsou nativní programové soubory vytvořeny. Nepodléhá další úpravě ze strany .NET nástrojů.

\n{2}{Tvorba programu v dotnet}

Následující část popisuje obecnou koncepci a strukturu projektu aplikace v dotnet. Součástí je postup pro tvorbu a vydání projektu. Blížší pozornost bude věnována tvorbě nativního AoT projektu.

\n{3}{Struktura aplikačních zdrojů}

Základním strukturovaným prvkem v .NET aplikaci je projektový soubor. Jedná se o XML soubor disponující příponou \emph{.csproj}. V rámci něj dochází ke konfiguraci a deklaraci, jak bude .NET CLI s projektem pracovat. Zároveň jsou zde definováný závislosti na další projekty a knihovny. Mezi základní charakteristiky běžně určené v projektovém souboru patří verze .NET, verze projektu/assembly, seznam závislostí, konfigurace pro buildování, testování a publikaci.

Pro tvorbu složitějších aplikací je možné využít více projektových souborů, které jsou následně propojeny. Tento způsob je využíván především v případě větších aplikací, které jsou rozděleny do více částí. Propojení a vazby mezi více projekty v aplikaci je definované pomocí tzv. solution souboru. Jedná se o kontejnerový soubor s příponou \emph{.sln}, jenž popisuje závislosti mezi projektovými soubory, konfigurace sestavení a nasazení a správu pomocných souborů.

\n{3}{Obecný postup}

\begin{enumerate}
    \item \textbf{Nastavení vývojového prostředí}: Sestává z instalace sady nástrojů .NET SDK.
    
    \item \textbf{Vytvoření projektu}: Pomocí příkazu \texttt{dotnet new} nebo skrze GUI IDE je vytvořen nový projekt a solution soubor. Součástí je výběr typu projektu, jazyka, frameworku a dalších konfiguračních parametrů.
    
    \item \textbf{Programování}: Sestává z tvorby kódu aplikace, testování a ladění.

    \item \textbf{Správa závislostí}: Pomocí nástrojů .NET CLI je možno referencovat balíčky a knihovny v rámci projektu.
    
    \item \textbf{Kompilace}: Kompilace aplikace probíhá pomocí příkazu \texttt{dotnet build}, který převede vysokoúrovňový kód do IL. V případě AoT dochází k dodatečné kompilace do nativního kódu dle cílové architektury.
    
    \item \textbf{Publikování}: Použitím příkazu příkazu \texttt{dotnet publish} dochází k vydání aplikace, tedy specifickému sestavení v konfigurovaném nastavení.
\end{enumerate}

\n{3}{Tvorba nativního programu}

Pro tvorbu nativního programu v .NET je nutné využít speciálního atributu \emph{PublishAoT} v projektovém souboru. Tento atribut je zodpovědný za konfiguraci projektu pro nativní AoT kompilaci. Při jeho použití je nutné specifikovat cílovou architekturu, pro kterou je nativní kód vytvářen. Po kompilaci kódu do IL dochází k dodatečné kompilaci do nativního kódu, která dodává další konzolový výstup s informacemi o průběhu kompilace.

Vzhledem k tomu, že nativní AoT kompilace je v .NETu stále vývojově nezralá, samotný proces kompilace, tak jako analýza kompilovaného kódu není dostatečně informativní. Za účelem přenesení vysokoúrovňových konceptů a formálních zápisů v C\# je při kompilace prováděno široké spektrum transformací a gerování kódu.

\n{4}{Deklarace unmanaged rozhraní}

\n{4}{Trimming}

\n{3}{Přehled podpory}

\n{4}{Funkcionalita}

Následující přehled představuje rozsah funkcionality implementované v rámci .NET frameworku 8.0, konkrétně APS.NET k datu zvěřejnění práce.

\begin{itemize}
    \item \textbf{REST minimal API}
    \item \textbf{gRPC API}
    \item \textbf{JWT Authentication}
    \item \textbf{CORS}
    \item \textbf{HealthChecks}
    \item \textbf{HttpLogging}
    \item \textbf{Localization}
    \item \textbf{OutputCaching}
    \item \textbf{RateLimiting}
    \item \textbf{RequestDecompression}
    \item \textbf{ResponseCaching}
    \item \textbf{ResponseCompression}
    \item \textbf{Rewrite}
    \item \textbf{StaticFiles}
    \item \textbf{WebSockets}
\end{itemize}

\n{4}{Cíle kompilace}

.NET poskytuje podporu pro kompilaci zdrojového kódu v režimu AoT pouze pro určité operační systémy:

\begin{itemize}
    \item \textbf{Windows} - plná podpora
    \item \textbf{Linux} - plná podpora
    \item \textbf{macOS} - plná podpora
    \item \textbf{Android} - částečná podpora
    \item \textbf{iOS} - částečná podpora
    \item \textbf{WebAssembly} - částečná podpora
\end{itemize}

\n{2}{Srovnání}

\n{3}{JIT}

\n{4}{Výhody}

Mezi hlavní výhody se řadí zprostředkování následujícího:

\begin{itemize}
    \item  \textbf{Reflexe} - CLR umožňuje využívat reflexi, která umožňuje získat informace o kódu za běhu aplikace. Tímto je umožněno vytvářet aplikace, které jsou schopny měnit své chování za běhu.
    \item \textbf{Dynamické načítání} - CLR umožňuje dynamicky načítat knihovny za běhu aplikace. Tímto je umožněno vytvářet aplikace, které jsou schopny měnit své chování za běhu.
    \item \textbf{Větší bezpečnost} - CLR zajišťuje, že aplikace nemůže přistupovat k paměti, která jí nebyla přidělena. Tímto je zajištěna bezpečnost aplikace a zabráněno chybám, které by mohly vést k pádu aplikace.
    \item \textbf{Správa paměti} - CLR zajišťuje správu paměti pomocí GC. Tímto je zajištěno, že paměť je uvolněna vždy, když ji aplikace již nepotřebuje. Tímto je zabráněno tzv. memory leakům, které by mohly vést k pádu aplikace.
    \item \textbf{Větší přenositelnost} - CLR zajišťuje, že aplikace je spustitelná na všech operačních systémech, na kterých je dostupné běhové prostředí CLR.
\end{itemize}

\n{4}{Nevýhody}

Zatímco za nevýhody CLR se dá považovat:

\begin{itemize}
    \item  \textbf{Výkonnost} - I když určité optimalizace jsou prováděny pro konkrétní systém a architekturu, výkon CLR je nižší než výkon nativního kódu. Dalším výkonnostním měřítkem je rychlost startu aplikace, která je pro CLR vyšší než v případě nativního kódu.
    \item \textbf{Operační paměť} - CLR využívá více operační paměti, jak pro aplikaci, tak i pro běhové prostředí.
    \item \textbf{Velikost aplikace} - Přítomnost CLR nehraje zásádní roli v případě monolitických aplikací, ale v případě mikroslužeb je nutné CLR přidat ke každé službě. Tímto se zvyšuje velikost jedné aplikační instance.
\end{itemize}

\n{3}{AoT}

\n{4}{Výhody}

Mezi hlavní výhody se řadí zprostředkování následujícího:

\begin{itemize}
    \item  \textbf{Výkonnost} - CLR umožňuje využívat reflexi, která umožňuje získat informace o kódu za běhu aplikace. Tímto je umožněno vytvářet aplikace, které jsou schopny měnit své chování za běhu.
    \item \textbf{Paměťová zátěž} - CLR umožňuje dynamicky načítat knihovny za běhu aplikace. Tímto je umožněno vytvářet aplikace, které jsou schopny měnit své chování za běhu.
\end{itemize}

\n{4}{Nevýhody}

Zatímco za nevýhody CLR se dá považovat:

\begin{itemize}
    \item  \textbf{Absence nástrojů z CLR} - Mnoho nástrojů, které jsou dostupné v CLR, nejsou dostupné v AoT kompilaci. Mezi tyto nástroje patří například reflexe, dynamické načítání knihoven a další.
    \item \textbf{Absence dynamického načítání} - například Assembly.LoadFile.
    \item \textbf{Bez generování kódu za běhu} - například System.Reflection.Emit.
    \item \textbf{Žádné C++/CLI} - např. System.Runtime.InteropServices.WindowsRuntime
    \item \textbf{Windows: absence COM} - např. System.Runtime.InteropServices.ComTypes
    \item \textbf{Vyžaduje trimming (ořezávání)} - má určitá omezení, je však klíčový pro rozumnou velikost výsledného programu
    \item \textbf{Kompilace do jediného souboru} 
    \item \textbf{Připojení běhových knihoven} - požadované běhové knihovny jsou součástí výsledného aplikačního souboru. To zvyšuje velikost samoteného programu ve srovnání s aplikacemi závislými na frameworku.
    \item \textbf{System.Linq.Expressions} - výsledný kód používá svou interpretovanou podobu, která je pomalejší než run-time generovaný kompilovaný kód.
    \item \textbf{Kompatibilita knihoven s AoT} - né všechny knihovny runtime jsou plně anotovány tak, aby byly kompatibilní s Native AoT. To znamená, že některá varování v knihovnách runtime nejsou pro koncové vývojáře použitelná.
\end{itemize}

\n{2}{Závěr}

xx

%%%%%%%%%%%%%%%%%%%%%%%%%%%%%%%%%%%%%%%%%%%%%%%%%%%%%%%%%%%%%%%%%%%%%%%%%%%%%%%%%
%                                 MICROSERVICE                                  %
%%%%%%%%%%%%%%%%%%%%%%%%%%%%%%%%%%%%%%%%%%%%%%%%%%%%%%%%%%%%%%%%%%%%%%%%%%%%%%%%%

\n{1}{Microservice architektura}
Při vývoji softwaru je možné využít z několika architektur. Jednou z těchto architektur je monolitická architektura. V monolitické architektuře je celá aplikace rozdělena do několika vrstev, které jsou využívány k oddělení logiky aplikace.

Oproti tomu microservice architektura je založena na principu oddělení aplikace do několika samostatných služeb. Každá z těchto služeb je zodpovědná za určitou část funkcionality aplikace. Služby jsou navzájem nezávislé a komunikují mezi sebou pomocí definovaných rozhraní. \cite{Richardson2018}

\n{2}{Historie}
Původ microservice architektury nelze přesně definovat, důležitý moment však nastal v roce 2011, kdy Martin Fowler publikoval článek \textit{Microservices} na svém blogu. V tomto článku popsal výhody a nevýhody této architektury a zároveň popsal způsob, jakým je možné tuto architekturu využít. Dalším popularizačním momentem pro popularizaci bylo vydání knihy \textit{Building Microservices} od Sama Newmana v roce 2015. Tato kniha popisuje způsob, jakým je možné využít microservice architekturu v praxi.

Opravdový přelom přišel postupně, nástupem a popularizací virtualizace a kontejnerizace v průběhu let 2013 až 2015. Tímto bylo umožněno vytvářet a spouštět mikroslužby v izolovaných prostředích. Tímto bylo umožněno vytvářet mikroslužby, které jsou nezávislé na operačním systému a hardwaru, na kterém jsou spouštěny. Nejdůležitější v tomto ohledu je nepochybně projekt Docker, který byl vydán v roce 2013. Díky Dockeru bylo možno jednoduše definovat, vytvářet a spouštět kontejnerizované aplikace.

\n{2}{Popis}

Architektura mikroslužeb rozděluje složité softwarové aplikace na menší, spravovatelné části, které lze vyvíjet, nasazovat a škálovat nezávisle.

\n{3}{Virtualizace a kontejnerizace}

Virtualizace a kontejnerizace jsou klíčové technologie, které umožňují architekturu mikroslužeb. Virtualizace umožňuje provozovat více operačních systémů na jednom fyzickém hardwarovém hostiteli, čímž se snižuje počet potřebných fyzických strojů a zvyšuje efektivita využití zdrojů. Kontejnerizace jde ještě o krok dále tím, že zabalí aplikaci a její závislosti do kontejneru, který může běžet na libovolném serveru Linux nebo Windows. Tím je zajištěno, že aplikace funguje jednotně i přes rozdíly v prostředí nasazení.

Kontejnerizace je obzvláště důležitá pro mikroslužby, protože zapouzdřuje každou mikroslužbu do vlastního kontejneru, což usnadňuje její nasazení, škálování a správu nezávisle na ostatních. Synonymem kontejnerizace se staly nástroje jako Docker, které nabízejí ekosystém pro vývoj, odesílání a provoz kontejnerových aplikací.

\n{3}{Orchestrace}

S rozšiřováním mikroslužeb a kontejnerů se jejich správa stává složitou. Nástroje pro orchestraci pomáhají automatizovat nasazení, škálování a správu kontejnerů. Mezi oblíbené orchestrační nástroje patří Kubernetes, Docker Swarm a Mesos. Zejména Kubernetes se stal de facto standardem, který poskytuje robustní rámec pro nasazení, škálování a provoz kontejnerových aplikací v clusteru strojů. Řeší vyhledávání služeb, vyvažování zátěže, sledování přidělování prostředků a škálování na základě výkonu pracovní zátěže.

\n{3}{Základní principy}

\n{4}{Komunikace}

Mikroslužby spolu komunikují prostřednictvím rozhraní API, obvykle prostřednictvím protokolů HTTP/HTTPS, i když pro aplikace citlivější na výkon lze použít i jiné protokoly, například gRPC. Komunikační vzory zahrnují synchronní požadavky (např. RESTful API) a asynchronní zasílání zpráv (např. pomocí brokerů zpráv jako RabbitMQ nebo Kafka). Tím je zajištěno volné propojení mezi službami, což umožňuje jejich nezávislý vývoj a nasazení.

\n{4}{Škálování}

Architektura mikroslužeb zvyšuje škálovatelnost. Služby lze škálovat nezávisle, což umožňuje efektivnější využití zdrojů a zlepšuje schopnost systému zvládat velké objemy požadavků. Běžně se používá horizontální škálování (přidávání dalších instancí služby), které usnadňují nástroje pro kontejnerizaci a orchestraci.

\n{4}{Odolnost}

Robustnosti mikroslužeb je dosaženo pomocí strategií, jako jsou přerušovače, záložní řešení a opakované pokusy, které pomáhají zabránit tomu, aby se selhání jedné služby kaskádově přeneslo na ostatní. Izolace služeb také znamená, že problémy lze omezit a vyřešit s minimálním dopadem na celý systém. Kromě toho jsou kontroly stavu a monitorování nezbytné pro včasné odhalení a řešení problémů.

\n{4}{Vývoj}

Mikroslužby umožňují agilní vývojové postupy. Týmy mohou vyvíjet, testovat a nasazovat služby nezávisle, což umožňuje rychlejší iteraci a zpětnou vazbu. Nedílnou součástí jsou pipelines pro kontinuální integraci a doručování (CI/CD), které umožňují automatizované testování a nasazení. Tento přístup podporuje kulturu DevOps a podporuje užší spolupráci mezi vývojovými a provozními týmy.

\n{3}{Serverless a mikroslužby}

Serverless je model vývoje aplikací, který umožňuje vývojářům psát a nasazovat kód bez starostí o infrastrukturu. Tento model je založen na konceptu funkcí jako služby (FaaS), které jsou jednotlivé kusy kódu, které jsou spouštěny na základě událostí. Serverless a mikroslužby se často používají společně, protože oba modely podporují škálovatelnost, agilitu a odolnost. Serverless může být výhodný pro mikroslužby, které jsou založeny na událostech, jako jsou zpracování obrázků, zpracování zpráv nebo plánování úloh.

\n{2}{Testování}

Testování mikroslužeb je klíčové pro zajištění kvality a spolehlivosti systému. Mikroslužby lze testovat na několika úrovních, včetně jednotkových testů, integračních testů a testů end-to-end. Jednotkové testy se zaměřují na testování jednotlivých komponent služby, zatímco integrační testy testují komunikaci mezi službami. Testy end-to-end testují celý systém z pohledu uživatele. Automatizované testování je klíčové pro rychlé a spolehlivé nasazení.

\n{2}{Výhody a nevýhody}

\n{3}{Výhody}

\n{4}{Zvýšená agilita} 

Mikroslužby umožňují rychlé, časté a spolehlivé poskytování rozsáhlých a komplexních aplikací. Týmy mohou aktualizovat určité oblasti aplikace, aniž by to mělo dopad na celý systém, což umožňuje rychlejší iterace.

\n{4}{Škálovatelnost}

Služby lze škálovat nezávisle, což umožňuje přesnější přidělování zdrojů na základě poptávky. To usnadňuje zvládání proměnlivého zatížení a může zlepšit celkovou efektivitu aplikace.

\n{4}{Odolnost} 

Decentralizovaná povaha mikroslužeb pomáhá izolovat selhání na jedinou službu nebo malou skupinu služeb, čímž zabraňuje selhání celé aplikace. Techniky, jako jsou jističe, zvyšují odolnost systému.

\n{4}{Technologická rozmanitost}

Týmy si mohou vybrat nejlepší nástroj pro danou práci a podle potřeby používat různé programovací jazyky, databáze nebo jiné nástroje pro různé služby, což vede k potenciálně optimalizovanějším řešením.

\n{4}{Flexibilita nasazení}

Mikroslužby lze nasazovat nezávisle, což je ideální pro kontinuální nasazení a integrační pracovní postupy. To také umožňuje průběžné aktualizace, modrozelené nasazení a kanárkové verze, což snižuje prostoje a rizika.

\n{4}{Modularita}

Tato architektura zvyšuje modularitu, což usnadňuje pochopení, vývoj, testování a údržbu aplikací. Týmy se mohou zaměřit na konkrétní obchodní funkce, což zvyšuje produktivitu a kvalitu.

\n{3}{Nevýhody}

\n{4}{Komplexnost} 

Správa více služeb na rozdíl od monolitické aplikace přináší složitost při nasazování, monitorování a řízení komunikace mezi službami.

\n{4}{Správa dat}

Konzistence dat mezi službami může být náročná, zejména pokud si každá mikroslužba spravuje vlastní databázi. Implementace transakcí napříč hranicemi vyžaduje pečlivou koordinaci a vzory jako Saga.

\n{4}{Zpoždění sítě}

Komunikace mezi službami po síti přináší zpoždění, které může ovlivnit výkonnost aplikace. Ke zmírnění tohoto jevu jsou nutné efektivní komunikační protokoly a vzory.

\n{4}{Provozní režie}

S počtem služeb roste potřeba orchestrace, monitorování, protokolování a dalších provozních záležitostí. To vyžaduje další nástroje a odborné znalosti.

\n{4}{Složitost vývoje a testování}

Mikroslužby sice zvyšují flexibilitu vývoje, ale také komplikují testování, zejména pokud jde o testování end-to-end, které zahrnuje více služeb.

\n{4}{Integrace služeb} Zajištění bezproblémové spolupráce služeb vyžaduje robustní správu API, řízení verzí a strategie zpětné kompatibility.

\n{2}{Závěr}

Architektura mikroslužeb je metoda vývoje softwarových systémů, které jsou rozděleny do malých, nezávislých služeb komunikujících prostřednictvím přesně definovaných rozhraní API. Tyto služby jsou vysoce udržovatelné a testovatelné, volně provázané, nezávisle nasaditelné a organizované podle obchodních schopností. Tento přístup k architektuře umožňuje organizacím dosáhnout větší agility a škálování jejich aplikací.

%%%%%%%%%%%%%%%%%%%%%%%%%%%%%%%%%%%%%%%%%%%%%%%%%%%%%%%%%%%%%%%%%%%%%%%%%%%%%%%%%
%                                  MONITORING                                   %
%%%%%%%%%%%%%%%%%%%%%%%%%%%%%%%%%%%%%%%%%%%%%%%%%%%%%%%%%%%%%%%%%%%%%%%%%%%%%%%%%

\n{1}{Monitorování aplikace}

Monitorování aplikací je klíčovým aspektem moderního vývoje a provozu softwaru, který týmům umožňuje sledovat výkon, stav a celkové chování aplikací v reálném čase. Zahrnuje shromažďování, analýzu a interpretaci různých typů dat a informací, které zajišťují hladký a efektivní chod aplikací a umožňují rychle identifikovat a řešit případné problémy.

\n{2}{Druhy dat}

Pro efektivní monitorování aplikace je nezbytné porozumět různým typům dat a informací, které lze shromažďovat:

\n{3}{Logy}

Protokoly jsou záznamy o událostech, ke kterým dochází v rámci aplikace nebo jejího provozního prostředí. Poskytují podrobné, časově označené záznamy o činnostech, chybách a transakcích, které mohou vývojáři a provozní týmy použít k řešení problémů, pochopení chování aplikace a zlepšení spolehlivosti systému.

\n{3}{Traces}

Trasy se používají ke sledování toku požadavků v aplikaci, zejména v distribuovaných systémech, kde jedna transakce může zahrnovat více služeb nebo komponent. Sledování pomáhá identifikovat úzká místa, pochopit problémy s latencí a zlepšit celkový výkon aplikací.

\n{3}{Metriky}

Metriky jsou kvantitativní údaje, které poskytují přehled o výkonu a stavu aplikace. Mezi běžné metriky patří doba odezvy, využití systémových prostředků (CPU, paměť, diskové I/O), chybovost a propustnost. Sledování těchto metrik pomáhá při proaktivním ladění výkonu a plánování kapacity.

\n{2}{Sběr dat}

Efektivita monitorování aplikací do značné míry závisí na schopnosti efektivně shromažďovat relevantní data.

\n{3}{Collectory}

Kolektory jsou nástroje nebo agenti, kteří shromažďují data z různých zdrojů v rámci aplikace a jejího prostředí. Mohou být nasazeny jako součást infrastruktury aplikace nebo mohou být provozovány jako externí služby. Kolektory jsou zodpovědné za shromažďování protokolů, stop a metrik a za předávání těchto dat do monitorovacích řešení, kde je lze analyzovat a vizualizovat. Efektivní sběr dat je nezbytný pro monitorování v reálném čase a pro zajištění toho, aby shromážděná data přesně odrážela stav a výkon aplikace.

\n{2}{Vizualizace dat}

Vizualizace dat je klíčovým aspektem monitorování aplikací, který umožňuje rychle porozumět stavu a chování aplikací. Vizualizace může zahrnovat různé typy grafů, tabulek, dashboardů a dalších nástrojů, které umožňují zobrazit data v uživatelsky přívětivé podobě. Vizualizace dat umožňuje týmům identifikovat vzory, problémy a příležitosti, které by jinak mohly zůstat skryty v datech.

\n{2}{Implementace monitorování}

Implementace monitorování aplikací zahrnuje několik klíčových kroků, včetně definice klíčových metrik, výběru monitorovacích nástrojů, nasazení kolektorů a vizualizaci dat. Týmy by měly také vytvořit procesy pro řešení problémů, které byly identifikovány prostřednictvím monitorování, a pro využití dat k plánování kapacity a optimalizaci výkonu.

\n{3}{Sběr dat v monitorovaných službách}

Implementace sběru dat zahrnuje inkorporaci funkcionality monitorování a zprostředkování dat v rámci předdefinovaného rozhraní. Sběr je realizován zpravidla sérií čítačů a zapisovačů, které jsou využívány k získávání dat z různých zdrojů. Takto sbíraná datá jsou kategorizována a značkována pro identifikaci.

Realizace monitorování je zajištěna buďto použitím existujících implementací v rámci sw knihoven nebo vytvořením vlastní implementace dle potřeb aplikace a monitorovacích protokolů.

\n{3}{Nasazení služeb pro správu a kolekci dat}

Nasazení služeb pro správu a kolekci dat je zajištěno pomocí nástrojů, které jsou schopny zprostředkovat sběr dat z různých zdrojů a zároveň zajišťují jejich zpracování a zobrazení. Tímto je zajištěno, že data jsou zpracována a zobrazena v reálném čase.

\n{3}{Vizualizace dat}

Vizualizace dat je zajištěna pomocí nástrojů, které jsou schopny zobrazit data v uživatelsky přívětivé podobě. Tímto je zajištěno, že data jsou zobrazena v reálném čase a jsou přehledná a srozumitelná.

\n{2}{Konfigurace}

Konfigurace monitorování je zajištěna pomocí konfiguračních souborů, které definují chování monitorovacích nástrojů a sběr dat. Ovlivnit chování monitorovacího systému může být provedeno jak na straně monitorovacích nástrojů, respektive služeb, tak i na straně aplikací a služeb, které jsou monitorovány.

\n{2}{Závěr}

Monitorování aplikací je nezbytným nástrojem pro vývoj a provoz moderních softwarových systémů. Zahrnuje shromažďování, analýzu a interpretaci různých typů dat a informací, které umožňují týmům sledovat výkon, stav a chování aplikací v reálném čase. Tímto je zajištěno, že aplikace jsou spolehlivé, výkonné a efektivní.


\cast{Praktická část}

%%%%%%%%%%%%%%%%%%%%%%%%%%%%%%%%%%%%%%%%%%%%%%%%%%%%%%%%%%%%%%%%%%%%%%%%%%%%%%%%%
%                             Tvorba tech stacku                                %
%%%%%%%%%%%%%%%%%%%%%%%%%%%%%%%%%%%%%%%%%%%%%%%%%%%%%%%%%%%%%%%%%%%%%%%%%%%%%%%%%

\n{1}{Tvorba tech stacku}

Za účelem důkladného testování výkonu a škálovatelnosti mikroslužeb byl vytvořen tech stack, který zahrnuje technologie pro kontejnerizaci, orchestraci, persistenci, komunikaci, monitorování a testování. 

\n{2}{Požadavky na SW}

Aplikace pro svůj účel nezávislého testování výkonu a škálovatelnosti mikroslužeb vyžaduje několik požadavků, které jsou rozděleny na funkční a nefunkční.

\n{3}{Funkční požadavky}

Funkční požadavky popisují, jak má aplikace fungovat a jaké funkce má poskytovat.

\n{4}{Sběr a vizualizace dat}

Aplikace musí být schopna sbírat a vizualizovat data o výkonu a škálovatelnosti mikroslužeb. To zahrnuje sběr a vizualizaci metrik, protokolů a tras.

\n{4}{Testování scénářů}

Aplikace musí být schopna provádět testování scénářů, které simuluje zátěž na mikroslužby a zjišťuje, jak se chovají za různých podmínek.

\n{4}{Konfigurace aplikace}

Aplikace musí být schopna konfigurovat testovací scénáře, které se mají provést, a způsob, jakým se mají provést.

\n{3}{Nefunkční požadavky}

\n{4}{Výkon}

Implementace aplikace, respektive jejich služeb, musí být schopna zvládnout zátěž, která je na ně kladena. To zahrnuje schopnost zvládnout požadavky na výkon a škálovatelnost.

\n{2}{Požadavky na HW}

Hardware, na kterém bude aplikace provozována, musí výkonnostně dostačovat pro provozování testovacích scénářů a sběr a vizualizaci dat. Týká se to primárně počtu jader, velikosti paměti a rychlosti diskového I/O. Provozované služby mají určitou základní režii, která se musí brát v potaz.

\n{2}{Výběr technlogií}

Součástí tvorby tech stacku je výběr technologií, které budou použity pro implementaci aplikace. Výběr technologií je závislý na požadavcích na aplikaci a HW, na kterém bude aplikace provozována.

\n{3}{Kontejnerizace a orchestrace}

Základním prvkem nasazení aplikace je kontejnerizace a orchestrace. Kontejnerizace zajišťuje, že aplikace bude spouštěna v izolovaném prostředí, které je nezávislé na hostitelském systému. Orchestrace zajišťuje, že aplikace bude spouštěna na dostupných zdrojích a bude schopna zvládnout zátěž, která je na ni kladena.

Pro kontejnerizaci byla zvolena technologie Docker. Docker je open-source platforma pro kontejnerizaci aplikací, která umožňuje vytvářet, spouštět a spravovat kontejnery.

Pro orchestraci byla vybrána technologie Kubernetes. Kubernetes je open-source platforma pro orchestraci kontejnerů, která umožňuje automatizovat nasazování, škálování a správu aplikací. Kubernetes je schopný pracovat s kontejnery, které jsou vytvořeny pomocí Dockeru.

\n{3}{Konfigurace nasazení}

Pro konfiguraci nasazení byla zvolena technologie Helm. Helm je open-source platforma pro správu balíčků, která umožňuje vytvářet, spravovat a nasazovat balíčky. Helm je schopný pracovat s balíčky, které jsou vytvořeny pomocí Kubernetes.

Definice balíčků je řešena pomocí konfiguračních souborů, které jsou použity již při tvorbě obecného obrazu. V rámci Helm je základním prvkem chart, který obsahuje definici balíčku a šablonu, která je použita pro generování konfigurace.

\n{3}{Persistenční vrstva}

Pro persistenci relačních dat byla zvolena technologie PostgreSQL. PostgreSQL je open-source relační databázový systém, který je schopný zvládnout velké množství dat a zároveň poskytovat vysoký výkon.

\n{3}{Komunikační protokoly}

Pro komunikaci mezi službami byl zvolen protokol HTTP. Verze HTTP/2 byla zvolena pro její schopnost zvládnout vysoký počet požadavků a zároveň poskytovat vysoký výkon.

\n{3}{Monitorovací nástroje}

Pro monitorování aplikace byl zvolen Grafana observability stack pro jeho pokrytí komplexní škály monitorovacích dat. Grafana observability stack zahrnuje nástroje pro sběr, vizualizaci a analýzu dat.

\n{4}{Grafana}

Open-source platforma pro vizualizaci a analýzu dat. Grafana umožňuje vizualizovat data z různých zdrojů, včetně časových řad, protokolů a tras.

\n{4}{Prometheus}
 
Open-source systém pro sběr a vizualizaci metrik. Prometheus umožňuje sbírat metriky z různých zdrojů, včetně aplikací, systémů a služeb.

\n{4}{Loki}

Open-source systém pro sběr a vizualizaci protokolů. Loki umožňuje sbírat protokoly z různých zdrojů, včetně aplikací, systémů a služeb.

\n{4}{Tempo}

Open-source systém pro sběr a vizualizaci tras. Tempo umožňuje sbírat trasy z různých zdrojů, včetně aplikací, systémů a služeb.

\n{4}{OpenTelemetry}

je open-source systém pro sběr a vizualizaci metrik, protokolů a tras. OpenTelemetry umožňuje sbírat metriky, protokoly a trasy z různých zdrojů, včetně aplikací, systémů a služeb.

\n{3}{Testovací nástroje}

\n{4}{K6}

Nástroj pro výkonové testování, který umožňuje vývojářům testovat výkon svých aplikací. K6 umožňuje vývojářům vytvářet a spouštět testy, které simuluji reálné uživatelské scénáře. Tímto je zajištěno, že aplikace je schopna zvládnout požadavky uživatelů. K6 je nástroj, který je možné využít pro testování mikroslužeb, protože umožňuje vývojářům vytvářet testy, které simuluji reálné uživatelské scénáře.

\n{3}{Testovací služby}

Pro implementaci testovacích služeb byl zvolena technologie dotnet, konkrétně jazyk C\#. Dotnet je open-source platforma pro vývoj a provozování aplikací, která umožňuje vytvářet výkonné a škálovatelné aplikace. Služby budou implementovány jako mikroslužby, které budou spouštěny v kontejnerech. Služby jsou vytvořeny ve dvou verzích, které se liší v použitém způsobu kompilace, a to JIT a AoT.

\n{2}{Návrh a implementace testovacích služeb}

\n{3}{Předpoklady služeb}

Služby musí být implementovány tak, aby v obou kompilačních verzích poskytovaly totožnou funkcionalitu. Jejich chování musí být konfigurovatelné na úrovni kontejneru, který je spouští. Zároveň musí sbírat data o svém chování a poskytovat je monitorovacím nástrojům.

AoT kompilované služby budou otestovány s ohledem na možné kompilační optimalizace, které ovlivňují výsledný program. Toto chování je ovlivňeno atributem OptimizationPreference, který je součástí konfigurace služby.

\n{3}{Implementace služeb}

\n{4}{SRS - Signal Readings Service}

Služba v systému hraje roli čtecího zařízení, které čte data ze zdroje a poskytuje je ostatním službám. Tato služba simulu základní kámen celého systému, značně ovlivňuje výkon a škálovatelnost celého systému. Očekává se velké množství požadavků na tuto službu.

Za účelem zjednodušení implementace není využito čtení dat ze skutečného zdroje, ale jsou generována náhodná data. Načež data jsou následně poskytována se simulovaným zdržením, časově založenému na měření skutečného zdržení systému při čtení dat ze vzdáleného zdroje u obdobného systému. Tato služba je implementována jako REST API (TODO: pokud konečná implementace bude gRPC, změň tuto sekci), které poskytuje data ve formátu JSON. (TODO: Obrázek návrhu architektury a rozhraní služby).

\n{4}{FUS - File Upload Service}

Služba v systému hraje roli zapisovacího zařízení, které zapisuje data do zdroje. Tato služba hraje roli méně vytíženého služby, která nemá značný vliv na fungování systému jako celku. Požadavky, jenž musí vyřídit nejsou kritické a nutné řešit s minimální odezvou.

Služba je implementována s REST API rozhraním. (TODO: Obrázek návrhu architektury a rozhraní služby).

\n{3}{BPS - Batch Processing Service}

Služba v systému hraje roli zpracovávajícího zařízení, které zpracovává data z jiných služeb. Tato služba hraje roli služby, která je závislá na ostatních službách a zpracovává data z nich. Reaguje na požadavek o hromadném zpracování při předem definovaném splnění podmínek.

\n{3}{FRS - Fast Response Service}

Služba v systému hraje roli rychlého zpracovávajícího zařízení, které zpracovává data z jiných služeb. Tato služba hraje roli služby, která je závislá na kritických systémech 3. strany a potřebují v co nejkratším čase odpovídat.

\n{2}{Konfigurace aplikace}

\n{3}{Konfigurace služeb}

\n{4}{Nginx}

Pro nginx je dodatečná konfigurace dodána pomocí souboru nginx.conf jenž je namountován do kontejneru. Tento soubor obsahuje konfiguraci pro nginx, která je použita při spuštění kontejneru.

Základní pravidla směrování

\begin{itemize}
    \item / - cesta na statickou hlavní stránku-rozcestník aplikace
    \item /grafana - směrování na Grafanu
    \item /fus - směrování na FUS
    \item /srs - směrování na SRS
    \item /bps - směrování na BPS
    \item /frs - směrování na FRS
\end{itemize}

\n{4}{LGTM - Monitorovací stack}

LGTM jakožto monitorovací stack zároveň konfiguruje veškeré monitorovací nástroje. Značnou část konfigurace představuje propojení nástrojů a tato konfigurace je řešena pomocí konfiguračních souborů, které jsou použity již při tvorbě obecného obrazu.

Dodatečná konfigurace je řešena podle proměnných prostředí a týká se pouze malé množiny nastavení specifickýh pro správný běh monitorovacích nástrojů v celém stacku.

\begin{itemize}
    \item \textbf{GF\_SERVER\_ROOT\_URL} - nastavení URL, na které bude Grafana dostupná. Toto nastavení je důležité pro správné směrování požadavků na Grafanu.
    \item \textbf{GF\_SERVER\_SERVE\_FROM\_SUB\_PATH} - nastavení, které určuje, zda bude Grafana dostupná z podadresáře v URL. Toto nastavení je důležité pro správné směrování požadavků na Grafanu.
    \item \textbf{GF\_AUTH\_ANONYMOUS\_ENABLED} - nastavení, které určuje, zda bude povoleno anonymní přihlášení do Grafany.
\end{itemize}

\n{4}{Postgres}

X

\n{4}{SGS - Signal Generation Service}

\n{4}{FUS - File Upload Service}

\n{4}{BPS - Batch Processing Service}

\n{4}{FRS - Fast Response Service}

\n{4}{K6}

\n{3}{Nastavení uživatelského rozhraní}

%%%%%%%%%%%%%%%%%%%%%%%%%%%%%%%%%%%%%%%%%%%%%%%%%%%%%%%%%%%%%%%%%%%%%%%%%%%%%%%%%
%                             Testování scénářů                                 %
%%%%%%%%%%%%%%%%%%%%%%%%%%%%%%%%%%%%%%%%%%%%%%%%%%%%%%%%%%%%%%%%%%%%%%%%%%%%%%%%%

\n{1}{Testování scénářů}

Testování scénářů je klíčovou součástí testování výkonu a škálovatelnosti mikroslužeb. Scénáře jsou definovány jako soubor kroků, které mají být provedeny, a jsou použity k simulaci zátěže na mikroslužby. Scénáře jsou vytvořeny pomocí testovacích nástrojů, které umožňují vytvářet a spouštět testy, které simuluji reálné uživatelské scénáře.

\n{2}{Požadavky na scénáře}

Scénáře musí být vytvořeny tak, aby simulovali reálné uživatelské scénáře. To znamená, že musí být vytvořeny tak, aby obsahovaly kroky, které mají být provedeny, a musí být vytvořeny tak, aby obsahovaly data, která mají být použita.

\n{2}{Popis scénářů}

Následující sekce obsahuje popis scénářů, které byly vytvořeny pro testování výkonu a škálovatelnosti mikroslužeb kompilovaných JIT a AoT.

\n{3}{Scénář 1 - TBS}

\n{3}{Scénář 2 - TBS}

\n{2}{Zpracování a vizualizace dat}

Po provedení testování scénářů je nutné zpracovat a vizualizovat data, která byla získána. To zahrnuje zpracování dat, která byla získána z testování scénářů, a zpracování dat, která byla získána z monitorovacích nástrojů.

\n{3}{Monitorování v reálném čase}

Monitorování v reálném čase je klíčovou součástí testování výkonu a škálovatelnosti mikroslužeb. Monitorování v reálném čase umožňuje sledovat výkon a škálovatelnost mikroslužeb v reálném čase.

\n{3}{Sběr historických dat}

Sběr historických dat je klíčovou součástí testování výkonu a škálovatelnosti mikroslužeb. Sběr historických dat umožňuje sledovat výkon a škálovatelnost mikroslužeb v čase.

% \n{2}{Obrázek}
% Obrázek \ref{fig:logo} prezentuje logo Fakulty aplikované informatiky.

% % Obrázek lze vkládat pomocí následujícího zjednodušeného stylu, nebo klasickým LaTex způsobem
% % Pozor! Obrázek nesmí obsahovat alfa kanál (průhlednost). Jde to totiž proti požadovanému standardu PDF/A.
% \obr{Popisek obrázku}{fig:logo}{0.5}{graphics/logo/fai_logo_cz.png}


% \n{2}{Tabulka}
% Tabulka \ref{tab:priklad} obsahuje dva řádky a celkem 7 sloupců.

% % Tabulku lze vkládat pomocí následujícího zjednodušeného stylu, nebo klasickým LaTex způsobem
% \tab{Popisek tabulky}{tab:priklad}{0.65}{|l|c|c|c|c|c|r|}{
%   \hline
%    & 1 & 2 & 3 & 4 & 5 & Cena [Kč] \\ \hline
%   \emph{F} & (jedna) & (dva) & (tři) & (čtyři) & (pět) & 300 \\ \hline
% }


% \n{2}{Citování}
% Následuje ukázka odkazování na různé zdroje:
% \begin{itemize}
% 	\item kniha \cite{HRW1997},
% 	\item kapitola v knize \cite{Delorme2006},
% 	\item článek v odborném žurnálu \cite{Bourreau2006},
% 	\item konferenční příspěvek \cite{Judish1999},
% 	\item doktorská práce \cite{Valente2005},
% 	\item technická zpráva \cite{Fralick1997},
% 	\item webová stránka \cite{WWWCST}.
% \end{itemize}

\n{1}{Analýza aplikace}

Tato kapitola se zabývá analýzou aplikace z hlediska vývoje, výstupu a výkonu. Využívá k tomu definovanou metodiku a scénáře testování. Výsledky jsou důkladně analyzovány a závěry shrnuty v jednotlivých sekcích.

\n{2}{Architektura}

Výsledná architektura aplikace je založena na mikroslužbách. Splňuje předem definované funkční a nefunkční požadavky. V případě testovaných služeb, zapojuje základní množinu systémových knihoven a knihoven 3. stran. Po straně telemetrie, implementuje sběr a zpracování dat z různých zdrojů. Výsledná data jsou následně zpracována a uložena do databáze, dle druhu dat. Veškeré dostupné zdroje jsou uživatelsky přívětivě vizualizovány v rámci webového aplikace Grafana. Aplikační stack je testovatelný a nasaditelný na všech hlavních platformách (po sestavení se zacílením na vybranou architekturu a využitím variant služeb třetích stran s cílovou architekturou).


\n{2}{Vývojový proces}

Následující sekce popisuje vývojový proces, tak jak se týkal testovaných služeb. Vývojový proces byl založen na experimentaci a snaze využít co nejvíc dostupných knihoven a nástrojů, za cenu nutnosti řešení problémů, případně změny implementace.

\obr{Ukázka kódu s vyzualizací direktiv dle konfigurace}{fig:codesample}{1}{graphics/images/code-visual-sample.png}

\n{3}{JIT}

Vývojový proces pro kompilaci služeb JIT se zacílením na .NET runtime probíhal standarntím způsobem. Veškeré dostupné knihovny a nástroje byly plně kompatibilní s JIT kompilací. Nedošlo k žádným nepředpokládaným problémům.

Znatelný rozdíl oproti běžnému vývoji byl výběr technologií, který přihlížel k potencionální kompatibilitě s AOT a tedy řešení, které inherntně vyžadovala funkce rezervované pro využití .NET runtime, byly ihned zavrženy.

\n{4}{Výhody}

Mezi hlavní výhody se řadí zprostředkování následujícího:

\begin{itemize}
    \item  \textbf{Reflexe} - CLR umožňuje využívat reflexi, která umožňuje získat informace o kódu za běhu aplikace. Tímto je umožněno vytvářet aplikace, které jsou schopny měnit své chování za běhu.
    \item \textbf{Dynamické načítání} - CLR umožňuje dynamicky načítat knihovny za běhu aplikace. Tímto je umožněno vytvářet aplikace, které jsou schopny měnit své chování za běhu.
    \item \textbf{Větší bezpečnost} - CLR zajišťuje, že aplikace nemůže přistupovat k paměti, která jí nebyla přidělena. Tímto je zajištěna bezpečnost aplikace a zabráněno chybám, které by mohly vést k pádu aplikace.
    \item \textbf{Správa paměti} - CLR zajišťuje správu paměti pomocí GC. Tímto je zajištěno, že paměť je uvolněna vždy, když ji aplikace již nepotřebuje. Tímto je zabráněno tzv. memory leakům, které by mohly vést k pádu aplikace.
    \item \textbf{Větší přenositelnost} - CLR zajišťuje, že aplikace je spustitelná na všech operačních systémech, na kterých je dostupné běhové prostředí CLR.
\end{itemize}

\n{4}{Nevýhody}

Zatímco za nevýhody CLR se dá považovat:

\begin{itemize}
    \item  \textbf{Výkonnost} - I když určité optimalizace jsou prováděny pro konkrétní systém a architekturu, výkon CLR je nižší než výkon nativního kódu. Dalším výkonnostním měřítkem je rychlost startu aplikace, která je pro CLR vyšší než v případě nativního kódu.
    \item \textbf{Operační paměť} - CLR využívá více operační paměti, jak pro aplikaci, tak i pro běhové prostředí.
    \item \textbf{Velikost aplikace} - Přítomnost CLR nehraje zásádní roli v případě monolitických aplikací, ale v případě mikroslužeb je nutné CLR přidat ke každé službě. Tímto se zvyšuje velikost jedné aplikační instance.
\end{itemize}

\n{3}{AOT}

Kompilace do nativního kódu probíhala s průběžnými problémy. Podpora ze strany knihoven 3. stran ve spoustě případů neodopvídala deklarovaným možnostem. Vývojový proces byl značně zpomalován nutností řešení problémů, které byly způsobeny nedostatečnou podporou. Experimentace s řešeními často vyústila v nutnost změny implementace, případě v implementaci zcela vlastní.

\n{4}{Výhody}

Mezi hlavní výhody se řadí zprostředkování následujícího:

\begin{itemize}
    \item  \textbf{Výkonnost} - CLR umožňuje využívat reflexi, která umožňuje získat informace o kódu za běhu aplikace. Tímto je umožněno vytvářet aplikace, které jsou schopny měnit své chování za běhu.
    \item \textbf{Paměťová zátěž} - CLR umožňuje dynamicky načítat knihovny za běhu aplikace. Tímto je umožněno vytvářet aplikace, které jsou schopny měnit své chování za běhu.
\end{itemize}

\n{4}{Nevýhody}

Zatímco za nevýhody CLR se dá považovat:

\begin{itemize}
    \item  \textbf{Absence nástrojů z CLR} - Mnoho nástrojů, které jsou dostupné v CLR, nejsou dostupné v AOT kompilaci. Mezi tyto nástroje patří například reflexe, dynamické načítání knihoven a další.
    \item \textbf{Absence dynamického načítání} - například Assembly.LoadFile.
    \item \textbf{Bez generování kódu za běhu} - například System.Reflection.Emit.
    \item \textbf{Žádné C++/CLI} - např. System.Runtime.InteropServices.WindowsRuntime
    \item \textbf{Windows: absence COM} - např. System.Runtime.InteropServices.ComTypes
    \item \textbf{Vyžaduje trimming (ořezávání)} - má určitá omezení, je však klíčový pro rozumnou velikost výsledného programu
    \item \textbf{Kompilace do jediného souboru} 
    \item \textbf{Připojení běhových knihoven} - požadované běhové knihovny jsou součástí výsledného aplikačního souboru. To zvyšuje velikost samoteného programu ve srovnání s aplikacemi závislými na frameworku.
    \item \textbf{System.Linq.Expressions} - výsledný kód používá svou interpretovanou podobu, která je pomalejší než run-time generovaný kompilovaný kód.
    \item \textbf{Kompatibilita knihoven s AOT} - né všechny knihovny runtime jsou plně anotovány tak, aby byly kompatibilní s Native AOT. To znamená, že některá varování v knihovnách runtime nejsou pro koncové vývojáře použitelná.
\end{itemize}

\n{2}{Výstup služeb}

Samotný proces nativní AOT a JIT kompilace je různě výkonnostně náročný. Při tvorbě obrazu službeb, ale i kompilace je hlavní náročná operace \emph{restore}, která stahuje potřebné závislosti a balíčky pro projekt. Proces kompilace je vysoce závislý na specifickém HW, SW a přítomnosti závislostí. Pro účely testování byly potřebné NuGet balíčky nacachovány v systému. Následující tabulka zobrazuje přehled časové náročnosti kompilace služeb pro oba kompilační cíle. K získání času výstupu bylo využity diagnostického režimu příkazu \emph{dotnet}. Pro AOT byl použit příkaz \emph{dotnet publish -v d -c Release-AOT -r osx-x64}, pro získání výstupu JIT byl použit příkaz \emph{dotnet publish -v d -c Release-JIT -r osx-x64 --self-contained false}.


\tab{Čas kompilace služeb}{tab:priklad}{0.65}{|l|c|c|r|}{
  \hline
    & JIT (s) & AOT (s) & AOT \% nárůst \\ \hline
  \emph{SRS} & 01.99 & 19.49 & 979.3 \\ \hline
  \emph{FUS} & 03.85 & 30.36 & 788.5 \\ \hline
  \emph{BPS} & 02.02 & 20.74 & 1026.7 \\ \hline
  \emph{EPS} & 01.85 & 20.05 & 1083.7 \\ \hline
}

Velikost samotného výstupního programu je dle očekávání výrazně menší v případě JIT kompilace. To je dáno tím, že výstupní program je závislý na .NET runtime, který poskytuje dodatečnou obecnou funkcionalitu a vytváří nativní kód včetně generování typů až za běhu aplikace. Následující tabulka zobrazuje velikost služeb pro oba kompilační cíle. Pro vytvoření výstupů na základě JIT byl použit příkaz \emph{dotnet publish -c Release-JIT -r osx-x64 /p:PublishSingleFile=true --self-contained false}, pro vytvoření výstupů AOT byl použit příkaz \emph{dotnet publish -c Release-JIT -r osx-x64 /p:PublishSingleFile=true --self-contained false}.

\tab{Velikost programu služeb}{tab:priklad}{0.65}{|l|c|c|r|}{
  \hline
    & JIT (MB) & AOT (MB) & AOT \% nárůst \\ \hline
  \emph{SRS} & 05.70 & 21.40 & 375.4 \\ \hline
  \emph{FUS} & 12.40 & 28.40 & 229.0 \\ \hline
  \emph{BPS} & 06.00 & 21.80 & 363.3 \\ \hline
  \emph{EPS} & 06.00 & 21.70 & 361.6 \\ \hline
}

Sestavení obrazu je závislé na přípravu prostředí, vyhodnocení a stažení závislostí, kompilaci a publikování aplikace. Výstupné obrazy jsou založené na linuxovém systému, Alpine s .NET runtime v případě JIT výstupu služby, zredukované Ubuntu v případě nativního AOT výstupu. Z pohledu použitelnosti výsledného obrazu služeb má smysl měřit velikost výstupního obrazu. Následující tabulka zobrazuje velikost obrazu služeb pro oba kompilační cíle. Použitý příkaz je \emph{docker build -t <service>:<tag> -f Dockerfile-<target> .}, kdy \emph{<target>} představuje vybranou kompilační metodu AOT nebo JIT. Před každým sestavením byl obraz a cache smazány. I přes toto opatření není zaručena konzistentní časová náročnost sestavení obrazu.

\tab{Velikost obrazu služeb}{tab:priklad}{0.65}{|l|c|c|r|}{
  \hline
    & JIT (MB) & AOT (MB) & AOT \% zmenšení \\ \hline
  \emph{SRS} & 121.97 & 31.41 & 74.3 \\ \hline
  \emph{FUS} & 134.36 & 38.32 & 71.5 \\ \hline
  \emph{BPS} & 122.39 & 31.40 & 74.3 \\ \hline
  \emph{EPS} & 122.26 & 31.74 & 74.0 \\ \hline
}

\n{3}{Vývojové prostředí}

K vývoji byl použit IDE Rider od společnosti JetBrains. Vyzkoušena byla rovněž i práce ve Visual Studio 2022 Community Edition a Visual Studio Code s doporučenými rozšířeními od Microsoft. Všechna vývojová prostředí jsou kompatibilní, co se týče procesu kompilace respektive sestavení, jelikož to se odehrává pomocí CLI .NET.

Samotný vývoj s ohledem na práci s direktivami pro různé kompilace byl značně zjednodušen vizualicemi, jenž poskytovala vývojová prostředí Rider a Visual Studio. Obdobně byla v těchto IDE zjednodušena i analýza a hledání chyb díky integraci referencí na kód generovaný na pozadí pro kompatibilitu s AOT. V tomto ohledu Visual Studio Code zaostávalo. S ohledem na aktivní vývoj a podporu, jenž je ze strany Microsoft poskytována podpoře vývoje .NET ve Visual Studio Code (po diskontuaci produktu Visual Studio pro Mac), lze očekávat, že se tato situace v budoucnu změní.

\n{3}{Knihovny třetích stran}

Pro zjednodušení procesu vývoje a využití existující funkcionality byly využity knihovny třetích stran. Následující seznam obsahuje knihovny, které byly využity použity v rámci vývoje a zda byly kompatibilní s AOT kompilací.

\begin{itemize}
  \item \textbf{Entity Framework} - Entity framework se pyšní vysokou kompatibilitou s AOT kompilací. V rámci vývoje nebyly zaznamenány problémy, avšak následné testování se ukázalo problematické. EF jakožto plnohodnotný ORM framework stopuje stav objektu a jeho změny. Toto chování bohužel vyžaduje dynamické generování kódu, což je v rozporu s možnostmi AOT kompilovaného kódu. Vypnutí této funkcionality je pouze částečné, neb EF stále vyžaduje reflexi při vkládání nových entit do databáze.
  \item \textbf{Fluent Migrator} - Fluent Migrator je knihovna, která umožňuje verzování databáze pomocí kódu. V rámci testování bylo zjištěno, že knihovna využívá reflexi pro načítání migrací. Toto chování je v rozporu s AOT kompilací a výsledkem je chyba při spuštění migrace. Problém byl vyřešen vytvořením vlastního minimalistického migrátoru, který nepoužívá reflexi.
  \item \textbf{Grpc} - Vytváření rozhrání a modelů pro gRPC komunikaci vyžadovalo využití přístupu model first. Tento přístup využívá generátorů pro tvorbu kódu, definijucího kódového rozhraní pro .NET. Tímto je dosaženo vygenerování veškerého potřebného kódu v době kompilace a je zajištěna kompatibila s AOT. Pro definici modelu code first ovšem kombatibila s AOT není zajištěna.
  \item \textbf{Párování konfigurace} - V rámci systémové .NET knihovny je umožněno volání API, jenž načte data ze sjednocení stavu proměnných prostředi a konfiguračního souboru. Součástí API je volání metody mapující tuto konfiguraci na předem definovaný objekt. Toto chování dle dostupných informací není v rozporu s AOT kompilací a volání relevantního kódu neprodukuje AOT warning. Z testování však vyplynulo, že mapování konfigurace ne objekt bylo problematické a neprobíhalo správně. Z toho důvodu je v případě AOT kompilace za pomocí deriktivy použité přímé načtení jednotlivých hodnot z konfigurace, dle stromomvého klíče.
\end{itemize}

\n{2}{Analýza testování}

Následující sekce se zabývá analýzou testovacích scénářů a výsledků testování. Testování bylo provedeno na základě předem definované metodiky. Podkladem testů byly definované scénáře, které byly vytvořeny s ohledem na funkční a nefunkční požadavky. Při testování byl nezávisle na spuštěný test zaznaménáván stav hostitelského systému s ohledem na spuštěné kontejnery a využití systémových prostředků. Samotné služby využívaly předem definované metry ve frameworku ASP.NET pro dodatečnou diagnostiku a monitorování. Výsledky testování byly zaznamenány a analyzovány.

\n{3}{Scénář 1 - Výkonnost komunikace}

První scénář se zabíral jednoduchou funkcionalitou dotazu na healthcheck endpoint a meřením výkonu kestrel serveru u odpovědí na požadavky skrze REST API. Testování přineslo rozdílné výkonostní výsledky mezi JIT a AOT kompilací. Dle předpokladu služby s nativním kódem využívaly méně času procesoru. Paměťová stopa však u nich byla větší. Konečně, AOT služby byly schopné v průměru rychleji odpovídat. Čistá rychlost zpracování požadavku a odpovědi není v mnoha případech kritickým faktorem. Avšak v případě velkého množství požadavků, může být rozdíl v řádech milisekund zásadní.

\tab{Průměrné využití zdrojů a doba odpovědi healthcheck služeb}{tab:service_metrics}{1.0}{|l|r|r|r|r|}{
  \hline
    Služba - Režim & CPU (ms) & IO (ns) & Paměť (MB) & Doba požadavku (ms) \\ \hline \hline
  \emph{SRS-AOT} & 3.41 & 0.550 & 41.1 & 1.61 \\ \hline
  \emph{SRS-JIT} & 9.69 & 0.453 & 41.3 & 3.84 \\ \hline
  \emph{FUS-AOT} & 1.99 & 0.825 & 52.5 & 1.27 \\ \hline
  \emph{FUS-JIT} & 7.62 & 0.458 & 39.3 & 2.22 \\ \hline
  \emph{BPS-AOT} & 1.21 & 0.425 & 37.9 & 2.57 \\ \hline
  \emph{BPS-JIT} & 9.24 & 0.550 & 36.3 & 1.96 \\ \hline
  \emph{EPS-AOT} & 2.47 & 0.451 & 36.5 & 2.07 \\ \hline
  \emph{EPS-JIT} & 6.63 & 0.686 & 35.3 & 3.09 \\ \hline
}

\n{3}{Scénář 2 - Přístup k perzistenci}

Scénář se zabýval výkonností přístupu k persistenci, respektive zachystením reálného scénáře, kdy jsou data získávána a ukládána do databáze. Faktorem byla jak samotná rychlost služby v ohledu komunikace a serializace dat, tak rychlost zpracování požadavku databází. Ve výsledku je vidět výrazný rozdíl ve využití zdrojů mezi AOT a JIT verzi služby, kdy první jmenovaná je výrazně efektivnější.

\tab{Průměrné využití zdrojů službou FUS a doba odpovědi stažení a nahrání souboru}{tab:service_metrics}{1.0}{|l|r|r|r|r|}{
  \hline
  Služba - Režim & CPU (ms) & IO (ns) & Paměť (MB) & Doba požadavku (ms) \\ \hline \hline
  \emph{FUS-AOT} & 1.9 & 2.208 & 29.3 & 4.18 \\ \hline
  \emph{FUS-JIT} & 16.7 & 2.000 & 60.9 & 8.05 \\ \hline
}

V případě doby odpovědi služby je velmi znatelný rozdíl služby kompilované JIT kdy její hodnota činila 93.6 ms. Následkem JIT kompilace potřebného kódu při prvním volání byla tato doba výrazně vyšší než v dalších voláních. Oproti tomu AOT varianta služby měla i při prvním volání odpoveď srovnatelnou s průměrným voláním a to 11.8 ms.


\tab{Průměrné využití GC službou FUS}{tab:service_metrics}{1.0}{|l|r|r|r|}{
  \hline
  Služba - Režim & Alokovaná paměť (MB) & Doba běhu (ms) & Velikost objektů (MB) \\ \hline \hline
  \emph{FUS-AOT} & 25.8 & 25.9 & 15.2 \\ \hline
  \emph{FUS-JIT} & 14.0 & 5.3 & 12.9  \\ \hline
}

Přítomnost vygenerovaných typů a funkcionality v nativní AOT verzi služby má za výsledek větší alokace paměti, jenž jsou následně uvolněny, a větší doba běhu GC.

\n{3}{Scénář 3 - Výpočetní zátěž}

Za účelem zjištění výkonnosti služeb, jejich potencionálně odlišné využití systémového API byl otestován scénář výpočetní zátěže. Na jednotlivé služby byly vysílány požadavky na výpočet 40-tého Fibonacciho čísla rekurzivní metodou. Výsledky testování ukázaly konsistnetně vyšší počet zpracovaných požadavků v případě AOT kompilace. 

TODO: Foto dashboard + tabulka výsledků + stručné vysvětlení

\n{3}{Scénář 4 - Vzájemná komunikace služeb}

Komplexnější situace pro aplikaci byla simulována ve čtvrtém scénáři. Zde na základě požadavku na EPS byla vyvolána událost do RabbitMQ, načež byla zpracována službou BPS. Ta na jejím základě stáhla patřičný záznam pomocí RPC z FUS a provedla simulaci zpracování dat výpočtem Fibonacciho čísla. Situace simulovala kombinaci synchronní a asynchronní komunikace mezi službami doplněnou o výpočetní zátěž. Výsledky přiblížily služby v obou kompilačních režimech a ukázaly pohled bližší reálnému nasazení, kdy čistě výkkonostní rozdíly kompilací nehrají tak zásadní roli. 

TODO: Foto dashboard + tabulka výsledků

\n{3}{Scénář 5 - Rychlost odpovědi služby po startu}

Simulaci serveless nasazení byla vyvolána v tomto scénáři. Jednotlivé varianty služby SRS byly v rámci testu zpuštěny, kontrolovány než se dostaly do stavu \emph{healthy} a následně nad nimi zavolán dotaz na generovaná data. Výsledky ukázaly, že služba kompilované nativním AOT způsobem byly mnohem rychleji dostupné a odpovídaly na požadavky dříve, než služba kompilované do .NET runtime.

TODO: Foto dashboard + tabulka výsledků

\n{2}{Závěr analýzy}

Na základě výsledků vývoje, výstupu a testování služeb lze odpovědět na definované hypotézy následujícím způsobem:

\begin{itemize}
  \item \textbf{Hypotéza 1} - Hypotéza, že vývoj služeb s jak AOT, tak JIT kompilací je v rámci podporované funkcionality systémových knihoven a ASP.NET možný s podobným API se ukázal jako ne zcela pravdivý. Při vývoji nastaly komplikace se serializací konfigurace, na které bylo nutné reagovat využitím odlišného API. Zároveň tento způsob serializace nebyl kompilátorem označen jako potencionálně problematický. Další problémy nastaly s využitím Entity Framework. Tento ORM využívá pro provádění operací nad databází tzv. tracking, který zaznamená změny nad aplikačními objekty a podle nich tvoří výsledné databázové operace. Vypnutím trackingu bylo umožněno se na datové entity dotázat a aktualizovat je. Operace vložení nové entity však bez trackingu nebyla možná. Pro knihovny 3. stran lze obecně říci, že podpora AOT kompilace není vždy úplně zřejmá a i v situacích kdy AOT varování jsou implementovány, lze očekávat chybné chování.
  \item \textbf{Hypotéza 2} - Výsledky ukazují, že služby napsané v nativním kódu se výrazněji rychleji spouští jak na hostitelských systémech, tak ve virtualizovaném prostředí. Zároveň binární velikosti samotných aplikací jsou mnohonásobně větší, než je tomu u služeb vyžadující .NET runtime. To je ovšem kompenzováno při virtualizovaném spuštění, kdy obraz služby pro vytvoření plnohodnotného kontejneru vyžaduje mnohem méně závislotí z hlediska paměti. Výsledné obrazy jsou tedy menší a rychleji spustitelné. Hypotéza byla potvrzena.
  \item \textbf{Hypotéza 3} - Na základě dostupných metrik bylo potvrzeno, že obecně služby kompilované do nativního kódu poskytují vyšší výkon a jsou méně paměťově náročné než služby kompilované pro .NET runtime. Tento fakt je způsoben rozdílem v době, kdy se část generují typy a část funkcionality aplikace. Pro .NET runtime za běhu a pro nativní AOT při sestavení. Zároveň bylo ale pozorováno zvýšené využití GC v případě služeb kompilovaných do nativního kódu. I přes tuto dodatečnou režii byly nativní služby efektivnější a hypotéza byla potvrzena.
\end{itemize}

\nn{Závěr}

V rámci diplomové práce byly analyzovány kompilační režimy JIT a nativní AOT na platformě .NET. První částí byla rešerše, ve které byly popsány základní principy fungování platformy .NET, jejich kompilačních režimů a cílů kompilace. Následně byla popsána architektura microservice, která slouží jako primární zacílení nativních AOT aplikací a která poskytuje vzor pro testovací nasazení. V neposlední řadě byla popsána problematika testování, telemetrie a monitorovacích řešení.

Praktická část se zabývala vývojem testovacích služeb a testovací platformy.Dále byly popsány nástroje, které umožňují vytváření nativních AOT aplikací na platformě .NET. V rámci rešerše byly také popsány nástroje, které umožňují měření výkonu aplikací.

V analytické části byly výsledky praktické části popsány a vyhodnoceny. Zhodnocení vývoje probíhalo ve třech režimech: analýza vývoje, výstupu a výkonu.

Výsledkem práce je komplexní analýza použití kompilačních režimů JIT a nativní AOT. Vývojový proces při kompilaci do nativního AOT kódu se ukázal nepřívětivý. Primárně podpora knihoven 3. stran a princip interceptorů a generátorů má za vinu subjektivně neintuitivní proces debugování kódu. Samotný programový výstup vyšel dle očekávání. V porovnání byly obrazy nativních služeb výrazně paměťově efektivnější. Výsledky testování ukázaly, že na platformě .NET nativní AOT aplikace mají obecně srovnatelný výkon jako aplikace v režimu JIT. Rozdíl je znatelný v situacích, kdy je nutno využít velké množství instancí stejné služby (plyne z velikosti obrazu) a v situacích, kdy je pro systém rozhodující rychlost zpracování služby včetně spuštění (Serverless platformy). Výsledky výkonnostního testování byly zaznamenány a zpracovány do dashboardů a grafů, které jsou připraveny v uživatelské rozhraní platformy Grafana.

Dále byla vytvořena sada testovacích služeb, které slouží jako ukázka možností platformy. V neposlední řadě pro účely analýzy byla vytvořena testovací platforma, která umožňuje vytváření a nasazování testovacích služeb v kompilačním režimu JIT a nativní AOT.

Služby kompilované do nativního AOT kódu přináší specifické výkonnostní výhody za cenu kompatibility API. Vývoj kódu je s ohledem na zažité postupy a praktiky v .NET nestandartní. Využití interceptorů a generátorů je odebrána část inciativy z rukou vývojáře a vytváří se na pozadí kompilace v .NET další úroveň abstrakce. Podpora knihoven a frameworků třetích stran je omezena a nelze se spolehnout na jejich plnou funkčnost. Tím připadá na vývojáře zodpovědnost za implementaci vlastních řešení, která by jinak byla dostupná.

Většina výhod, jenž z .NET plyne souvisí s možnostmi jeho runtime prostředí. Nativní AOT kompilace má smysl ve specifických případech, jenž plynou z nutnosti rychlosti spuštění a velikosti výstupu aplikace (s přihlednutím k velikosti .NET runtime). Případy konkurenční výhody pro AOT kompilaci staví na předpokladu a tím je zájem či potřeba mít zdrojové kódy v .NET, respektive jazyce C\#. Při rozmanitém technologickém přístupu, kdy je vývojář, respektive zapojený tým schopen přijmout jiný jazyk a framework, jsou výhody AOT kompilace ztraceny, zatímco nedostatky jsou zvýrazněny. Tento předpoklad je relativně v rozporu s požadavky na poskytování cloudových služeb, kdy je očekáváno silné technické a vědomostní zázemí a flexibilní přístup k technologiím.

Nativní AOT kompilace má oproti JIT kompilaci nesporné výhody za splnění uričtých podmínek na požadavky vůči nasazení, kódu a vývojového týmu. Zaplňuje specifickou díru v portfoliu technologií platformy .NET, která je klíčová pro kompletní řešení cloudové platformy pouze s použitím této platformy. Vývojáři, kteří se rozhodnou pro AOT kompilaci, by měli být obeznámeni s těmito specifiky a měli by být schopni je zohlednit v návrhu a implementaci řešení.

V návaznosti na platformu, která v práci vznikla v rámci výkonnostního testování služeb, je možné doplnit implementaci dalších služeb, případně rozšířit stávající. Podle vzoru současného řešení lze dodat další funkcionalitu, nastavit další zdroje telemetrie, případně rozšířit možnosti vizualizace dat. Z pohledu uživatelské přívětivosti se nabízí rozšíření o webovou aplikaci využívající princip Docker outside of Docker (DooD). Tímto by bylo možné zjednodušit spouštění konkrétních testovacích scénářů v GUI. V rámci webové aplikace by bylo možné nastavit parametry testování, spustit konkrétní test a prokliknout se odkazem na relevantní dashboard v Grafaně. S ohledem na citlivé data a přístupy, které aplikace zprostředkovává, se nabízí rozšíření o autentizaci a autorizaci v případě vystavení stacku v síti. Jelikož grafické rozhraní aplikace je založeno na aplikaci Grafana, jež je schopna připojit se k externím zprostředkovatelům autentizace, bylo by vhodné zapojit službu jako Keycloak pro sjednocení autentifikace napříč stackem.


XXX Petr radí


tahle XXX XXX



\OdsazovaniOdstavcuStop


% ============================================================================ %
\seznamlit{
%   Na toto místo je třeba vložit veškeré citované bibliografické položky.
\bibitem{Price2023}PRICE, Mark J. \emph{Apps and Services with .NET 8. Second Edition}. Packt, 2023. ISBN 978-1837637133.
\bibitem{Kokosa2018}KOKOSA, K. \emph{Pro .NET Memory Management: For Better Code, Performance, and Scalability}. For Professionals By Professionals. Apress, New York, 2018. ISBN 978-1484240267.
\bibitem{Richter2012}RICHTER, J. \emph{CLR via C\#: The Common Language Runtime for .NET Programmers}. 4th ed. Microsoft Press, Redmond, Wash., 2012. ISBN 978-0735667457.
\bibitem{Pflug2023}PFLUG, Kenny. Native AOT with ASP.NET Core - Overview [online]. 2023 [cit. 2024-02-23]. Available from: https://www.thinktecture.com/en/net/native-aot-with-asp-net-core-overview/
\bibitem{Martin2018}MARTIN, Robert C. \emph{Clean architecture: a craftsman's guide to software structure and design}. Robert C. Martin series. London, England: Prentice Hall, [2018]. ISBN 978-0134494166.
\bibitem{Richardson2018}RICHARDSON, C. \emph{Microservices Patterns: With Examples in Java}. O'Reilly Media, Sebastopol, Calif., 2018. ISBN 978-1617294549.
\bibitem{Nickoloff2019}NICKOLOFF, J.; KUENZIL, S. \emph{Docker in Action}. 2nd ed. Manning Publications, Greenwich, CT, 2019. ISBN 978-1617294761.
\bibitem{Garrison2017}GARRISON, J.; NOVA, K. \emph{Cloud Native Infrastructure: Designing, Building, and Running Scalable Microservices Applications}. 1st ed. O'Reilly Media, Sebastopol, Calif., 2017. ISBN 978-1491984307.
\bibitem{Gammelgaard2021}GAMMELGAARD, C. H. \emph{Microservices for .NET Developers: A Hands-On Guide to Building and Deploying Microservices-Based Applications Using .NET Core}. 2nd ed. Apress, 2021, ISBN 978-1617297922.
\bibitem{Lock2021}LOCK, A. \emph{ASP.NET Core in Action}. 2nd ed. Manning Publications, Greenwich, CT, 2021. ISBN 978-1617298301.

}
% Pro generování literatury lze alternativně použít i příkaz "\seznamlitbib", 
% který se postará o plnohodnotné vkládání referencí pomocí "bibliography". 
% V takovém případě se využívají bibliografické údaje uložené v souboru 
% tex/literatura.bib. Ty se automaticky upravuji dle zvolené citační normy 
% (v šabloně je nastavena česká norma).
%\seznamlitbib


% ============================================================================ %
% ============================================================================ %
% Encoding: UTF-8 (žluťoučký kůň úpěl ďábelšké ódy)
% ============================================================================ %

\seznamzkr

\begin{tabular}{ll}
PC & Personal Computer \\
OS & Operační Systém \\
HW & Hardware \\
SW & Software \\
CPU & Central Processing Unit \\
RAM & Random Access Memory \\
XML & Extensible Markup Language \\
JIT & Just in Time \\
AOT & Ahead of Time \\
CLI & Command Line Interface \\
CLR & Common Language Runtime \\
IL & Intermediate Language \\
API & Application Programming Interface \\
RPC & Remote Procedure Call \\
SDK & Software Development Kit \\
IDE & Integrated Development Environment \\
GUI & Graphical User Interface \\
\end{tabular}

% ============================================================================ %
 % Seznam zkratek


% ============================================================================ %
\seznamobr  % Seznam je generován automaticky


% ============================================================================ %
\seznamtab  % Seznam je generován automaticky


% ============================================================================ %
% ============================================================================ %
% Encoding: UTF-8 (žluťoučký kůň úpěl ďábelšké ódy)
% ============================================================================ %

\listofappendices

\priloha{Obrázková příloha}

Příloha obsahuje obrázky vytvořeny v průběhu vývoje, testování a analýzy. Obrázky jsou v případě diagramů vytvořeny pomocí nástroje Graphviz a Python knihovny Diagrams mingrammer, v případě dashboardů se jedná o foto obrazovky. \\

\begin{figure}
    \centering
    \includegraphics[width=1\textwidth]{graphics/images/scenario1-dashboard.png}
    \caption{Scénář 1 - Grafana dashboard}
    \label{fig:scenario1dashboard}
\end{figure}

% ============================================================================ %
 % Prilohy


% ============================================================================ %

\end{document}

% ============================================================================ %
