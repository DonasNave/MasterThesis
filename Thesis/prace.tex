% ============================================================================ %
%
%           Šablona bakalářské/diplomové práce
%
% Autor:    Ing. Jozef Říha (2006-05-04), od té doby šablonu udržuje
%           Ing. Pavel Tomášek, Ph.D. (tomasek@utb.cz)
%
% Verze:    2021-05-04
%
% Kódování: UTF-8 (kontrolní řetězec: žluťoučký kůň úpěl ďábelšké ódy)
%
% Sazba:    pdflatex prace.tex && pdflatex prace.tex
%           (nutné dvakrát pro korektní vložení citací a jiných referencí),
%           v případě umístění literatury do externího bib souboru je třeba volat
%           pdflatex prace.tex && bibtex prace && pdflatex prace.tex && pdflatex prace.tex
%
% Tip:      Ve správně vysázeném českém textu by na konci řádku neměla zůstant
%           samotná jednopísmenná předložka či spojka. Na takové místo se vkládá
%           nezalomitelná mezera pomocí symbolu ~. Existuje program, který umí
%           zpracovat celý TeX dokument najednou podle českých konvencí:
%           http://petr.olsak.net/ftp/olsak/vlna/
%
% Pozor:    Vzhledem k požadovanému standardu PDF/A nesmí vložené obrázky 
%           obsahovat alfa kanál (průhlednost).
%
% ============================================================================ %


\documentclass[a4paper,12pt]{article}

% Definice vzhledu a nastavení se načítá z následujícího souboru (netřeba editovat)
\input{tex/UTB.tex}

% Uživatelské definice -- upravte dle požadavků
\nastavfakultu{FAI}
	% FAI  -- pro Fakultu aplikované informatiky
	% FAME -- pro Fakultu managementu a ekonomiky
	% FHS  -- pro Fakultu humanitních studií
	% FLKR -- pro Fakultu logistiky a krizového řízení
	% FMK  -- pro Fakutlu mutimediálních komunikací
	% FT   -- pro Fakultu technologickou
	% UNI  -- pro Univerzitní institut
\nastavtyp{DP}
	% BP   -- bakalářská práce
	% DP   -- diplomová práce
\nastavrok{2024}
	% zadejte rok místo "xxxx"
\nastavjazyk{CZ}
	% CZ   -- práce bude v českém jazyce
	% EN   -- práce bude v anglickém jazyce

% Lze přidat vertikalni odsazeni nad (prvni parametr) a pod (druhy parametr)
% obrázky, tabulky i rovnice/soustavy rovnic
\nastavmezerukolemobrazku{0mm}{0mm}
\nastavmezerukolemtabulek{0mm}{0mm}
\nastavmezerukolemrovnic{0mm}{0mm}

\nastavautora{Bc. Noe Švanda}
\nastavnazevcz{Analýza služeb kompilovaných v režimu Ahead-of-Time a Just-In-Time na platformě .NET}
\nastavnazeven{Název práce anglicky (max. 2 řádky)} % Jen u anglicky psané práce
\nastavabstraktcz{Tato diplomová práce analyzuje možnosti kompilace služeb v režimu Ahead-of-Time a Just-In-Time na platformě .NET. Zaměřuje se na srovnání vývojového procesu, výstupu a výkonu služeb kompilovaných v obou režimech. Za tímto účelem je vytvořena aplikace pro testování scénářů a monitorování výsledků. Výsledkem práce je analýza vývojového postupu, programového výstupu a výkonnostních dat služeb. Na jejich základě jsou porovnány výhody a nevýhody obou režimů kompilace a analyzována jejich použitelnost.}
\nastavabstrakten{This thesis analyses the possibilities of compiling services in Ahead-of-Time and Just-In-Time mode on the .NET platform. It focuses on comparing the development process, output and performance of services compiled in both modes. To this end, an application is created to test the scenarios and monitor the results. The result of the work is an analysis of the development process, program output and service performance data. Based on these, the advantages and disadvantages of both compilation modes are compared and their applicability is analyzed.}
\nastavklicovaslovacz{kompilace, Ahead-of-Time, Just-In-Time, mikroslužba, obraz, metrika, testování}
\nastavklicovaslovaen{compilation, Ahead-of-Time, Just-In-Time, microservice, image, metric, testing}

% Následující příkaz nastaví standard PDF/A-1b
\aplikujpdfa


% ============================================================================ %
\begin{document}

\titulnistrana

\zadani

\prohlaseni

\abstraktaklicovaslova

% ============================================================================ %

\clearpage

\thispagestyle{empty}
Děkuji svému vedoucímu práce, doc. Ing. Petru Šilhavému, Ph.D., za jeho cenné rady, trpělivost a ochotu věnovat mi svůj čas. Dále bych chtěl poděkovat své rodině a přátelům za podporu a pochopení během mého studia.

% ============================================================================ %
\obsah  % Obsah je generován automaticky


% ============================================================================ %
\OdsazovaniOdstavcuStart % Nastaví odsazování odstavců dle zvoleného jazyka

\nn{Úvod}
Programovací jazyky jsou základním kamenem softwarovévo vývoje respektive celého moderního světa v období informací. Představují způsob, kterým vývojář komunikuje s virtuálním prostředím OS a následně HW rozhraním. Vývoj výkonu HW, znalostí a zkušeností vývojářů a požadavků na vyvíjené systémy byl hnacím strojem technologického rozvoje. Postupným vývojem přicházeli další a další variace programovacích jazyků, některé rozdílné inkrementálně, jiné zcela diametrálně. Významným mezníkem v přístupu k tvorbě a běhu strojového kódu je vznik virtuálních strojů, které umožňují běh kódu nezávisle na HW. Tento přístup umožňuje vývojářům psát kód v jazyce, který je jim přirozený a následně jej spouštět na různých platformách.

Dotnet je platforma od společnosti Microsoft, která umožňuje vytvářet kód určený pro následnou kompilaci za běhu (Just-in-Time, dále JIT) a spuštění pomocí tzv. běhového prostředí (Common Language Runtime, dále CLR), jenž operuje jako virtuální stroj. Jedná se o relativně vyvinutou a zkušenou platformu s využitím v mnoha projektech a firmách. Přesto právě na této platformě byla dodatečně vyvinuta možnost pro PC platformy kompilace do nativního kódu (Ahead-of-Time, dále AOT), který je spouštěn přímo na OS a konkrétní architektuře HW. Tato funkce přichází do období rozmachu vývoje a migrace nativních cloudových řešení. Ty charakterizuje snaha dodávat pouze nezbytnou část infrastruktury a zpoplatnit reálnou dobu běhu systému s režií. Právě v prostředí cloudu mají nastávat situace, kdy bude využití služeb zkompilovaných do nativního kódu výhodnější. V kterých případech však opravdu takto napsaný program exceluje či selhává? A lze kvantifikacovat rozdíly mezi kopilací pro běhové prostředí JIT a AOT kompilací?

Tato práce se zabývá porovnáním vývojového procesu, charakteristik a výkonu JIT a AOT kompilovaných služeb na platformě .NET. Cílem je zjistit, zda a v jakých případech je možné využít AOT kompilace pro zvýšení výkonu a zlepšení chování aplikací. Výsledkem práce je kvantifikace, respektive srovnání výkonu a chování JIT a AOT kompilace na platformě Dotnet. Na základě těchto výsledků je možné posoudit a doporučit vhodné případy pro využití AOT kompilace.

\cast{Teoretická část}
%%%%%%%%%%%%%%%%%%%%%%%%%%%%%%%%%%%%%%%%%%%%%%%%%%%%%%%%%%%%%%%%%%%%%%%%%%%%%%%%%
%                                    .NET                                       %
%%%%%%%%%%%%%%%%%%%%%%%%%%%%%%%%%%%%%%%%%%%%%%%%%%%%%%%%%%%%%%%%%%%%%%%%%%%%%%%%%

\n{1}{Platforma .NET}

Platforma .NET od společnosti Microsoft představuje komplexní sadu nástrojů k vývoji aplikací v podporovaných jazycích. Tato platforma je multiplatformní a umožňuje vývoj pro operační systémy jako Windows, Linux, macOS ale i pro mobilní platformy a zařízení Internet of Things (dále IoT). Vývojáři mohou využívat nástroje pro vývoj webových aplikací, desktopových aplikací, mobilních aplikací a dalších. Platforma .NET je postavena na dvou hlavních nástrojích. Prvním z nich je \textit{CLR}, běhové prostředí zodpovídající za běh aplikací. Druhým nástrojem je \textit{.NET CLI} (Command Line Interface, dále CLI), konzolový nástroj, zodpovědný za interakci s dílčími nástroji platformy .NET.

\n{2}{Historie}

Využití runtime prostředí, respektive v originální podobě virtuálního stroje, má historický původ. V dřívějších dobách byli programátoři limitování nutností kompilace kódu do nativní reprezentace přímo pro architekturu systému. Kód vytvořen pro jednu konkrétní architekturu se zpravidla neobešel bez modifikací, pokud měl fungovat i na odlišné architektuře.

V průběhu 90. let 20. století představila společnost Sun-Microsystems virtuální stroj Java Virtual Machine (dále JVM). Jedná se o komponentu runtime prostředí Javy, která zprostředkovává spuštění specifického kódu, správu paměti, vytváření tříd a typů a další. Kompilací Javy do tzv. bytecode (Intermediate Language, dále IL), tedy provedením mezikroku v procesu transformace zdrojového kódu do strojového kódu, je získána reprezentace programu, jenž běží na každém zařízení s implementovaným JVM. V rámci JVM dochází k finálním krokům mezi které patří interpretace (JIT kompilace) bytecode do nativního kódu pro cílovou architekturu systému. 

Microsoft v reakci na JVM vydal v roce 2000 první .NET Framework, který umožňoval spouštět kód v jazyce C\# na operačním systému Windows. Cílem prvních verzí .NET Framework nebylo primárně umožnit vývoj pro různé zařízení a operační systémy, ale zprostředkovat lepší nástroje pro vývoj aplikací. V roce 2014 byla vydána první multiplatformní verze platformy .NET. Ta nese název .NET Core a umožňovala spouštět kód v jazyce C\# na operačních systémech Windows, Linux a macOS. 

TODO: Cite

\n{2}{Architektura}

Platforma .NET je postavena na několika klíčových komponentách, které zajišťují běh aplikací a poskytují nástroje pro vývoj aplikací. \cite{netdocs} Mezi nejdůležitější komponenty patří:

\begin{itemize}
    \item \textbf{Common Language Runtime} - CLR je základním kamenem .NET a poskytuje běhové prostředí pro spouštění aplikací na platformě. Překládá IL do nativního kódu, spravuje alokaci paměti a garbage collection (dále GC), zajišťuje zpracování výjimek (exceptions). CLR také kontroluje datové typy, interoperabilitu a zprostředkovává služby nezbytné pro spouštění nejrůznějších aplikací .NET.
    \item \textbf{.NET CLI} - Všestranný nástroj pro vývoj, kompilaci a nasazení aplikací .NET prostřednictvím rozhraní příkazové řádky. Podporuje širokou škálu operací, od vytváření projektů a správy závislostí až po testování a publikování aplikací. Prostředí .NET CLI je multiplatformní a umožňuje sjednocení rozhraní uživatelských nástrojů pro vývoj aplikací .NET.
    \item \textbf{Microsoft Build} - Microsoft Build (dále MSBuild) je engine používaný v platformě .NET, který umožňuje sestavovat aplikace a vytvářet balíčky pro nasazení. Tento nástroj používá k organizaci a řízení procesu sestavení projektový soubor \emph{csproj} na bázi Extensible Markup Language (dále XML). Tím je zajištěna kontrola nad kompilací a průběhem sestavení. V rámci procesu sestavení lze doplnit vlastní úlohy a cíle kompilace, což poskytuje flexibilitu sestavení pro komplexní procesy ve velkých projektech.
    \item \textbf{nástroje .NET Software Development Kit} (dále SDK) - Soubor nástrojů a knihoven podporujících vývoj, debugging a testování aplikací .NET. Zahrnují různé CLI a GUI nástroje, které pomáhají vývojářům spravovat práci s kódem, optimalizovat výkon a zajistit kvalitní výstup programu v platformě .NET.
    \item \textbf{Roslyn} - Roslyn je sada kompilátorů platformy .NET, která poskytuje bohaté Application Programming Interface (dále API) pro analýzu kódu. Umožňuje vývojářům používat implementace kompilátorů jazyka C\# a VB.NET jako služby. Roslyn zlepšuje výkonnost vývojářů poskytnutím funkcí jako je refaktoring, generování kódu a sémantická analýza.
    \item \textbf{NuGet} - Správce balíčků pro platformu .NET. dodává standardizovanou metodu správy externích knihoven, na nichž závisí aplikace v .NET. Zjednodušuje proces inkorporace knihoven, systémových i třetích stran, do projektu. Rovněž spravuje závislosti, čímž zajišťuje, že projekty zůstávají aktuální a kompatibilní. Tento nástroj je téměř nezbytný pro vývoj na platformě .NET, neboť umožňuje modulární vývoj softwaru.
\end{itemize}


\n{2}{Frameworky a technologie}

Platforma .NET poskytuje mnoho frameworků a technologií pro vývoj aplikací. Jednotlivé frameworky plní různé role a poskytují různé úrovně funkcionality pro vývoj aplikací. Z hlediska struktury a účelu je lze kategorizovat následujícím způsobem. \cite{netdocs}

\begin{itemize}
    \item \textbf{.NET} - Hlavní framework platformy. .NET je robustní framework pro vývoj softwaru. Podporuje tvorbu a provoz moderních aplikací a služeb. Původně známý jako .NET Framework a primárně zaměřen na prostředí Windows, s příchodem .NET Core a novějších verzí se vyvinul v modulární platformu s open-source zdrojovým kódem známou jednoduše jako .NET. Umožňuje vývojářům vytvářet aplikace, které jsou škálovatelné, výkonné a multiplatformní.
    \item \textbf{ASP.NET} - Robustní framework pro vytváření webových aplikací a služeb. Je součástí ekosystému .NET navržený tak, aby umožňoval vývoj vysoce výkonných, dynamických webových stránek, API a webových aplikací v reálném čase. ASP.NET podporuje jak webové formuláře, tak architekturu Model-View-Controller (dále MVC). S uvedením ASP.NET Core byl framework přepracován pro cloudovou nasazením škálevatolnost, vývoj napříč platformami a vysoký výkon. Poskytuje komplexní základ pro vývoj moderních webových aplikací, které lze spustit jak na Linuxu, Windows tak macOS. ASP.NET Core také představuje Blazor, který umožňuje vývojářům používat C\# při vývoji webu, což dále zvyšuje všestrannost ekosystému. Vývojářům, kteří chtějí využít .NET pro vývoj webu, poskytuje ASP.NET komplexní a flexibilní sadu nástrojů pro vytváření všech řešení, od malých webů až po složité webové platformy.
    \item \textbf{MAUI} - Moderní specializovaný framework pro vývoj aplikací napříč platformami v rámci ekosystému .NET. Umožňuje vývojářům vytvářet aplikace pro Android, iOS, macOS a Windows z jedné kódové základny. Zakládá na populárních konceptech z Xamarin.Forms a zároveň rozšiřuje jeho možnosti na desktopové aplikace. .NET MAUI zjednodušuje vývojový proces tím, že poskytuje jednotnou sadu nástrojů pro vývoj uživatelského rozhraní na všech platformách s možností přístupu k funkcím specifickým pro platformu v případě potřeby. Podporuje moderní vývojové vzory a nástroje, včetně MVVM, datové vazby a asynchronního programování, což usnadňuje vytváření sofistikovaných a citlivých aplikací. Předchůdcem MAUI je platforma Xamarin, která sloužila pro vytváření mobilních aplikací na platformě .NET. \cite{Libery2023}
    \item \textbf{Blazor} - Specializovaný framework v rámci ekosystému .NET, který zprostředkovává vývojářům tvorbu interaktivních webových uživatelských rozhraní pomocí C\# namísto JavaScriptu. Blazor může běžet na serveru (Blazor Server), kde zpracovává požadavky a komunikuje s uživatelským rozhraním pomocí knihovny SignalR, která zabezpečuje websocket komunikaci. Nebo také v prohlížeči skrz WASM, kdy dochází k přeložení C\# kódu do nativního kódu WASM a je spouštěn přímo ve webovém prohlížeči vedle tradičních webových technologií, jako jsou HTML a CSS. Umožňuje vývojářům využít znalosti .NET pro komplexní vývoj webových aplikací a vytvářet bohaté webové aplikace běžící na straně klienta v prohlížeči. Architektura Blazor je založená na komponentách a usnadňuje jejich opětovné použití pro tvorbu uživatelského rozhraní. Zároveň Blazor podporuje modulární vývojový přístup a poskytuje možnost vyvíjet webové aplikace v ekosystému .NET.
\end{itemize}

\n{3}{Knihovny}

Knihovny představují soubor funkcí a tříd, které mohou být použity při vývoji ve více aplikacích. Typicky představují logicky oddělenou a obecnou část funkcionality aplikace. Umožňují distribuovat běžnou funkcionalitu napříč různými projekty. Knihovny v .NET mohou být tvořeny binárnimi Dynamic Link Library (DLL) soubory nebo organizované jako samostatný projekt v rámci kontejneru projektů zvaném solution. K distribuci knihoven obecně dochází pomocí balíčků NuGet. Pomocí stejnojmenného nástroje jsou knihovny zabaleny a sdíleny přes internetová úložiště. 

Běžnou praxí tvůrce platformy, programovacího jazyka nebo frameworku je poskytnutí sad knihoven, které usnadňují vývoj aplikací. Zároveň tyto knihovny zpravidla implementují nejběžnější funkcionality, které mohou programátoři vyžadovat. Typicky se jedná o přístup k souborovému systému, síti, databázím, grafickému rozhraní a další. \cite{Price2023}

Následující seznam obsahuje některé z nejběžněji používaných knihoven v .NET:

\begin{itemize}
    \item \textbf{System} - Poskytuje základní třídy, typy a rozhraní, které umožňují a podporují širokou škálu operací na úrovni systému, jako jsou vstupy a výstupy (IO), vlákna, kolekce, diagnostika a další. Je nezbytná prakticky pro každou aplikaci .NET.
    \item \textbf{System.IO} - Dodáva funkcionalitu čtení z datových proudů, souborů a zápis do nich a práci se souborovým systémem.
    \item \textbf{System.Net} - Obsahuje třídy a abstrakce pro síťovou komunikaci a elektronickou poštu.
    \item \textbf{System.Data} - Zprostředkovává přístup k datovým zdrojům, jako jsou databáze nebo XML soubory, a obsahuje ADO.NET pro přístup k vybraným databázovým serverům.
    \item \textbf{System.Collections} - Rozhraní a třídy, které definují různé kolekce objektů, jako jsou seznamy, fronty, bitová pole, hašovací tabulky a slovníky.
    \item \textbf{System.Linq} - Zaštiťuje dotazování nad kolekcemi objektů pomocí Language Integrated Query (dále LINQ).
    \item \textbf{System.Threading} - Umožňuje správu vláken, synchronizační primitiva nebo například thread pool. Podporuje vývoj paralelizovaných aplikacích.
    \item \textbf{System.Security} - Spravuje ověřování, autorizaci a šifrování, a je základem pro vývoj bezpečných aplikací.
    \item \textbf{Entity Framework} - Object-Relational Mapping (dále ORM) framework, který umožňuje vývojářům pracovat s databázemi pomocí objektově orientovaného přístupu. Poskytuje abstrakci nad databázovými systémy a umožňuje vývojářům pracovat s daty pomocí objektů a tříd. \cite{netdocs}
\end{itemize}

Kromě knihoven poskytovaných společností Microsoft existuje mnoho knihoven třetích stran. Za vývojem těchto knihoven mohou stát vývojařské komunity nebo být vydány velkými společnostmi. Běžně tyto knihovny navazují na sadu funkcí poskytovaných Microsoftem a rozšiřují je o další novou funkcionalitu, nebo portují známé existující projekty do platformy .NET. Mezi příklady nejznámějších knihovny třetích stran v .NET patří Dapper, AutoMapper, Newtonsoft.Json a další.

\n{2}{Nástroje .NET}

Platforma .NET zprostředkovává širokou sadu nástrojů za účelem tvorby, sestavení a spuštění aplikace. Mezi nejdůležitější lze zařadit následující:

\n{3}{IDE}

Neméně důležitým prvkem vývoje aplikací je integrované vývojové prostředí (dále IDE). I když není povinné, pro spoustu vývojářů je jeho použití neodmyslitené. IDE je nástroj, který zprostředkovává vývoj aplikací, správu projektů, debuggování a další. IDE poskytuje uživatelské rozhraní, které umožňuje vývojářům vytvářet, upravovat a testovat kód. Zprostředkovává nástroje pro správu projektů, jako jsou sestavení, testování a publikace. Umožňuje provádět různorodé operace nad aplikací, jako je refaktorování kódu, hledání chyb a ladění.

Jedním z nejpoužívanějších IDE je Visual Studio, vyvíjené společností Microsoft. Visual Studio poskytuje prvotřídní podporu pro vývoj na platformě .NET. Mezi další populární IDE patří Visual Studio Code a JetBrains Rider.

\n{3}{Balíčky}

Pro jednoduchou distribuci knihoven, jak systémových tak třetích stran, je využíván nástroj pro správu balíčků NuGet. Projekt, jenž má být distribuován je buďto opatřen atributem \emph{<PackageOnBuild>} a sestaven nebo je využito příkazu \emph{dotnet pack}.

Takto vytvořené balíčky lze distribuovat např. přes NuGet.org, což je veřejný repozitář knihoven, který je dostupný pro všechny vývojáře. Možná je také implementace vlastních řešení. Distribuované knihovny jsou jednoduše importovatelné do projektu a umožňují snadnou správu závislostí.

\n{3}{Dokumentace}

Dokumentace je důležitou součástí vývoje aplikací. Poskytuje informace o tom, jak používat nástroje a technologie, které jsou součástí platformy .NET. Dokumentace obsahuje informace o API, knihovnách, nástrojích a dalších součástech platformy .NET. Dokumentace je dostupná online na oficiálních webových stránkách platformy .NET a obsahuje podrobné informace o mnoha aspektech vývoje aplikací. K nalezení je na webu \url{https://docs.microsoft.com/en-us/dotnet/}.

\n{3}{Jazyky}

Základem aplikace je zdrojový kód, který je v případě platformy .NET reprezentován nejčastěji jedním z podporovaných jazyků, Mezi nejčastěji využívané patří VisualBasic.NET (dále VB.NET), C\# a F\#.

\begin{itemize}
    \item \textbf{C\#} - Představuje všestranný, objektově orientovaný jazyk navržený tak, aby umožnil vývojářům vytvářet širokou škálu bezpečných a robustních aplikací, které běží na platformě .NET. Kombinuje sílu a flexibilitu C++ s jednoduchostí jazyka Visual Basic.
    \item \textbf{F\#} - Funkční jazyk, který také podporuje imperativní a objektově orientované programování. Primárně je vhodný pro vědecké aplikace a aplikace náročné na data. Zakládá na silném typování, umožňuje stručný, robustní a výkonný kód.
    \item \textbf{VB.NET} - Historický programovací jazyk vyvinutý společností Microsoft. VB, představený v roce 1991, byl navržen jako uživatelsky přívětivé programovací prostředí založené na jazyce BASIC; jeho drag-and-drop rozhraní umožňovalo snadné vytváření GUI. Tento přístup zpřístupnil programování širšímu okruhu uživatelů a kladl důraz na rychlý vývoj aplikací (RAD). VB.NET je moderní verze jazyka Visual Basic, která je implementována na platformě .NET.
\end{itemize}


\n{2}{Aplikační struktura}

Základním stavebním prvkem aplikace v .NET je soubor projektu. Jedná se o soubor na bázi XML disponující příponou \emph{.csproj}. V rámci něj dochází ke konfiguraci a deklaraci, jak bude .NET CLI, respektive nástroje sestavení, s projektem pracovat. Zároveň jsou zde definované závislosti na další projekty a knihovny. Mezi základní charakteristiky běžně určené v projektovém souboru patří verze .NET, verze projektu/assembly, seznam závislostí, konfigurace pro buildování, testování a publikaci.

Pro tvorbu složitějších aplikací je možné využít více projektových souborů. Tyto souboury jsou seskupeny pomocí speciálního solution souboru. Jedná se o soubor sloužící ke kontejnerizaci a provázání veškerých projektových a pomocných souborů, jako jsou konfigurace setavení, pomocné skripty a další. Disponuje příponou \emph{.sln}. \cite{Price2023}

Mezi další běžně používané známé konfigurační soubory patří následující:

\begin{itemize}
    \item \textbf{appsettings.json} - obsahuje nastavení aplikace
    \item \textbf{launchsettings.json} - deklaruje konfiguraci pro spuštění aplikace
    \item \textbf{Directory.Build.props} - zprostředkovává globální nastavení atributů pro všechny projekty v solution
    \item \textbf{Directory.Build.targets} - obsahuje globální nastavení cílů kompilace pro všechny projekty v solution
    \item \textbf{NuGet.config} - nastavení pro balíčkovací správce NuGet
\end{itemize}

Dalším příkladem projektového souboru je \emph{.fsproj} pro F\# projekty, \emph{.vbproj} pro Visual Basic projekty a \emph{.nuspec} pro balíčkovací soubory NuGet. Speciálně pro IDE Visual Studio je využíván soubor \emph{.dcsproj}, který obsahuje nastavení pro spustění a debuggování aplikace spuštěné v Docker kontejneru. 

\n{2}{Kompilace zdrojového kódu}

Kompilace je proces transformace zdrojového kódu do jiné podoby. Kód je zpravidla kompilován do podoby bližší cílové architektuře, ať je touto architekturou OS, případně konkrétní HW, nebo runtime prostředí (virtuální stroj). V rámci platformy .NET jsou k dispozici dva hlavní režimy kompilace zdrojového kódu: kompilace pro běhové prostředí (CLR) a kompilace do nativního kódu přímo pro cílovou architekturu (Native AOT). \cite{Price2023}

\n{3}{Cíle kompilace}

Cílem kompilace je převést zdrojový kód do podoby, kterou je možné spustit na cílovém zařízení. Platforma .NET podporuje zacílit na několik cílových zařízení, jako jsou desktopové počítače, mobilní zařízení nebo cloudové služby. Mezi podporované cíle patří:

\begin{itemize}
    \item \textbf{Desktopové počítače} - Zacílení na platformy PC probíhá několika způsoby. Aplikace obvykle běží na CLR, kde je kód kompilován do jazyka IL a poté spouštěn prostřednictvím .NET runtime, přičemž je za běhu převeden na nativní kód. Pro situace, kdy .NET runtime není nebo nemůže být přítomen lze využít kompilace AOT pro vygenerování nativního kódu.
    \item \textbf{Mobilní zařízení} - Pro mobilní vývoj poskytuje .NET MAUI, sadu nástrojů, které umožňují vývojářům psát nativní aplikace pro Android, iOS a Windows. Využívá se zde stejné sdílené kódové základny .NET. Umožňuje vývojářům používat knihovny C\# a .NET k vytváření mobilních aplikací. Tyto aplikace jsou kompilovány specificky pro každou platformu a mohou využívat nativní funkce zařízení.
\end{itemize}

\n{3}{Obecné postupy kompilace}

Proces kompilace zahrnuje několik charakteristických postupů, které jsou přizpůsobeny optimalizaci výkonu aplikací při vývoji i za běhu. Jejich cílem je zvýšení výkonu a zabezpečení aplikací, ale také zajištění větší kompatibility a efektivity programu napříč platformami. Mezi tyto postupy kompilace patří:

\begin{itemize}
    \item \textbf{Linkování} - Kompilátor během procesu sestavení aplikace propojuje množství zdrojových souborů a dll, aby vytvořily jeden spustitelný soubor nebo knihovnu. To zahrnuje řešení odkazů mezi různými dll a integraci všech požadovaných zdrojů.
    \item \textbf{Optimalizace} - Různé optimalizace IL a nativního kódu pro zlepšení výkonu. Tyto optimalizace probíhají jak během kompilace IL, tak během kompilace JIT nebo AOT do nativního kódu.
    \item \textbf{Tree shaking a trimming} - V moderních aplikacích .NET odstraňují nástroje, jako je IL Linker, během procesu sestavování nepoužívaný kód z konečné sestavy. Toto \emph{protřepávání stromu} snižuje velikost aplikace tím, že vylučuje nepotřebné knihovny a cesty ke kódu.
    \item \textbf{Vytváření metadat a manifestů} - Překladače .NET také vytvářejí metadata a soubory manifestů, které popisují obsah a závislosti sestav, což má zásadní význam pro verzování, zabezpečení a rozlišení sestav za běhu.
\end{itemize}

\n{3}{Kompilace pro CLR}

Standartním výstupem sestavení aplikace v platformě .NET je transformace zdrojového kódu z vybraného podporovaného jazyka do assembly v jazyce IL. Tento výstupní IL se v .NET konkrétně navývá Common Intermediate language (CIL) nebo také Microsoft Intermediate Language (MSIL). IL je jazyk nezávislý na platformě, který je následně kompilován do nativního kódu za běhu aplikace.

CLR je zodpovědný za interpretaci IL kódu a jeho kompilaci do nativního kódu. Kompilace do nativního kódu je prováděna v rámci běhu aplikace, což zajišťuje, že kód je optimalizován pro konkrétní architekturu systému. V případě jazyka C\# na platformě Windows slouží ke kompilaci spustitelný soubor \emph{csc.exe}. 

Výstupem kompilace pro CLR je assembly s popisnými metadaty a IL (v případě režimu Ready to Run i částečně nativním) kódem. Assembly typicky disponují příponou \emph{.dll}, případně jsou zabaleny do spustilného souboru dle cílové platformy a výstupu. Takovýto výstup je následně připraven buďto ke spuštění za pomocí CLR, případně pro využití a referenci při tvorbě dalšího .NET kódu. Kód IL je sada instrukcí nezávislá na procesoru, kterou může spustit běhové prostředí .NET (CLR). \cite{Richter2012}

Speciálním případem JIT kompilace je aplikace R2R. Zdrojový kód je při sestavení zkompilován do podoby nativního kódu pomocí nástroje crossgen, čímž vzniknou sestavy Ready To Run (dále R2R). Za běhu se sestavy R2R načtou a spustí s minimální kompilací JIT, protože většina kódu je již v nativní podobě. CLR může přesto JIT kompilovat některé části kódu, které nelze staticky zkompilovat předem. Využití je v aplikacích, které potřebují zkrátit dobu spouštění, ale zachovat určitou funkcionalitu nebo úroveň optimalizace poskytovanou JIT kompilací.

\n{3}{Kompilace do nativního kódu}

Přímá nativní AOT kompilace je proces, při kterém je kód kompilován do nativního kódu cílové architektury. Děje se tak v procesu sestavení programu ze zdrojového kódu. V případě platformy .NET je tato funkcionalita dostupná při použití jazyka C\# a speciálních projektových atributů. 

Jedná se o funkcionalitu, jenž prošla několika iteracemi. První možnosti sestavení aplikace v nativním kódu na .NET platformě byly aplikace Universal Windows Platform. Jednalo se o aplikace využívající specifické rozhraní, nativní pro produkty Microsoft. S verzí frameworku .NET 7 byly rozšířeny možnosti sestavení aplikace jako do podoby nativního kódu i pro další architektury a typy aplikací. Tato nová funkcionalita získala vyráznější podporu v roce 2023 s vydáním frameworku .NET 8. Filozofie společnosti Microsoft ohledně AOT kompilace je, že vývojáři by měli mít možnost využít AOT kompilace v platformě .NET, pokud je to pro daný scénář vhodné. Scénáře kladoucí takovéto požadavky se vyskytují především v cloudovém nasazení, respektivně při implementaci cloudové infrastruky universálně s využitím platformy .NET. \cite{Pflug2023}

Výstupem nativní AOT kompilace je v rámci .NET spustitelný soubor. Soubor nabývá formátu dle cílové architektury, jenž byla definována v procesu sestavení. Takto vytvořený soubor je možné spustit přímo bez potřeby CLR.

\n{2}{Běh kódu}

Spuštění, respektive běh kódu na HW počítačového zařízení vyžaduje instrukční sadu, které daná architektura rozumí, tedy nativní kód. V případě nativní AOT kompilace v .NET tento kód získáme již při sestavení aplikace. Při využití kompilace do IL je nutné kód získat pomocí jednoho z kompilačních způsobů podporovaného CLR. Výsledná nativní reprezentace se v obou případech spouští zavoláním vstupní metody v binárním souboru dle specifikace architektury.

\n{3}{CLR}

 CLR je běhové prostředí frameworku .NET. Poskytuje spravované prostředí pro spouštění aplikací .NET. Podporuje více programovacích jazyků, včetně jazyků C\#, VB.NET a F\#, a umožňuje jejich bezproblémovou spolupráci. Spravuje paměť prostřednictvím automatického GC, který pomáhá předcházet únikům paměti a optimalizuje využití prostředků. CLR také zajišťuje typovou bezpečnost a ověřuje, zda jsou všechny operace typově bezpečné, aby se minimalizovaly chyby při programování.

CLR je zodpovědný za několik důležitých funkcí, které zvyšují produktivitu vývojářů a výkon aplikací.

\begin{itemize}
    \item \textbf{Správa paměti} - CLR spravuje alokaci a dealokalizaci paměti, čímž zajišťuje efektivní využití systémových prostředků a zabraňuje únikům paměti (memory leaks). Obsahuje GC, který automaticky přiděluje a sbírá paměť obsazenou objekty.
    \item \textbf{Bezpečnost} - CLR poskytuje komplexní model zabezpečení, který pomáhá chránit aplikace před neoprávněným přístupem, poškozením dat a dalšími bezpečnostními hrozbami. Vynucuje zásady zabezpečení, jako je zabezpečení přístupu ke kódu (Code Access Security, dále CAS) a zabezpečení založené na rolích. Zajišťuje, aby kód byl spouštěn s příslušnými oprávněními na základě svého původu a úrovně důvěryhodnosti, čímž chrání citlivá data a systémové prostředky.
    \item \textbf{Zpracování vyjímek} - Zpracování výjimek v CLR zahrnuje detekci, propagaci a zpracování chyb a stavů, které mohou nastat během provádění programu. Mechanismus vyjímek umožňuje elegantně řešit neočekávané situace a zachovat stabilitu aplikace.
    \item \textbf{Generování typů} - CLR podporuje dynamické generování typů za běhu, což aplikacím umožňuje za běhu dynamicky vytvářet a manipulovat typy dle potřeby. To dává možnost scénářům, jako je dynamické generování kódu, kompilace kódu za běhu a dynamické vytváření objektů.
    \item \textbf{Reflexe} - Reflexe umožňuje validaci a manipulaci typů, attributů a metadat načtených z dll za běhu. Umožňuje vývojářům dotazovat se a upravovat typy a jejich attributy za běhu, dynamicky volat metody a přistupovat k informacím o metadatech. Díky tomu mají aplikace .NET využívající běhové prostředí výrazné možnosti introspekce a přizpůsobení.
\end{itemize}

Klíčovými vlasnostmi CLR jsou multiplatformnost kódu, reflexe, optimalizace kódu pro konkrétní architekturu a bezpečnost. CLR nabízí mechanismy, jako je CAS, které zabraňují neoprávněným operacím. Kompilace JIT znamená, že kód zprostředkujícího jazyka je zkompilován do nativního kódu těsně před spuštěním, což zajišťuje optimální výkon na cílovém hardwaru. CLR usnadňuje zpracování chyb v různých jazycích a poskytuje konzistentní přístup k řešení výjimek. Navíc obsahuje nástroje pro ladění a profilování, které vývojářům pomáhají efektivně identifikovat a odstraňovat problémy s výkonem. Aby mohl být kód z IL reprezentace spuštěn na systému, respektive HW stroje, musí být dodatečně kompilován. Za tímto účelem existuje v CLR několik technik, které mají využití v specifických scénářích. \cite{Richter2012}

\n{3}{Nativní kód}

Běh nativního kódu je závislý na konkrétní architektuře systému, pro které jsou nativní programové soubory vytvořeny. Nepodléhá další úpravě ze strany .NET nástrojů. Spuštění probíhá nativním příkazem operačního systému, který zprostředkuje spuštění programu.

\n{2}{Tvorba programu v .NET}

Následující část popisuje obecnou koncepci a strukturu projektu aplikace v .NET. Součástí je postup pro tvorbu a vydání projektu. Blížší pozornost bude věnována tvorbě nativního AOT projektu.

\n{3}{Obecný postup}

Vytvoření aplikace v .NET sestává z několika kroků, které zahrnují nastavení vývojového prostředí, tvorbu projektu, programování, správu závislostí, kompilaci a publikaci. Postup je následující:

\begin{enumerate}
    \item \textbf{Nastavení vývojového prostředí}: Sestává z instalace sady nástrojů .NET SDK.
    
    \item \textbf{Vytvoření projektu}: Pomocí příkazu \texttt{dotnet new} nebo skrze GUI IDE je vytvořen nový projekt a solution soubor. Součástí je výběr typu projektu, jazyka, frameworku a dalších konfiguračních parametrů.
    
    \item \textbf{Programování}: Sestává z tvorby kódu aplikace, testování a ladění.

    \item \textbf{Správa závislostí}: Pomocí nástrojů .NET CLI je možno referencovat balíčky a knihovny v rámci projektu.
    
    \item \textbf{Kompilace}: Kompilace aplikace probíhá pomocí příkazu \texttt{dotnet build}, který převede vysokoúrovňový kód do IL. V případě AOT dochází k dodatečné kompilace do nativního kódu dle cílové architektury.
    
    \item \textbf{Publikování}: Použitím příkazu příkazu \texttt{dotnet publish} dochází k vydání aplikace, tedy specifickému sestavení v konfigurovaném nastavení. \cite{Price2023}
\end{enumerate}

\n{3}{Tvorba nativního programu}

Pro tvorbu nativního programu v .NET je nutné využít speciálního atributu \emph{PublishAoT} v projektovém souboru. Tento atribut je zodpovědný za konfiguraci projektu pro nativní AOT kompilaci. Při jeho použití je nutné specifikovat cílovou architekturu, pro kterou je nativní kód vytvářen. Po kompilaci kódu do IL dochází k dodatečné kompilaci do nativního kódu, která dodává další konzolový výstup s informacemi o průběhu kompilace. Výstupem je spustitelný soubor, který je možné spustit na cílovém zařízení bez potřeby CLR. Následující seznam obsahuje klíčové termíny pro tvorbu nativního AOT programu v .NET:

\begin{itemize}
    \item \textbf{EmitCompilerGeneratedFiles} - Pokud je v projektu zapnut atribut EmitCompilerGeneratedFiles, kompilátor generuje soubory, které obsahují podrobné informace o stavu zkompilované aplikace. Ty zahrnují i meziprodukty nebo výpisy strojového kódu, které jsou cenné pro procest ladění a analýzy výstupního produktu. Pomáhají při zacílení na kompilaci do nativního AOT lépe pochopit, jak je vysokoúrovňový kód překládán do strojového kódu.
    \item \textbf{Deklarace unmanaged rozhraní} - Deklarace unmanaged nebo také nespravovaného rozhraní zahrnuje definování způsobů jakým jednotlivá rozhraní komunikují. Spravované rozhraní představuje zdrojový kód vytvářené aplikace .NET. Nespravované rozhraní jsou funkce operující mimo .NET aplikaci, například v C/C++ knihovně. Pomocí deklarací je specifikován způsob volání. U nativních AOT aplikací je důležité, aby tato rozhraní byla přesně definována a dodržována, protože jakýkoli nesoulad nebo chyba v deklaraci může vést k chybám za běhu, které se hůře diagnostikují a opravují kvůli absenci runtime prostředí a dynamickým funkcím.
    \item \textbf{Trimming} - Nebo také ořezání je proces odstranění nepotřebného kódu a zdrojů z aplikace během sestavení, za účelem snížení velikosti a zvýšení výkonu. V rámci platformy .NET představuje trimming jednu z technik postupu \emph{tree shaking}, kdy překladač analyzuje, které části zdrojového kódu jsou skutečně používány, a zbytek vyloučí z výstupního produktu. To je zvláště důležité pro aplikace, kde je úložiště nebo paměť omezená. Ořezávání pomáhá při optimalizaci a zajišťuje, aby aplikace obsahovala pouze nezbytné části a závislosti. \cite{netdocs}
\end{itemize}

\n{3}{Přehled podpory}

Následující přehled představuje rozsah funkcionality implementované v rámci .NET frameworku k datu vydání verze 8.0.0. \cite{aspnetdocs}

\begin{itemize}
    \item \textbf{REST minimal API} - Tvorba minimalistých služeb implementujících REST API.
    \item \textbf{gRPC API} - Komunikace mezi službami pomocí protokolu gRPC.
    \item \textbf{JSON Web Token Authentication} - Autentizace a autorizace za pomocí JSON Web Token (dále JWT) tokenů.
    \item \textbf{CORS} - Konfigurace Cross-Origin Resource Sharing.
    \item \textbf{HealthChecks} - Monitorování stavu aplikace.
    \item \textbf{HttpLogging} - Logování HTTP požadavků.
    \item \textbf{Localization} - Lokalizace aplikace.
    \item \textbf{OutputCaching} - Cachování výstupu.
    \item \textbf{RateLimiting} - Omezení počtu požadavků.
    \item \textbf{RequestDecompression} - Dekomprese požadavků.
    \item \textbf{ResponseCaching} - Cachování odpovědí.
    \item \textbf{ResponseCompression} - Komprese odpovědí.
    \item \textbf{Rewrite} - Přepisování adresy Uniform Resource Locator (dále URL).
    \item \textbf{StaticFiles} - Poskytování statických souborů.
    \item \textbf{WebSockets} - Komunikace pomocí WebSockets.
\end{itemize}

%%%%%%%%%%%%%%%%%%%%%%%%%%%%%%%%%%%%%%%%%%%%%%%%%%%%%%%%%%%%%%%%%%%%%%%%%%%%%%%%%
%                                 MICROSERVICE                                  %
%%%%%%%%%%%%%%%%%%%%%%%%%%%%%%%%%%%%%%%%%%%%%%%%%%%%%%%%%%%%%%%%%%%%%%%%%%%%%%%%%

\n{1}{Microservice architektura}

Při vývoji softwaru je možné aplikovat různé architektury a návrhové vzory. Za základní vysoce rozšířenou architekturu lze považovat monolitickou architekturu. V rámci ní je celá aplikace rozdělena do několika vrstev, které jsou využívány k oddělení logiky. Tato architektura je jednoduchá na vývoj a nasazení, ale může být obtížně škálovatelná a udržitelná s rostoucí složitostí aplikace. Monolitická architektura je založena na principu, že celá aplikace je spuštěna jako jednotná instance, jenž sdílí stejný kód a zdroje. Tento styl nasazení aplikace je vystaven riziku problémů s výkonem, škálovatelností a odolností.

Fundamentálně opačná je architektura microservice. Ta je založena na principu oddělení aplikace do několika samostatných služeb. Každá z těchto služeb je zodpovědná za určitou část funkcionality aplikace. Služby jsou navzájem nezávislé a komunikují mezi sebou pomocí definovaných rozhraní. Tím je zajištěno, že každá služba může být vyvíjena, nasazována a škálována nezávisle na ostatních. Tato architektura umožňuje vývojářům pracovat na menších a jednodušších částech aplikace, což zvyšuje produktivitu a umožňuje rychlejší iterace. Díky nezávislosti služeb je také možné dosáhnout vyšší odolnosti a škálovatelnosti aplikace. \cite{Martin2018}

\n{2}{Historie}

Původ microservice architektury nelze přesně definovat, důležitý moment však nastal v roce 2011, kdy Martin Fowler publikoval článek \textit{Microservices} na svém blogu. V tomto článku popsal výhody a nevýhody této architektury a zároveň popsal způsob, jakým je možné tuto architekturu využít. Dalším popularizačním momentem pro popularizaci bylo vydání knihy \textit{Building Microservices} od Sama Newmana v roce 2015. Tato kniha popisuje způsob, jakým je možné využít microservice architekturu v praxi.

Opravdový přelom přišel postupně, nástupem a popularizací virtualizace a kontejnerizace v průběhu let 2013 až 2015. Tímto bylo umožněno vytvářet a spouštět mikroslužby v izolovaných prostředích. Tímto bylo umožněno vytvářet mikroslužby, které jsou nezávislé na operačním systému a hardwaru, na kterém jsou spouštěny. Nejdůležitější v tomto ohledu je nepochybně projekt Docker, který byl vydán v roce 2013. Díky Dockeru bylo možno jednoduše definovat, vytvářet a spouštět kontejnerizované aplikace. 

\n{2}{Základní principy}

V oblasti architektury microservice existuje několik základních principů, které tento přístup odlišují od tradičnějších softwarových architektur. Tyto principy nejsou jen teoretické, mají přímý dopad na to, jak jsou služby vyvíjeny, nasazovány a udržovány. Jejich využití přispívá k tvorbě vysoce flexibilní, škálovatelné a odolné architektuře. \cite{Martin2018} \cite{Richardson2018}

\begin{itemize}
    \item \textbf{Decentralizace} - Jeden z hlavních principů microservice architektury. Její význam je, že každá služba zodpovídá za určitou část funkcionality aplikace. Služby jsou navzájem nezávislé a komunikují mezi sebou pomocí definovaných rozhraní. Tím je zajištěno, že každá služba může být vyvíjena, nasazována a škálována nezávisle na ostatních. Díky decentralizaci a izolaci služeb je možné minimalizovat dopad chyb na celý systém.
    \item \textbf{Odolnost} - Robustnost microservice architektury je definována schopností systému zůstat v provozu i přes výskyt chyb v jeho dílčích částech. To znamená, že pokud jedna služba selže, zbytek systému může pokračovat v provozu. Toho je dosaženo použitím vzorů jako je \emph{Circuit Breaker}, tedy využitím stabilních proxy služeb pro vzájemnou komunikaci služeb. Součástí odolnosti je využití vzoru opakovaného volání požadavku (\emph{Retry Policy}), jenž je obvykle doplněno o nastavení mechanismů chování na základě času nebo počtu neúspěšných volání. 
    \item \textbf{Virtualizace} - Virtualizace reprezentuje vzor provozování více operačních systémů na jednom fyzickém hardwarovém hostiteli. Tím snižuje počet potřebných fyzických strojů a zvyšuje efektivita využití zdrojů.
    \item \textbf{Kontejnerizace} - Kontejnerizace představuje proces zabalení aplikace a jejich závislosti do kontejneru. Kontejner představuje základní spustitelnou jednotku microservice architektury. Je založen na minimalistický obraz OS, k němuž jsou dodány potřebné nástroj, knihovny a výstup aplikace. Umožňuje spustit aplikaci ve vybraném prostředí s vybranou konfigurací při co nejmenší režii. Synonymem kontejnerizace je nástroj Docker, který nabízí ekosystém pro vývoj, správu a provoz kontejnerových aplikací.
    \item \textbf{Orchestrace} - Rozšířováním počtu služeb respektive kontejnerů se jejich správa stává složitou. Nástroje pro orchestraci pomáhají automatizovat nasazení, škálování a správu kontejnerů. Mezi oblíbené orchestrační nástroje patří Kubernetes, Docker Swarm a Mesos. Zejména Kubernetes se stal de facto standardem, který poskytuje robustní rámec pro nasazení, škálování a provoz kontejnerových aplikací v clusteru strojů. Řeší vyhledávání služeb, vyvažování zátěže, sledování přidělování prostředků a škálování na základě výkonu pracovní zátěže.
    \item \textbf{Škálování} - Architektura mikroslužeb zvyšuje škálovatelnost. Služby lze škálovat nezávisle, což umožňuje efektivnější využití zdrojů a zlepšuje schopnost systému zvládat velké objemy požadavků. Běžně se používá horizontální škálování, které usnadňují nástroje pro kontejnerizaci a orchestraci. To funguje na principu vytváření dodatečných instancí škálované služby a rozložení zátěže mezi tyto instance dle definovaných mechanismů (např. dle váhy, rychlosti odpovědi nebo hashe IP adresy).
\end{itemize}

\n{2}{Komponenty}

Architektura mikroslužeb rozkládá aplikace do menších, oddělených služeb, z nichž každá plní samostatnou funkci. Pro efektivní správu těchto služeb, zejména v distribuovaném prostředí, se používá několik základních komponent. Tato část se zabývá klíčovými architektonickými komponentami, které usnadňují robustní provoz, komunikaci a škálovatelnost mikroslužeb. \cite{Williams2023}

\n{3}{Obecné komponenty}

\begin{itemize}
    \item \textbf{API Gateway} - Brána, která slouží jako vstupní bod pro komunikaci s mikroslužbami. Zajišťuje směrování požadavků, autentizaci, autorizaci, zabezpečení a další funkce, které jsou společné pro všechny služby. API Gateway může také poskytovat další funkce, jako jsou cachování, transformace zpráv a řízení toku dat. Tím zjednodušuje a centralizuje správu komunikace mezi klienty a mikroslužbami.
    \item \textbf{Service Discovery} - Mechanismus, který umožňuje mikroslužbám dynamicky najít a komunikovat s ostatními službami v systému. To je důležité pro dynamické škálování, nasazování a správu služeb. Service Discovery může být implementován pomocí centrálního registru služeb nebo distribuovaného protokolu.
    \item \textbf{Load Balancer} - Služba rozděluje provoz mezi několik instancí stejné služby, aby se zajistila rovnoměrná zátěž a zvýšila odolnost proti chybám. Load Balancer může být implementován jako hardwareové zařízení nebo softwarová služba, která poskytuje rozhraní pro konfiguraci a správu zátěže.
\end{itemize}

\n{3}{Komunikační systémy}

Mikroslužby spolu komunikují skrze rozhraní prostřednictvím vybraných protokolů, nástrojů a vzorů. Mezi nejčastěji využívané patří:

\begin{itemize}
    \item \textbf{REpresentational State Transfer} (dále REST) - Představuje vysoce rozšířenou možnost komunikace mezi mikroslužbami. Využívají se při ní standardní metody protokolu Hypertext Transfer Protocol (dále HTTP) k provádění operací na rozhraní identifikovaným prostřednictvím adresy URL. Díky bezstavové povaze je rozhraní REST vysoce škálovatelné a vhodné pro veřejně přístupné služby. Má širokou podporu na různých platformách a v různých jazycích, což pomáhá zajistit interoperabilitu v rozmanitém ekosystému mikroslužeb.
    \item \textbf{Remote Procedure Call} (dále RPC) - Komunikační metoda používaná v distribuovaných systémech, včetně mikroslužeb, kdy program způsobí, že se procedura spustí v jiném adresním prostoru (obvykle na jiné virtualizované ve sdílené síti). Tato technika abstrahuje složitost síťové komunikace do jednoduchosti volání lokální funkce nebo metody. Mezi běžné implementace RPC patří generic RPC (dále gRPC), Thrift anebo Apache Avro. \cite{Sazanavets2022}
    \item \textbf{Message Broker} - Jedná se o komunikační vzor kdy broker - prostředník, spravuje asynchronní komunikaci mezi mikroslužbami pomocí front zpráv. Tato metoda odděluje mikroslužby tím, že jim umožňuje publikovat zprávy do fronty, aniž by znaly podrobnosti o tom, které služby je budou spotřebovávat. Mezi běžné zprostředkovatele zpráv patří RabbitMQ, Apache Kafka a AWS SQS. Tato komunikační architektura zvyšuje odolnost proti chybám, protože zprostředkovatel zpráv může zajistit, že zprávy nebudou ztraceny při přenosu, i když je spotřebitelská služba dočasně nedostupná. 
\end{itemize}

\n{3}{Databáze}

V microservice architektuře si každá služba obvykle spravuje vlastní databázi, což je přístup, který podtrhuje princip decentralizované správy dat. Tato izolace pomáhá službám být volně provázané a nezávisle nasaditelné, přičemž každé databázové schéma je přizpůsobeno konkrétním potřebám služby. V závislosti na případu použití mohou služby používat různé typy databází. Structured Query Language (dále SQL) pro transakční data vyžadující silnou konzistenci a vlastnosti Atomocity Consistency Isolation Duratibility (dále ACID). Nebo Not only SQL (dále NoSQL) pro flexibilnější možnosti ukládání dat, které nabývají velkých objemů, nejsou definovány schématy nebo mají specifickou vazbu například na čas. Tato různorodost databázových technologií přináší výzvy, jako je jednotný přístup k různým datovým zdrojům pomocí abstrakce ve vzoru \emph{Repository}. Další častou problematikou je udržování konzistence dat napříč službami a distribuovanými transakcemi. K řešení se využívájí specifické vzory, jako je například \emph{Saga}.

\n{3}{Bezpečnost}

Bezpečnost v architektuře microservice je velmi důležitá, protože distribuovaná povaha těchto systémů přináší mnoho zranitelných míst. Bezpečnostní prvky se zaměřují na ochranu dat při přenosu i v klidovém stavu a zajišťují, že k službám a datům mají přístup pouze oprávněné subjekty. Mezi klíčové strategie patří implementace API Gateway s vestavěnými bezpečnostními prvky, jako je ověřování, autorizace a terminace SSL. Zásadní význam mají systémy správy identit a přístupu (Identity and Access Management, dále IAM), často integrované s tokeny Open Authorization (dále OAuth) a JWT pro správu identit uživatelů a řízení přístupu na základě definovaných zásad. Zajištění šifrované komunikace mezi službami pomocí protokolů, jako je Transport Layer Security (dále TLS), může navíc chránit před odposlechem a manipulací. Zásadní jsou také účinné strategie logování, auditování a monitorování, které poskytují možnost odhalovat bezpečnostní hrozby, reagovat na ně a zmírňovat je v reálném čase. Každá z těchto složek hraje klíčovou roli při vytváření bezpečného ekosystému služeb a umožňuje robustní obranné mechanismy proti interním i externím bezpečnostním rizikům.

\n{2}{Testování}

Testování mikroslužeb je klíčové pro zajištění kvality a spolehlivosti systému. Mikroslužby lze testovat na několika následujících úrovních:

\begin{itemize}
    \item \textbf{Jednotkové testy} - Testují jednotlivé komponenty služby, jako jsou třídy, metody a funkce. Cílem je ověřit, že jednotlivé části fungují správně a splňují požadavky.
    \item \textbf{Integrační testy} - Testují komunikaci mezi službami a ověřují, že služby spolupracují správně. Zjišťují, jestli služby komunikují správně a zda jsou data přenášena a zpracovávána správně.
    \item \textbf{End-to-end testy} - Testují celý systém z pohledu uživatele. Cílem je ověřit, že systém funguje správně a splňuje požadavky uživatele.
    \item \textbf{Smoke testy} - Testují základní funkce systému, aby se ověřilo, že je systém správně sestaven, dokáže se spustit a provést základní operace.
    \item \textbf{Load testy} - Testují výkonnost systému za zátěžových podmínek. Cílem je ověřit, že systém je schopen zvládnout požadavky uživatelů a udržet výkon při zátěži.
    \item \textbf{Penetrační testy} - Testují bezpečnost systému a identifikují potenciální bezpečnostní chyby. Cílem je odhalit slabá místa v systému a zlepšit jeho odolnost proti útokům.
\end{itemize}

Automatizované testování je klíčové pro rychlé a spolehlivé nasazení. Pomáhá odhalit chyby a problémy v raných fázích vývoje a minimalizuje riziko selhání v produkci. Testování microservice architektury je však složitější než testování monolitických aplikací, protože služby jsou distribuované a navzájem závislé. To vyžaduje efektivní kombinované strategie testování. Automatizace pomáhá zjednodušit komplexní testovací strategie a zajišťuje, že jednotlivé části aplikace jsou spolehlivé. \cite{Newman2015}

\n{2}{Výhody a nevýhody}

Mezi hlavní výhody microservice architekture lze zařadit:

\begin{itemize}

\item \textbf{Přizpůsobitelnost} - Mikroslužby umožňují rychlé, modulární a spolehlivé poskytování rozsáhlých a komplexních aplikací. Týmy mohou aktualizovat určité oblasti aplikace, aniž by to mělo dopad na celý systém, což umožňuje rychlejší iterace.

\item \textbf{Škálovatelnost} - Služby lze škálovat nezávisle, což umožňuje přesnější přidělování zdrojů na základě aktuálního stavu systému. Tím je řešena problematika proměnlivého zatížení aplikace.

\item \textbf{Odolnost} - Decentralizovaná povaha služeb pomáhá izolovat selhání na jedinou službu nebo malou skupinu služeb, čímž zabraňuje selhání celé aplikace.

\item \textbf{Technologická rozmanitost} - Týmy si mohou vybrat nejlepší nástroj pro danou práci a podle potřeby používat různé programovací jazyky, databáze nebo jiné nástroje pro různé služby, což vede k optimalizovanějším řešením.

\item \textbf{Flexibilita nasazení} - Mikroslužby lze nasazovat nezávisle, což je ideální pro vzory CI/CD. Umožňuje průběžné aktualizace při minimalizaci prodlevy a minimalizaci rizika.

\item \textbf{Modularita} - Microservice architektura zvyšuje modularitu, což usnadňuje vývoj, testování a údržbu aplikací. Týmy se mohou zaměřit na konkrétní doménovou logiku, což zvyšuje produktivitu a kvalitu. Rovněž umožňuje geograficky dislokované nasazení.

\end{itemize}

Zatímco mezi nevýhody patří:

\begin{itemize}

\item \textbf{Komplexnost} - Správa více služeb na rozdíl od monolitické aplikace přináší složitost při nasazování, monitorování a řízení komunikace mezi službami.

\item \textbf{Správa dat} - Konzistence dat mezi službami může být náročná, zejména pokud si každá mikroslužba spravuje vlastní databázi. Implementace transakcí napříč rozhraními vyžaduje pečlivou koordinaci.

\item \textbf{Zpoždění sítě} - Komunikace mezi službami po síti přináší zpoždění, které může ovlivnit výkonnost aplikace. Ke zmírnění tohoto jevu jsou nutné efektivní komunikační protokoly a vzory.

\item \textbf{Provozní režie} - S počtem služeb roste potřeba orchestrace, monitorování, protokolování a dalších provozních záležitostí. To vyžaduje další nástroje a odborné znalosti.

\item \textbf{Složitost vývoje a testování} - Mikroslužby sice zvyšují flexibilitu vývoje, ale také komplikují testování, zejména pokud jde o testování \emph{end-to-end}, které zahrnuje více služeb.

\item \textbf{Integrace služeb} - Zajištění bezproblémové spolupráce služeb vyžaduje robustní správu API, řízení verzí a strategie zpětné kompatibility.

\end{itemize}

\n{2}{Nasazení založené na mikroslužbách}

Efektivní nasazení mikroslužeb je klíčové pro využití jejich potenciálních výhod, jako je škálovatelnost, flexibilita a odolnost. Tato část se zabývá různými strategiemi nasazení, které jsou pro mikroslužby obzvláště vhodné, zejména v prostředí cloud-native. Tyto strategie zajišťují, že mikroslužby lze efektivně spravovat a škálovat, dynamicky reagovat na změny zatížení a minimalizovat prostoje. \cite{Williams2023}

\n{3}{Strategie}

Existuje několik strategií nasazení, které jsou v microservice architektuře aplikovatelné:

\begin{itemize}
    \item \textbf{Jedna služba na hostitele} - Strategie zahrnuje nasazení každé služby na vlastní server, ať už virtuální, nebo fyzický. Tento přístup zjednodušuje ladění a izolaci služeb, ale může vést k nedostatečnému využití zdrojů a vyšším nákladům.
    \item \textbf{Více služeb na jednoho hostitele} - Nasazení více služeb na jednom hostiteli maximalizuje využití zdrojů a snižuje náklady. Vyžaduje však pečlivou správu, aby nedocházelo ke konfliktům a aby se služby vzájemně nerušily.
    \item \textbf{Instance služby na kontejner} - Moderní nasazení mikroslužeb často používají kontejnery (například Docker) pro umístění jednotlivých služeb. Kontejnery poskytují odlehčené, konzistentní prostředí pro každou službu, zjednodušují nasazení a škálování v různých prostředích a zajišťují, že každá služba má splněny své závislosti bez konfliktů.
\end{itemize}

\n{3}{Cloud-native nasazení}

Microservice architektura je obzvláště vhodná pro nativní cloudová prostředí, která podporují jejich dynamickou povahu. Příklady strategií cloud-native nasazení zahrnují:

\begin{itemize}
    \item \textbf{Kontejnery a orchestrace} - Nástroje jako například Kubernetes orchestrují kontejnerové služby a řídí jejich životní cyklus od nasazení až po ukončení. Kubernetes se stará o škálování, vyrovnávání zátěže a obnovu. Umožňuje také deklarativní konfiguraci a  definici infrastruktury formou kódu, což zjednodušuje správu a automatizaci nasazení. Využití kontejnerů a orchestrace umožňuje rychlé a spolehlivé nasazení mikroslužeb při maximální kontrole nad prostředím.
    \item \textbf{Mikroslužby na platformě jako služba} - Platform as a Service (dále PaaS) je typ nasazení poskytovující prostředí, kde lze mikroslužby snadno nasadit, škálovat a spravovat bez nutnosti starat se o základní infrastrukturu. Poskytovatel cloudu je zodpovědný za provoz a správu platformy, což uživatelům umožňuje soustředit se na vývoj aplikací.
    \item \textbf{Serverless} - Bezserverové výpočetní modely umožňují nasazení mikroslužeb jako funkcí (Function as a Service, dále FaaS), které se spouštějí v reakci na události. Poskytovatel cloudu spravuje prostředí, v němž jsou nasazeny a dodává rozhraní pro jejich konfiguraci. Tento model je prezentován jako vysoce škálovatelný a nákladově efektivní, protože zdroje jsou spotřebovávány pouze během provádění funkcí. \cite{Garrison2017}
\end{itemize}

%%%%%%%%%%%%%%%%%%%%%%%%%%%%%%%%%%%%%%%%%%%%%%%%%%%%%%%%%%%%%%%%%%%%%%%%%%%%%%%%%
%                                  MONITORING                                   %
%%%%%%%%%%%%%%%%%%%%%%%%%%%%%%%%%%%%%%%%%%%%%%%%%%%%%%%%%%%%%%%%%%%%%%%%%%%%%%%%%

\n{1}{Monitorování aplikace}

Monitorování aplikací je klíčovým aspektem moderního vývoje a provozu softwaru, který týmům umožňuje sledovat výkon, stav a celkové chování aplikací v reálném čase. Zahrnuje shromažďování, analýzu a interpretaci různých typů dat a informací, které zajišťují hladký a efektivní chod aplikací a umožňují rychle identifikovat a řešit případné problémy.

\n{2}{Cíle monitorování}

Cílem monitorování v kontextu mikroslužeb je poskytnout využitelné informace v několika klíčových oblastech:

\begin {itemize}
    \item \textbf{Výkonnost systému} - Monitorování se snaží zachytit kritické výkonnostní metriky, jako je latence, propustnost a chybovost. Tyto metriky pomáhají pochopit, jak dobře služby fungují za normálních podmínek a při zátěži.
    \item \textbf{Využití zdrojů} - Je důležité sledovat využití systémových prostředků včetně procesoru, paměti a diskových I/O. Poznatky o využití zdrojů pomáhají při optimalizaci výkonu aplikací a při plánování škálovacích operací.
    \item \textbf{Stav a dostupnost služeb} - Sledování stavu a dostupnosti jednotlivých mikroslužeb zajišťuje, že lze rychle identifikovat a odstranit případné problémy, aby byla zachována integrita a spolehlivost systému.
    \item \textbf{Dopad na vývoj a nasazení} - Ačkoli je monitorování spíše kvalitativní, může také poskytnout zpětnou vazbu o dopadu různých strategií nasazení a sestavení na výkon systému. To zahrnuje míru úspěšnosti nasazení, problémy vyplývající z nových nasazení a chování nových funkcí v ostrém prostředí.
\end{itemize}

\n{2}{Druhy dat}

Pro efektivní monitorování aplikace je nezbytné porozumět různým typům dat a informací, které lze shromažďovat:

\n{3}{Logy}

Protokoly jsou záznamy o událostech, ke kterým dochází v rámci aplikace nebo jejího provozního prostředí. Poskytují podrobné, časově označené záznamy o činnostech, chybách a transakcích, které mohou vývojáři a provozní týmy použít k řešení problémů, pochopení chování aplikace a zlepšení spolehlivosti systému.

\n{3}{Traces}

Trasy se používají ke sledování toku požadavků v aplikaci, zejména v distribuovaných systémech, kde jedna transakce může zahrnovat více služeb nebo komponent. Sledování pomáhá identifikovat úzká místa, pochopit problémy s latencí a zlepšit celkový výkon aplikací.

\n{3}{Metriky}

Metriky jsou kvantitativní údaje, které poskytují přehled o výkonu a stavu aplikace. Mezi běžné metriky patří doba odezvy, využití systémových prostředků (CPU, paměť, diskové I/O), chybovost a propustnost. Sledování těchto metrik pomáhá při proaktivním ladění výkonu a plánování kapacity.

\n{2}{Sběr dat}

Efektivita monitorování aplikací do značné míry závisí na schopnosti efektivně shromažďovat relevantní data.

\n{3}{Collectory}

Kolektory jsou nástroje nebo agenti, kteří shromažďují data z různých zdrojů v rámci aplikace a jejího prostředí. Mohou být nasazeny jako součást infrastruktury aplikace nebo mohou být provozovány jako externí služby. Kolektory jsou zodpovědné za shromažďování protokolů, stop a metrik a za předávání těchto dat do monitorovacích řešení, kde je lze analyzovat a vizualizovat. Efektivní sběr dat je nezbytný pro monitorování v reálném čase a pro zajištění toho, aby shromážděná data přesně odrážela stav a výkon aplikace.

\n{2}{Analýza a interpretace}

\n{3}{Vizualizace dat}

Vizualizace dat je klíčovým aspektem monitorování aplikací, který umožňuje rychle porozumět stavu a chování aplikací. Vizualizace může zahrnovat různé typy grafů, tabulek, dashboardů a dalších nástrojů, které umožňují zobrazit data v uživatelsky přívětivé podobě. Vizualizace dat umožňuje týmům identifikovat vzory, problémy a příležitosti, které by jinak mohly zůstat skryty v datech.

\n{2}{Implementace monitorování}

Implementace monitorování aplikací zahrnuje několik klíčových kroků, včetně definice klíčových metrik, výběru monitorovacích nástrojů, nasazení kolektorů a vizualizaci dat. Týmy by měly také vytvořit procesy pro řešení problémů, které byly identifikovány prostřednictvím monitorování, a pro využití dat k plánování kapacity a optimalizaci výkonu.

\n{3}{Sběr dat v monitorovaných službách}

Implementace sběru dat zahrnuje inkorporaci funkcionality monitorování a zprostředkování dat v rámci předdefinovaného rozhraní. Sběr je realizován zpravidla sérií čítačů a zapisovačů, které jsou využívány k získávání dat z různých zdrojů. Takto sbíraná datá jsou kategorizována a značkována pro identifikaci.

Realizace monitorování je zajištěna buďto použitím existujících implementací v rámci sw knihoven nebo vytvořením vlastní implementace dle potřeb aplikace a monitorovacích protokolů.

\n{3}{Nasazení služeb pro správu a kolekci dat}

Nasazení služeb pro správu a kolekci dat je zajištěno pomocí nástrojů, které jsou schopny zprostředkovat sběr dat z různých zdrojů a zároveň zajišťují jejich zpracování a zobrazení. Tímto je zajištěno, že data jsou zpracována a zobrazena v reálném čase.

\n{3}{Vizualizace dat}

Vizualizace dat je zajištěna pomocí nástrojů, které jsou schopny zobrazit data v uživatelsky přívětivé podobě. Tímto je zajištěno, že data jsou zobrazena v reálném čase a jsou přehledná a srozumitelná.

\n{2}{Konfigurace}

Konfigurace monitorování je zajištěna pomocí konfiguračních souborů, které definují chování monitorovacích nástrojů a sběr dat. Ovlivnit chování monitorovacího systému může být provedeno jak na straně monitorovacích nástrojů, respektive služeb, tak i na straně aplikací a služeb, které jsou monitorovány.

\n{2}{Závěr}

Monitorování aplikací je nezbytným nástrojem pro vývoj a provoz moderních softwarových systémů. Zahrnuje shromažďování, analýzu a interpretaci různých typů dat a informací, které umožňují týmům sledovat výkon, stav a chování aplikací v reálném čase. Tímto je zajištěno, že aplikace jsou spolehlivé, výkonné a efektivní.


\cast{Praktická část}
%%%%%%%%%%%%%%%%%%%%%%%%%%%%%%%%%%%%%%%%%%%%%%%%%%%%%%%%%%%%%%%%%%%%%%%%%%%%%%%%%
%                   Testovací aplikace                              %
%%%%%%%%%%%%%%%%%%%%%%%%%%%%%%%%%%%%%%%%%%%%%%%%%%%%%%%%%%%%%%%%%%%%%%%%%%%%%%%%%

\n{1}{Testovací aplikace}

Praktická část této práce se zaměřuje na vytvoření testovací aplikace postavené na microservice architektuře. Cílem je vytvořit, otestovat a analyzovat služby využívající JIT a nativní AOT kompilace. Rozsah funkcionality a chování aplikace je definováno množinou funkčních a nefunkčních požadavků. Dále jsou vybrány konkrétní technologie a nástroje, jenž v aplikaci doplní .NET služby. Jejich účelem je zprostředkování platformy monitorování, dodání perzistence, hostování webového rozhraní a zprostředkování testování. Samotný postup testování je definován metodikou, která zahrnuje také definici hypotéz a je podrobně popsána v následující kapitole. Tyto hypotézy jsou následně ověřeny v analytické sekci práce. Ověření probíhá v rámci experimentů, které jsou provedeny na testovací aplikaci. Data z testů jsou zprostředkovány pro analýzu a vyhodnocení. Aplikace je nasazena v kontejnerizovaném prostředí a vytvořena s ohledem na rozšiřitelnost pro otestování konkrétní doménové problematiky.

\n{2}{Požadavky na SW}

Na aplikaci jsou pro splnění účelu analýzy vývoje, výstupu a výkonu služeb kladeny přímé i nepřímé požadavky. Je klíčové navrhnout řešení, vybrat technologie a provést implementaci včetně konfigurace s ohledem na tyto požadavky. Následující část této sekce je kategorizována do funkčních a nefunkčních požadavků na SW.

\n{3}{Funkční požadavky}

Funkční požadavky definují chování, funkce a vlastnosti, které musí aplikace poskytovat. Přímo souvisejí s doménovými požadavky a zahrnují specifikace, jako je zpracování dat, provádění výpočtů nebo podpora konkrétních procesů. Funkční požadavky popisují očekávané operace systému, včetně vstupů, chování a výstupů. Jsou tak klíčové pro vývoj a testování. V případě testovací aplikace jsou požadavky se zacílením na splnění testovacích scénářů a minimalistickou simulaci scénářů. Následující seznam funkčních požadavků byl definován pro testovací aplikaci.

\begin{itemize}
  \item \textbf{Healthchecks} - Služby musí implementovat na REST API healtcheck endpoint, který bude poskytovat informace o stavu služby. Endpoint musí být dostupný na standardní adrese \emph{/health}. Návratová hodnota bude triviálně formou řetězce \emph{Healthy} a vracet HTTP Code 200 v případě, že je služba dostupná. Dostupnost je definována schopností služby přijímat požadavky.
  \item \textbf{SwaggerUI} - Pro vizualizaci a testování REST API služeb musí být implementováno grafické rozhraní SwaggerUI. SwaggerUI musí být dostupné na standardní adrese \emph{/swagger} a musí zobrazovat dostupné endpointy a umožňovat jejich testování. SwaggerUI je implementováno pouze v konfiguraci JIT kompilace a režimu Debug.
  \item \textbf{Perzistence souborů} - Aplikace musí umožňovat ukládání libovolného souboru do perzistentního úložiště. Soubor musí být uložen do PostgreSQL databáze a musí být možné ho následně stáhnout. Pro ukládání a čtení souborů musí být implementováno rozhraní REST API. Specificky pro čtení souborů musí být implementováno i gRPC rozhraní.
  \item \textbf{Generování signálů} - Aplikace musí být schopna generovat náhodné signály. Signál musí obsahovat název, jednotku a hodnotu. Pro získání generovaných signálů musí být implementováno rozhraní REST API.
  \item \textbf{Výpočet n-tého Fibonacciho čísla} - Je požadováno, aby aplikace poskytovala funkcionalitu výpočtu Fibonacciho čísla rekurzivní metodou. Tato neefektivní metoda má za účel vytvořit zátěž na systém. Pro volání výpoču a získání výsledku musí být implementováno rozhraní rozhraní API.
  \item \textbf{Asynchronní komunikace} - Aplikace musí být schopna asynchronně zpracovávat data z jiných služeb. To zahrnuje vyvolání události a její následnou konzumaci vzorem publish - subscribe. Pro implementaci asynchronní komunikace bude využito RabbitMQ. Samotné vyvolání události musí být k dispozici pomocí požadavku na REST API.
  \item \textbf{Sběr telemetrických dat z .NET služeb} - Aplikace musí být schopna sbírat a ukládat telemetrická data z .NET služeb. To zahrnuje metriky, logy a traces. Data musí být dostupná v reálném čase. Veškerá data budou strukturována dle zásad OpenTelemetry a sbírány na gRPC rozhraní této služby. Sbíraná data budou určena podstatou funkcionality služby a doplněna o množinu dostupných a relevantních metrik zprostředkovaných knihovnami OpenTelemetry.
  \item \textbf{Monitorování kontejnerů a systému} - Aplikace musí být schopna sbírat a vizualizovat data o výkonu, škálovatelnosti kontejnerů a hostitelském systému. To zahrnuje sběr a vizualizaci výkonnostních metrik. Data musí být dostupná v reálném čase a musí být perzistentně ukládána.
  \item \textbf{Testování scénářů} - Aplikace musí být schopna provádět testování scénářů, které simulují běh systému a zátěž na mikroslužby. Testovací scénáře musí být jednoduše vytvořitelné pomocí skriptů. Spouštění scénářů nad aplikací musí být možno více způsoby, napřímo pomocí nástroje nativně běžícího na hostitelském sytému, ale také kontejnerizovaným nasazením nástroje. Testovací scénáře musí být konfigurovatelné a spustitelné v manuálním a automatizovaném režimu. 
  \item \textbf{Vizualizace dat} - V rámci aplikace musí být dostupné grafické rozhraní pro vizualizaci metrik a testovacích dat. Vizualizace musí být dostupná v reálném čase a umožnit zobrazení historických dat. Je nutné, aby aplikace podporovala seskupení a filtraci dat podle druhu, značek a času. Zároveň je požadováno, aby uživatelé mohli jednoduše připojit různé zdroje dat a vytvářet vizualizace ve vlastní režii. Přístup k funkcionalitě vizualizace musí být řešen přes webové rozhraní.
  \item \textbf{Směrování} - Přístup k aplikaci bude řešen pomocí reverzní proxy. Ta bude mít vystavené rozhraní z orchestračního nástroje a její indexovací stránka bude odkazovat na vizualizační nástroj monitorovacích dat.
  \item \textbf{Konfigurace aplikace} - V rámci aplikace musí být možnost konfigurovat chování nasazených služeb. To se týká jak konfigurace komunikace mezi službami, tak i konfigurace monitorovacích nástrojů. Nastavení musí být uloženo v konfiguračních souborech ve standardním formátu dle konvencí dané služby nebo nástroje.
\end{itemize}

\n{3}{Nefunkční požadavky}

Nefunkční požadavky specifikují celkové vlastnosti systému. Definují atributy kvality, které musí systém splňovat. Nefunkční požadavky mohou zahrnovat omezení týkající se návrhu a implementace aplikace, jako jsou bezpečnostní standardy, soulad s právními a regulačními směrnicemi, doba odezvy při zpracování dat, kapacita pro souběžné uživatele, integrita dat a mechanismy převzetí služeb při selhání. Mají zásadní význam pro zajištění životaschopnosti a efektivity aplikace v provozním prostředí. Ovlivňují celkový uživatelský dojem, výkonnost systému a splnění regulačních podmínek. Následující seznam nefunkčních požadavků byl definován pro testovací aplikaci.

\begin{itemize}
  \item \textbf{Použitelnost} - Aplikace musí být snadno použitelná a přístupná pro uživatele. To zahrnuje snadnou konfiguraci a nasazení aplikace na specifickém HW a OS. Aplikace musí být dostupná na webovém rozhraní a standardních portech.
  \item \textbf{Udržitelnost} - Aplikace musí být udržitelná a snadno rozšiřitelná. To zahrnuje dodržení praktik čistého kódu a vhodných návrhových vzorů. Implementace služeb musí být založena na principu SOLID a Don't Repeat Yourself (dále DRY). Dodržování SOLID principu zajišťuje testovatelnost, rozšiřitelnost a dlouhodobou udržitelnost aplikace. Zatímco použitím DRY principu je zabráněno tvorbě duplicitního kódu. Vytvořený kód, konfigurace a skripty musí být řádně dokumentovány.
  \item \textbf{Testovatelnost} - Aplikace musí být snadno testovatelná. To zahrnuje zprostředkování nástrojů a API pro možnost definice a konfigurace vlastních testovacích scénářů. Testování musí být automatizovatelné a poskytovat možnost perzistentního ukládání výsledků testů.
  \item \textbf{Přívětivost} - Aplikace musí být přívětivá pro uživatele. To zahrnuje snadnou navigaci, přehlednost a intuitivní ovládání. Vizuální stránka aplikace musí být jasná a přehledná.
\end{itemize}

\n{2}{Požadavky na HW}

Hardware, na kterém bude aplikace provozována, musí výkonnostně dostačovat pro běh testovacích scénářů a sběr a vizualizaci dat. Týká se to primárně počtu jader, velikosti paměti a rychlosti diskového I/O. Provozované služby mají určitou základní režii, která se musí brát v potaz.

\n{2}{Organizace a správa zdrojů}

Pro správu souborů práce byl zvolen verzovací systém (Source Control Management, dále SCM) Git. Git je open-source nástroj, který umožňuje spravovat a sdílet soubory. Byl zvolen pro svou schopnost vést historii v rámci větví a s ohledem na dostupná serverová úložiště. Za účelem jednoduché organizace souborů bylo zvoleno řešení monorepozitáře. Monorepozitář je repozitář, který obsahuje veškeré soubory projektu, ale také relevantní dokumentaci, obrázky, podpůrné nástroje a zdrojové soubory diplomové práce. Následující struktura adresářů byla zvolena pro organizaci souborů.

\begin{itemize}
    \item \textbf{Documentation} - Zahrnuje podpůrné soubory aplikační dokumentace.
    \item \textbf{Source} - Obsahuje zdrojové soubory aplikace, nasazení a konfigurace.
    \item \textbf{Thesis} - Uchovává text diplomové práce, zdrojové soubory pro vytvoření práce v LaTeX a práci samotnou ve formátu pdf.
\end{itemize}

Pro sdílení veškerých souborů souvisejících s prací a jejich sdílení byl vybrán GitHub, jakožto server pro hostování repozitáře. GitHub je platforma pro verzování souborů a projektů. Navíc poskytuje dodatečné funkce jako je CI/CD, správa dokumentace a další. Repozitář projektu je veden jako veřejný s licencí MIT a je dostupný na adrese \url{https://github.com/DonasNave/MasterThesis}.

\n{2}{Návrh a implementace testovacích služeb}

Následující pasáž se zabírá návrhem a implementací testovacích služeb, které budou využity pro analýzu vývoje a výkonu jednotlivých kompilací AOT a JIT v rámci .NET. Služby jsou implementovány jako mikroslužby a podporují kontejnerizované nasazení v microservice architektuře. Každá služba reprezentuje jednu dílčí funkcionalitu a má definované rozhraní pro komunikaci s ostatními službami. Pro implementaci služeb byla vybrána z podstaty práce technologie .NET, konkrétně jazyk C\#. Verze frameworku byla zvolena .NET 8.0 (konkrétně 8.0.4) jakožto jediná verze oficiálně podporující nativní AOT kompilaci pro framework ASP.NET. Jazyk C\# je použit ve verzi 12.0.

\n{3}{Architektura}

Architektura testovacích služeb byla vytvořena s cílem minimalisticky simulovat scénáře v aplikaci. Pro splnění funkčních požadavků bylo zvoleno následující rozdělení zodpovědnosti .NET služeb:

\begin{itemize}
    \item \textbf{SRS - Signal reading service} - Simuluje roli čtecího zařízení. Generuje signály a poskytuje je ostatním službám.
    \item \textbf{FUS - File Upload Service} - Zprostředkovává datové perzistentní zapisovací zařízení. Čte nebo zapisuje soubory do PostgreSQL databáze.
    \item \textbf{BPS - Business Processing Service} - Reaguje na události publikované v RabbitMQ, do kterého je napojena jako subscriber. Provádí výpočet Fibonacciho čísla.
    \item \textbf{EPS - Event Publishing Service} - Slouží k vyvolání události, která je následně zpracována jinými službami. Je přihlášena do RabbitMQ jako publisher.
\end{itemize}

Nativního AOT kompilace kódu je deklarována použitím atributu \emph{PublishAoT} v projektovém souboru. Za účelem zajištění co největší podobnosti služeb zacílených na AOT a JIT kompilaci, bude využito vymezení konstantních hodnot v rámci projektu. Konstanty \emph{JIT} a \emph{AOT} budou využity pro rozlišení chování služeb v rámci obou kompilačních verzích. S použitím direktiv kompilátoru a zmíněných konstant bude v nutných případech docíleno rozdílného volání API při snaze zachovat totožnou funkcionalitu.

\n{3}{Očekávání vývojového procesu}

Na základě podporované funkcionality, tak jak je definována týmem .NET a popsána v rámci rešerše, je očekáváno, že vývojový proces bude probíhat bez výrazných problémů a bude možné vytvořit služby, které budou schopny zvládnout definované funkční a nefunkční požadavky. Podpora třetích stran byla předem prozkoumána v rámci dostupných dokumentací vybraných knihoven .NET. Konkrétní podoba a rozsah této podpory budou plně ověřitelné až po implementaci a otestování služeb.

\n{3}{Organizace souborů}

Organizace zdrojových souborů služeb, knihoven a pomocných souborů je řešena v rámci hlavního adresáře \emph{DTA} obsahujícího .NET solution soubor, pomocné soubory a solution složky s konkrétními projekty služeb a knihoven. Následující graf popisuje strukturu adresáře projektu.

\obr{Struktura adresáře projektu}{fig:projectstructure}{0.3}{graphics/images/folder-structure-app.png}

Každá z vyvinutých služeb využívá konkrétní .NET SDK \emph{Microsoft.NET.Sdk.Web}, které umožňuje využít třídu \emph{WebApplication} pro registraci a konfiguraci funkcionality služby a zároveň poskytuje konfigurovatelný Kestrel server pro běh programu. Za účelem zajištění jednotného přístupu k logování, metrikám a konfiguraci byly vytvořeny společné knihovny, které jsou využity ve všech službách. Následující graf vizualizuje ukázkou strukturu adresáře služby. Následný seznam popisuje vybrané složky a soubory.

\obr{Struktura adresáře služby}{fig:servicestructure}{0.3}{graphics/images/folder-structure-service.png}

\begin{itemize}
  \item \textbf{Api} - Obsahuje implementaci rozhraní služby.
  \item \textbf{Extensions} - Implementuje extension metody specifické pro doménu služby.
  \item \textbf{Monitoring} - Obsahuje statickou třídu, která drží reference na počítadla metrik.
  \item \textbf{Service} - Ve složce jsou implementovány služby, které provádějí doménovou logiku služby.
  \item \textbf{Properties} - Drží konfiguraci pro spuštění služby.
  \item \textbf{Program.cs} - Obsahuje definici a konfiguraci služby, včetně jejího vstupního bodu.
  \item \textbf{appsettings.json} - Konfigurační soubor služby.
  \item \textbf{Dockerfile-AOT} - Soubor pro tvorbu Docker obrazu pro AOT kompilaci.
  \item \textbf{Dockerfile-JIT} - Soubor pro tvorbu Docker obrazu pro JIT kompilaci.
\end{itemize}

Součástí řešení je společná konfigurace, která je využita ve všech službách. Ta je řešena jedna na úrovni solution souboru a Directory.Build.props souboru. Týká se jednotné distribuce projektových atributů pro verzi, kompatibilitu s AOT, vynucení konkrétních pravidel pro kód a analyzéry.

\n{3}{Knihovny třetích stran}

Pro implementaci funkcionality aplikace byly využity následující knihovny třetích stran:

\begin{itemize}
  \item \textbf{Npgsql} - Npgsql je open-source ADO.NET provider pro PostgreSQL, který umožňuje komunikaci s PostgreSQL databází. Npgsql poskytuje základní balíček funkcí pro vytvoření připojení na základě standardizovaného řetězce pro připojení. Tento balíček sice není plně kompatibilní s AOT kompilací, funkce které jsou využity v rámci aplikace jsou avšak kompatibilní.
  \item \textbf{Dapper} - ORM knihovna pro .NET, která umožňuje mapovat databázové struktury na C\# objekty a provádět dotazy na databázi. Dapper.AOT je dílčí knihovna, která umožňuje provádět dotazy na databázi s podporou AOT kompilace. Toho je zajištěno tím, že Dapper.AOT generuje kód pro mapování objektů v době kompilace. Využívá k tomu interceptorů a generátorů pro zachování totožného API jak v případě kódu pro JIT kompilaci. Samotný balíček Dapper.AOT obsahuje další knihovnu Dapper.Advisor, která pomáhá s analýzou zdrojového kódu včetně dotazů na databázi.
  \item \textbf{OpenTelemetry} - OpenTelemetry zprostředkovává množinu knihoven pro sběr, zpracování a export telemetrických dat. V rámci dodaného API je možno registrovat vlastní metriky, logy a traces, ale také nastavit export vybraných dat systémových knihoven a třetích stran. To se týká vybraných knihoven, které zprostředkovávají vlastní implementaci metrik OpenTelemetry.
  \item \textbf{Grpc} - Knihovny pro implementaci komunikace pomocí protokolu HTTP/2 a gRPC. Konkrétně jsou využity Grpc.AspNetCore v případě serveru, Grpc.Net.Client pro klienta a Google.Protobuf s Grpc.Tools pro generování modelů v přístupu code first.
  \item \textbf{RabbitMQ} - Asynchronní komunikace a implementace publish subscribe vzoru je umožněna knihovnou RabbitMQ.Client. S její pomocí aplikace komunikují s brokerem, vytváří fronty, dochází k přihlášení nebo odběru zpráv a jejich publikování.
  \item \textbf{Swagger} - Grafické rozhraní pro vizualizaci a testování REST API služeb. Swagger je využit pouze v kombinaci konfigurací \emph{JIT Debug}. K tomuto účelou jsou využity knihovny Swashbuckle.AspNetCore a Microsoft.AspNetCore.OpenApi.
\end{itemize}

\n{3}{Společné knihovny}

V rámci zjednodušení tvorby služeb, jednotné implementace a konfigurace, ale také z důvodu zajištění některé základní klíčové funkcionality, byly vytvořeny společné knihovny. Tyto knihovny obsahují společné třídy, rozhraní a konfigurace, které jsou použity ve všech službách. Následující výčet popisuje oblasti funkcionality, které jsou zprostředkovány společnými knihovnami.

\begin{itemize}
  \item \textbf{Perzistence} - Pro implementaci perzistence byla vytvořena pomocná knihovna DTA.Extensions.Postgres, která poskytuje pomocnou funkcionalitu pro zajištění existence databáze pro službu, dle konfigurace v řetězci pro připojení.
  \item \textbf{Migrace} - Zajištění migrace databáze bylo implementováno po vlastní ose minimalistickým migrátorem v knihovně DTA.Migrator. Tato knihovna poskytuje základní funkcionalitu pro vytvoření databáze, vytvoření tabulek a indexů, ale také zajištění migrace dat a verzování změn.
  \item \textbf{Telemetrie} - Knihovna DTA.Extensions.Telemetry zprostředkovává extensions metody pro jednotnou a jednoduchou registraci sběru a export telemetrických dat napříč službami.
  \item \textbf{Modely} - Knihovna DTA.Models obsahuje společné modely, které jsou využity ve službách. Je tím docílena dostupnost datových struktur a rozhraní aplikace napříč všemi službami.
  \item \textbf{Obecná funkcionalita} - Za účelem sjednocení funkcionality využité napříč všemi službami jsou v rámci DTA.Extensions.Common knihovny implementovány specifické extension metody. Poskytnuta je funkcionalita pro extrakci názvů a verzí z metadat služby.
\end{itemize}

\n{3}{SRS - Signal reading service}

Za účelem simulace funkce čtecího zařízení byla vytvořena služba SRS. Tato služba poskytuje základní rozhraní pro získání dat signálu včetně jednotek formou REST API. Pro zjednodušení implementace není využito čtení dat ze skutečného zdroje, ale jsou generována náhodná data. Načež data jsou následně poskytována se zdržením simulujícím čtení dat ze vzdáleného zdroje. Služba poskytuje následující rozhraní

\begin{itemize}
    \item \textbf{GET /api/signals/\{int:amount\}} - Vygeneruje zadané množství náhodných signálů
\end{itemize}

\n{3}{FUS - File Upload Service}

Služba v systému hraje roli rozhraní k persistentnímu uložišti, v rámci kterého čte a zapisuje data. Jakožto úložiště je využito PostgreSQL databáze. Služba využívá vlastní databázovou instanci a spravuje vlastní tabulky pomocí migrací definovaných SQL skripty. Pro přístup k perzistence dat je využito knihovny knihovny Dapper, která umožňuje mapování databázových struktur na C\# objekty a vytváření a provádění dotazů na databázi. SRS poskytuje rozhraní formou REST API pro zápis a čtení dat. Daty je myšlen libovolný soubor v libovolném formátu. Samotná podstata nahraných dat není pro službu důležitá, ale to že jsou uložena do databáze. Za účelem sehrání testovacích scénářů poskytuje služba také gRPC rozhraní, které je zajištěno na dedikovaném portu. V rámci gRPC komunikace slouží FUS jako server, který zpracovává volání vzdálené procudery. Služba poskytuje následující rozhraní:

\begin{itemize}
    \item \textbf{GET /api/file/download/\{int:id\}} - Stáhne soubor podle zadaného ID.
    \item \textbf{POST /api/file/upload} - Nahraje soubor do systému. Soubor je předán jako multipart/form-data.
    \item \textbf{gRPC Operation FileServer.GetFile} - Stáhne soubor podle zadaného objektu s ID.
\end{itemize}

\n{3}{BPS - Business Processing Service}

Pro splnění role vytvoření zátěže na systém je vytvořena BPS. Tato služba získává data pomocí volání gRPC, konzumuje jako subsciber událost a provádí náročnou výpočetní operaci, kterou simuluje obtížnou doménovou operaci. Konkrétně implementaváno je neefiktivní rekurzivní výpočet zadaného čísla Fibonacciho posloupnosti. BPS se po spuštění přihlašuje k odběru zpráv na předem definovaný kanál \emph{simulated} na službe RabbitMQ. Po získání zprávy volá vzdálenou proceduru nad FUS. Následně provádí výpočet 40-tého Fibonacciho čísla. Tato hodnota je pevná s ohledem na její jediné využití a zacílení pro scecifický testovací scénář. Služba poskytuje následující rozhraní:

\begin{itemize}
    \item \textbf{GET /api/processFibonacci/\{int:degree\}} - Vypočítá číslo z Fibonacciho posloupnosti na zadané pozici náročným rekurzivním způsobem.
    \item \textbf{Event subscribed: <queue-name>\_simulated} - Přihlášení k odběru zpráv v rámci kanálu na službě RabbitMQ.
\end{itemize}

\n{3}{EPS - Event Publishing Service}

Jednoduchá služba umožňující vyvolat událost ve frontě a docílit spuštění dodatečných operací v aplikaci. V systému simuluje roli uživatele vyvolávajícího událost. Služba poskytuje následující rozhraní:

\begin{itemize}
    \item \textbf{GET api/simulateEvent/\{int:id\}} - Vyvolá simulovanou událost s daným ID.
    \item \textbf{Event published: <queue-name>\_simulated} - Vyvolá údalost se zprávou obsahující identifikator na konfigurovaném kanálu do služby RabbitMQ.
\end{itemize}

Následující diagram znázorňuje přímé závislosti testovacích služeb na další nástroje.

\obr{Diagram .NET služeb a závislých služeb}{fig:logo}{0.75}{graphics/diagrams/services-architecture.png}

\n{2}{Monitorování aplikace}

Za účelem monitorování aplikace byla vybraná množina nástrojů. Tyto nástroje umožňují sběr, perzistenci a vizualizaci metrik, traces a logů. Klíčové bylo zajistit možnost sledovat dění uvnitř aplikace, ale i v rámci hostitelského systému. Následující pasáž se zabývá výběrem a implementací monitorovacích nástrojů.

\n{3}{Grafana observability stack}

Pro monitorování aplikace byl zvolen Grafana Observability stack pro jeho pokrytí komplexní škály monitorovacích dat. Zahrnuje specifické nástroje pro sběr, vizualizaci a analýzu dat. Zprostředkovává jednoduchou možnost propojení dílčích nástrojů a konfiguraci datových zdrojů. V neposlední řadě poskytuje rozsáhlé možnosti vizualizace. Následující nástroje jsou součástí Grafana Observability stacku.

\begin{itemize}
  \item \textbf{Grafana} - Grafana je open source webová aplikace pro analýzu a interaktivní vizualizaci dat. Poskytuje možnost sestavit dashboard z komponent jako jsou grafy, tabulky a další. Jedná se o velmi populární technologii v doménách serverové infrastruktury a monitorování. Grafana umožňuje sjednotit monitorovací služby a zobrazit data v reálném čase. Podporuje širokou škálu datových zdrojů, jako jsou Prometheus, InfluxDB, Tempo, Loki nebo PostgreSQL, což umožňuje jednoduchou konfiguraci a připojení monitorovacích dat aplikace. Zároveň nativně podporuje datový zdroj, kde ukládá data testovací nástroj aplikace. Kombinací dat z různých zdrojů umožňuje vytvářet komplexní pohled na celý systém.
  \item \textbf{Prometheus} - Open-source monitorovací systém. Shromažďuje a ukládá metriky jako time-series data a umožňuje se na ně dotazovat pomocí vlastního výkonného jazyka PromQL. Jeho architektura podporuje více modelů získávání dat, což je využito při napojení více zdrojů v aplikaci. Dále umožňuje přímé stahování metrik z cílových služeb nebo kolektorů, odesílání metrik přes gateway a zprostředkování notifikací.
  \item \textbf{Loki} - Škálovatelný agregátor logů. Na rozdíl od obdobných systémů pro agregaci logů, jenž indexují všechna data, Loki indexuje pouze metadata, přičemž ukládá celá data logu efektivním způsobem. Loki je navržen tak, aby jednoduše spolupracoval s Grafanou a umožňuje rychle vyhledávat a vizualizovat logy.
  \item \textbf{Tempo} - Poskytuje jednoduše ovladatelný open-source nástroj pro sledování distribuovaných požadavků. Tempo podporuje ukládání a načítání traces. Na rozdíl od mnoha jiných nástrojů pro traces, nevyžaduje Tempo žadné předem definované schéma. Je navržen tak, aby se bezproblémově integroval s Prometheus, Grafanou a aby jednoduše přijímal data z OpenTelemetry kolektoru.
  \item \textbf{OpenTelemetry} - Open source kolektor monitorovacích dat. Poskytuje jednotný způsob sběru, zpracování a exportu dat. Je konfigurovatelný a podporuje více pipeline, které mohou upravovat telemetrická data při jejich sběru. Výrazně zjednodušuje instrumentaci služeb, protože umožňuje agregovat a exportovat metriky, traces a logy do různých analytických a monitorovacích nástrojů. Poskytuje podporu pro export dat do Prometheus, Tempo i Loki.
\end{itemize}

Implementace Grafana Observability stacku je zajištěna pomocí obrazu nazvaného dta-lgtm a sestrojeného po vzoru Grafana LGTM (Loki, Grafana, Tempo a Mimir). Grafana LGTM kombinuje množinu monitorovacích nástrojů v rámci jediného obrazu s předchystanou konfigurací. Tím je odstíněna část konfigurace monitorovacího stacku a zjednodušen proces nasazení. Obraz použitý v rámci práce využívá kombinaci dříve zmíněných technologií (Loki, Tempo, Grafana, Prometheus a OpenTelemetry) zabalených a předkonfigurovaných v rámci Docker obrazu. Tím je v aplikaci zajištěno, že potřebná konfigurace pro vzájemné propojení nástrojů a datových zdrojů je předpřipravena. Stejně tak jsou součástí vytvořeného obrazu vlastní monitorovací dashboardy pro vizualizace.

V rámci aplikace mají jednotlivé služby nastaven export svých logů, traces a metrik do OpenTelemetry, respektive na adekvátní rozhraní dta-lgtm. Služby využívají existujících metrů a logů, ale také vytváří vlastní metriky a logy. Vlastní metriky zahrnují informace o počtu a druhu provedených operací. Z předpřipravených metrik, ať systémových nebo třetích strán jsou využity následující instrumentace:

\begin{itemize}
  \item \textbf{System.Runtime} - Metriky běhového prostředí .NET.
  \item \textbf{System.Net.Http} - Metriky HTTP dotazů.
  \item \textbf{Microsoft.AspNetCore.Hosting} - Metriky hostovacího prostředí ASP.NET Core.
  \item \textbf{Microsoft.AspNetCore.Server.Kestrel} - Metriky serveru Kestrel.
  \item \textbf{Npgsql} - Metriky klientské knihovny PostgreSQL.
\end{itemize}

\n{3}{Monitorování hostitelského systému}

Monitorování hostitelského systému poskytuje pro aplikaci klíčové informace o využití zdrojů a výkonu, jak ze samotnéhé systému, tak i z jednotlivých kontejnerů. Pro monitorování Docker kontejnerů je využit nástroj CAdvisor. CAdvisor je schopen monitorovat kontejnery běžící na Dockeru, Kubernetes nebo jiných kontejnerových platformách. Poskytuje informace o využití procesoru, paměti, sítě a diskového I/O z pohledu hostitelského systému. Dalším využitým nástrojem je NodeExporter. NodeExporter je nástroj pro sběr metrik z hostitelského systému. Obdobně jako CAdvisor poskytuje informace o využití procesoru, paměti, sítě a diskového I/O. Data z obou zmíněných nástrojů jsou exportována do Prometheus, kde jsou následně zpracována a zprostředkována Grafaně.

\n{2}{Testovací nástroje}

Za účelem testování monitorovacího stacku byl vybrán nástroj K6. Jedná se o moderní open-source nástroj pro zátěžové testování. Slouží k vytváření a spouštění výkonnostních testů nad aplikací. Nabízí bohaté API pro těchto testů v jazyce JavaScript. Umožňuje psát komplexní scénáře napodobující reálný provoz systému nebo simulovat hraniční situace. K6 podporuje různé systémové metriky, jako je doba odezvy, propustnost a chybovost. Nabízí šiřoké možnosti rozšíření skrze API, což umožňuje přizpůsobení a integraci s dalšími nástroji pro komplexní sledování výkonu. K6 obsahuje nativní integraci exportu výsledku testování do databáze InfluxDB ve verzi 1. Pro zjednodušení procesu testování a zajištění opakovatelnosti testovacích scénářů byly vytvořeny skripty pro spuštění testů. Tyto skripty zajišťují opakovatelné spouštění testů v rámci Docker kontejneru s nastavitelnými parametry. Skripty jsou vytvořeny v jazyce Bash a využívají nástroje K6 pro spuštění testů a zpracování výsledků. Zároveň pracují s Docker Compose orchestrací pro spouštění a vypínání služeb dle potřeby testů.

\n{2}{Nasazení aplikace}

Následující pasáž popisuje náležitosti nasazení aplikace, včetně obecného popisu finální struktury nasazení, využitých nástrojů kontejnerizace a orchestrace, konkrétních verzí obrazů a použitou konfiguraci.

\n{3}{Přehled řešení}

Řešení aplikace sestává z následujících sekcí a jednotlivých obrazů služeb a verzí, definovaných v rámci Docker Compose souboru.

\begin{itemize}
    \item \textbf{Testovací služby} - Aplikace obsahuje testovací .NET služby FUS, SRS, BPS a EPS. Tyto služby jsou vytvořeny ve dvou kompilačních verzích - AOT a JIT. Každá služba je vytvořena jako obraz s názvem \emph{dta-<service-name>} a značkou \emph{<compilatioinMode>-latest}.
    \item \textbf{Komunikace} - Komunikační kanál mezi službami je zajištěn pomocí RabbitMQ. Pro RabbitMQ je využit obraz \emph{rabbitmq:3-management-alpine}.
    \item \textbf{Monitorovací nástroje} - Monitorování zajišťuje Grafana Observability stack implementovaný v rámci obrazu \emph{dta-lgtm:latest}, jenž obsahuje OpenTelemetry, Prometheus, Loki, Tempo a Grafanu. Pro měření výkonu hostitelského systému a export těchto dat jsou využity služby NodeExporter a CAdvisor s obrazy \emph{node-exporter:latest} a \emph{cadvisor-arm64:v0.49.1}.
    \item \textbf{Perzistence} - Pro perzistenci dat je využita PostgreSQL databáze. Pro PostgreSQL je využit obraz \emph{postgres:latest}. Ukládání metrik je zajištěno pomocí InfluxDB a obrazu \emph{influxdb:1.8.10}.
    \item \textbf{Směrování} - Funkci reverzní proxy zajišťuje Nginx ve verzi obrazu \emph{nginx:latest}.
\end{itemize}

\obr{Diagram nasazení aplikace}{fig:app}{0.85}{graphics/images/dta.drawio.png}

\n{3}{Kontejnerizace a orchestrace}

Kontejnerizace služeb je zajištěna pomocí nástroje Docker. Docker je open-source platforma poskytující ekosystém pro správu kontejnerů. Jednotlivé služby jsou vytvořeny jako obrazy podle definicí Dockerfile. 

Orchestrace aplikace je zprostředkována rovněž nástrojem Docker, konkrétně formou \emph{compose} utility. Ta umožňuje jednoduše nasadit a spravovat větší množství služeb. Definice nasazení aplikace je sepsána v souboru \emph{compose.yaml}. V něm lze nalézt následující části:

\begin{itemize}
  \item \textbf{volumes} - V této sekci jsou definovány všechny volume, které jsou využity v rámci stacku. Volume jsou definovány názvem a využity pro ukládání dat služeb. Konkrétně jsou využity úložiště pro data Grafany, OpenTelemetry, Prometheus a InfluxDB.
  \item \textbf{networks} - Konfigurace pro vnitřní síť aplikace, která je využita pro komunikaci mezi službami. Síť je definována názvem a typem. Pro potřeby aplikace se využívá síť s názvem \emph{stack-network} a typ \emph{bridge}, jenž funguje jako síťový most.
  \item \textbf{services} - Definuji sekci pro jednotlivé služby, které jsou součástí stacku. Každá služba je definována názvem služby - \emph{container\_name}, názvem obrazu - \emph{image}, v případě lokálně sestavených služeb také definicí sestavení - \emph{build}. Dále je definováno, jaké porty jsou mapovány z kontejneru do hostitelského systému - \emph{ports}, jaké volume jsou připojeny k kontejneru - \emph{volumes}, použité síťové rozhraní - \emph{networks}, závislosti služby - \emph{depends\_on} a proměnné prostředí - \emph{environment}. V případech testovaných služeb je uvedena dodatečná konfigurační sekce nasazení - \emph{deploy}, jenž limituje dostupné zdroje paměti a procesoru.
\end{itemize}

\n{3}{Konfigurace nasazení}

Za účelem běhu aplikace je klíčové správné nastavení konfigurace. Konfigurace je řešena na různých úrovních. Základní úroveň představuje soubor \emph{compose.yaml}. V tomto souboru je dílčí konfigurace jednotlivých služeb řešena v rámci konfiguračních souborů, které jsou do služeb připojeny z hostitelského OS, a proměnných prostředí. 

Nastavení směrování v rámci stacku je řešeno konfigurací proxy služby Nginx. Za tímto účelem obsahuje dva klíčové soubory, rozcestník v podobně statického \emph{index.html} souboru a konfigurační \emph{nginx.conf} soubor se směrovacími pravidly. Oba zmíněné soubory jsou do služby připojeny formou mapování virualizovaného repozitáře kontejneru. Směrovací pravidla jsou následující:

\begin{itemize}
    \item / - cesta na statickou hlavní stránku-rozcestník aplikace
    \item /grafana - směrování na Grafanu
\end{itemize}

Nastavení telemetrie spočívá v definici rozhraní, nastavení chování služeb a systémů z niž se telemetrická data sbírají, jejich cíl pro zpracování, správu a vizualizaci. Značnou část konfigurace představuje propojení nástrojů aplikace v rámci obrazu dta-lgtm, k čemuž slouží konfigurační soubory. Ty obsahují výchozí minimalistickou konfiguraci pro jednotlivé nástroje. Výstupem dta-lgtm je individuální kontejner a zmíněná konfigurace je z podstaty interní a není potřeba s ní manipulovat s ohledem na požadavky práce. Mezi dodatečné konfigurace telemetrie napříč službami patří:

\begin{itemize}
    \item \textbf{Testované služby} - Veškeré testovací služby mají nastavený endpoint pro export telemetrických dat. Toto nastavení je zprostředkováno proměnnou prostředí \emph{OpenTelemetrySettings\_ExporterEndpoint}.
    \item \textbf{LGTM} - Konfigurace pro LGTM je řešena pomocí proměnných prostředí. Je definována adresa a cesta, z které je k dispozici Grafana. Dále je umožněno anonymní přihlášení do Grafany.
    \item \textbf{Cadvisor} - Nastavení služby představuje dodatečné nastavení volumes, které jsou připojeny k kontejneru. Připojením systémových souborů je zajištěno sběr dat o využití systémových zdrojů. Toto nastavení je závislé na OS hostitelského stroje.
\end{itemize}

Jednotlivé služby mají vlastní dodatečné konfigurace, které jsou řešeny pomocí kombinace dle standardizovaného postupu pro konfiguraci .NET aplikací, a to konfiguračního souboru \emph{appsettings.json} a proměnných prostředí. V první řadě řadě je použita konfigurace ze souboru, načež je přepsána odpovídajícími hodnotami proměnných prostředí. Každá služba má definovaný specifický prefix pro identifikaci proměnných prostředí. Následující seznam popisuje dodatečné konfigurace jednotlivých služeb.

\begin{itemize}
  \item \textbf{FUS - File Upload Service} - Obsahuje connection string (přístupový řetězec) pro připojení do databáze PostgreSQL.
  \item \textbf{BPS - Batch Processing Service} - Disponuje konfigurací pro připojení k RabbitMQ a konkrétní frontě.
  \item \textbf{EPS - Event Publishing Service} - Drží informaci o rozhraní RabbitMQ, konkrétní frontě a gRPC rozhraní služby FUS.
\end{itemize}

Služby využité pro perzistentní ukládání dat jsou konfigurovány pomocí proměnných prostředí. Jedná se o následující služby:

\begin{itemize}
  \item \textbf{PostgreSQL} - Jsou definovány údaje pro uživatele databáze a název databáze.
  \item \textbf{InfluxDB} - Proměnné prostředí definují název databáze, uživatelské údaje a povolení přihlášení pomocí http.
\end{itemize}

Definice uživatelského rozhraní, respektive dostupných dashboardů, je dána při sestavení obrazu LGTM. V rámci něj jsou předdefinovány hodnoty pro připojení zdrojů dat, tj. Prometheus, Loki, Tempo a InfluxDb. Patřičné dashboardy zobrazující relevantní data pro různé scénáře systému byly předem připraveny a jsou k dispozici po otevření Grafany anonymním uživatelem. Následující seznam popisuje klíčové soubory konfigurace uživatelského rozhraní.

\begin{itemize}
  \item \textbf{grafana-dashboards.json} - Definuje dostupné dashboardy v Grafaně. Dashboardy jsou definovány v JSON formátu a obsahují definici panelů, zdrojů dat a dalších parametrů.
  \item \textbf{grafana-datasources.json} - Obsahuje  zdroje dat z kterých Grafana, respektive dashboardy, čerpají data.
\end{itemize}

%%%%%%%%%%%%%%%%%%%%%%%%%%%%%%%%%%%%%%%%%%%%%%%%%%%%%%%%%%%%%%%%%%%%%%%%%%%%%%%%%
%                             Testování scénářů                                 %
%%%%%%%%%%%%%%%%%%%%%%%%%%%%%%%%%%%%%%%%%%%%%%%%%%%%%%%%%%%%%%%%%%%%%%%%%%%%%%%%%

\n{1}{Testování scénářů}

Testování scénářů je klíčovou součástí testování výkonu mikroslužeb. Scénáře jsou definovány jako soubor kroků, které mají být provedeny, a jsou použity k simulaci zátěže na mikroslužby. Scénáře jsou vytvořeny pomocí testovacích nástrojů, které umožňují vytvářet a spouštět testy, které simuluji reálné uživatelské scénáře.

\n{2}{Předpoklad scénářů}

Scénáře musí být vytvořeny tak, aby simulovali reálné uživatelské scénáře. To znamená, že musí být vytvořeny tak, aby obsahovaly kroky, které mají být provedeny, a musí být vytvořeny tak, aby obsahovaly data, která mají být použita.

\n{3}{Očekávání výkonnosti služeb}

Pro výsledné obrazy služeb kompilovaných AOT do nativního kódu je očekáváno, že budou zabírat výrazně menší paměť, než je tomu u ekvivalentních obrazů služeb kompilovaných pro JIT. Předpoklad zakládá na požadavku běhového prostředí, kdy nativní služby vyžadují pouze malou množinu knihoven nad OS Alpine. Oproti tomu služby kompilované do JIT vyžadují nad OS Alpine běžící .NET runtime, i když zredukovaný o nepoužívané knihovny. Samotná velikost výstupních souborů služeb je předpokládaná větší u nativního výstupu, než u ekvivalentního JIT výstupu. To je dáno tím, že nativní výstup obsahuje dodatečné třídy, konstrukce a funkce, které substituují běhové prostředí .NET.

Další očekávání se týká rychlosti odezvy služeb po spuštění. Předpokládá se, že služby kompilované do nativního kódu budou mít rychlejší odezvu než služby kompilované do JIT. To je dáno tím, že nativní kód je přímo spustitelný na cílovém systému, zatímco JIT kód vyžaduje dodatečný čas na kompilaci a optimalizaci.

\n{2}{Definice scénářů}

Scénáře jsou vytvořeny jako množina javascriptových souborů splňujících požadavku API nástroje K6. Každý scénář je definován přes jeden nebo více scriptových souborů. Tyto soubory obsahují kroky, které mají být provedeny, a data, která mají být použita. Pro sjednocení obecných nastavení jsou vytvořený konfigurační soubory, které jsou využity ve více scénářích. Pro zjednodušené a automatizované spuštění testovacích scénářů jsou definovány runner skripty, které zajišťují spuštění testů spolu se správou orchestrace.

Pro dodatečnou identifikaci dat jednotlivých scénářů je užito InfluxDB tagů, které jsou přidány k jednotlivým voláním v testech. Tím je zajištěno, že data z jednotlivých scénářů jsou jednoznačně identifikována a lze je následně zpracovat. 

\begin{itemize}
    \item \textbf{dta\_service} - Značka pro identifikaci služby, která je testována. Má standardní formát hodnot \emph{Služba-Kompilační režim}, kdy služba může nabývat hodnot \emph{SRS, FUS, BPS, EPS} a kompilační režim nabývá hodnot \emph{JIT, AOT}.
    \item \textbf{test\_scenario} - Značka pro identifikaci scénáře, který je testován. Má standardní formát hodnot \emph{scenario 
    + číslo}.
    \item \textbf{test\_id} - Identifikátor konkrétního testovacího scénáře. Nabývá libovolné hodnoty a slouží pro identifikaci konkrétních instancí, tedy spuštění testovacího scénáře.
\end{itemize}

Každý scénař má definován vlastní dashboard v Grafaně, který je využit pro sledování výsledků testů v reálném čase. Zároveň je součástní každého scénáře readme soubor, jenž podrobněji popisuje jednotlivé kroky a data, která jsou využita.

\n{2}{Popis scénářů}

Následující sekce obsahuje popis scénářů, které byly vytvořeny pro testování výkonu a škálovatelnosti mikroslužeb kompilovaných JIT a AoT. Ke každému scénáři patři odpovídající sada souborů scriptů a konfigurací. Rovněž každý scénář disponuje vlastním interaktivním dashboardem v Grafaně, který umožňuje sledovat výsledky testů v reálném čase.

\n{3}{Scénář 1 - schopnost odpovídat služeb}

Scénář 1 je zaměřen na schopnost mikroslužeb odpovídat na požadavky. K tomuto účelu je využit základní endpoint \emph{/health}, který informuje o stavu služby. Scénář je vytvořen tak, aby simuloval zátěž na mikroslužby a zjišťoval, zda jsou schopny odpovídat na požadavky.

Jelikož healthcheck endpoint je triviální ve své implementaci, nehraje roli další režie spojená se zpracováním logiky požadavku. Tímto je zajištěno, že se otestuje maximální vliv jednotlivých nasazení na výkon a škálovatelnost mikroslužeb.

Scénář se dělí na více kroků, aby při každém byl zjištěn dostatek zdrojů pro v systému pro testovanou službu. Krok je proveden vždy po určitém časovém intervalu, který je definován v konfiguračním souboru testu.

\obr{Diagram scénáře 1}{fig:logo}{0.8}{graphics/diagrams/stack-scenario1.png}

\n{4}{Relevantní služby}

\begin{itemize}
    \item \textbf{SRS, FUS, BPS, EPS} - všechny služby s definovaným healthcheck endpointem
\end{itemize}

\n{4}{Průběh scénáře}

\begin{itemize}
    \item \textbf{Krok 1} - Spuštění služeb v rámci stacku
    \item \textbf{Krok 2} - Na služby jsou zasílány požadavky na healthcheck endpoint. Charakter požadavků je stupňující se k konfigurovanému maximu, načež zase klesá.
    \item \textbf{Krok 3} - Služby ukončují svoji činnost a zasílají data o provedeném testu
\end{itemize}

\n{3}{Scénář 2 - přístup k perzistenci}

Cílem tohoto scénáře je otestovat schopnost poradit si s vysokým množství asynchroních operací přístupu k datům. Scénář se pokouší identifikovat dodatečné režie spojené s přístupem k perzistenci a zjišťuje, zda jsou služby schopny zpracovat vysoký počet požadavků na databázi. Zejména je cílem pozorovat potencionál rozdíl v přístupu AOT a JIT zkompilované služby k systémovému API.

\obr{Diagram scénáře 2}{fig:logo}{0.9}{graphics/diagrams/stack-scenario2.png}

\n{4}{Relevantní služby}

\begin{itemize}
    \item \textbf{FUS} - služba pro přístup k perzistenci na databázi Postgres
\end{itemize}

\n{4}{Průběh scénáře}

\begin{itemize}
    \item \textbf{Krok 1} - Služba je spuštěna v rámci stacku
    \item \textbf{Krok 2} - Na službu jsou zasílány požadavky na zápis i čtení dat z perzistentního úložiště. Charakter požadavků je stupňující se k konfigurovanému maximu, načež zase klesá.
    \item \textbf{Krok 3} - Služba ukončuje svoji činnost a zasílá data o prevedeném testu
\end{itemize}

\n{3}{Scénář 3 - zátěž zpracování dat}

Cílem tohoto scénáře je otestovat schopnost mikroslužeb v jednotlivých kompilacích zpracovat náročnější operace. Scénář se zaměřuje na samotnou podstatu přístupu k vnitřnímu systémového API, efektivitě jeho využití a další režii, která by mohla být odlišná mezi JIT a AOT kompilací.

Předmětem scénář jsou dva výpočetně náročné algoritmy - faktoriál a Fibonacciho posloupnost. Tyto algoritmy jsou implementovány v rámci služby a jsou volány zvenčí. Scénář je vytvořen tak, aby simuloval zátěž na výpočetní jednotku a prozkoumal tak potencionální výkonnostní rozdíly v rámci přístupu k systémovému API a organizaci instrukcí.

\obr{Diagram scénáře 3}{fig:logo}{0.9}{graphics/diagrams/stack-scenario3.png}

\n{4}{Relevantní služby}

\begin{itemize}
    \item \textbf{BPS} - služba, která poskytuje rozhraní a logiku pro výpočet faktoriálu a Fibonacciho posloupnosti
\end{itemize}

\n{4}{Průběh scénáře}

\begin{itemize}
    \item \textbf{Krok 1} - Služba je spuštěna v rámci stacku
    \item \textbf{Krok 2} - Na službu jsou zasílány požadavky na výpočet faktoriálu a Fibonacciho posloupnosti. Charakter požadavků je stupňující se k konfigurovanému maximu, načež zase klesá.
    \item \textbf{Krok 3} - Služba ukončuje svoji činnost a zasílá data o prevedeném testu
    
\end{itemize}

\n{3}{Scénář 4 - komunikace mezi službami}

Tento scénář je zaměřen na rychlost a zátěž celkového systému při splnění požadavků vyžadující komunikaci mezi službami. Scénář je vytvořen tak, aby vyvolal událost z jedné služby, která je zpracována jinou službou. Pro splnění události je potřeba dat z perzistentního úložiště, která jsou získána ze třetí služby.

\obr{Diagram scénáře 4}{fig:logo}{1}{graphics/diagrams/stack-scenario4.png}

\n{4}{Relevantní služby}

\begin{itemize}
    \item \textbf{FUS} - služba hraje roli serveru, na něž se dotáže klient gRPC voláním. Následně přistupuje k perzistenci pro získání dat k splnění volání.
    \item \textbf{BPS} - poslouchá nad předem definovanou frontou a vyčkává na zprávu pro zpracování. V momentu přijetí zprávy, zpracovává vyvolanou událost a získává data ze vzdáleného volání z FUS.
    \item \textbf{EPS} - na základě přijatého volání přes REST API, zasílá služba EPS zprávu do předem definované fronty, na niž naslouchá BPS.
\end{itemize}

\n{4}{Průběh scénáře}

\begin{itemize}
    \item \textbf{Krok 1} - Služby jsou spuštěny v rámci stacku
    \item \textbf{Krok 2} - Do služby EPS je zaslán požadavek na zpracování dat. 
    \item \textbf{Krok 3} - Služba EPS zprávu zasílá do fronty, na kterou naslouchá služba BPS. 
    \item \textbf{Krok 4} - Služba BPS zprávu zpracovává a získává data ze vzdáleného volání na službu FUS. 
    \item \textbf{Krok 5} - Služba FUS získává data z perzistence a zasílá je zpět službě BPS. 
    \item \textbf{Krok 6} - Služba BPS zpracovává data.
    \item \textbf{Krok 7} - Služby ukončují svoji činnost a zasílají data o prevedeném testu
\end{itemize}

\n{3}{Scénář 5 - rychlost odpovědi po startu služby}

Cílem tohoto scénáře je otestovat rychlost spuštění služby. Scénář testuje, jak rychle je služba schopna odpovědět na požadavek po spuštění. V rámci testu jsou testovány různé endpointy, které jsou volány po spuštění služby.

Základem scénáře je pomocí CLI příkazů vyvolat spuštění služby a ihned po jejím spuštění zaslat požadavek na získání dat.

\obr{Diagram scénáře 5}{fig:logo}{0.9}{graphics/diagrams/stack-scenario5.png}

\n{4}{Relevantní služby}

\begin{itemize}
    \item \textbf{SRS} - služba je testována pro svoji nutnost serializace vygenerované datové odpovědi. Vyžaduje určitou množinu operací, jenž se podepíše dále nad rychlostí odpovědi služby a více přiblíží reálnému scénáři.
    \item \textbf{FUS} - za účelem otestování rychlosti odpovědi služby s ohledem na vazbu do dalšího systému je využita i služba FUS. S jejím přístupem k persistentnímu úložišti přiblíží scénář, kdy je nutné pro zpracování odpovědi nejen nastartovat službu, ale i získat data ze vzdáleného zdroje.
\end{itemize}

\n{4}{Průběh scénáře}

\begin{itemize}
    \item \textbf{Krok 1} - Služba je spuštěn v rámci stacku
    \item \textbf{Krok 2} - V návaznosti na spuštění služby je zaslán požadavek na získání dat.
    \item \textbf{Krok 3} - Služba SRS/FUS zpracovává požadavek a zprostředkovává data.
    \item \textbf{Krok 4} - Data jsou zaslána zpět klientovi.
    \item \textbf{Krok 5} - Služba ukončuje svoji činnost.
\end{itemize}

\n{2}{Spouštění scénářů}

Jednotlivé scénáře jsou spouštěny dle definice v příslušném readme souboru. Jedná se o sekvenci instrkcí/příkazů pro přípravu požadovaného stavu systému a spuštění K6 testu v rámci kontejneru.

\n{2}{Zpracování a vizualizace dat}

Po provedení testování scénářů je nutné zpracovat a vizualizovat data, která byla získána. To zahrnuje zpracování dat, která byla získána z testování scénářů, a zpracování dat, která byla získána z monitorovacích nástrojů.

\n{3}{Monitorování v reálném čase}

Monitorování v reálném čase je klíčovou součástí testování výkonu a škálovatelnosti mikroslužeb. Umožňuje sledovat výkon a škálovatelnost mikroslužeb při běhu testů.

Toho je docíleno využitím dashboardů v Grafaně, důkladnou konfigurací a zobrazením metrik, kterých sběr je implementován v rámci mikroslužeb.

Dalším aspektem monitorování v reálném čase je zobrazení výsledků testů v reálném čase. Toho je rovněž docíleno pomocí specifických dashboardů v Grafaně, které integrují data z K6 testovacího nástroje a zaslané do InfluxDb. Díky propojení Grafany s InfluxDb je možné sledovat výsledky testů v reálném čase.

\n{3}{Sběr historických dat}

Historická data jsou automaticky ukládána do jednotlivých databází při sběru. Po propagaci telemetrických dat do jednotného collectoru OpenTelemetry jsou data dále poskytována službám Loki, Tempo a Prometheus. Ty jedna jednotlivá telemetrická data zpracují, zároveň ale slouží jako jejich persistence. Data z výsledků testů jsou ukládána do InfluxDb.


\cast{Analytická část}
\n{1}{Analýza aplikace}

Tato kapitola se zabývá analýzou aplikace z hlediska vývoje, výstupu a výkonu. Využívá k tomu definovanou metodiku a scénáře testování. Výsledky jsou důkladně analyzovány a závěry shrnuty v jednotlivých sekcích.

\n{2}{Architektura}

Výsledná architektura aplikace je založena na mikroslužbách. Splňuje předem definované funkční a nefunkční požadavky. V případě testovaných služeb, zapojuje základní množinu systémových knihoven a knihoven 3. stran. Po straně telemetrie, implementuje sběr a zpracování dat z různých zdrojů. Výsledná data jsou následně zpracována a uložena do databáze, dle druhu dat. Veškeré dostupné zdroje jsou uživatelsky přívětivě vizualizovány v rámci webového aplikace Grafana. Aplikační stack je testovatelný a nasaditelný na všech hlavních platformách (po sestavení se zacílením na vybranou architekturu a využitím variant služeb třetích stran s cílovou architekturou).


\n{2}{Vývojový proces}

Následující sekce popisuje vývojový proces, tak jak se týkal testovaných služeb. Vývojový proces byl založen na experimentaci a snaze využít co nejvíc dostupných knihoven a nástrojů, za cenu nutnosti řešení problémů, případně změny implementace.

\obr{Ukázka kódu s vyzualizací direktiv dle konfigurace}{fig:codesample}{1}{graphics/images/code-visual-sample.png}

\n{3}{JIT}

Vývojový proces pro kompilaci služeb JIT se zacílením na .NET runtime probíhal standarntím způsobem. Veškeré dostupné knihovny a nástroje byly plně kompatibilní s JIT kompilací. Nedošlo k žádným nepředpokládaným problémům.

Znatelný rozdíl oproti běžnému vývoji byl výběr technologií, který přihlížel k potencionální kompatibilitě s AOT a tedy řešení, které inherntně vyžadovala funkce rezervované pro využití .NET runtime, byly ihned zavrženy.

\n{4}{Výhody}

Mezi hlavní výhody se řadí zprostředkování následujícího:

\begin{itemize}
    \item  \textbf{Reflexe} - CLR umožňuje využívat reflexi, která umožňuje získat informace o kódu za běhu aplikace. Tímto je umožněno vytvářet aplikace, které jsou schopny měnit své chování za běhu.
    \item \textbf{Dynamické načítání} - CLR umožňuje dynamicky načítat knihovny za běhu aplikace. Tímto je umožněno vytvářet aplikace, které jsou schopny měnit své chování za běhu.
    \item \textbf{Větší bezpečnost} - CLR zajišťuje, že aplikace nemůže přistupovat k paměti, která jí nebyla přidělena. Tímto je zajištěna bezpečnost aplikace a zabráněno chybám, které by mohly vést k pádu aplikace.
    \item \textbf{Správa paměti} - CLR zajišťuje správu paměti pomocí GC. Tímto je zajištěno, že paměť je uvolněna vždy, když ji aplikace již nepotřebuje. Tímto je zabráněno tzv. memory leakům, které by mohly vést k pádu aplikace.
    \item \textbf{Větší přenositelnost} - CLR zajišťuje, že aplikace je spustitelná na všech operačních systémech, na kterých je dostupné běhové prostředí CLR.
\end{itemize}

\n{4}{Nevýhody}

Zatímco za nevýhody CLR se dá považovat:

\begin{itemize}
    \item  \textbf{Výkonnost} - I když určité optimalizace jsou prováděny pro konkrétní systém a architekturu, výkon CLR je nižší než výkon nativního kódu. Dalším výkonnostním měřítkem je rychlost startu aplikace, která je pro CLR vyšší než v případě nativního kódu.
    \item \textbf{Operační paměť} - CLR využívá více operační paměti, jak pro aplikaci, tak i pro běhové prostředí.
    \item \textbf{Velikost aplikace} - Přítomnost CLR nehraje zásádní roli v případě monolitických aplikací, ale v případě mikroslužeb je nutné CLR přidat ke každé službě. Tímto se zvyšuje velikost jedné aplikační instance.
\end{itemize}

\n{3}{AOT}

Kompilace do nativního kódu probíhala s průběžnými problémy. Podpora ze strany knihoven 3. stran ve spoustě případů neodopvídala deklarovaným možnostem. Vývojový proces byl značně zpomalován nutností řešení problémů, které byly způsobeny nedostatečnou podporou. Experimentace s řešeními často vyústila v nutnost změny implementace, případě v implementaci zcela vlastní.

\n{4}{Výhody}

Mezi hlavní výhody se řadí zprostředkování následujícího:

\begin{itemize}
    \item  \textbf{Výkonnost} - CLR umožňuje využívat reflexi, která umožňuje získat informace o kódu za běhu aplikace. Tímto je umožněno vytvářet aplikace, které jsou schopny měnit své chování za běhu.
    \item \textbf{Paměťová zátěž} - CLR umožňuje dynamicky načítat knihovny za běhu aplikace. Tímto je umožněno vytvářet aplikace, které jsou schopny měnit své chování za běhu.
\end{itemize}

\n{4}{Nevýhody}

Zatímco za nevýhody CLR se dá považovat:

\begin{itemize}
    \item  \textbf{Absence nástrojů z CLR} - Mnoho nástrojů, které jsou dostupné v CLR, nejsou dostupné v AOT kompilaci. Mezi tyto nástroje patří například reflexe, dynamické načítání knihoven a další.
    \item \textbf{Absence dynamického načítání} - například Assembly.LoadFile.
    \item \textbf{Bez generování kódu za běhu} - například System.Reflection.Emit.
    \item \textbf{Žádné C++/CLI} - např. System.Runtime.InteropServices.WindowsRuntime
    \item \textbf{Windows: absence COM} - např. System.Runtime.InteropServices.ComTypes
    \item \textbf{Vyžaduje trimming (ořezávání)} - má určitá omezení, je však klíčový pro rozumnou velikost výsledného programu
    \item \textbf{Kompilace do jediného souboru} 
    \item \textbf{Připojení běhových knihoven} - požadované běhové knihovny jsou součástí výsledného aplikačního souboru. To zvyšuje velikost samoteného programu ve srovnání s aplikacemi závislými na frameworku.
    \item \textbf{System.Linq.Expressions} - výsledný kód používá svou interpretovanou podobu, která je pomalejší než run-time generovaný kompilovaný kód.
    \item \textbf{Kompatibilita knihoven s AOT} - né všechny knihovny runtime jsou plně anotovány tak, aby byly kompatibilní s Native AOT. To znamená, že některá varování v knihovnách runtime nejsou pro koncové vývojáře použitelná.
\end{itemize}

\n{2}{Výstup služeb}

Samotný proces nativní AOT a JIT kompilace je různě výkonnostně náročný. Při tvorbě obrazu službeb, ale i kompilace je hlavní náročná operace \emph{restore}, která stahuje potřebné závislosti a balíčky pro projekt. Proces kompilace je vysoce závislý na specifickém HW, SW a přítomnosti závislostí. Pro účely testování byly potřebné NuGet balíčky nacachovány v systému. Následující tabulka zobrazuje přehled časové náročnosti kompilace služeb pro oba kompilační cíle. K získání času výstupu bylo využity diagnostického režimu příkazu \emph{dotnet}. Pro AOT byl použit příkaz \emph{dotnet publish -v d -c Release-AOT -r osx-x64}, pro získání výstupu JIT byl použit příkaz \emph{dotnet publish -v d -c Release-JIT -r osx-x64 --self-contained false}.


\tab{Čas kompilace služeb}{tab:priklad}{0.65}{|l|c|c|r|}{
  \hline
    & JIT (s) & AOT (s) & AOT \% nárůst \\ \hline
  \emph{SRS} & 01.99 & 19.49 & 979.3 \\ \hline
  \emph{FUS} & 03.85 & 30.36 & 788.5 \\ \hline
  \emph{BPS} & 02.02 & 20.74 & 1026.7 \\ \hline
  \emph{EPS} & 01.85 & 20.05 & 1083.7 \\ \hline
}

Velikost samotného výstupního programu je dle očekávání výrazně menší v případě JIT kompilace. To je dáno tím, že výstupní program je závislý na .NET runtime, který poskytuje dodatečnou obecnou funkcionalitu a vytváří nativní kód včetně generování typů až za běhu aplikace. Následující tabulka zobrazuje velikost služeb pro oba kompilační cíle. Pro vytvoření výstupů na základě JIT byl použit příkaz \emph{dotnet publish -c Release-JIT -r osx-x64 /p:PublishSingleFile=true --self-contained false}, pro vytvoření výstupů AOT byl použit příkaz \emph{dotnet publish -c Release-JIT -r osx-x64 /p:PublishSingleFile=true --self-contained false}.

\tab{Velikost programu služeb}{tab:priklad}{0.65}{|l|c|c|r|}{
  \hline
    & JIT (MB) & AOT (MB) & AOT \% nárůst \\ \hline
  \emph{SRS} & 05.70 & 21.40 & 375.4 \\ \hline
  \emph{FUS} & 12.40 & 28.40 & 229.0 \\ \hline
  \emph{BPS} & 06.00 & 21.80 & 363.3 \\ \hline
  \emph{EPS} & 06.00 & 21.70 & 361.6 \\ \hline
}

Sestavení obrazu je závislé na přípravu prostředí, vyhodnocení a stažení závislostí, kompilaci a publikování aplikace. Výstupné obrazy jsou založené na linuxovém systému, Alpine s .NET runtime v případě JIT výstupu služby, zredukované Ubuntu v případě nativního AOT výstupu. Z pohledu použitelnosti výsledného obrazu služeb má smysl měřit velikost výstupního obrazu. Následující tabulka zobrazuje velikost obrazu služeb pro oba kompilační cíle. Použitý příkaz je \emph{docker build -t <service>:<tag> -f Dockerfile-<target> .}, kdy \emph{<target>} představuje vybranou kompilační metodu AOT nebo JIT. Před každým sestavením byl obraz a cache smazány. I přes toto opatření není zaručena konzistentní časová náročnost sestavení obrazu.

\tab{Velikost obrazu služeb}{tab:priklad}{0.65}{|l|c|c|r|}{
  \hline
    & JIT (MB) & AOT (MB) & AOT \% zmenšení \\ \hline
  \emph{SRS} & 121.97 & 31.41 & 74.3 \\ \hline
  \emph{FUS} & 134.36 & 38.32 & 71.5 \\ \hline
  \emph{BPS} & 122.39 & 31.40 & 74.3 \\ \hline
  \emph{EPS} & 122.26 & 31.74 & 74.0 \\ \hline
}

\n{3}{Vývojové prostředí}

K vývoji byl použit IDE Rider od společnosti JetBrains. Vyzkoušena byla rovněž i práce ve Visual Studio 2022 Community Edition a Visual Studio Code s doporučenými rozšířeními od Microsoft. Všechna vývojová prostředí jsou kompatibilní, co se týče procesu kompilace respektive sestavení, jelikož to se odehrává pomocí CLI .NET.

Samotný vývoj s ohledem na práci s direktivami pro různé kompilace byl značně zjednodušen vizualicemi, jenž poskytovala vývojová prostředí Rider a Visual Studio. Obdobně byla v těchto IDE zjednodušena i analýza a hledání chyb díky integraci referencí na kód generovaný na pozadí pro kompatibilitu s AOT. V tomto ohledu Visual Studio Code zaostávalo. S ohledem na aktivní vývoj a podporu, jenž je ze strany Microsoft poskytována podpoře vývoje .NET ve Visual Studio Code (po diskontuaci produktu Visual Studio pro Mac), lze očekávat, že se tato situace v budoucnu změní.

\n{3}{Knihovny třetích stran}

Pro zjednodušení procesu vývoje a využití existující funkcionality byly využity knihovny třetích stran. Následující seznam obsahuje knihovny, které byly využity použity v rámci vývoje a zda byly kompatibilní s AOT kompilací.

\begin{itemize}
  \item \textbf{Entity Framework} - Entity framework se pyšní vysokou kompatibilitou s AOT kompilací. V rámci vývoje nebyly zaznamenány problémy, avšak následné testování se ukázalo problematické. EF jakožto plnohodnotný ORM framework stopuje stav objektu a jeho změny. Toto chování bohužel vyžaduje dynamické generování kódu, což je v rozporu s možnostmi AOT kompilovaného kódu. Vypnutí této funkcionality je pouze částečné, neb EF stále vyžaduje reflexi při vkládání nových entit do databáze.
  \item \textbf{Fluent Migrator} - Fluent Migrator je knihovna, která umožňuje verzování databáze pomocí kódu. V rámci testování bylo zjištěno, že knihovna využívá reflexi pro načítání migrací. Toto chování je v rozporu s AOT kompilací a výsledkem je chyba při spuštění migrace. Problém byl vyřešen vytvořením vlastního minimalistického migrátoru, který nepoužívá reflexi.
  \item \textbf{Grpc} - Vytváření rozhrání a modelů pro gRPC komunikaci vyžadovalo využití přístupu model first. Tento přístup využívá generátorů pro tvorbu kódu, definijucího kódového rozhraní pro .NET. Tímto je dosaženo vygenerování veškerého potřebného kódu v době kompilace a je zajištěna kompatibila s AOT. Pro definici modelu code first ovšem kombatibila s AOT není zajištěna.
  \item \textbf{Párování konfigurace} - V rámci systémové .NET knihovny je umožněno volání API, jenž načte data ze sjednocení stavu proměnných prostředi a konfiguračního souboru. Součástí API je volání metody mapující tuto konfiguraci na předem definovaný objekt. Toto chování dle dostupných informací není v rozporu s AOT kompilací a volání relevantního kódu neprodukuje AOT warning. Z testování však vyplynulo, že mapování konfigurace ne objekt bylo problematické a neprobíhalo správně. Z toho důvodu je v případě AOT kompilace za pomocí deriktivy použité přímé načtení jednotlivých hodnot z konfigurace, dle stromomvého klíče.
\end{itemize}

\n{2}{Analýza testování}

Následující sekce se zabývá analýzou testovacích scénářů a výsledků testování. Testování bylo provedeno na základě předem definované metodiky. Podkladem testů byly definované scénáře, které byly vytvořeny s ohledem na funkční a nefunkční požadavky. Při testování byl nezávisle na spuštěný test zaznaménáván stav hostitelského systému s ohledem na spuštěné kontejnery a využití systémových prostředků. Samotné služby využívaly předem definované metry ve frameworku ASP.NET pro dodatečnou diagnostiku a monitorování. Výsledky testování byly zaznamenány a analyzovány.

\n{3}{Scénář 1 - Výkonnost komunikace}

První scénář se zabíral jednoduchou funkcionalitou dotazu na healthcheck endpoint a meřením výkonu kestrel serveru u odpovědí na požadavky skrze REST API. Testování přineslo rozdílné výkonostní výsledky mezi JIT a AOT kompilací. Dle předpokladu služby s nativním kódem využívaly méně času procesoru. Paměťová stopa však u nich byla větší. Konečně, AOT služby byly schopné v průměru rychleji odpovídat. Čistá rychlost zpracování požadavku a odpovědi není v mnoha případech kritickým faktorem. Avšak v případě velkého množství požadavků, může být rozdíl v řádech milisekund zásadní.

\tab{Průměrné využití zdrojů a doba odpovědi healthcheck služeb}{tab:service_metrics}{1.0}{|l|r|r|r|r|}{
  \hline
    Služba - Režim & CPU (ms) & IO (ns) & Paměť (MB) & Doba požadavku (ms) \\ \hline \hline
  \emph{SRS-AOT} & 3.41 & 0.550 & 41.1 & 1.61 \\ \hline
  \emph{SRS-JIT} & 9.69 & 0.453 & 41.3 & 3.84 \\ \hline
  \emph{FUS-AOT} & 1.99 & 0.825 & 52.5 & 1.27 \\ \hline
  \emph{FUS-JIT} & 7.62 & 0.458 & 39.3 & 2.22 \\ \hline
  \emph{BPS-AOT} & 1.21 & 0.425 & 37.9 & 2.57 \\ \hline
  \emph{BPS-JIT} & 9.24 & 0.550 & 36.3 & 1.96 \\ \hline
  \emph{EPS-AOT} & 2.47 & 0.451 & 36.5 & 2.07 \\ \hline
  \emph{EPS-JIT} & 6.63 & 0.686 & 35.3 & 3.09 \\ \hline
}

\n{3}{Scénář 2 - Přístup k perzistenci}

Scénář se zabýval výkonností přístupu k persistenci, respektive zachystením reálného scénáře, kdy jsou data získávána a ukládána do databáze. Faktorem byla jak samotná rychlost služby v ohledu komunikace a serializace dat, tak rychlost zpracování požadavku databází. Ve výsledku je vidět výrazný rozdíl ve využití zdrojů mezi AOT a JIT verzi služby, kdy první jmenovaná je výrazně efektivnější.

\tab{Průměrné využití zdrojů službou FUS a doba odpovědi stažení a nahrání souboru}{tab:service_metrics}{1.0}{|l|r|r|r|r|}{
  \hline
  Služba - Režim & CPU (ms) & IO (ns) & Paměť (MB) & Doba požadavku (ms) \\ \hline \hline
  \emph{FUS-AOT} & 1.9 & 2.208 & 29.3 & 4.18 \\ \hline
  \emph{FUS-JIT} & 16.7 & 2.000 & 60.9 & 8.05 \\ \hline
}

V případě doby odpovědi služby je velmi znatelný rozdíl služby kompilované JIT kdy její hodnota činila 93.6 ms. Následkem JIT kompilace potřebného kódu při prvním volání byla tato doba výrazně vyšší než v dalších voláních. Oproti tomu AOT varianta služby měla i při prvním volání odpoveď srovnatelnou s průměrným voláním a to 11.8 ms.


\tab{Průměrné využití GC službou FUS}{tab:service_metrics}{1.0}{|l|r|r|r|}{
  \hline
  Služba - Režim & Alokovaná paměť (MB) & Doba běhu (ms) & Velikost objektů (MB) \\ \hline \hline
  \emph{FUS-AOT} & 25.8 & 25.9 & 15.2 \\ \hline
  \emph{FUS-JIT} & 14.0 & 5.3 & 12.9  \\ \hline
}

Přítomnost vygenerovaných typů a funkcionality v nativní AOT verzi služby má za výsledek větší alokace paměti, jenž jsou následně uvolněny, a větší doba běhu GC.

\n{3}{Scénář 3 - Výpočetní zátěž}

Za účelem zjištění výkonnosti služeb, jejich potencionálně odlišné využití systémového API byl otestován scénář výpočetní zátěže. Na jednotlivé služby byly vysílány požadavky na výpočet 40-tého Fibonacciho čísla rekurzivní metodou. Výsledky testování ukázaly konsistnetně vyšší počet zpracovaných požadavků v případě AOT kompilace. 

TODO: Foto dashboard + tabulka výsledků + stručné vysvětlení

\n{3}{Scénář 4 - Vzájemná komunikace služeb}

Komplexnější situace pro aplikaci byla simulována ve čtvrtém scénáři. Zde na základě požadavku na EPS byla vyvolána událost do RabbitMQ, načež byla zpracována službou BPS. Ta na jejím základě stáhla patřičný záznam pomocí RPC z FUS a provedla simulaci zpracování dat výpočtem Fibonacciho čísla. Situace simulovala kombinaci synchronní a asynchronní komunikace mezi službami doplněnou o výpočetní zátěž. Výsledky přiblížily služby v obou kompilačních režimech a ukázaly pohled bližší reálnému nasazení, kdy čistě výkkonostní rozdíly kompilací nehrají tak zásadní roli. 

TODO: Foto dashboard + tabulka výsledků

\n{3}{Scénář 5 - Rychlost odpovědi služby po startu}

Simulaci serveless nasazení byla vyvolána v tomto scénáři. Jednotlivé varianty služby SRS byly v rámci testu zpuštěny, kontrolovány než se dostaly do stavu \emph{healthy} a následně nad nimi zavolán dotaz na generovaná data. Výsledky ukázaly, že služba kompilované nativním AOT způsobem byly mnohem rychleji dostupné a odpovídaly na požadavky dříve, než služba kompilované do .NET runtime.

TODO: Foto dashboard + tabulka výsledků

\n{2}{Závěr analýzy}

Na základě výsledků vývoje, výstupu a testování služeb lze odpovědět na definované hypotézy následujícím způsobem:

\begin{itemize}
  \item \textbf{Hypotéza 1} - Hypotéza, že vývoj služeb s jak AOT, tak JIT kompilací je v rámci podporované funkcionality systémových knihoven a ASP.NET možný s podobným API se ukázal jako ne zcela pravdivý. Při vývoji nastaly komplikace se serializací konfigurace, na které bylo nutné reagovat využitím odlišného API. Zároveň tento způsob serializace nebyl kompilátorem označen jako potencionálně problematický. Další problémy nastaly s využitím Entity Framework. Tento ORM využívá pro provádění operací nad databází tzv. tracking, který zaznamená změny nad aplikačními objekty a podle nich tvoří výsledné databázové operace. Vypnutím trackingu bylo umožněno se na datové entity dotázat a aktualizovat je. Operace vložení nové entity však bez trackingu nebyla možná. Pro knihovny 3. stran lze obecně říci, že podpora AOT kompilace není vždy úplně zřejmá a i v situacích kdy AOT varování jsou implementovány, lze očekávat chybné chování.
  \item \textbf{Hypotéza 2} - Výsledky ukazují, že služby napsané v nativním kódu se výrazněji rychleji spouští jak na hostitelských systémech, tak ve virtualizovaném prostředí. Zároveň binární velikosti samotných aplikací jsou mnohonásobně větší, než je tomu u služeb vyžadující .NET runtime. To je ovšem kompenzováno při virtualizovaném spuštění, kdy obraz služby pro vytvoření plnohodnotného kontejneru vyžaduje mnohem méně závislotí z hlediska paměti. Výsledné obrazy jsou tedy menší a rychleji spustitelné. Hypotéza byla potvrzena.
  \item \textbf{Hypotéza 3} - Na základě dostupných metrik bylo potvrzeno, že obecně služby kompilované do nativního kódu poskytují vyšší výkon a jsou méně paměťově náročné než služby kompilované pro .NET runtime. Tento fakt je způsoben rozdílem v době, kdy se část generují typy a část funkcionality aplikace. Pro .NET runtime za běhu a pro nativní AOT při sestavení. Zároveň bylo ale pozorováno zvýšené využití GC v případě služeb kompilovaných do nativního kódu. I přes tuto dodatečnou režii byly nativní služby efektivnější a hypotéza byla potvrzena.
\end{itemize}

\nn{Závěr}

V rámci diplomové práce byly analyzovány kompilační režimy JIT a nativní AOT na platformě .NET. První částí byla rešerše, ve které byly popsány základní principy fungování platformy .NET, jejich kompilačních režimů a cílů kompilace. Následně byla popsána architektura microservice, která slouží jako primární zacílení nativních AOT aplikací a která poskytuje vzor pro testovací nasazení. V neposlední řadě byla popsána problematika testování, telemetrie a monitorovacích řešení.

Praktická část se zabývala vývojem testovacích služeb a testovací platformy.Dále byly popsány nástroje, které umožňují vytváření nativních AOT aplikací na platformě .NET. V rámci rešerše byly také popsány nástroje, které umožňují měření výkonu aplikací.

V analytické části byly výsledky praktické části popsány a vyhodnoceny. Zhodnocení vývoje probíhalo ve třech režimech: analýza vývoje, výstupu a výkonu.

Výsledkem práce je komplexní analýza použití kompilačních režimů JIT a nativní AOT. Vývojový proces při kompilaci do nativního AOT kódu se ukázal nepřívětivý. Primárně podpora knihoven 3. stran a princip interceptorů a generátorů má za vinu subjektivně neintuitivní proces debugování kódu. Samotný programový výstup vyšel dle očekávání. V porovnání byly obrazy nativních služeb výrazně paměťově efektivnější. Výsledky testování ukázaly, že na platformě .NET nativní AOT aplikace mají obecně srovnatelný výkon jako aplikace v režimu JIT. Rozdíl je znatelný v situacích, kdy je nutno využít velké množství instancí stejné služby (plyne z velikosti obrazu) a v situacích, kdy je pro systém rozhodující rychlost zpracování služby včetně spuštění (Serverless platformy). Výsledky výkonnostního testování byly zaznamenány a zpracovány do dashboardů a grafů, které jsou připraveny v uživatelské rozhraní platformy Grafana.

Dále byla vytvořena sada testovacích služeb, které slouží jako ukázka možností platformy. V neposlední řadě pro účely analýzy byla vytvořena testovací platforma, která umožňuje vytváření a nasazování testovacích služeb v kompilačním režimu JIT a nativní AOT.

Služby kompilované do nativního AOT kódu přináší specifické výkonnostní výhody za cenu kompatibility API. Vývoj kódu je s ohledem na zažité postupy a praktiky v .NET nestandartní. Využití interceptorů a generátorů je odebrána část inciativy z rukou vývojáře a vytváří se na pozadí kompilace v .NET další úroveň abstrakce. Podpora knihoven a frameworků třetích stran je omezena a nelze se spolehnout na jejich plnou funkčnost. Tím připadá na vývojáře zodpovědnost za implementaci vlastních řešení, která by jinak byla dostupná.

Většina výhod, jenž z .NET plyne souvisí s možnostmi jeho runtime prostředí. Nativní AOT kompilace má smysl ve specifických případech, jenž plynou z nutnosti rychlosti spuštění a velikosti výstupu aplikace (s přihlednutím k velikosti .NET runtime). Případy konkurenční výhody pro AOT kompilaci staví na předpokladu a tím je zájem či potřeba mít zdrojové kódy v .NET, respektive jazyce C\#. Při rozmanitém technologickém přístupu, kdy je vývojář, respektive zapojený tým schopen přijmout jiný jazyk a framework, jsou výhody AOT kompilace ztraceny, zatímco nedostatky jsou zvýrazněny. Tento předpoklad je relativně v rozporu s požadavky na poskytování cloudových služeb, kdy je očekáváno silné technické a vědomostní zázemí a flexibilní přístup k technologiím.

Nativní AOT kompilace má oproti JIT kompilaci nesporné výhody za splnění uričtých podmínek na požadavky vůči nasazení, kódu a vývojového týmu. Zaplňuje specifickou díru v portfoliu technologií platformy .NET, která je klíčová pro kompletní řešení cloudové platformy pouze s použitím této platformy. Vývojáři, kteří se rozhodnou pro AOT kompilaci, by měli být obeznámeni s těmito specifiky a měli by být schopni je zohlednit v návrhu a implementaci řešení.

V návaznosti na platformu, která v práci vznikla v rámci výkonnostního testování služeb, je možné doplnit implementaci dalších služeb, případně rozšířit stávající. Podle vzoru současného řešení lze dodat další funkcionalitu, nastavit další zdroje telemetrie, případně rozšířit možnosti vizualizace dat. Z pohledu uživatelské přívětivosti se nabízí rozšíření o webovou aplikaci využívající princip Docker outside of Docker (DooD). Tímto by bylo možné zjednodušit spouštění konkrétních testovacích scénářů v GUI. V rámci webové aplikace by bylo možné nastavit parametry testování, spustit konkrétní test a prokliknout se odkazem na relevantní dashboard v Grafaně. S ohledem na citlivé data a přístupy, které aplikace zprostředkovává, se nabízí rozšíření o autentizaci a autorizaci v případě vystavení stacku v síti. Jelikož grafické rozhraní aplikace je založeno na aplikaci Grafana, jež je schopna připojit se k externím zprostředkovatelům autentizace, bylo by vhodné zapojit službu jako Keycloak pro sjednocení autentifikace napříč stackem.


XXX Petr radí


tahle XXX XXX



\OdsazovaniOdstavcuStop


% ============================================================================ %
\seznamlit{
%   Na toto místo je třeba vložit veškeré citované bibliografické položky.

% 1
\bibitem{Troelsen2003}TROELSEN, Andrew. \emph{C\# and the .NET Platform}. Second edition. Apress, 2003. ISBN 9781590590553.

% 2
\bibitem{Richter2012}RICHTER, J. \emph{CLR via C\#: The Common Language Runtime for .NET Programmers}. 4th ed. Microsoft Press, Redmond, Wash., 2012. ISBN 978-0735667457.

% 3
\bibitem{netdocscli}MICROSOFT. \emph{.NET CLI overview}. Online. Microsoft Learn. Dostupné z: https://learn.microsoft.com/en-us/dotnet/core/tools/. [cit. 2024-05-01].

% 4
\bibitem{netdocsmsbuild}MICROSOFT. \emph{MSBuild}. Online. Microsoft Learn. Dostupné z: https://learn.microsoft.com/en-us/dotnet/core/tools/. [cit. 2024-05-02].

% 5
\bibitem{Price2023c8}PRICE, Mark J. \emph{C\# 12 and .NET 8 – Modern Cross-Platform Development Fundamentals}. Eight Edition. Packt Publishing, 2023. ISBN 978-1-83763-587-0.

% 6
\bibitem{Harrison2017}HARRISON, Nick. \emph{Code Generation with Roslyn}. Apress, 2017. ISBN 978-1-4842-2211-9.

% 7
\bibitem{Williams2023}WILLIAMS, Trevoir. \emph{Microservices Design Patterns in .NET}. 1st Edition. Packt Publishing, 2023. ISBN 978-1-80461-030-5.

% 8
\bibitem{Danylko2023}DANYLKO, Jonathan R. \emph{ASP.NET 8 Best Practices}. Packt Publishing, 2023. ISBN 978-1-83763-713-3.

% 9
\bibitem{aspnetdocs}MICROSOFT CORPORATION. \emph{ASP.NET Documentation} [online]. [cit. 2024-03-18]. Dostupné z: https://learn.microsoft.com/en-us/aspnet/core/?view=aspnetcore-8.0

% 10
\bibitem{Libery2023}LIBERY, Jesse a Rodrigo JUAREZ. \emph{.NET MAUI for C\# Developers}. Packt Publishing, 2023. ISBN 978-1-83763-169-8.

% 11
\bibitem{Alls2023}ALLS, Jason. \emph{Clean Code with C\#}. Second edition. Packt Publishing, 2023. ISBN 978-1-83763-519-1.

% 12
\bibitem{Bock2016}BOCK, Jason. \emph{.NET Development Using the Compiler API. Second edition}. Apress, 2016. ISBN 978-1-4842-2111-2.

% 13
\bibitem{netdocsr2r}MICROSOFT. \emph{ReadyToRun Compilation}. Online. Microsoft Learn. Dostupné z: https://learn.microsoft.com/en-us/dotnet/core/deploying/ready-to-run. [cit. 2024-05-02].

% 14
\bibitem{Pflug2023}PFLUG, Kenny. \emph{Native AOT with ASP.NET Core - Overview} [online]. 2023 [cit. 2024-02-23]. Available from: https://www.thinktecture.com/en/net/native-aot-with-asp-net-core-overview/

% 15
\bibitem{Price2023}PRICE, Mark J. \emph{Apps and Services with .NET 8. Second Edition}. Packt Publishing, 2023. ISBN 978-1837637133.

% 16
\bibitem{netdocssg}MICROSOFT. \emph{Source Generators}. Online. Microsoft Learn. Dostupné z: https://learn.microsoft.com/en-us/dotnet/csharp/roslyn-sdk/source-generators-overview. [cit. 2024-05-02].

% 17
\bibitem{Richardson2018}RICHARDSON, C. \emph{Microservices Patterns: With Examples in Java}. O'Reilly Media, Sebastopol, Calif., 2018. ISBN 978-1617294549.

% 18
\bibitem{dockerdocs}DOCKER INC. \emph{Docker Docs} [online]. 2013 [cit. 2024-03-13]. Dostupné z: https://docs.docker.com

% 19
\bibitem{Gammelgaard2021}GAMMELGAARD, C. H. \emph{Microservices for .NET Developers: A Hands-On Guide to Building and Deploying Microservices-Based Applications Using .NET Core}. 2nd ed. Apress, 2021, ISBN 978-1617297922.

% 20
\bibitem{Newman2015}NEWMAN, Sam. \emph{Building microservices}. Sebastopol, CA: O´Reilly Media, [2015]. ISBN 1491950358.

% 21
\bibitem{Sazanavets2022}SAZANAVETS, Fiodar. \emph{Microservice Communication in .NET Using gRPC}. Packt Publishing, 2022. ISBN 978-1-80323-643-8.

% 22
\bibitem{richardsonsaga}RICHARDSON, Chris. \emph{Pattern: Saga}. Online. Dostupné z: https://microservices.io/patterns/data/saga.html. [cit. 2024-05-01].

% 23
\bibitem{Garrison2017}GARRISON, J.; NOVA, K. \emph{Cloud Native Infrastructure: Patterns for Scalable Infrastructure and Applications in a Dynamic Environment}. 1st ed. O'Reilly Media, Sebastopol, Calif., 2017. ISBN 978-1491984307.

% 24
\bibitem{Riedesel2021}RIEDESEL, Jamie. \emph{Software Telemetry: Reliable logging and monitoring}. Manning, 2021. ISBN 978-1617298141.

% 25
\bibitem{Majors2022}MAJORS, Charity; FONG-JONES, Liz a MIRANDA, George. Observability Engineering: Achieving Production Excellence. O'Reilly Media, 2022. ISBN 978-1492076445.

% 26
\bibitem{Molkova2023}MOLKOVA, Liudmila a Sergey KANZHELEV. \emph{Modern Distributed Tracing in .NET}. Packt Publishing, 2023. ISBN 978-1-83763-613-6.

% 27
\bibitem{Blanco2023}BLANCO, Daniel Gomez. \emph{Practical OpenTelemetry: Adopting Open Observability Standards Across Your Organization}. Apress, 2023. ISBN 978-1484290750.

% 28
\bibitem{Chapman2023}CHAPMAN, Rob a Peter HOLMES. \emph{Observability with Grafana}. Packt Publishing, 2023. ISBN 978-1-80324-800-4.

% 29
\bibitem{netdocsnativeguide}MICROSOFT. \emph{Native AOT deployment}. Online. Microsoft Learn. Dostupné z: https://learn.microsoft.com/en-us/dotnet/core/deploying/native-aot/?tabs=net7%2Clinux-ubuntu. [cit. 2024-05-04].

% 30
\bibitem{Saliture2023}SALITURO, Eric. \emph{Learn Grafana 10.x}. Second Edition. Packt Publishing, 2023. ISBN 978-1-80323-108-2.

% 31
\bibitem{Marcotte2024}MARCOTTE, Carl-Hugo. \emph{Architecting ASP.NET Core Applications}. Third Edition. Packt Publishing, 2024. ISBN 9781805123385.

% 32
\bibitem{Akinshin2019}AKINSHIN, Andrey. \emph{Pro .NET Benchmarking}. Apress Berkeley, CA, 2019. ISBN 978-1-4842-4941-3.

% 33
\bibitem{Kokosa2018}KOKOSA, K. \emph{Pro .NET Memory Management: For Better Code, Performance, and Scalability}. For Professionals By Professionals. Apress, New York, 2018. ISBN 978-1484240267.

% 34
\bibitem{Ramel2022}.NET 7 Preview 3 Is All About Native AOT. RAMEL, David. Visual Studio Magazine [online]. 2022 [cit. 2024-03-19]. Dostupné z: https://visualstudiomagazine.com/articles/2022/04/15/net-7-preview-3.aspx

% 35
\bibitem{Martin2018}MARTIN, Robert C. \emph{Clean architecture: a craftsman's guide to software structure and design}. Robert C. Martin series. London, England: Prentice Hall, 2018. ISBN 978-0134494166.

% 36
\bibitem{Lock2021}LOCK, A. \emph{ASP.NET Core in Action}. 2nd ed. Manning Publications, Greenwich, CT, 2021. ISBN 978-1617298301.

% 37
\bibitem{k6manual}PFLB, INC. \emph{User Manual for k6, an Open Source Tool for Load Testing}. PFLB [online]. 2021 [cit. 2024-04-01]. Dostupné z: https://pflb.us/blog/k6-user-manual/

% 38
\bibitem{Garverick2023}GARVERICK, Joshua a Omar Dean MCIVER. Implementing Event-Driven Microservices Architecture in .NET 7. Packt Publishing, 2023. ISBN 978-1-80323-278-2.

% 39
\bibitem{netdocs}MICROSOFT CORPORATION. \emph{.NET Documentation} [online]. [cit. 2024-03-18]. Dostupné z: https://learn.microsoft.com/en-us/dotnet/

% 40
\bibitem{grafanadocs}
RAINTANK, INC. \emph{Grafana Labs - Technical Documentation} [online]. [cit. 2024-03-22]. Dostupné z: https://grafana.com/docs/

% 41
\bibitem{nginxdocs} F5, INC. \emph{NGINX Product Documentation} [online]. [cit. 2024-04-12]. Dostupné z: https://docs.nginx.com

% 42
\bibitem{postgredocs}THE POSTGRESQL GLOBAL DEVELOPMENT GROUP. \emph{PostgreSQL Documentation} [online]. [cit. 2024-04-14]. Dostupné z: https://www.postgresql.org/docs/

% 43
\bibitem{influxdocs}INFLUXDATA, INC. \emph{InfluxDB v1 Documentation} [online]. [cit. 2024-04-14]. Dostupné z: https://docs.influxdata.com/influxdb/v1/

% 44
\bibitem{Esposio2024}ESPOSIO, Dino. \emph{Microservices Design Patterns in .NET}. Pearson Education, 2024. ISBN 978-0-13-820336-8.

% 45
\bibitem{Nickoloff2019}NICKOLOFF, J.; KUENZIL, S. \emph{Docker in Action}. 2nd ed. Manning Publications, Greenwich, CT, 2019. ISBN 978-1617294761.

}

% Pro generování literatury lze alternativně použít i příkaz "\seznamlitbib", 
% který se postará o plnohodnotné vkládání referencí pomocí "bibliography". 
% V takovém případě se využívají bibliografické údaje uložené v souboru 
% tex/literatura.bib. Ty se automaticky upravuji dle zvolené citační normy 
% (v šabloně je nastavena česká norma).
%\seznamlitbib


% ============================================================================ %
% ============================================================================ %
% Encoding: UTF-8 (žluťoučký kůň úpěl ďábelšké ódy)
% ============================================================================ %

\seznamzkr

\begin{tabular}{ll}
PC & Personal Computer \\
OS & Operační Systém \\
HW & Hardware \\
SW & Software \\
CPU & Central Processing Unit \\
RAM & Random Access Memory \\
XML & Extensible Markup Language \\
JIT & Just in Time \\
AOT & Ahead of Time \\
CLI & Command Line Interface \\
CLR & Common Language Runtime \\
IL & Intermediate Language \\
API & Application Programming Interface \\
RPC & Remote Procedure Call \\
SDK & Software Development Kit \\
IDE & Integrated Development Environment \\
GUI & Graphical User Interface \\
\end{tabular}

% ============================================================================ %
 % Seznam zkratek


% ============================================================================ %
\seznamobr  % Seznam je generován automaticky


% ============================================================================ %
\seznamtab  % Seznam je generován automaticky


% ============================================================================ %
% ============================================================================ %
% Encoding: UTF-8 (žluťoučký kůň úpěl ďábelšké ódy)
% ============================================================================ %

\listofappendices

\priloha{Obrázková příloha}

Příloha obsahuje obrázky vytvořeny v průběhu vývoje, testování a analýzy. Obrázky jsou v případě diagramů vytvořeny pomocí nástroje Graphviz a Python knihovny Diagrams mingrammer, v případě dashboardů se jedná o foto obrazovky. \\

\begin{figure}
    \centering
    \includegraphics[width=1\textwidth]{graphics/images/scenario1-dashboard.png}
    \caption{Scénář 1 - Grafana dashboard}
    \label{fig:scenario1dashboard}
\end{figure}

% ============================================================================ %
 % Prilohy


% ============================================================================ %

\end{document}

% ============================================================================ %
